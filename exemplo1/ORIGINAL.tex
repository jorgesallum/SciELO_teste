\selectlanguage{brazilian}
\begin{abstract}
\par O texto ora apresentado enfoca a reflexão acerca dos rebatimentos da desigualdade
 racial nos espaços do sistema educacional, que se configuram fundamentais, tanto para
 a reprodução, quanto para o enfrentamento da condição desigual. Dessa forma,
 considera-se o potencial da escola para o processo de construção da igualdade racial
 e o compromisso ético do Serviço Social na construção de outra ordem societária,
 buscando-se discutir as contribuições da inserção do assistente social na política
 pública de educação.

\par \noindent \\
\footnotesize{\textit{Palavras Chave}: Desigualdade racial, Política pública de educação, Serviço Social, Racial inequality, Public education policy, Social Work}
\par \noindent \\
\footnotesize{Recebido em: 29/07/2014;} \footnotesize{Aceito em: 24/11/2014.} 
\end{abstract}
\section*{Introdução}\par O presente texto, elaborado a partir de construções teóricas que serão apresentadas em
 tese de doutorado, propõe a discussão acerca das contribuições do assistente social no
 enfrentamento da questão racial, expressão da questão social, dentro dos espaços
 escolares.\par O processo de constituição socioeconômica, política e cultural do Brasil,
 intrinsecamente relacionado ao desenvolvimento do sistema do capital, implica
 particularidades, relacionadas especialmente à formação de uma colônia de exploração
 mercantil no país. A invasão dos portugueses ao território, até então ocupado pelas
 inúmeras tribos indígenas, é parte da estratégia para conquista do mercado e
 desenvolvimento do capital mercantil na esfera internacional.\par Essa análise é bastante cara para a reflexão aqui proposta, uma vez que a sociedade
 escravista que se forma no Brasil tem caráter mercantil e, portanto, o seu
 desenvolvimento e sua decadência estão diretamente relacionados aos ciclos de
 expansão/retração do sistema capitalista. Mesmo com a abolição oficial do sistema
 escravista permaneceram elementos característicos dessa configuração, o que vai
 contribuir para a permanência (reconfigurada por vezes) do quadro de desigualdade racial
 que vulnerabiliza, explora e oprime a população negra no Brasil. Tal processo de
 exploração/opressão caracteriza as relações capitalistas de produção.\par Tem-se um sistema que se funda sob as bases da acumulação e, portanto, as instituições
 da sociedade fundada sob essa orientação comprometem-se para assegurar ou legitimar essa
 ordem. A escola (todos os níveis de ensino formalizado) é emblemática para esse
 entendimento, uma vez que se constitui como uma das instituições privilegiadas para a
 formação de quadros que assegurem o fortalecimento do capital, tanto aqueles detentores
 dos meios de produção, quanto os trabalhadores que serão expropriados no processo de
 produção da riqueza. Dentro dessa lógica, na trajetória histórica da política de
 educação brasileira é possível identificar uma escola para a classe que detém os meios
 de produção e outra para a classe que vive do trabalho.\par Contudo, dialeticamente, a escola pode ser também o espaço para as ações com vistas à
 transformação do que está posto e particularmente, conforme enfocado neste texto, para o
 enfrentamento da desigualdade racial.\par Nesse contexto, é possível destacar a inserção do assistente social nos espaços
 escolares, processo em curso no país e que, na análise aqui discutida, irá contribuir no
 enfrentamento proposto, na medida em que a profissão tem como um de seus compromissos
 éticos a defesa de um projeto societário que aponta para a construção de outra ordem
 societária, sem exploração/opressão de classe, gênero, raça e etnia. Dessa forma, a
 partir de suas competências ético-políticas, teórico-metodológicas e técnico-operativas,
 os profissionais de Serviço Social podem contribuir significativamente para o trabalho
 de promoção da igualdade racial dentro dos espaços escolares.\par Partindo desses pressupostos iniciais, o texto ora apresentado busca analisar brevemente
 o processo histórico da desigualdade racial no Brasil e seus rebatimentos no sistema
 educacional, para apontar a escola enquanto instituição reprodutora da desigualdade, mas
 também, dialeticamente, espaço privilegiado para o enfrentamento e, nessa análise,
 destacando-se a inserção profissional do assistente social enquanto agente no
 fortalecimento desse processo.\subsection*{1. Desigualdade racial no Brasil: a escola e o reflexo desse panorama}\par A desigualdade racial no Brasil pode ser visualizada na contemporaneidade a partir de
 diversas dimensões, entre elas, o acesso à educação. A situação de vulnerabilidade
 vivenciada pela população negra tem no âmbito do acesso e permanência no sistema
 educacional uma de suas expressões mais significativas. Considerando inicialmente a
 dimensão da alfabetização, segundo dados sistematizados no \textit{relatório anual
 das desigualdades raciais no Brasil 2009-2010} (Paixão, 2010, p. 207), em
 1988 a taxa de analfabetismo entre a população branca acima de 15 anos era de 12,1\%,
 e em 2008 passa a ser de 6,2\%. Entre a população negra acima de quinze anos, essa
 taxa era de 28,6\% em 1988, e em 2008 passa a ser de 13,6\%. Ou seja, mesmo diante de
 considerável avanço, a taxa de analfabetismo da população negra em 2008 ainda é maior
 do que a registrada entre a população branca em 1988.\par Ampliando a análise para os anos de estudos, o mesmo relatório aponta que em 2008,
 entre os homens brancos com mais de quinze anos de idade, a média de anos de estudo
 foi de 8,2, enquanto entre os homens pretos e pardos com mais de quinze anos de
 idade, a média de anos de estudo foi de 6,3. Já entre as mulheres na mesma faixa
 etária, a média de anos de estudo foi de 8,3 entre as brancas e 6,7 entre as negras
 (pretas e pardas).\par Esse panorama contemporâneo não pode, contudo, ser percebido descolado do processo de
 formação econômica, cultural, social e política do país, que remete ao modo de
 produção escravista mercantil, essencial para o desenvolvimento da \textit{colônia
 Brasil}, mas que se insere também entre os pré-requisitos da eclosão
 capitalista modernizadora (Fernandes, 2011, p.
 363).\par De acordo com a análise de Fernandes (2011) a
 ocupação do território (que depois passa a ser chamado de Brasil) por Portugal deu-se
 não pela necessidade de povoar, mas antes pela premência de produzir, já que a
 escravidão colonial dava suporte material ao capitalismo comercial na Europa. Tal
 escravidão é caracterizada principalmente por ser mercantil, ou seja,\par [...] o escravo não só constitui uma mercadoria; é a principal mercadoria de uma
 vasta rede de negócios (que vai da captura e do tráfico de escravos e à forma de
 trabalho), a qual conta, durante muito tempo, como um dos nervos ou a mola mestra
 da acumulação do capital mercantil. (Fernandes,
 2011, p. 365)\par Identifica-se uma formação onde a colônia (Brasil) é parte do sistema econômico da
 metrópole (Portugal), mas também de toda a rede de centros econômicos mundiais da
 época (Inglaterra, por exemplo) e, nessa análise, o proprietário do escravizado não
 era o detentor exclusivo do excedente gerado pelo trabalho. Tal excedente entrava no
 sistema de apropriação do capital mercantil, constituído por dimensões não apenas
 econômicas, mas legais, políticas e fiscais. Mesmo sendo o escravizado "propriedade
 econômica do senhor", a escravidão mercantil não era um "negócio privado", mas tinha
 como finalidade produzir e reproduzir rendimentos econômicos gerados nas transações
 comerciais. A esses rendimentos, Fernandes
 (2011) denomina butim, cujo significado remete aos bens materiais tomados
 de escravos ou prisioneiros durante um ataque ou guerra, ou ainda o resultado de um
 roubo, de uma pilhagem.\par Esse butim, no plano em que se dava a partilha colonial dos frutos da pilhagem,
 perdia qualquer ligação com suas origens. Aí, nem a produção escravista nem a
 propriedade do senhor contavam para qualquer efeito. O que importava eram as
 "mercadorias" e as "riquezas" que entravam, através desse singular rateio -
 provavelmente o mais odioso tipo de pilhagem da história humana - na circulação
 engendrada pelo capital mercantil (Fernandes,
 2011, p. 373).\par Com a transição do período colonial (onde se constituem as bases para a formação de
 um capitalismo dependente no Brasil), a escravidão mercantil continua a existir,
 porém os senhores agora têm como desafio adequar a força de trabalho escravo às
 configurações das formas de produção, que já não se limitavam às plantações. Esse
 desenvolvimento econômico, ao mesmo tempo em que dependente do sistema escravista,
 traz também a necessidade de outras formas de trabalho, contribuindo para a
 desagregação do sistema que existia e que até então havia possibilitado a formação da
 riqueza da colônia e, principalmente, do Império, cumprindo a função de fator de
 acumulação de capital.\par Historicamente, esse fator tornou-se ineficaz, sendo substituído por formas de
 acumulação mais apropriadas e rentáveis ao desenvolvimento do capital, o que vai
 "condenar" a escravidão mercantil ao desaparecimento (Fernandes, 2011, p. 420-421).\par Entre os elementos que se destacam nessa análise, é possível apontar que a
 escravização da população negra no Brasil não apenas serviu, mas foi ainda
 fundamental, para o desenvolvimento do sistema capitalista no país e que, portanto,
 as desigualdades apontadas anteriormente constroem-se e reconstroem-se dentro e
 intrinsecamente relacionadas a tal sistema.\par Nesses termos, o enfrentamento da desigualdade racial na peculiaridade brasileira
 precisa ser compreendido no contexto de superação da ordem social vigente, assentada
 sobre as bases do capital, e as ações a serem desenvolvidas são sempre possibilidades
 presentes, remetendo a um projeto macro de construção de outra ordem societária.
 Assim, não são apenas ações culturais, sociais, mas econômicas e políticas, estando
 imbricadas no ideário de transformação da realidade que está posta.\par Tal construção articula-se às mais diversas e amplas lutas sociais, e é a partir do
 horizonte ético da emancipação que acreditamos na contribuição que pode se originar
 no contexto da escola, compreendida na perspectiva da concepção de Gramsci (Nosella, 2010) como instituição responsável pelo
 ensino formal e que, nesse texto, refere-se à escola pública (em todos os níveis)
 dirigida pelos pressupostos da política pública de educação.\par A concepção de escola que possibilita a reflexão acerca da igualdade racial é aquela
 onde a educação não é puramente tecnicista, utilitária, ou, citando Gramsci (apud
 Nosella, 2010, p. 50), uma "escola
 interessada".\par Trata-se antes de uma "escola desinteressada", integrada, unitária, onde apreende-se
 tanto aspectos gerais do desenvolvimento da vida em sociedade (cultura geral), quanto
 preparação para o mercado de trabalho. Ou seja, não se trata da negação pura e
 simples da escola técnica, aquela destinada a preparar para o mercado de trabalho,
 mas antes da necessidade de que essa escola se construa na unidade, na integração
 entre espaço escolar e realidade objetiva.\par Todavia, o que é possível apontar é a existência de uma dicotomia entre o que se
 trabalha através do currículo escolar e a realidade concreta dos educandos. Tal cisão
 é sobremodo evidente na escola brasileira. A formação histórico-cultural do país,
 assentada sobre o entendimento da superioridade de uns sobre outros, transforma a
 riqueza da diversidade em abismal desigualdade, e nesse horizonte, a narrativa
 histórica exclui e/ou torna invisíveis aqueles de quem é tirado o direito de
 expressar-se e contar a história da qual foram parte, para qual contribuíram, e que
 só é assim porque eles dela participaram.\par No Brasil, em todas as dimensões de sua constituição de nação, estão presentes
 aspectos fundamentais das culturas que compuseram essa gente, esses costumes, essa
 língua portuguesa cheia de \textit{milongas, calundus, batuques, abacaxis e
 torós}. Nesse território desenvolve-se uma forma de civilização que, ainda
 que buscasse copiar a metrópole, não era extensão de Portugal; mesmo com o elevado
 número de africanos, não era um "pedacinho de África", e também já não era mais a
 terra milenarmente ocupada pelos indígenas.\par A partir dessa análise, e considerando inicialmente as influências africanas,
 indígenas e portuguesas, apontar no Brasil apenas uma delas como preponderante é
 desconsiderar todo o complexo sobre o qual foram tecidas as cores e formas desta
 sociedade. A história repetida nas versões oficiais relega indígenas e africanos a
 segundo plano, confina os homens africanos às lavouras "do lado de fora", e as
 mulheres "às cozinhas", como se indígenas e africanos houvessem apenas dado "uma
 contribuição" para a formação brasileira, e não efetivamente povos que contribuíram
 para que o Brasil fosse desse jeito e não de outro.\par Além disso, essa história assim contada e recontada reduz toda a diversidade étnica
 das inúmeras tribos que habitavam o país a uma única denominação, os "índios", e
 todas as nações africanas trazidas forçadamente para o Brasil como os "escravos".\par Havia no continente africano uma multiplicidade cultural significativa, e mesmo entre
 os africanos trazidos para o Brasil havia inúmeros dialetos, culturas, religiões.
 Reduzir essa riqueza à denominação única de "escravos africanos" é a primeira
 demonstração do quanto a presença negra foi compactada, desfocada e diminuída na
 história do país.\par Tal desfoque é evidenciado no currículo escolar, que privilegia a "história oficial"
 e, portanto, não negra e não indígena. O que se tem assistido no Brasil é a
 constituição de uma escola (desde a pré-escola até a universidade) para a "classe que
 vive do trabalho" e a continuidade daquela destinada a classe historicamente
 detentora dos meios de produção.\par Na crítica de Gramsci (apud Nosella, 2010, p.
 50) é preciso a escola desinteressada, aquela que permite o desenvolvimento integral
 do educando, ou seja,\par uma escola que dê à criança a possibilidade de se formar, de se tornar homem, de
 adquirir aqueles critérios gerais necessários para o desenvolvimento do caráter.
 Uma escola humanista, em suma, assim como a entendiam os antigos e mais próximos
 homens do Renascimento. Uma escola que não hipoteque o futuro do garoto, nem
 obrigue sua vontade, sua inteligência, sua consciência e informação a se mover na
 bitola de um trem com estação marcada. Uma escola de liberdade e livre-iniciativa,
 e não uma escola de escravidão e de mecanicidade.\par Essa é a concepção de escola que contempla a igualdade racial e mais, apenas nessa
 escola é possível essa construção. O educando deve ter acesso a saberes que lhe
 permitam situar-se no mundo de forma que, as relações estabelecidas estejam
 assentadas sob o ideário humanista e, assim, intrinsecamente relacionadas à
 igualdade.\par Todavia, essa proposta não pode ser limitada a uma ou outra instituição de ensino que
 resolva adotar um currículo diferenciado, ou mesmo cumprir efetivamente dispositivos
 legais como a Lei n. 10.639, de 9 de janeiro de 2003, que estabelece a
 obrigatoriedade da temática "História e cultura africana" no currículo oficial da
 educação no Brasil. Tal concepção de escola só se efetiva enquanto elemento de
 transformação da realidade que está posta a partir da primazia da responsabilidade do
 Estado na condução dessas ações. Ou seja, torna-se necessário uma política pública de
 educação que contemple em sua formulação, implementação, execução e avaliação,
 aspectos que remetam à igualdade racial.\par Sem a pretensão de oferecer respostas ou medidas prontas para serem implementadas, a
 proposta deste texto é a busca por refletir sobre possibilidades e caminhos a serem
 construídos a partir da escola, e é nesse sentido que a inserção do assistente social
 na política pública de educação pode possibilitar algumas trajetórias.\subsection*{2. Igualdade racial e a inserção do Serviço Social na política pública de
 educação no Brasil}\par A proposta de educação que possibilite o enfrentamento da desigualdade racial remete
 à reflexão acerca das significações do conceito de igualdade na sociedade que se
 constitui sobre os fundamentos do modo de produção capitalista. Não se trata da
 igualdade cujo propósito maior seja permitir que todos consumam igualmente. Trata-se
 antes do entendimento de que todos os seres humanos têm direitos iguais, e devem ser
 asseguradas condições objetivas para que possam acessar e usufruir desses
 direitos.\par Igualdade deveria constituir-se, dessa forma, princípio para a vida social, mas é
 possível observar que, historicamente, foi se transfigurando e reduzindo-se apenas à
 busca de garantia para condições iguais de consumo. A reflexão acerca da necessidade
 de uma política de educação para igualdade assenta-se sobre a perspectiva do acesso e
 usufruto de direitos, que são comuns a todos os seres humanos, mas que na sociedade
 contemporânea estão acessíveis a alguns e inalcançáveis para outros, o que
 objetivamente torna aqueles que têm acesso aos direitos, "sujeitos sociais", e
 aqueles que não os têm, invisíveis.\par O processo histórico cultural de formação da sociedade brasileira, as configurações
 do modo de produção capitalista, as transformações ocorridas no mundo do trabalho,
 entre outras dimensões, contribuem para a construção do que denomina-se desigualdade
 racial, que no país desprivilegia a população negra, mesmo que a vigente Constituição
 da República Federativa do Brasil estabeleça que todos são iguais.\par A igualdade estabelecida na lei não existe de fato, o que implica o processo
 histórico de exclusão de determinados segmentos da população do direito de
 participação, de decisão ou, ainda, da liberdade, entendida minimamente como
 possibilidade concreta de escolha.\par A instituição da escola no Brasil é um processo marcado, desde suas origens, pela
 desigualdade, que também historicamente se visualiza no país. Tem-se como marco do
 processo de sistematização da oferta de ensino no Brasil a presença dos jesuítas
 através da Companhia de Jesus, que em 1570 já contava com cinco escolas de nível
 elementar e três de nível médio, perfazendo oito estabelecimentos de ensino, voltados
 exclusivamente para a educação dos homens e com o direcionamento para o sacerdócio ou
 a advocacia (Cotrim e Parisi, 1982, p.
 260).\par A partir da segunda metade do século XVIII, com o avanço dos ideais do Iluminismo por
 toda a Europa, Portugal busca modernizar a aprendizagem tanto nas metrópoles quanto
 nas colônias. Assim, torna o ensino mais utilitário, buscando diminuir o
 direcionamento da Companhia de Jesus, o que implicou a expulsão dos jesuítas dos
 domínios de Portugal em 1759.\par Todavia, no Brasil, efetivamente não alterou o panorama educacional, visto que a
 estrutura da Companhia de Jesus já estava solidificada no país, e Portugal não
 ofereceu nenhuma outra proposta significativa, permanecendo o sistema educacional sem
 grandes alterações durante todo o período imperial. Cotrim e Parisi (1982, p. 267) destacam:\par Era uma educação de fachada, ornamental, acadêmica, desvinculada da realidade
 social. A instrução era apenas uma forma de demonstração de
 \textit{status}, servindo, quando muito, para auxiliar a ascensão do
 indivíduo ao exercício da atividade política.\par O processo da Revolução de 1930, na esteira da Proclamação da República em 1889,
 marca a história do Brasil, visto que a partir dele muda-se de forma profunda e
 definitiva a face do país, que nesse período inicia sua transição de uma sociedade
 predominantemente agrária para uma nação industrial e moderna (Aggio, Barbosa e Coelho, 2002, p. 16).\par Destaca-se nesse período a criação do então Ministério da Educação e Saúde (1937) e a
 difusão dos novos ideais para a educação, particularmente por meio das Constituições
 federais de 1934 e 1937, que asseguraram que os brasileiros tinham o direito a
 receber pela família e poder público, a educação "elementar" e o ensino técnico e
 profissionalizante.\par A ampliação das vagas nas escolas com a progressiva expansão dos níveis de ensino
 constitui processo em curso no país, especialmente a partir dos anos 1980, no período
 pós-ditadura militar, quando se busca assegurar para todas as crianças brasileiras o
 acesso à escola, oportunizando inclusive condições objetivas para tal, como merenda
 escolar, livro didático. A partir de meados da década de 1990 pode-se verificar que
 essa "preocupação" estende-se também para os adolescentes, no sentido de garantir a
 ampliação dos anos de estudo para a população brasileira. E finalmente, a partir dos
 anos 2000, é possível observar que amplia-se o acesso ao ensino superior,
 principalmente pela expansão das instituições privadas de ensino, bem como a
 implementação de programas estatais para garantia de acesso à universidade (sobretudo
 às instituições privadas) para jovens oriundos de famílias de baixa renda.\par Todavia, essa ampliação do acesso a todos os níveis de ensino formal no Brasil leva
 às indagações acerca da qualidade dessa oferta. Ou seja, buscou-se ampliar o número
 de vagas, mas um dos questionamentos é se esse crescimento foi acompanhado da
 qualidade do ensino ou se foram reduzidos os investimentos em qualificação
 profissional, material de apoio, condições físicas das salas de aula, para garantir
 apenas o aumento do número de vagas.\par O que significou a ampliação das oportunidades de acesso? É preciso pensar que por
 vezes, sob o discurso da garantia de maior número de vagas, reduziram-se os
 investimentos nas demais dimensões, o que implicou a queda significativa da qualidade
 do ensino.\par Contudo, a reflexão acerca da qualidade na educação não se limita ao aspecto da
 ampliação do número de vagas. Os comparativos (quase sempre assentados sob a
 perspectiva do senso comum) nos quais busca-se cotejar a educação formal ofertada na
 contemporaneidade, com aquela de décadas atrás, acabam por incorrer no erro de não
 considerar o processo histórico onde se insere a educação formal no Brasil. Nas
 décadas de 1970, 1980, propunha-se uma educação que desse conta da demanda da
 sociedade daquele período, e esperar ou propor a educação naqueles mesmos moldes,
 muito mais que um saudosismo deslocado, implica uma leitura com equívocos
 significativos.\par O processo de modernização vivenciado na sociedade brasileira (em todas as suas
 dimensões) não comporta os moldes de educação das décadas passadas. Por outro lado,
 não se encontram alternativas/propostas para a oferta de um sistema de educação que
 garanta oportunidades iguais de desenvolvimento, de construção de cidadania, de
 entendimento do humano para além da perspectiva utilitarista. Daí a necessidade da
 compreensão da amplitude do significado de \textit{construção da igualdade}
 na contemporaneidade e que no espaço da escola é demasiadamente complexa e ampla para
 limitar-se às relações entre educador e educando, ao lócus exclusivo da sala de
 aula.\par O espaço escolar enquanto possibilidade para a igualdade racial não pode limitar-se à
 ação desses dois agentes (educador e educando), podendo e devendo ser ampliado e
 aberto para a contribuição de outras áreas, entre elas o Serviço Social.\par A inserção do Serviço Social no contexto da escola no Brasil pode ser discutida
 dentro da concepção da política pública de educação, e essa reflexão não é recente no
 âmbito da profissão. Contudo, ganha destaque apenas a partir do início de 2000, com a
 elaboração de pareceres, formação de comissões para estudos e reflexões, grupos de
 trabalho, além da realização do Seminário Nacional de Serviço Social na Educação
 (2012) e da elaboração do documento intitulado \textit{Subsídios para a atuação de
 assistentes sociais na política de educação} (CFESS, 2012).\par No espaço da política pública de educação, os assistentes sociais devem atuar em
 consonância com o projeto ético-político da profissão, na defesa da igualdade,
 princípio fundamental para a proposta emancipatória defendida pelo Serviço
 Social.\par Conforme Martins (2012, p. 45), o trabalho
 profissional do Serviço Social nos espaços da política pública de educação ocorre a
 partir de três eixos: a dimensão socioeducativa da profissão, a democratização da
 educação e a articulação entre essa política e as demais. Na análise aqui proposta,
 ganha destaque a dimensão socioeducativa do trabalho do Serviço Social, uma vez que o
 profissional, a partir desse aspecto, pode de forma efetiva intervir no processo de
 enfrentamento da desigualdade racial.\par O assistente social na Educação poderá atuar com todos os membros da comunidade
 escolar, tendo a possibilidade de mobilizar um processo reflexivo que envolve a
 percepção objetiva da vida social, e da vida de cada indivíduo e das condições
 sociais e históricas que norteiam a sociedade. Esta atividade propicia a
 politização em torno de diversos temas que perpassam o ambiente escolar e social
 (Martins, 2012, p. 46).\par A mobilização para a reflexão, apontando para a percepção das condições sociais e
 históricas, conforme destaca Martins no trecho citado, indica, em nossa percepção, as
 possibilidades para as contribuições do trabalho do assistente social no processo de
 enfrentamento da desigualdade racial no espaço da escola. É nesse processo de
 mobilização, reflexão e politização que o Serviço Social constrói as estratégias para
 o enfrentamento da questão racial (expressão da questão social, objeto de trabalho da
 profissão) no ambiente escolar, que se configura enquanto espaço reprodutor da
 desigualdade, mas também lócus privilegiado para a negação dessa realidade e
 construção do novo.\par Nessa dimensão, a proposta de educação que orienta a práxis dos assistentes sociais
 não pode ser aquela que reproduz as relações sociais presentes na sociedade, mas,
 antes, uma educação orientada para a promoção do ser humano enquanto sujeito
 coletivo.\par A educação pode ser considerada um espaço privilegiado para o enriquecimento ou
 empobrecimento do gênero humano. Assim, na perspectiva de fortalecimento do
 projeto ético-político, o trabalho do/a assistente social na política de educação
 pressupõe a referência a uma concepção de educação emancipadora, que possibilite
 aos indivíduos sociais o desenvolvimento de suas potencialidades e capacidades
 como gênero humano (CFESS, 2012).\par Com esse direcionamento, o profissional de Serviço Social, ao inserir-se nos
 diferentes espaços sócio-ocupacionais que demarcam os diversos níveis e modalidades
 de ensino, no âmbito da política de educação, deve voltar-se para a garantia da
 qualidade que vai além de estratégias para permanência na escola, diminuição de
 índices de repetência, mesmo de alfabetização formal.\par A qualidade da educação defendida pelo Serviço Social está intrinsecamente
 relacionada com a construção de outra ordem societária e, portanto, não pode
 prescindir do compromisso profissional do assistente social com a emancipação,
 enquanto demanda política inerente a liberdade, valor ético central, conforme
 preconiza o Código de Ética Profissional.\par A qualidade da educação, aqui referida, ao mesmo tempo em que envolve uma densa
 formação intelectual com domínio de habilidades cognitivas e conteúdos formativos,
 também engloba a produção e disseminação de um conjunto de valores e práticas
 sociais alicerçadas no respeito à diversidade humana e aos direitos humanos, na
 livre orientação e expressão sexual, na livre identidade de gênero, de cunho não
 sexista, não racista e não homofóbica/lesbofóbica/transfóbica, fundamentais à
 autonomia dos sujeitos singulares e coletivos e ao processo de emancipação humana
 (CFESS, 2012).\par A escola no Brasil que visa à qualidade não pode, considerando as peculiaridades de
 formação do país, abrir mão do trabalho que objetive o enfrentamento da desigualdade
 racial e que aponta para a emancipação humana. O trabalho profissional do assistente
 social relaciona-se a esse compromisso qualitativo, na medida em que, comprometido
 com o projeto ético-político da profissão, caminha na construção de outra ordem
 societária. No âmbito da discussão proposta neste texto, considera-se fundamental a
 inserção do Serviço Social na política pública de educação, assegurando as condições
 necessárias para que se desenvolva uma proposta de articulação dos diversos atores do
 lócus escolar.\par Pensar o trabalho profissional nesse espaço e com esse direcionamento requer o
 questionamento acerca das competências profissionais, uma vez que o trabalho
 profissional deve estar comprometido com a \textit{defesa intransigente dos direitos
 humanos e a recusa do arbítrio e do autoritarismo}.\par Esse compromisso requer a \textit{defesa do aprofundamento da democracia, enquanto
 socialização da participação política e da riqueza socialmente produzida; o
 posicionamento em favor da equidade e justiça social, assegurando a universalidade
 de acesso os bens e serviços relativos aos programas sociais, bem como sua gestão
 democrática}. Esses princípios destacados levam à reflexão acerca do
 compromisso ético-profissional, que não é diferenciado no lócus da política pública
 de educação e que pressupõem competências que se relacionam às dimensões teórica,
 técnica, ética e política do trabalho profissional, desenvolvidas mesmo no cenário de
 determinações colocadas pelas configurações atuais do sistema do capital e seus
 rebatimentos nas relações não apenas econômicas, mas também sociais, políticas e
 culturais.\par Segundo Iamamoto (1998), a realidade atual da
 globalização, a alteração das relações entre Estado e sociedade civil, o agudizamento
 da questão social em suas múltiplas manifestações provocam a profissão para a
 intervenção crítica frente à realidade. Assim, o Serviço Social encontra-se às voltas
 com o desafio de desenvolver o trabalho profissional articulado às atuais
 configurações da sociedade, o que significa ampliar também os espaços para a práxis
 profissional transformadora, entre eles o lócus escolar.\par Considerando que o trabalho profissional do assistente social desenvolve-se nessa
 realidade que se modifica constantemente, ele só tem sentido na história da sociedade
 na qual está inserido, sendo que "pensar o Serviço Social na contemporaneidade requer
 os olhos abertos para o mundo contemporâneo para decifrá-lo e participar de sua
 recriação" (Iamamoto, 1998, p. 19). As
 transformações ocorridas e ainda em curso na realidade brasileira exponenciam o
 desafio de decifrar a realidade para nela intervir de forma crítica, colocando para o
 Serviço Social, enquanto categoria profissional comprometida com a transformação
 social, a necessidade de cada vez mais desenvolver seu trabalho de forma radicalmente
 articulada com as demandas da contemporaneidade.\par Dessa forma, delineia-se o trabalho do assistente social inserido na política pública
 de educação e que pode ter, como uma das dimensões de sua intervenção, ações
 direcionadas para o enfrentamento da desigualdade racial, assentadas sob a
 perspectiva da emancipação humana em uma análise que entende os limites de tal
 construção na ordem societária burguesa vigente, mas que também visualiza as
 possibilidades que tal ação representa.\par Emancipação configura-se assim como necessidade, ponto de partida/chegada no processo
 de construção da ordem societária com valores distintos do capital, na medida em que
 potencializa protagonismos sociopolíticos dos sujeitos em condição de subalternidade.
 Trata-se ainda da necessidade de instituição de outros padrões éticos para a vida em
 sociedade, de valores que estejam fundamentados na condição humana como elementos
 essenciais para a construção de projetos societários.\par Para o Serviço Social evidencia-se, portanto, a necessidade de se garantir direitos,
 implementar políticas, desenvolver ações que tenham como compromisso a alteração do
 quadro de relações raciais desiguais. Esse é um dos compromissos expressos no projeto
 ético-político da profissão e que encontra no espaço das instituições escolares
 possibilidades de intervenção frente à realidade.\par Conforme destacado em \textit{Subsídios para atuação de assistentes sociais na
 política pública de educação} (CFESS,
 2012), a educação configura-se como elemento essencial na construção de uma
 sociedade justa e igualitária, defendida pelo Serviço Social. Todavia, tem sido
 historicamente utilizada para a manutenção da ordem hegemônica do capital, mas pode
 também servir de estratégia para possibilitar a contra-hegemonia, e é nesse contexto
 que o Serviço Social encontra os elementos para desenvolver sua práxis profissional,
 que, conforme apontado nesse texto, pode contribuir para o enfrentamento da
 desigualdade racial.\section*{Conclusão}\par A aproximação do Serviço Social com a política pública de educação não é recente,
 podendo-se apontar que, desde a sua institucionalização, a profissão está tanto inserida
 nos espaços escolares (ainda que de início voltada para a prática assistencialista),
 como participando das discussões e movimentos para elaboração, implementação e execução
 da referida política. Contudo, é a partir dos anos 2000 que o Serviço Social passa a
 discutir de forma mais efetiva a inserção profissional do assistente social na política
 pública de educação.\par Dessa forma, é possível apontar avanços significativos na reflexão que a categoria
 profissional de assistentes sociais vem construindo acerca da inserção nessa política
 pública. Todavia, é preciso ressaltar que a trajetória de consolidação da educação,
 enquanto política pública, é ainda repleta de avanços e retrocessos.\par É nessa trajetória de conquistas e limitações da política de educação que se insere a
 discussão acerca do enfrentamento da desigualdade racial e o trabalho do assistente
 social. Pensar essa construção é, portanto, necessidade da sociedade brasileira como
 está constituída na contemporaneidade e que dessa forma deve estar presente nas
 propostas que se relacionam com a política pública de educação.
\balance
\pagebreak\onecolumn
\begin{biblio}[Referências bibliográficas]
\tit{AGGIO,} A.; BARBOSA, A. S.; COELHO, H. M. F. Política e sociedade no
 Brasil (1930-1964). São Paulo: Annablume, 2002.
\tit{COTRIM,} G. V.; PARISI, M. Fundamentos da educação: história e filosofia
 da educação. 6. ed. São Paulo: Saraiva, 1982.
\tit{CFESS }(Conselho Federal de Serviço Social). Subsídios para a atuação dos
 assistentes sociais na política de educação. Brasília: CFESS, 2012.
\tit{FERNANDES,} F. A sociedade escravista no Brasil. 1976. In: IANNI, O.
 (Org.). Florestan Fernandes: sociologia crítica e militante. 2. ed. São Paulo:
 Expressão Popular, 2011.
\tit{IAMAMOTO,} M. V. O Serviço Social na contemporaneidade: trabalho e
 formação profissional. São Paulo: Cortez, 1998.
\tit{MARTINS,} E. B. C. O Serviço Social no âmbito da política educacional:
 dilemas e contribuições da profissão na perspectiva do projeto ético político. In:
 SILVA, M. M. J. (Org.). Serviço Social na educação: teoria e prática. Campinas: Papel
 Social, 2012.
\tit{NOSELLA,} P. A escola de Gramsci. 4. ed. São Paulo: Cortez,
 2010.
\tit{PAIXÃO,} Marcelo et al. Relatório anual das desigualdades raciais no
 Brasil 2009-2010. Rio de Janeiro: Garamond, 2010.
\end{biblio}

\medskip\par\noindent
\footnotesize{This is an Open Access article distributed under the terms of the Creative
 Commons Attribution Non-Commercial License which permits unrestricted
 non-commercial use, distribution, and reproduction in any medium, provided the
 original work is properly cited.}
