\selectlanguage{english}
\noindent{}HERRERO, M.B., 
ARROSSI, S., 
RAMOS, S., 
BRAGA, J.U. \textbf{Spatial analysis of the tuberculosis treatment dropout, Buenos Aires, Argentina} Revista de Saúde Pública. 49(00): 1-9. \url{http://dx.doi.org/10.1590/S0034-8910.2015049005391}\bigskip{}

{
\renewcommand{\abstractname}{Abstract}
\begin{abstract}
\begin{spacing}{0.93}
 \textbf{OBJECTIVE:}
 Identify spatial distribution patterns of the proportion of nonadherence to tuberculosis treatment and its associated factors.
\par{} \textbf{METHODS:}
 We conducted an ecological study based on secondary and primary data from municipalities of the metropolitan area of Buenos Aires, Argentina. An exploratory analysis of the characteristics of the area and the distributions of the cases included in the sample (proportion of nonadherence) was also carried out along with a multifactor analysis by linear regression. The variables related to the characteristics of the population, residences and families were analyzed.
\par{} \textbf{RESULTS:}
 Areas with higher proportion of the population without social security benefits (p = 0.007) and of households with unsatisfied basic needs had a higher risk of nonadherence (p = 0.032). In addition, the proportion of nonadherence was higher in areas with the highest proportion of households with no public transportation within 300 meters (p = 0.070).
\par{} \textbf{CONCLUSIONS:}
 We found a risk area for the nonadherence to treatment characterized by a population living in poverty, with precarious jobs and difficult access to public transportation.

\end{spacing}
\vspace*{1.9mm}
\fontsize{9}{10.8}\selectfont{\textit{Keywords:} Tuberculosis, drug therapy, Medication Adherence, Socioeconomic Factors, Health Inequalities, Ecological Studies}
\end{abstract}
}

\noindent{}HERRERO, M.B., 
ARROSSI, S., 
RAMOS, S., 
BRAGA, J.U. \textbf{Análisis espacial del abandono del tratamiento de tuberculosis, Buenos Aires, Argentina} Revista de Saúde Pública. 49(00): 1-9. \url{http://dx.doi.org/10.1590/S0034-8910.2015049005391}\bigskip{}

{
\renewcommand{\abstractname}{Resumen}
\begin{abstract}
\begin{spacing}{0.93}
 \textbf{OBJETIVO:}
 Identificar patrones de distribución espacial de la proporción de la no-adherencia al tratamiento de la tuberculosis y sus factores asociados.
\par{} \textbf{METODOS:}
 Estudio ecológico con datos secundarios y primarios en municipios seleccionados del Área Metropolitana de Buenos Aires. Se realizó un análisis exploratorio de las características del área y de las distribuciones de los casos incluidos en la muestra (proporción de no-adherencia) y un análisis de múltiples factores por regresión lineal. Se analizaron variables referidas a las características de la población, las viviendas y los hogares.
\par{} \textbf{RESULTADOS:}
 Las áreas con mayor proporción de población que no realizaba aportes jubilatorios (p = 0,007) y con mayor proporción de hogares con necesidades básicas insatisfechas según capacidad de subsistencia presentaron mayor riesgo de no-adherencia (p = 0,032). La proporción de no-adherencia fue más elevada en las áreas con mayor proporción de viviendas sin servicio de transporte público a menos de 300 m (p = 0,070).
\par{} \textbf{CONCLUSIONES:}
 Existe un área de riesgo para la no-adherencia al tratamiento, caracterizada por tener una población que vive en condiciones de pobreza y precariedad laboral, con dificultades de acceso al servicio de transporte público.

\end{spacing}
\vspace*{1.9mm}
\fontsize{9}{10.8}\selectfont{\textit{Keywords:} Tuberculosis, quimioterapia, Cumplimiento de la Medicación, Factores Socioeconómicos, Desigualdades en la Salud, Estudios Ecológicos}
\end{abstract}
}

\vspace*{-1.6mm}
{\noindent\fontsize{9}{10.8}\selectfont{Received: 18/2/2014;
Accepted: 2/11/2014.}}
\begin{multicols}{2}
\section*{INTRODUCTION}
\par{}Although tuberculosis (TB) is a curable disease that can be prevented, it is an important public health issue in Argentina. Each year, more than 10,000 new cases and more than 800 deaths caused by this disease are reported. The geographic distribution of TB in the country is not uniform as in the rest of the world.\protect\footnote{ Instituto Nacional de Enfermedades Respiratorias “Dr. Emilio Coni”. Notificación de casos de tuberculosis en la República Argentina. Período 1980-2011. Buenos Aires: Ministerio de Salud; 2012.} The nonadherence to treatment is considered one of the main obstacles for the control of the disease due to the consequences of its discontinuation, associated with the social vulnerability of patients.\textsuperscript{\textsuperscript{7}}\par{}TB persists as a public health problem, despite the low cost of its diagnosis and treatment. These measures are part of the strategy of the directly observed treatment, short-course (DOTS) recommended by the World Health Organization (WHO) to reduce the nonadherence to the treatment,\textsuperscript{\textsuperscript{25}} which were adopted in Argentina and implemented with the \textit{Programa Nacion}\textit{al de Control de la }\textit{Tuberculosis} (PNCTB – National Tuberculosis Control Program).\protect\footnote{ Zerbini EV, Darnaud RMH, Prieto VG. Programa Nacional de Control de la Tuberculosis: Normas Técnicas 2008. 3. ed. Santa Fé: Instituto Nacional de Enfermedades Respiratorias Dr. Emilio Coni; 2008.} Although the implementation of the DOTS strategy has been carried out in the country for the last 10 years, the proportion of cases who have gave up treatment was 12.0\%\allowbreak{} in 2010, one of the highest in recent years.\protect\footnote{ Instituto Nacional de Enfermedades Respiratorias “Dr. Emilio Coni”. Resultado del tratamiento de la tuberculosis pulmonar ED(+) en la República Argentina. Período 1980-2010. Buenos Aires: Ministerio de Salud; 2012.}\par{}Studies address the treatment adherence from a focus based on environmental factors\textsuperscript{\textsuperscript{12}}\textsuperscript{,}\textsuperscript{\textsuperscript{24}} and the individual factors related to the patient.\textsuperscript{\textsuperscript{1}}\textsuperscript{,}\textsuperscript{\textsuperscript{4}}\textsuperscript{,}\textsuperscript{\textsuperscript{5}}\textsuperscript{,}\textsuperscript{\textsuperscript{13}}\textsuperscript{,}\textsuperscript{\textsuperscript{16}}\textsuperscript{,}\textsuperscript{\textsuperscript{18}}\textsuperscript{,}\textsuperscript{\textsuperscript{22}}\par{}The occurrence of TB and its consequences to health are related to the social conditions.\textsuperscript{\textsuperscript{20}} To understand its behavior in a territory and its determinants it is essential to establish equitable actions that aim at reducing inequalities and improve adherence to the treatment. \protect\footnote{ Acosta LSW. O mapa de Porto Alegre e a tuberculose: distribuição espacial e determinantes sociais [dissertation]. Porto Alegre (RS): Faculdade de Medicina da UFRGS; 2008.} The ecological studies aims to identify, based on social characteristics and on the territory, relations with the distribution of diseases and health outcomes, considering the different hierarchical levels of the determinants.\textsuperscript{\textsuperscript{3}}\textsuperscript{,}\textsuperscript{\textsuperscript{9}}\textsuperscript{,}\textsuperscript{\textsuperscript{14}}\par{}Despite the importance of such studies, in Argentina no studies can be found about the characteristics of social groups and the area where they live and the relationship with the nonadherence to the TB treatment.\par{}The objective of this study was to identify patterns of spatial distribution of the proportion of nonadherence to the tuberculosis treatment and its associated factors.
\section*{METHODS}
\par{}This spatial-ecological study was conducted in seven municipalities of the Sixth Health Region (6\textsuperscript{th} HR) in the Buenos Aires metropolitan area (BAMA) (where there are 116 census fractions – Figure 1): Almirante Brown, Avellaneda, Berazategui, Esteban Echeverría, Ezeiza, Lomas de Zamora and Quilmes. The two other municipalities that are also part of the 6\textsuperscript{th} HR (Lanus and Florencio Varela) could not be included because the locations did not have an Ethics Committee to evaluate the cross-sectional protocol of the study that provides the georeferenced cases (adherence and nonadherence).\protect\footnote{ Arrossi S, Herrero MB, Faccia K, Greco A, Ramirez Lijó S, Aizemberg L et al. Evaluación de los factores predictivos de la no-adherencia al tratamiento de la tuberculosis en municipios seleccionados del área metropolitana de Buenos Aires: estudio colaborativo multicéntrico. Buenos Aires: Ministerio de Salud de la Nación; 2008 (ECM 2008).}\par{}
\par
{
\centering{
\includegraphics[width=\maxwidth{0.5\textwidth}]{not-found.png}
}
\captionof{figure}{\textbf{Figure 1:} \textit{Study area: selected municipalities from Sixth Health Region (6th HR) and census fractions. Buenos Aires, Argentina, 2001.}} 
}
\par
\par{}The 6\textsuperscript{th} HR has about 3,653,000 inhabitants, and it is the most populated region of Buenos Aires.\protect\footnote{ Ministerio de Salud de la Provincia de Buenos Aires (ARG). Diagnóstico de las Regiones Sanitarias 2007-2008, La Plata, Buenos Aires; 2008.} It also concentrates 13.0\%\allowbreak{} of all reported cases of TB in the country and it is the sanitary region that has the largest number of TB cases in the province every year, with the highest dropout index (25.0\%\allowbreak{}) and the lowest DOTS coverage (12.0\%\allowbreak{}).\protect\footnote{ Instituto Nacional de Enfermedades Respiratorias “Dr. Emilio Coni”. Resultado del tratamiento de la tuberculosis pulmonar ED(+) en la República Argentina. Período 1980-2010. Buenos Aires: Ministerio de Salud; 2012.}\par{}The database and mapping of the National Census of Population and Households (2001), of the \textit{Insti}\textit{tuto Nacional de Est}\textit{adísticas y Censos} (INDEC – National Institute of Statistics and Censuses) were used as a secondary data source.\protect\footnote{ INDEC. Instituto Nacional de Estadísticas y Censos [Argentina]. Buenos Aires; 2001 [cited 2015 May 21]. Available from: \textit{http:\fshyp{}\fshyp{}www.indec.gov.ar}} All cases reported, from households in the municipal districts selected by the 6\textsuperscript{th} HR and treated at health services located in the region in 2007, were referenced. This was possible because these individuals participated in a study that aims to identify the foreknowledge of the nonadherence to tuberculosis treatment in these municipalities.\textsuperscript{\textsuperscript{1}} We also calculated the proportion of nonadherence to the TB treatment for the census fractions (analysis units in this study) of the municipalities of 6\textsuperscript{th} HR.\par{}The information has been grouped into three types of indicators according to the census classification.\protect\footnote{ INDEC. Instituto Nacional de Estadísticas y Censos [Argentina]. Buenos Aires; 2001 [cited 2015 May 21]. Available from: \textit{http:\fshyp{}\fshyp{}www.indec.gov.ar}} The characteristics of the area were considered according to the presence of wastewater treatment; electricity per household; gas network; at least one block paved; regular waste collection service at least twice a week; public transportation within 300 m. The proportion of households was considered according to the type of the pavement’s predominant material, water supply system, presence or absence of public water network, and type of health service.\par{}We considered the following: proportion of households grouped according to overcrowding (three or more persons per room); lack of basic needs (overcrowding, housing, sanitation, education, and subsistence capacity); index of household material privation (IHMP);\protect\footnote{ According to the National Institute of Statistics and Censuses (INDEC), and the Index of Household Material Privation (IHMP) it is a variable that identifies the residences according to their material deprivation in two dimensions – material and patrimonial resources. In relation to the Unsatisfied Basic Needs (UBN), the households with this characteristic have at least one of the following indicators of deprivation: overcrowding (more than 3 persons per room); housing (living in an improper location [leased place, hotel or pension room, shack, place without rooms], not considering house, apartment and farm); health conditions (without water-closet); school attendance (with at least one school-age child [6-12 years] who do not attend school); subsistence capacity (with four or more individuals per family unit, whose responsible has not concluded the third grade of Elementary School).} economic situation of the family; and the presence of refrigerator, freezer, landline or cell phone, microwave, computer with Internet connection, kitchen with sink and piped water in the residence.\par{}We considered the proportion of population according to sex, age, health plan, marital status and literacy. Moreover, we considered the ratio of individuals according to educational level, state of activity and of retirement contribution (contributes and is discounted; does not contribute nor receive discount; no remuneration).\par{}The statistical software package Stata 10.0 and two geographic information systems, ArcView 3.2 and GeoDA 8, were used to elaborate maps and perform the spatial analysis. The dropout rate was calculated by dividing the number of cases of nonadherence by the total number of patients who have started the treatment in each unit of analysis (census fraction). The Bayesian and Freeman-Tukey square-root transformations, empirical for these measures, were calculated having as a reference the set of fractions of the municipal census. The thematic maps with these proportions were elaborated to choose the most appropriate way to present the spatial distribution patterns.\par{}The exploratory analysis of the area characteristics and the nonadherence ratio distribution were also performed. The multifactorial analysis was performed with linear regression. In this model, the independent variables were the sociodemographic and socioeconomic characteristics of the groups and the areas related to dropout cases. The dependent variable was the “nonadherence”. The variables applied in the multiple linear regression model were those that had a significant association (p < 0,20) in the bivariate analysis. The final model included variables with a significance level of p = 0.05 and those considered essential for the explanatory model.\par{}The study protocol was approved by the Ethics Committee of each hospital included.
\section*{RESULTS}
\par{}The city of Avellaneda had residences with better overall conditions and available basic services. It was also the city with the lowest variations for each indicator of the census fractions. On the other hand, the city of Ezeiza, with the worst situation concerning most part of the analyzed indicators, had high variations in the census fractions. The distribution of the population was more homogeneous among municipalities, although the city of Avellaneda generally has the best situation regarding indicators (Table 1).\par{}

\end{multicols}


\begin{landscape}
\ctable[
  caption = {\textbf{Table 1:} \textit{Characteristics of the area, residences, households and population related to the cases of tuberculosis. Sixth Health Region (6th HR), Buenos Aires, Argentina, 2001.}}, 
  width=\linewidth, pos = ht, left, long
]
{p{0.28\linewidth}p{0.08\linewidth}p{0.08\linewidth}p{0.08\linewidth}p{0.08\linewidth}p{0.08\linewidth}p{0.08\linewidth}p{0.09\linewidth}p{0.08\linewidth}}
{}
{ \\\hline
\multicolumn{1}{L{0.28\linewidth}}{\textbf{Characteristic}}
 & \multicolumn{8}{L{0.67\linewidth}}{\textbf{Existence (\%\allowbreak{}) }} \\\hline 
\multicolumn{1}{L{0.28\linewidth}}{\textbf{AB}}
 & \multicolumn{1}{L{0.08\linewidth}}{\textbf{AV}}
 & \multicolumn{1}{L{0.08\linewidth}}{\textbf{BZ}}
 & \multicolumn{1}{L{0.08\linewidth}}{\textbf{EE}}
 & \multicolumn{1}{L{0.08\linewidth}}{\textbf{EZ}}
 & \multicolumn{1}{L{0.08\linewidth}}{\textbf{LZ}}
 & \multicolumn{1}{L{0.08\linewidth}}{\textbf{QM}}
 & \multicolumn{1}{L{0.09\linewidth}}{\textbf{Total}} \\\hline 
\multicolumn{8}{L{0.87\linewidth}}{\textbf{}} \\\hline 
\multicolumn{1}{L{0.28\linewidth}}{\textbf{\%\allowbreak{}}}
 & \multicolumn{1}{L{0.08\linewidth}}{\textbf{\%\allowbreak{}}}
 & \multicolumn{1}{L{0.08\linewidth}}{\textbf{\%\allowbreak{}}}
 & \multicolumn{1}{L{0.08\linewidth}}{\textbf{\%\allowbreak{}}}
 & \multicolumn{1}{L{0.08\linewidth}}{\textbf{\%\allowbreak{}}}
 & \multicolumn{1}{L{0.08\linewidth}}{\textbf{\%\allowbreak{}}}
 & \multicolumn{1}{L{0.08\linewidth}}{\textbf{\%\allowbreak{}}}
 & \multicolumn{1}{L{0.09\linewidth}}{\textbf{\%\allowbreak{}}} \\\hline 
\multicolumn{9}{L{0.95\linewidth}}{Area and residences characteristics} \\\hline \multicolumn{1}{L{0.28\linewidth}}{ Electricity per household} & \multicolumn{1}{C{0.08\linewidth}}{96.0} & \multicolumn{1}{C{0.08\linewidth}}{100} & \multicolumn{1}{C{0.08\linewidth}}{98.3} & \multicolumn{1}{C{0.08\linewidth}}{98.9} & \multicolumn{1}{C{0.08\linewidth}}{97.5} & \multicolumn{1}{C{0.08\linewidth}}{98.7} & \multicolumn{1}{C{0.09\linewidth}}{95.3} & \multicolumn{1}{C{0.08\linewidth}}{97.2} \\\hline \multicolumn{1}{L{0.28\linewidth}}{ Paved Street} & \multicolumn{1}{C{0.08\linewidth}}{73.0} & \multicolumn{1}{C{0.08\linewidth}}{100} & \multicolumn{1}{C{0.08\linewidth}}{75.4} & \multicolumn{1}{C{0.08\linewidth}}{72.4} & \multicolumn{1}{C{0.08\linewidth}}{78.2} & \multicolumn{1}{C{0.08\linewidth}}{87.2} & \multicolumn{1}{C{0.09\linewidth}}{72.8} & \multicolumn{1}{C{0.08\linewidth}}{76.1} \\\hline \multicolumn{1}{L{0.28\linewidth}}{ Waterwaste and sewage treatment system} & \multicolumn{1}{C{0.08\linewidth}}{13.0} & \multicolumn{1}{C{0.08\linewidth}}{85.3} & \multicolumn{1}{C{0.08\linewidth}}{73.8} & \multicolumn{1}{C{0.08\linewidth}}{4.9} & \multicolumn{1}{C{0.08\linewidth}}{2.0} & \multicolumn{1}{C{0.08\linewidth}}{6.7} & \multicolumn{1}{C{0.09\linewidth}}{59.0} & \multicolumn{1}{C{0.08\linewidth}}{37.5} \\\hline \multicolumn{1}{L{0.28\linewidth}}{ Garbage collection service} & \multicolumn{1}{C{0.08\linewidth}}{87.3} & \multicolumn{1}{C{0.08\linewidth}}{100} & \multicolumn{1}{C{0.08\linewidth}}{95.3} & \multicolumn{1}{C{0.08\linewidth}}{89.8} & \multicolumn{1}{C{0.08\linewidth}}{97.2} & \multicolumn{1}{C{0.08\linewidth}}{92.9} & \multicolumn{1}{C{0.09\linewidth}}{88.7} & \multicolumn{1}{C{0.08\linewidth}}{91.5} \\\hline \multicolumn{1}{L{0.28\linewidth}}{ Gas pipeline network} & \multicolumn{1}{C{0.08\linewidth}}{63.6} & \multicolumn{1}{C{0.08\linewidth}}{97.8} & \multicolumn{1}{C{0.08\linewidth}}{87.9} & \multicolumn{1}{C{0.08\linewidth}}{64.0} & \multicolumn{1}{C{0.08\linewidth}}{62.0} & \multicolumn{1}{C{0.08\linewidth}}{75.3} & \multicolumn{1}{C{0.09\linewidth}}{69.7} & \multicolumn{1}{C{0.08\linewidth}}{72.2} \\\hline \multicolumn{1}{L{0.28\linewidth}}{ Electrical power installation} & \multicolumn{1}{C{0.08\linewidth}}{44.5} & \multicolumn{1}{C{0.08\linewidth}}{100} & \multicolumn{1}{C{0.08\linewidth}}{99.7} & \multicolumn{1}{C{0.08\linewidth}}{41.3} & \multicolumn{1}{C{0.08\linewidth}}{30.1} & \multicolumn{1}{C{0.08\linewidth}}{89.3} & \multicolumn{1}{C{0.09\linewidth}}{92.7} & \multicolumn{1}{C{0.08\linewidth}}{74.4} \\\hline \multicolumn{1}{L{0.28\linewidth}}{ Deprived dwelling} & \multicolumn{1}{C{0.08\linewidth}}{34.8} & \multicolumn{1}{C{0.08\linewidth}}{2.4} & \multicolumn{1}{C{0.08\linewidth}}{23.1} & \multicolumn{1}{C{0.08\linewidth}}{38.0} & \multicolumn{1}{C{0.08\linewidth}}{39.3} & \multicolumn{1}{C{0.08\linewidth}}{31.0} & \multicolumn{1}{C{0.09\linewidth}}{30.0} & \multicolumn{1}{C{0.08\linewidth}}{30.7} \\\hline \multicolumn{1}{L{0.28\linewidth}}{ Ceramic floors, flag stone or mosaic} & \multicolumn{1}{C{0.08\linewidth}}{48.1} & \multicolumn{1}{C{0.08\linewidth}}{75.8} & \multicolumn{1}{C{0.08\linewidth}}{56.2} & \multicolumn{1}{C{0.08\linewidth}}{41.3} & \multicolumn{1}{C{0.08\linewidth}}{38.4} & \multicolumn{1}{C{0.08\linewidth}}{53.0} & \multicolumn{1}{C{0.09\linewidth}}{52.5} & \multicolumn{1}{C{0.08\linewidth}}{50.4} \\\hline \multicolumn{1}{L{0.28\linewidth}}{ Water supply in the residence} & \multicolumn{1}{C{0.08\linewidth}}{64.3} & \multicolumn{1}{C{0.08\linewidth}}{76.7} & \multicolumn{1}{C{0.08\linewidth}}{78.7} & \multicolumn{1}{C{0.08\linewidth}}{58.7} & \multicolumn{1}{C{0.08\linewidth}}{52.3} & \multicolumn{1}{C{0.08\linewidth}}{69.1} & \multicolumn{1}{C{0.09\linewidth}}{74.2} & \multicolumn{1}{C{0.08\linewidth}}{69.0} \\\hline \multicolumn{1}{L{0.28\linewidth}}{ Public transportation within 300 m} & \multicolumn{1}{C{0.08\linewidth}}{90.2} & \multicolumn{1}{C{0.08\linewidth}}{100} & \multicolumn{1}{C{0.08\linewidth}}{89.9} & \multicolumn{1}{C{0.08\linewidth}}{85.2} & \multicolumn{1}{C{0.08\linewidth}}{81.8} & \multicolumn{1}{C{0.08\linewidth}}{89.9} & \multicolumn{1}{C{0.09\linewidth}}{84.9} & \multicolumn{1}{C{0.08\linewidth}}{87.2} \\\hline \multicolumn{9}{L{0.95\linewidth}}{Characteristics of households and population} \\\hline \multicolumn{1}{L{0.28\linewidth}}{ Kitchen with piped water} & \multicolumn{1}{C{0.08\linewidth}}{62.0} & \multicolumn{1}{C{0.08\linewidth}}{77.5} & \multicolumn{1}{C{0.08\linewidth}}{75.2} & \multicolumn{1}{C{0.08\linewidth}}{56.0} & \multicolumn{1}{C{0.08\linewidth}}{48.7} & \multicolumn{1}{C{0.08\linewidth}}{68.4} & \multicolumn{1}{C{0.09\linewidth}}{71.2} & \multicolumn{1}{C{0.08\linewidth}}{66.1} \\\hline \multicolumn{1}{L{0.28\linewidth}}{ Microwave oven} & \multicolumn{1}{C{0.08\linewidth}}{9.8} & \multicolumn{1}{C{0.08\linewidth}}{27.7} & \multicolumn{1}{C{0.08\linewidth}}{14.1} & \multicolumn{1}{C{0.08\linewidth}}{10.6} & \multicolumn{1}{C{0.08\linewidth}}{9.1} & \multicolumn{1}{C{0.08\linewidth}}{10.1} & \multicolumn{1}{C{0.09\linewidth}}{14.2} & \multicolumn{1}{C{0.08\linewidth}}{12.4} \\\hline \multicolumn{1}{L{0.28\linewidth}}{ Overcrowding} & \multicolumn{1}{C{0.08\linewidth}}{7.1} & \multicolumn{1}{C{0.08\linewidth}}{10.2} & \multicolumn{1}{C{0.08\linewidth}}{5.8} & \multicolumn{1}{C{0.08\linewidth}}{7.2} & \multicolumn{1}{C{0.08\linewidth}}{8.7} & \multicolumn{1}{C{0.08\linewidth}}{6.3} & \multicolumn{1}{C{0.09\linewidth}}{5.9} & \multicolumn{1}{C{0.08\linewidth}}{6.4} \\\hline \multicolumn{1}{L{0.28\linewidth}}{ Water-closet without flushing system or without water-closet} & \multicolumn{1}{C{0.08\linewidth}}{28.6} & \multicolumn{1}{C{0.08\linewidth}}{0.7} & \multicolumn{1}{C{0.08\linewidth}}{17.6} & \multicolumn{1}{C{0.08\linewidth}}{31.1} & \multicolumn{1}{C{0.08\linewidth}}{32.1} & \multicolumn{1}{C{0.08\linewidth}}{25.0} & \multicolumn{1}{C{0.09\linewidth}}{23.1} & \multicolumn{1}{C{0.08\linewidth}}{24.4} \\\hline \multicolumn{1}{L{0.28\linewidth}}{ Refrigerator (with or without freezer)} & \multicolumn{1}{C{0.08\linewidth}}{81.8} & \multicolumn{1}{C{0.08\linewidth}}{77.6} & \multicolumn{1}{C{0.08\linewidth}}{85.1} & \multicolumn{1}{C{0.08\linewidth}}{81.0} & \multicolumn{1}{C{0.08\linewidth}}{76.5} & \multicolumn{1}{C{0.08\linewidth}}{82.0} & \multicolumn{1}{C{0.09\linewidth}}{82.4} & \multicolumn{1}{C{0.08\linewidth}}{81.9} \\\hline \multicolumn{1}{L{0.28\linewidth}}{ Computer (with or without Internet)} & \multicolumn{1}{C{0.08\linewidth}}{8.9} & \multicolumn{1}{C{0.08\linewidth}}{23.6} & \multicolumn{1}{C{0.08\linewidth}}{13.7} & \multicolumn{1}{C{0.08\linewidth}}{10.1} & \multicolumn{1}{C{0.08\linewidth}}{9.2} & \multicolumn{1}{C{0.08\linewidth}}{9.7} & \multicolumn{1}{C{0.09\linewidth}}{12.5} & \multicolumn{1}{C{0.08\linewidth}}{11.6} \\\hline \multicolumn{1}{L{0.28\linewidth}}{ Literate population} & \multicolumn{1}{C{0.08\linewidth}}{84.1} & \multicolumn{1}{C{0.08\linewidth}}{91.7} & \multicolumn{1}{C{0.08\linewidth}}{85.3} & \multicolumn{1}{C{0.08\linewidth}}{83.9} & \multicolumn{1}{C{0.08\linewidth}}{83.3} & \multicolumn{1}{C{0.08\linewidth}}{86.7} & \multicolumn{1}{C{0.09\linewidth}}{85.8} & \multicolumn{1}{C{0.08\linewidth}}{85.2} \\\hline \multicolumn{1}{L{0.28\linewidth}}{ Population without health insurance} & \multicolumn{1}{C{0.08\linewidth}}{63.7} & \multicolumn{1}{C{0.08\linewidth}}{33.0} & \multicolumn{1}{C{0.08\linewidth}}{57.8} & \multicolumn{1}{C{0.08\linewidth}}{63.2} & \multicolumn{1}{C{0.08\linewidth}}{62.7} & \multicolumn{1}{C{0.08\linewidth}}{63.7} & \multicolumn{1}{C{0.09\linewidth}}{54.8} & \multicolumn{1}{C{0.08\linewidth}}{59.0} \\\hline \multicolumn{1}{L{0.28\linewidth}}{ Telephone (landline, mobile or both)} & \multicolumn{1}{C{0.08\linewidth}}{53.1} & \multicolumn{1}{C{0.08\linewidth}}{69.5} & \multicolumn{1}{C{0.08\linewidth}}{58.3} & \multicolumn{1}{C{0.08\linewidth}}{53.9} & \multicolumn{1}{C{0.08\linewidth}}{49.0} & \multicolumn{1}{C{0.08\linewidth}}{54.8} & \multicolumn{1}{C{0.09\linewidth}}{54.4} & \multicolumn{1}{C{0.08\linewidth}}{54.8} \\\hline \multicolumn{1}{L{0.28\linewidth}}{ No material deprivation in the residences} & \multicolumn{1}{C{0.08\linewidth}}{34.7} & \multicolumn{1}{C{0.08\linewidth}}{65.7} & \multicolumn{1}{C{0.08\linewidth}}{43.6} & \multicolumn{1}{C{0.08\linewidth}}{33.9} & \multicolumn{1}{C{0.08\linewidth}}{31.6} & \multicolumn{1}{C{0.08\linewidth}}{38.7} & \multicolumn{1}{C{0.09\linewidth}}{41.5} & \multicolumn{1}{C{0.08\linewidth}}{39.3} \\\hline \multicolumn{1}{L{0.28\linewidth}}{ With material deprivation in the residences} & \multicolumn{1}{C{0.08\linewidth}}{21.4} & \multicolumn{1}{C{0.08\linewidth}}{0.2} & \multicolumn{1}{C{0.08\linewidth}}{16.1} & \multicolumn{1}{C{0.08\linewidth}}{21.7} & \multicolumn{1}{C{0.08\linewidth}}{23.2} & \multicolumn{1}{C{0.08\linewidth}}{19.7} & \multicolumn{1}{C{0.09\linewidth}}{19.0} & \multicolumn{1}{C{0.08\linewidth}}{19.2} \\\hline \multicolumn{1}{L{0.28\linewidth}}{ With at least a UBN indicator} & \multicolumn{1}{C{0.08\linewidth}}{20.7} & \multicolumn{1}{C{0.08\linewidth}}{4.4} & \multicolumn{1}{C{0.08\linewidth}}{17.2} & \multicolumn{1}{C{0.08\linewidth}}{19.3} & \multicolumn{1}{C{0.08\linewidth}}{20.2} & \multicolumn{1}{C{0.08\linewidth}}{18.8} & \multicolumn{1}{C{0.09\linewidth}}{23.0} & \multicolumn{1}{C{0.08\linewidth}}{19.9} \\\hline \multicolumn{1}{L{0.28\linewidth}}{ With UBN per subsistence condition} & \multicolumn{1}{C{0.08\linewidth}}{7.0} & \multicolumn{1}{C{0.08\linewidth}}{1.4} & \multicolumn{1}{C{0.08\linewidth}}{6.2} & \multicolumn{1}{C{0.08\linewidth}}{4.9} & \multicolumn{1}{C{0.08\linewidth}}{5.8} & \multicolumn{1}{C{0.08\linewidth}}{18.3} & \multicolumn{1}{C{0.09\linewidth}}{21.1} & \multicolumn{1}{C{0.08\linewidth}}{11.8} \\\hline \multicolumn{1}{L{0.28\linewidth}}{ With one spouse unemployed or inactive} & \multicolumn{1}{C{0.08\linewidth}}{50.8} & \multicolumn{1}{C{0.08\linewidth}}{50.1} & \multicolumn{1}{C{0.08\linewidth}}{48.2} & \multicolumn{1}{C{0.08\linewidth}}{53.0} & \multicolumn{1}{C{0.08\linewidth}}{52.7} & \multicolumn{1}{C{0.08\linewidth}}{48.7} & \multicolumn{1}{C{0.09\linewidth}}{49.9} & \multicolumn{1}{C{0.08\linewidth}}{50.3} \\\hline \multicolumn{1}{L{0.28\linewidth}}{ Population that contributes to or receives retirement funds} & \multicolumn{1}{C{0.08\linewidth}}{12.6} & \multicolumn{1}{C{0.08\linewidth}}{27.7} & \multicolumn{1}{C{0.08\linewidth}}{15.2} & \multicolumn{1}{C{0.08\linewidth}}{13.0} & \multicolumn{1}{C{0.08\linewidth}}{13.6} & \multicolumn{1}{C{0.08\linewidth}}{13.2} & \multicolumn{1}{C{0.09\linewidth}}{17.3} & \multicolumn{1}{C{0.08\linewidth}}{15.1} \\\hline \multicolumn{1}{L{0.28\linewidth}}{ Unemployed population} & \multicolumn{1}{C{0.08\linewidth}}{17.5} & \multicolumn{1}{C{0.08\linewidth}}{12.3} & \multicolumn{1}{C{0.08\linewidth}}{18.7} & \multicolumn{1}{C{0.08\linewidth}}{16.0} & \multicolumn{1}{C{0.08\linewidth}}{14.5} & \multicolumn{1}{C{0.08\linewidth}}{19.8} & \multicolumn{1}{C{0.09\linewidth}}{17.3} & \multicolumn{1}{C{0.08\linewidth}}{17.3} \\\hline \multicolumn{9}{p\linewidth}{\textsuperscript{ } \footnotesize{Sources: Cross-sectional study (ECM, 2008)e and National Census of Population and Households, INDEC, 2001.\textsuperscript{g}}}\\\multicolumn{9}{p\linewidth}{\textsuperscript{ } \footnotesize{AB: Almirante Brown; AV: Avellaneda; BZ: Berazategui; EE: Esteban Echeverría; EZ: Ezeiza; LZ: Lomas de Zamora; QM: Quilmes; UBN: unsatisfied basic needs}}
}
\end{landscape}
\begin{multicols}{2}
\lipsum

\end{multicols}
