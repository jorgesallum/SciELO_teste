\selectlanguage{brazilian}
\noindent{}MARRONE, G. \textbf{Semiótica da cidade: corpos, espaços, tecnologias} Galáxia (São Paulo). 00(29): 28-43. \url{http://dx.doi.org/10.1590/1982-25542015122803}\bigskip{}

{
\renewcommand{\abstractname}{Resumo}
\begin{abstract}
\begin{spacing}{0.93}
A cidade é feita de espaços,\allowbreak{} corpos,\allowbreak{} tecnologias.\allowbreak{} A semiótica tem estudado há tempos todos estes três fenômenos de significação.\allowbreak{} Todavia,\allowbreak{} o estudo semiótico do espaço urbano quase nunca foi articulado pela semiótica dos objetos técnicos,\allowbreak{} nem pela do corpo.\allowbreak{} Entrelaçar estes três âmbitos,\allowbreak{} como este artigo busca fazer,\allowbreak{} aparece,\allowbreak{} portanto,\allowbreak{} como um gesto teórico tão urgente quanto necessário.\allowbreak{} Nestas páginas analisa-\allowbreak{}se um velho desenho animado de Walt Disney em que o personagem de Pateta muda radicalmente os próprios programas de ação e paixão,\allowbreak{} condicionado por estar no espaço urbano,\allowbreak{} como pedestre ou automobilista.\allowbreak{} Um ator,\allowbreak{} dois actantes,\allowbreak{} e consequentemente,\allowbreak{} dois espaços de significação diferentes.\allowbreak{}
\end{spacing}
\vspace*{1.9mm}
\fontsize{9}{10.8}\selectfont{\textit{Palavras-chave:} espaço urbano, corpos, espaços, tecnologias, espaços de significação, desenho animado}
\end{abstract}
}

\noindent{}MARRONE, G. \textbf{Semiotics of the city:\allowbreak{} bodies,\allowbreak{} spaces,\allowbreak{} technologies} Galáxia (São Paulo). 00(29): 28-43. \url{http://dx.doi.org/10.1590/1982-25542015122803}\bigskip{}

{
\renewcommand{\abstractname}{Abstract}
\begin{abstract}
\begin{spacing}{0.93}
Cities are made of spaces,\allowbreak{} bodies and technologies.\allowbreak{} Semiotics has been working for a long time on these phenomena of meaning.\allowbreak{} However,\allowbreak{} the semiotic study of urban spaces have rarely met both semiotics of technical objects and semiotics of body.\allowbreak{} Dealing with these three fields as this paper aims to do,\allowbreak{} seems to be an urgent and necessary theoretical move.\allowbreak{} In its pages an old Walt Disney’s cartoon is analyzed; the main character is Goofy,\allowbreak{} who changes his passion and action programs depending on he is pedestrian or driver.\allowbreak{} One actor,\allowbreak{} two actants and,\allowbreak{} as a consequence,\allowbreak{} two meaning spaces.\allowbreak{}
\end{spacing}
\vspace*{1.9mm}
\fontsize{9}{10.8}\selectfont{\textit{Keywords:} urban space, bodies, spaces, technologies, meaning spaces, cartoon}
\end{abstract}
}

\vspace*{-1.6mm}
{\noindent\fontsize{9}{10.8}\selectfont{Recebido em: 01/2015;
Aceito em: 03/2015.}}
\begin{multicols}{2}
\section*{Espaços urbanos e subjetividade}
\par{}Entre os estudiosos de espaços urbanos já é convenção difundida que a cidade não é exaurida na diferenciação de seus espaços (\allowbreak{}ruas,\allowbreak{} praças,\allowbreak{} jardins,\allowbreak{} rios e praias,\allowbreak{} terrenos vagos.\allowbreak{}.\allowbreak{}.\allowbreak{})\allowbreak{} e na articulação das coisas que a preenchem (\allowbreak{}edifícios,\allowbreak{} igrejas e monumentos,\allowbreak{} sinalética,\allowbreak{} propagandas,\allowbreak{} luzes,\allowbreak{} serviços diversos)\allowbreak{}.\allowbreak{} Insiste-\allowbreak{}se no fato que na construção de uma cidade – de qualquer dimensão e complexidade,\allowbreak{} densidade ou rarefação – são acima de tudo os cidadãos,\allowbreak{} sejam eles sedentários ou não,\allowbreak{} residentes ou de passagem,\allowbreak{} a trabalho ou turistas,\allowbreak{} que vivem os lugares urbanos,\allowbreak{} atravessam-\allowbreak{}nos de acordo com percursos variavelmente estabilizados,\allowbreak{} valorizando,\allowbreak{} desvalorizando e revalorizando continuamente estes lugares.\allowbreak{} De um lado estaria,\allowbreak{} então,\allowbreak{} o espaço,\allowbreak{} natural e construído,\allowbreak{} condição de possibilidades do ambiente urbano como forma ideal abstrata; de outro se situariam,\allowbreak{} ao contrário,\allowbreak{} as pessoas,\allowbreak{} sujeitos individuais e coletivos que,\allowbreak{} em determinado espaço,\allowbreak{} se encontram posicionados de diversas formas,\allowbreak{} constituindo sua substância social.\allowbreak{} O todo no interior de uma história e de uma memória que atenua as forças entrópicas do tempo,\allowbreak{} consolidando homens e coisas,\allowbreak{} construindo e mantendo retalhos de identidades.\allowbreak{}\par{}A perspectiva dos estudos semióticos,\allowbreak{} junto a diversas outras ciências humanas e sociais,\allowbreak{} tentou tornar,\allowbreak{} ao mesmo tempo,\allowbreak{} mais complexa e radical a questão:\allowbreak{} espaços e sujeitos não existem enquanto tais para então encontrar-\allowbreak{}se e conjungir-\allowbreak{}se,\allowbreak{} ora por vontade,\allowbreak{} ora por destino; muito diferentemente,\allowbreak{} eles se constituem reciprocamente,\allowbreak{} são os polos de uma relação que os precede e,\allowbreak{} fundando-\allowbreak{}os,\allowbreak{} os transcende.\allowbreak{} A cidade nasce na cansativa instituição e na manutenção histórica e identitária de tal relação.\allowbreak{} Ela não é a somatória de duas entidades autônomas,\allowbreak{} mas a forma relacional de seu recíproco constituir-\allowbreak{}se.\allowbreak{} Não existem espaços autônomos e sujeitos independentes que,\allowbreak{} em segunda instância,\allowbreak{} se reúnem mais ou menos casualmente em um determinado ambiente ou situação.\allowbreak{} Eles se realizam como sujeitos espaciais que,\allowbreak{} desde o início,\allowbreak{} se reúnem internamente nos seus corpos e lugares,\allowbreak{} traduzindo-\allowbreak{}os uns nos outros e produzindo assim novas formas de subjetividade.\allowbreak{} \par{}Mas o que queremos dizer quando falamos de sujeitos espaciais? A esta pergunta devemos responder com atenção.\allowbreak{} Se não,\allowbreak{} porque,\allowbreak{} precisamente,\allowbreak{} a semiótica não é a única ou a melhor disciplina a ocupar-\allowbreak{}se de ambientes urbanos como realidades sociais.\allowbreak{} Ela intervém antes sobre um \textit{status questionis} já discutido,\allowbreak{} explorado,\allowbreak{} atestado.\allowbreak{} E deve declarar,\allowbreak{} a partir de uma mesma correspondência teórica e analítica,\allowbreak{} as próprias intenções explorativas específicas e os mapas metodológicos que pretende colocar em jogo.\allowbreak{} Deve ser recordado então que – à diferença da sociologia,\allowbreak{} antropologia,\allowbreak{} psicologia,\allowbreak{} etc.\allowbreak{} – os sujeitos dos quais fala a teoria da significação são entidades abstratas e formais,\allowbreak{} posições sintáticas de corte narrativo que,\allowbreak{} à semelhança de seus parentes frásticos (\allowbreak{}os da análise lógica,\allowbreak{} para que entendamos)\allowbreak{},\allowbreak{} têm um papel muito preciso – desenvolver ou sofrer uma ação,\allowbreak{} experimentar uma paixão ou provocá-\allowbreak{}la a outros – de encontro a posições sintáticas concomitantes (\allowbreak{}objetos,\allowbreak{} antissujeitos,\allowbreak{} destinadores,\allowbreak{} etc.\allowbreak{})\allowbreak{} dentro de um plano de ação pré-\allowbreak{}constituído e sobre o panorama de um mecanismo estrutural geral de transformação.\allowbreak{} A subjetividade se constitui dentro de um programa narrativo,\allowbreak{} no quadro de certa projetualidade que,\allowbreak{} almejando um objetivo,\allowbreak{} faz sim com que entre o início e o fim de cada história haja um resíduo,\allowbreak{} uma diferença,\allowbreak{} talvez uma reviravolta,\allowbreak{} seguramente uma transformação.\allowbreak{} No fim das contas,\allowbreak{} ninguém é mais o mesmo.\allowbreak{}\par{}Assim,\allowbreak{} não devem ser confundidos os sujeitos como forças sintáticas (\allowbreak{}tecnicamente \textit{actantes})\allowbreak{} com as figuras do mundo,\allowbreak{} que,\allowbreak{} concretizando-\allowbreak{}os semanticamente,\allowbreak{} encarregam-\allowbreak{}se deles (\allowbreak{}\textit{atores})\allowbreak{}:\allowbreak{} que podem ser indivíduos e pessoas,\allowbreak{} mas também instituições coletivas,\allowbreak{} criaturas abstratas,\allowbreak{} feras,\allowbreak{} entidades espirituais,\allowbreak{} coisas,\allowbreak{} tecnologias.\allowbreak{} Se ao mesmo sujeito actante podem corresponder diversas fisionomias de atores (\allowbreak{}um tapete voador,\allowbreak{} nas fábulas,\allowbreak{} é um objeto que pode desenvolver o papel de sujeito)\allowbreak{},\allowbreak{} apresenta-\allowbreak{}se evidente que nossos sujeitos espaciais possam ter mais naturezas,\allowbreak{} consigam manifestar-\allowbreak{}se sob diversas e falsas aparências,\allowbreak{} espreitar ou se esconder tanto a partir de pessoas quanto de edifícios,\allowbreak{} de objetos ou de paisagens,\allowbreak{} de casas e coisas,\allowbreak{} multidões humanas e bairros inteiros,\allowbreak{} incluindo todas as disciplinas ou artes que materialmente constroem uma cidade – urbanística,\allowbreak{} engenharia,\allowbreak{} arquitetura,\allowbreak{} planificação territorial,\allowbreak{} etc.\allowbreak{} – e todos os objetos que,\allowbreak{} em uma cidade,\allowbreak{} vivem e se movimentam assim como os sujeitos humanos:\allowbreak{} automóveis,\allowbreak{} ônibus e transportes coletivos,\allowbreak{} trens e metrôs,\allowbreak{} motocicletas e bicicletas,\allowbreak{} carroças e riquexós,\allowbreak{} caminhões comuns e de reboque,\allowbreak{} scooters,\allowbreak{} patins e muitos outros.\allowbreak{}\par{}Tudo isso para dizer que,\allowbreak{} no fundo,\allowbreak{} de acordo com tal perspectiva teórica os automóveis são sujeitos espaciais,\allowbreak{} logo sociais,\allowbreak{} para todos os efeitos:\allowbreak{} se comportam e são interpretados como tais.\allowbreak{} E assim como esses,\allowbreak{} os outros meios de transporte que em um espaço urbano se encontram a existir e a consistir – incluindo os pés do pedestre,\allowbreak{} ou \textit{pé-\allowbreak{}móveis} como deveríamos mais exatamente chamá-\allowbreak{}los,\allowbreak{} que com as outras tecnologias de deslocamento na cidade se encontram a conviver e a conflitar-\allowbreak{}se.\allowbreak{} Além disso,\allowbreak{} para sermos mais precisos,\allowbreak{} os sujeitos espaciais que vagam pelos itinerários metropolitanos são sempre sujeitos híbridos,\allowbreak{} feitos de corpos e coisas,\allowbreak{} de pessoas e tecnologias,\allowbreak{} de substância humana e não humana:\allowbreak{} não existem pessoas +\allowbreak{} carros,\allowbreak{} mas motoristas em carros,\allowbreak{} automobilistas,\allowbreak{} assim como motociclistas,\allowbreak{} ciclistas,\allowbreak{} patinadores velozes,\allowbreak{} usuários de ônibus ou do metrô,\allowbreak{} etc.\allowbreak{} Assim como em um poema cavalheiresco,\allowbreak{} um paladino que por acaso se encontra dotado de uma espada invencível se torna de repente ele próprio invencível,\allowbreak{} assumindo de fato outros semblantes e novas oportunidades,\allowbreak{} ao menos até que a espada lhe seja eventualmente subtraída,\allowbreak{} de forma análoga um sujeito que antes andava – ou depois andará – a pé,\allowbreak{} não tem as mesmas oportunidades,\allowbreak{} a mesma vontade de fazer e de andar,\allowbreak{} as mesmas paixões,\allowbreak{} o mesmo caráter.\allowbreak{} A carteira de motorista,\allowbreak{} sabe-\allowbreak{}se,\allowbreak{} torna profundamente diferente a subjetividade do motorista,\allowbreak{} ou mesmo a reconstitui,\allowbreak{} caminhando para fundar um sujeito híbrido,\allowbreak{} metade humano metade não humano,\allowbreak{} corpo e tecnologia que,\allowbreak{} entrelaçando-\allowbreak{}se,\allowbreak{} se amalgamam em um único programa de ação e de paixão.\allowbreak{} Um tal híbrido,\allowbreak{} porém,\allowbreak{} nunca está sozinho,\allowbreak{} mônada mais ou menos diabólica,\allowbreak{} mas vive e opera em um contexto em que outros sujeitos humanos e não humanos,\allowbreak{} outros híbridos – semelhantes ou diferentes – interagem com ele,\allowbreak{} transformando-\allowbreak{}o ulteriormente e sendo por ele transformados.\allowbreak{} É demasiado fácil falar da antropomorfização do carro ou da mecanização do automobilista.\allowbreak{} O cenário citadino – de Limoges a Los Angeles,\allowbreak{} de Monteriggioni a Dubai – é muito mais intrincado,\allowbreak{} constitutiva e felizmente:\allowbreak{} de modo que qualquer sonho nostálgico,\allowbreak{} ânsia naturalística ou hipótese essencialista precisa obrigatoriamente – é o caso de dizê-\allowbreak{}lo – de ajuda divina.\allowbreak{}
\section*{Ânsias automobilísticas}
\par{}Isso é demonstrado e motivado,\allowbreak{} oferecendo-\allowbreak{}nos mais de uma pista de reflexão,\allowbreak{} por um saboroso desenho animado de Walt Disney de 1950,\allowbreak{} \textit{Motormania}\textsuperscript{2},\allowbreak{} em que Pateta-\allowbreak{}Goofy se encontra às voltas com o automóvel pelas ruas de uma típica pequena cidade norte-\allowbreak{}americana.\allowbreak{} As obras de arte são frequentemente impregnadas de teoria.\allowbreak{} Não necessariamente conscientes de seu alcance filosófico,\allowbreak{} elas manifestam em todo caso nas dobras de seus dispositivos textuais e discursivos,\allowbreak{} graças aos meios semióticos para sua disposição específica,\allowbreak{} uma própria e verdadeira teoria sobre o mundo humano e social,\allowbreak{} algum movimento conceitual no jogo estratégico das culturas – destinada a permanecer implícita e silenciosa,\allowbreak{} a menos que uma análise \textit{a posteriori},\allowbreak{} com os instrumentos de uma metalinguagem metodológica \textit{ad hoc},\allowbreak{} não queira e não saiba explicitá-\allowbreak{}la,\allowbreak{} traduzi-\allowbreak{}la,\allowbreak{} redizendo-\allowbreak{}a quase da mesma forma.\allowbreak{} Talvez não apenas as obras de arte tradicionais,\allowbreak{} plásticas e figurativas,\allowbreak{} mas também,\allowbreak{} como no caso que gostaríamos de brevemente ilustrar aqui,\allowbreak{} textos midiáticos de alguma espessura e profundidade,\allowbreak{} sejam eles anúncios publicitários,\allowbreak{} transmissões televisivas,\allowbreak{} filmes comerciais ou,\allowbreak{} precisamente,\allowbreak{} desenhos para crianças.\allowbreak{}\par{}A mensagem explícita e aparente,\allowbreak{} do texto em questão,\allowbreak{} faz evidente referência a Stevenson de doutor Jekill e Mr.\allowbreak{} Hide:\allowbreak{} por trás das pessoas comuns,\allowbreak{} os \textit{average men} da burguesia abastada,\allowbreak{} se esconde sempre uma alma obscura,\allowbreak{} para cada bom Jekill corresponde assim um terrível Hide.\allowbreak{} E mesmo o cidadão norte-\allowbreak{}americano médio,\allowbreak{} homem totalmente comum,\allowbreak{} pacífico,\allowbreak{} honesto e respeitável,\allowbreak{} esconde uma metamorfose próxima futura:\allowbreak{} basta que entre no carro para se tornar um indivíduo terrível,\allowbreak{} briguento,\allowbreak{} irritante,\allowbreak{} em meio a indivíduos violentos e ruins como ele.\allowbreak{} A partir disso tem-\allowbreak{}se a história de Mr.\allowbreak{} Walker (\allowbreak{}um \textit{pé-\allowbreak{}móvel} já pelo nome)\allowbreak{},\allowbreak{} personagem absolutamente dócil e respeitoso em relação à vivência civil,\allowbreak{} que logo que tira o estrondoso automóvel para fora da garagem,\allowbreak{} se transforma no opressor Mr.\allowbreak{} Wheeler,\allowbreak{} pronto para atropelar (\allowbreak{}\textit{nomen est omen} também aqui)\allowbreak{} qualquer um que apareça em sua frente.\allowbreak{} A cidade é o pano de fundo mais característico dessas contínuas metamorfoses entre Walker e Wheeler,\allowbreak{} típico duplo narrativo que entra e sai constantemente do carro saltitando ao mesmo tempo entre suas personalidades opostas.\allowbreak{} O espaço urbano se torna assim a clássica selva metropolitana onde automobilistas e automóveis travam uma guerra extrema e cruel,\allowbreak{} correm como se estivessem em Indianápolis,\allowbreak{} apreciando acidentes contínuos e catástrofes mais curiosas quando ocorrem ao outro,\allowbreak{} ou mesmo sofrendo os abusos gratuitos e os risos sarcásticos e sádicos que o mais forte da vez lhes inflige sem piedade.\allowbreak{}\par{}Uma análise um pouco mais estreita do texto convida,\allowbreak{} porém,\allowbreak{} a uma interpretação menos moralista e bondosa desse divertido desenho disneyano,\allowbreak{} e com ela propõe uma teoria sociossemiótica da cidade como lugar onde os híbridos – em suas contínuas transformações e desregramentos,\allowbreak{} figurações e reconfigurações – constituem absolutamente a norma.\allowbreak{} O espaço urbano,\allowbreak{} em suma,\allowbreak{} surge como dispositivo de construção e desconstrução incessante da subjetividade individual e coletiva,\allowbreak{} onde corpos,\allowbreak{} espaços e tecnologias – mesclando-\allowbreak{}se em hierarquias variáveis – aparecem como atores de graus semelhantes,\allowbreak{} sujeitos precisamente,\allowbreak{} dotados de programas análogos de ação e de paixão.\allowbreak{}
\section*{Segmentação textual}
\par{}Tentaremos segmentar o texto – de pouco mais de seis minutos – em diversas sequências narrativas,\allowbreak{} a partir das relações de disjunção e conjunção entre o corpo de Pateta e seu veículo,\allowbreak{} ou seja,\allowbreak{} das passagens actoriais entre Walker e Wheeler.\allowbreak{} Para cada uma dessas se identificará um espaço específico e um relativo percurso em seu interior,\allowbreak{} e com isso uma série de procedimentos de temporalização,\allowbreak{} aspectualização,\allowbreak{} agogia,\allowbreak{} algumas passagens tímicas e modais,\allowbreak{} escalas de tensão e intensidade e,\allowbreak{} consequentemente,\allowbreak{} como êxito semiótico de tudo isso,\allowbreak{} um barômetro passional muito movimentado.\allowbreak{} Propomos para isso a seguinte segmentação:\allowbreak{}
\begin{enumerate}[label=\arabic*,leftmargin=*]
\item \par{}O cemitério do carro:\allowbreak{} final
\item \par{}Walker 1:\allowbreak{} o jardim de casa
\item \par{}Wheeler 1:\allowbreak{} em direção à cidade
\item \par{}Walker 2:\allowbreak{} na cidade
\item \par{}Wheeler 2:\allowbreak{} o acidente
\item \par{}Walker-\allowbreak{}Wheeler:\allowbreak{} final
\end{enumerate}


\par{}Como já aparece evidente,\allowbreak{} a estrutura do texto é circular:\allowbreak{} a narração começa pelo fim (\allowbreak{}o carro no cemitério)\allowbreak{} e se reconecta ao fim nas últimas cenas (\allowbreak{}o carro se dirige ao cemitério)\allowbreak{}.\allowbreak{} O que faz imediatamente suspeitar que seja também e,\allowbreak{} sobretudo,\allowbreak{} o carro,\allowbreak{} e não apenas Walker\fshyp{}Wheeler,\allowbreak{} o protagonista da narrativa,\allowbreak{} o sujeito que causa as repentinas metamorfoses do \textit{average man} e lhe impõe as consequências apropriadas.\allowbreak{} Mais que um simples auxiliar por trás do qual se esconde um perigoso adversário,\allowbreak{} como poderia parecer à primeira vista,\allowbreak{} o carro é um ponto central fortíssimo da história,\allowbreak{} verdadeiro e próprio sujeito operador que ocasiona as transformações narrativas e se transforma ele mesmo.\allowbreak{} Neste sentido,\allowbreak{} semelhante ao sujeito duplamente humano que lhe é análogo.\allowbreak{} Mas vejamos um pouco mais no detalhamento sequência por sequência.\allowbreak{}
\section*{O cemitério do carro:\allowbreak{} final}
\par{}As primeiras imagens mostram o carro em péssimo estado,\allowbreak{} abandonado junto a vários outros em uma espécie de garagem-\allowbreak{}cemitério da qual,\allowbreak{} inferimos,\allowbreak{} nunca mais sairá.\allowbreak{} A tomada dinâmica se afasta,\allowbreak{} mostrando a amplitude do local em que o carro se encontra,\allowbreak{} e então a densa,\allowbreak{} triste companhia que o circunda.\allowbreak{} A voz externa nos informa,\allowbreak{} enquanto isso,\allowbreak{} com falsa profecia de tonalidade vagamente darwiniana,\allowbreak{} que o carro “está destinado à extinção rápida” uma vez “nas mãos de um indivíduo normal (\allowbreak{}\textit{average man})\allowbreak{}”.\allowbreak{} Por que tudo isso acontece? O filme pretende demonstrar com sua história edificante.\allowbreak{} E pergunta-\allowbreak{}se,\allowbreak{} antes de qualquer coisa,\allowbreak{} o que é um indivíduo normal? Distanciando a imagem sobre uma simples casa residencial qualquer,\allowbreak{} e passando assim para a sequência sucessiva por debreagem espacial e actorial,\allowbreak{} a voz em \textit{off} tenta responder rapidamente a essa segunda questão.\allowbreak{}
\section*{Walker 1:\allowbreak{} o jardim de casa}
\par{}Aprendemos que o \textit{average man},\allowbreak{} normal,\allowbreak{} uma vez que igual a qualquer outro (\allowbreak{}e todos os personagens da história serão representados como tantos outros Patetas,\allowbreak{} com fisionomias e caráteres idênticos uns aos outros)\allowbreak{},\allowbreak{} é na realidade uma “criatura de comportamentos estranhos e imprevisíveis”.\allowbreak{} Basta tomar um desses casos para demonstrá-\allowbreak{}lo.\allowbreak{} “Take the case of Mr.\allowbreak{} Walker”,\allowbreak{} diz a voz em \textit{off} introduzindo aquele que se tornará o herói da história [fig.\allowbreak{} 1],\allowbreak{} o qual sai alegremente de casa interpelando o espectador e ao mesmo tempo cumprimentando o narrador invisível.\allowbreak{} Os adjetivos e epítetos positivos que descrevem tal herói se esvaem:\allowbreak{} tranquilo,\allowbreak{} respeitável,\allowbreak{} de inteligência média,\allowbreak{} honesto,\allowbreak{} não faria mal a uma mosca.\allowbreak{} Além do que as imagens confirmam isso,\allowbreak{} enquanto ele cumprimenta passarinhos que gorjeiam e evita esmagar formigas invisíveis em seu caminho,\allowbreak{} que se protegem jovialmente.\allowbreak{}
\par
{
\centering{
\includegraphics[width=\maxwidth{0.5\textwidth}]{r1982-2553-gal-29-0028-gf01.tif}
}
\captionof{figure}{\textbf{Fig.\allowbreak{}1.\allowbreak{}:} \textit{Mr.\allowbreak{} Walker}} 
}
\par
\par{}Em um regime temporal de iteratividade – estamos em uma manhã qualquer de um belo dia qualquer do ano – Walker percorre o breve trecho que da saída de casa o leva até a porta da garagem.\allowbreak{} Um deslocamento mínimo no espaço que comporta uma mudança semântica e narrativa muito forte.\allowbreak{} A porta da garagem em breve se revelará de fato como uma espécie de porta para o inferno,\allowbreak{} atrás da qual,\allowbreak{} prensada até um nível irreal entre milhares de típicas coisas inúteis,\allowbreak{} repousa o carro maligno que,\allowbreak{} uma vez ligado,\allowbreak{} provocará a imediata transformação de Walker em Wheeler.\allowbreak{}\par{}Espaços,\allowbreak{} corpos e tecnologias,\allowbreak{} em suma,\allowbreak{} desde o início se constituem e se reconstituem em sua relação recíproca:\allowbreak{} se a casa é o espaço próprio de Walker,\allowbreak{} e a garagem o espaço próprio do carro,\allowbreak{} o percurso de um a outro implica uma colocação em continuidade dos dois espaços e,\allowbreak{} com isso,\allowbreak{} a constituição,\allowbreak{} por tradução,\allowbreak{} de um novo sujeito,\allowbreak{} dado pela conjunção do corpo ao veículo,\allowbreak{} que será,\allowbreak{} precisamente,\allowbreak{} Wheeler [fig.\allowbreak{} 2].\allowbreak{}
\par
{
\centering{
\includegraphics[width=\maxwidth{0.5\textwidth}]{r1982-2553-gal-29-0028-gf02.tif}
}
\captionof{figure}{\textbf{Fig.\allowbreak{} 2.\allowbreak{}:} \textit{Mr.\allowbreak{} Wheeler}} 
}
\par

\section*{Wheeler 1:\allowbreak{} em direção à cidade}
\par{}Para sermos precisos,\allowbreak{} a metamorfose não é tão automática dado que,\allowbreak{} como todas as metamorfoses sérias,\allowbreak{} é mais interessante pelo processo que comporta do que pelo resultado que obtém.\allowbreak{} A música fica mais alta,\allowbreak{} também o volume da voz do narrador,\allowbreak{} mas sobretudo,\allowbreak{} o motor do carro estrondeia cada vez mais forte,\allowbreak{} prestes a dar a partida e sair.\allowbreak{} Ora,\allowbreak{} ao aumento do barulho do motor corresponde a intensidade da transformação de Pateta,\allowbreak{} de Walker em Wheeler.\allowbreak{} O ritmo progressivo do motor de arranque do veículo é acompanhado pelo ritmo ascendente-\allowbreak{}descendente dos braços de Walker:\allowbreak{} ritmos em uníssono que se intensificam cada vez mais até que,\allowbreak{} com o fundo do desenho tornado vermelho escuro,\allowbreak{} o personagem assume os típicos traços do monstro:\allowbreak{} olhos amarelos,\allowbreak{} dentes afiados,\allowbreak{} um riso maligno infernal,\allowbreak{} garras,\allowbreak{} braços para o alto como que sinalizando a passagem seguinte para a ação maléfica.\allowbreak{} O corpo do humano torna-\allowbreak{}se um com o não humano:\allowbreak{} não é o primeiro que se transforma em função do segundo; em um olhar mais atento,\allowbreak{} ambos mudam profundamente e são ligados ao mesmo tempo,\allowbreak{} adquirindo a “sede de poder” que o enunciador designa com uma tonalidade fria e didática que,\allowbreak{} agora,\allowbreak{} contrasta intensamente com o \textit{pathos} enunciado.\allowbreak{} Assistiremos a outros fenômenos do gênero,\allowbreak{} espécies de rimas rítmicas e figurativas entre o corpo de Wheeler e o do carro.\allowbreak{}\par{}Neste ponto,\allowbreak{} o híbrido está constituído,\allowbreak{} a tecnologia está encarnada ou,\allowbreak{} o que seria o mesmo,\allowbreak{} o corpo se tornou máquina,\allowbreak{} e com isso tem-\allowbreak{}se a criação de um novo ser,\allowbreak{} um “monstro incontrolável”,\allowbreak{} um “demônio do volante” pronto para voar entre as vias periféricas em direção ao centro da cidade,\allowbreak{} cruzando e se embrenhando junto com outros tantos monstros que correm com ele.\allowbreak{} Atropelará um pedestre (\allowbreak{}um \textit{walker},\allowbreak{} como ele era até pouco antes)\allowbreak{},\allowbreak{} se chocará com um carro à saída da viela de casa [fig.\allowbreak{} 3],\allowbreak{} e depois partirá em direção à grande via de quatro pistas que se revela ser – como assinala a tomada do alto – uma espécie de videogame \textit{ante litteram}.\allowbreak{} O fato é que Wheeler,\allowbreak{} como declara a si mesmo,\allowbreak{} satisfeito,\allowbreak{} acredita ser “o dono da rua”,\allowbreak{} imagina assim que todas as vias – \textit{avenue,\allowbreak{} street,\allowbreak{} drive,\allowbreak{} boulevard,\allowbreak{} place,\allowbreak{} way,\allowbreak{} turnpike,\allowbreak{} detour},\allowbreak{} o caminho que sejam – portem o seu nome próprio porque,\allowbreak{} precisamente,\allowbreak{} no fundo lhe pertencem [fig.\allowbreak{} 4].\allowbreak{} Para conseguir o que quer,\allowbreak{} entre os carros enlouquecidos e enfurecidos,\allowbreak{} faz uma voz grossa,\allowbreak{} se impõe pela força,\allowbreak{} range os dentes e encontra espaço entre as buzinas insuportáveis e o asfalto quente.\allowbreak{} Expressões como “abram caminho”,\allowbreak{} “saiam da frente”,\allowbreak{} “eu quero passar”,\allowbreak{} “calem a boca” constituem o discurso regular do \textit{average man} no volante.\allowbreak{}
\par
{
\centering{
\includegraphics[width=\maxwidth{0.5\textwidth}]{r1982-2553-gal-29-0028-gf03.tif}
}
\captionof{figure}{\textbf{Fig.\allowbreak{} 3.\allowbreak{}:} \textit{Conflitos entre Wheelers}} 
}
\par

\par
{
\centering{
\includegraphics[width=\maxwidth{0.5\textwidth}]{r1982-2553-gal-29-0028-gf04.tif}
}
\captionof{figure}{\textbf{Fig.\allowbreak{} 4.\allowbreak{}:} \textit{Todas as estradas são de Wheeler}} 
}
\par
\par{}Ao ponto que,\allowbreak{} para desprezar os outros automobilistas,\allowbreak{} não é regra que a velocidade seja o melhor instrumento:\allowbreak{} desacelerar pode ser igualmente irritante,\allowbreak{} tática agógica sutil que ocupa o lugar da ânsia pela intensidade a todo custo.\allowbreak{} E essa é outra rima figurativa importante:\allowbreak{} enquanto o carro segue vagarosamente ao centro do \textit{boulevard} da cidade,\allowbreak{} impedindo a ultrapassagem dos outros,\allowbreak{} a capota se abre,\allowbreak{} e com ela o chapéu de Wheeler,\allowbreak{} como predispondo ambos ao eufórico “ar fresco” identificado pelo narrador.\allowbreak{} Assim,\allowbreak{} não é apenas o corpo do humano que deseja um pouco da brisa irascível que causa tanto nervosismo aos carros engarrafados e barulhentos,\allowbreak{} e aproveitar a música do rádio do carro,\allowbreak{} mas o híbrido em sua completude,\allowbreak{} ou melhor,\allowbreak{} a matéria que o constitui.\allowbreak{} Há uma espécie de curiosa intercorporeidade acolhida em um único invólucro,\allowbreak{} ao mesmo tempo humano e não humano:\allowbreak{} uma espécie de \textit{sensorium commune} que não é o sinestésico,\allowbreak{} do corpo antes de ser dividido em modalidades sensoriais,\allowbreak{} mas outro,\allowbreak{} que precisamente,\allowbreak{} equipara os dois seres que o senso comum tende a manter separados por princípio:\allowbreak{} o guia de um lado,\allowbreak{} o guiado de outro.\allowbreak{}\par{}Surge então um terceiro ator,\allowbreak{} tornando ainda mais complicado o cidadão híbrido:\allowbreak{} o rádio do carro,\allowbreak{} não um puro acessório com finalidade de proporcionar um prazer estético superficial e insensível,\allowbreak{} mas carga modal ulterior que se adjunge,\allowbreak{} traduzindo-\allowbreak{}o,\allowbreak{} junto ao corpo-\allowbreak{}máquina precedente.\allowbreak{} Uma coisa é dirigir um carro,\allowbreak{} outra é dirigi-\allowbreak{}lo com o rádio ligado,\allowbreak{} como sabemos bem,\allowbreak{} e como nos recorda sub-\allowbreak{}repticiamente nosso texto exemplar.\allowbreak{} É outro hábito discutível e totalmente diferente:\allowbreak{} muda o modo de dirigir,\allowbreak{} de pensar e valorizar o próprio fato de estar no carro:\allowbreak{} muda a afetividade,\allowbreak{} muda até mesmo,\allowbreak{} em certos casos,\allowbreak{} a relação de pressuposição entre programa de ação final e programa de uso:\allowbreak{} o rádio me acompanha enquanto vou para qualquer lugar de carro? Ou ao contrário,\allowbreak{} para qualquer lugar que eu vá,\allowbreak{} será apenas para ouvir o rádio no carro?\par{}Hoje os celulares,\allowbreak{} os \textit{iPods},\allowbreak{} as pequenas telas televisivas das quais são dotados os veículos atuais,\allowbreak{} tornam as coisas muito mais confusas,\allowbreak{} e os híbridos ainda mais encaixotados e instáveis.\allowbreak{} Na época de nosso desenho,\allowbreak{} em todo caso,\allowbreak{} não havia aparelhos para se envolver,\allowbreak{} como Wheeler sabe muito bem,\allowbreak{} interagindo com os inimigos nos outros carros – novo Sun Tzu – em uma espécie de tática de esgotamento,\allowbreak{} e obrigando-\allowbreak{}os todos a seguir,\allowbreak{} como punição,\allowbreak{} o ritmo doce e embalante da valsa de Strauss que o rádio propaga no ar.\allowbreak{} Isso dá lugar a uma ulterior rima entre o humano e o não humano.\allowbreak{} Logo que alguém,\allowbreak{} de trás,\allowbreak{} insulta Wheeler chamando-\allowbreak{}o de “suíno”,\allowbreak{} eis que o indivíduo – como um ato de desprezo a mais – assume as feições de um porco [fig.\allowbreak{} 5] e toca a buzina do carro que,\allowbreak{} em uma metamorfose simultânea (\allowbreak{}híbrido no híbrido)\allowbreak{},\allowbreak{} grunhe (\allowbreak{}\textit{oink!\allowbreak{}})\allowbreak{}.\allowbreak{} A série das transformações paralelas poderia assumir desta forma uma representação do seguinte tipo:\allowbreak{}\end{multicols}
\ctable[
  width=\textwidth, pos = ht, left, long
]
{p{0.54\textwidth}p{0.06\textwidth}p{0.36\textwidth}}
{}
{ \\\hline
\multicolumn{1}{L{0.54\textwidth}}{Carro => carro com rádio => buzina que grunhe} & \multicolumn{1}{L{0.06\textwidth}}{ } & \multicolumn{1}{L{0.36\textwidth}}{\textit{dimensão do não humano}} \\\hline \multicolumn{1}{L{0.54\textwidth}}{Wheeler => Wheeler que ouve o rádio => suíno} & \multicolumn{1}{L{0.06\textwidth}}{ } & \multicolumn{1}{L{0.36\textwidth}}{\textit{dimensão do humano}} \\\hline 
}
\begin{multicols}{2}

\par
{
\centering{
\includegraphics[width=\maxwidth{0.5\textwidth}]{r1982-2553-gal-29-0028-gf05.tif}
}
\captionof{figure}{\textbf{Fig.\allowbreak{} 5.\allowbreak{}:} \textit{Wheeler suíno}} 
}
\par
\par{}Para modificar novamente a disposição pragmática e passional,\allowbreak{} intervém na cena um novo ator da cidade:\allowbreak{} o semáforo.\allowbreak{} Bem quando o rádio havia predisposto Wheeler à desaceleração e a um relaxamento durativos (\allowbreak{}embora com o objetivo de irritar os outros híbridos idênticos)\allowbreak{},\allowbreak{} eis que o \textit{stop} do sinal vermelho provoca no sujeito homem\fshyp{}máquina um desespero neurótico pelo tempo que ele é obrigado a desperdiçar (\allowbreak{}“trinta segundos da vida perdidos!\allowbreak{} Por que comigo? Maldição,\allowbreak{} não é possível!\allowbreak{}”)\allowbreak{},\allowbreak{} tornando necessária a ativação de um programa de busca pelo tempo perdido.\allowbreak{} É então que o semáforo imediatamente se torna outra coisa:\allowbreak{} não um instrumento de regulação do tráfego urbano,\allowbreak{} mas algo muito similar a seu contrário:\allowbreak{} uma espécie de \textit{gongo} a partir do qual se empreende uma verdadeira corrida automobilística.\allowbreak{} Os carros se dispõem nervosamente ao longo da linha de partida [fig.\allowbreak{} 6],\allowbreak{} de modo que,\allowbreak{} assim que surge o sinal verde,\allowbreak{} todos disparam como loucos,\allowbreak{} até.\allowbreak{}.\allowbreak{}.\allowbreak{} o sinal vermelho seguinte,\allowbreak{} e assim continuando até o exaurimento nervoso,\allowbreak{} ou melhor,\allowbreak{} até a demasiado humana perda de controle do veículo.\allowbreak{} Wheeler bate contra o poste de outro sinal vermelho e então,\allowbreak{} como enésima variação,\allowbreak{} o carro quase destruído,\allowbreak{} em uma pausa obrigatória,\allowbreak{} torna-\allowbreak{}se lugar de passagem para um cortejo de passageiros que desce do ônibus parado ao lado [fig.\allowbreak{} 7].\allowbreak{} Não mais meio de transporte,\allowbreak{} portanto,\allowbreak{} mas um tipo de passarela formada inesperadamente e uma criativa bricolagem urbana.\allowbreak{} Até que o último passageiro,\allowbreak{} muito educado,\allowbreak{} feche a porta do carro cumprimentando o pobre Wheeler,\allowbreak{} derrotado pelo vergonhoso acontecimento que está suportando.\allowbreak{} Menos mal – como válvula de escape infantil – que há \textit{pé-\allowbreak{}móveis} desventurados para descontar:\allowbreak{} é ótimo espirrar água suja sobre eles,\allowbreak{} passando com velocidade sobre poças de água.\allowbreak{}
\par
{
\centering{
\includegraphics[width=\maxwidth{0.5\textwidth}]{r1982-2553-gal-29-0028-gf06.tif}
}
\captionof{figure}{\textbf{Fig.\allowbreak{} 6.\allowbreak{}:} \textit{Carros parados ao semáforo,\allowbreak{} posicionados em posição de partida}} 
}
\par

\par
{
\centering{
\includegraphics[width=\maxwidth{0.5\textwidth}]{r1982-2553-gal-29-0028-gf07.tif}
}
\captionof{figure}{\textbf{Fig.\allowbreak{}7.\allowbreak{}:} \textit{Carro-\allowbreak{}passarela para passageiros do ônibus}} 
}
\par
\par{}O último ato das aventuras de Wheeler na cidade é o reencontro do objeto de valor supremo:\allowbreak{} a vaga para estacionar.\allowbreak{} Uma ótima vaga para o carro se delineia no horizonte,\allowbreak{} reavivando nele a sede de domínio do homem sobre o homem.\allowbreak{} Impondo-\allowbreak{}se sobre outros automóveis desejosos de parar,\allowbreak{} não sem antes ter maltratado outros carros estacionados,\allowbreak{} Wheeler obtém finalmente a conjunção com o almejado objeto de desejo,\allowbreak{} sem compreender – todavia – o que virá a acontecer com sua própria identidade.\allowbreak{} Perdida,\allowbreak{} voltará a ser a partir daquele momento,\allowbreak{} com efeito,\allowbreak{} o virtuoso Mr.\allowbreak{} Walker,\allowbreak{} entre as calçadas da cidade.\allowbreak{} 
\section*{Walker 2:\allowbreak{} na cidade}
\par{}Com o carro diabólico estacionado,\allowbreak{} Wheeler assume subitamente as feições de Walker (\allowbreak{}nota-\allowbreak{}se que,\allowbreak{} enquanto a transformação de Walker em Wheeler é longa e espetacular,\allowbreak{} a inversa é oculta e imediata)\allowbreak{}.\allowbreak{} O espaço que é propriamente dele,\allowbreak{} apesar de si mesmo,\allowbreak{} é dessa vez o centro da cidade,\allowbreak{} por onde os carros passam seguindo suas vidas,\allowbreak{} enquanto os cidadãos normais passeiam tranquilos em compras ou simples caminhadas.\allowbreak{} Pelo menos a princípio.\allowbreak{} Na realidade,\allowbreak{} a guerra entre \textit{pé-\allowbreak{}móveis} e automóveis sempre continua.\allowbreak{} Basta que se proponha o problema de atravessar uma rua.\allowbreak{} É impossível ou – o que seria o mesmo – uma tentativa extremamente arriscada para qualquer transeunte:\allowbreak{} os carros o recusam,\allowbreak{} devolvendo-\allowbreak{}o imediatamente às calçadas,\allowbreak{} quase demarcando o território com limites espaciais muito fortes – de um lado os pedestres,\allowbreak{} de outro os carros – que uma vez cruzados,\allowbreak{} a própria integridade física é seriamente posta em risco.\allowbreak{} E quando Walker,\allowbreak{} concentrando-\allowbreak{}se,\allowbreak{} se carrega modalmente com o dito muito fácil “querer é poder”,\allowbreak{} e tenta atravessar mais uma vez,\allowbreak{} a reação é implacável:\allowbreak{} o logotipo do carro da vez,\allowbreak{} uma estrela,\allowbreak{} situado na ponta do capô frontal,\allowbreak{} se torna subitamente a mira de um fuzil,\allowbreak{} e assim é utilizado para melhor atingir o sujeito iludido.\allowbreak{} Os carros se tornam por fim animais ferozes que,\allowbreak{} com o capô dessa vez transformado em uma grande boca dentada,\allowbreak{} tentam agarrar o pobre Walker [fig.\allowbreak{} 8].\allowbreak{}
\par
{
\centering{
\includegraphics[width=\maxwidth{0.5\textwidth}]{r1982-2553-gal-29-0028-gf08.tif}
}
\captionof{figure}{\textbf{Fig.\allowbreak{}8.\allowbreak{}:} \textit{Walker perseguido por carros enfurecidos}} 
}
\par
\par{}Como um adjuvante inesperado,\allowbreak{} entra em cena o jornal.\allowbreak{} Ou melhor:\allowbreak{} o híbrido Walker que lê o jornal,\allowbreak{} atravessando a rua distraído.\allowbreak{} A leitura do cotidiano – embora traga a má notícia do aumento vertiginoso dos acidentes de trânsito – é para Wheeler algo muito similar à aquisição do meio mágico nas fábulas russas:\allowbreak{} o \textit{pé-\allowbreak{}móvel} que cruza o território do outro,\allowbreak{} neste caso do carro,\allowbreak{} quando imerso no jornal se torna como que invisível,\allowbreak{} certamente invulnerável,\allowbreak{} e milagrosamente consegue alcançar a calçada,\allowbreak{} enquanto os carros,\allowbreak{} atirando-\allowbreak{}se em torno sem ao menos atingir-\allowbreak{}lhe de raspão,\allowbreak{} acertam-\allowbreak{}se a si mesmos sucessivamente [fig.\allowbreak{} 9].\allowbreak{}
\par
{
\centering{
\includegraphics[width=\maxwidth{0.5\textwidth}]{r1982-2553-gal-29-0028-gf09.tif}
}
\captionof{figure}{\textbf{Fig.\allowbreak{}9.\allowbreak{}:} \textit{A mágica leitura do jornal salva Walker dos carros,\allowbreak{} que se colidem sucessivamente}} 
}
\par
\par{}Mas mais uma vez as coisas são mais complicadas do que parecem:\allowbreak{} os atores em jogo – na cidade – são sempre novos e continuamente imprevisíveis.\allowbreak{} Entra em cena um híbrido particular,\allowbreak{} de fato,\allowbreak{} metade menino e metade patinete,\allowbreak{} que se atira não pelas ruas urbanas,\allowbreak{} mas pelas calçadas,\allowbreak{} arremessando o pobre Walker de ponta cabeça com seu bem-\allowbreak{}amado jornal,\allowbreak{} além de zombar com prazer [fig.\allowbreak{} 10].\allowbreak{}
\par
{
\centering{
\includegraphics[width=\maxwidth{0.5\textwidth}]{r1982-2553-gal-29-0028-gf10.tif}
}
\captionof{figure}{\textbf{Fig.\allowbreak{}10.\allowbreak{}:} \textit{O menino no patinete marca seu veículo com outro Walker atropelado na calçada (\allowbreak{}ao fundo o jornal,\allowbreak{} já abandonado)\allowbreak{}}} 
}
\par
\par{}Assim,\allowbreak{} o único lugar seguro,\allowbreak{} a única âncora de salvação é o próprio carro,\allowbreak{} para o qual Walker retorna sendo um pouco menos \textit{walker} do que antes,\allowbreak{} pois destruído,\allowbreak{} agora caminha a quatro patas.\allowbreak{} E eis que inevitavelmente,\allowbreak{} assim que liga o motor,\allowbreak{} a metamorfose,\allowbreak{} digamos,\allowbreak{} básica,\allowbreak{} é representada,\allowbreak{} e Mr.\allowbreak{} Walker reassume as feições e paixões do terrível Mr.\allowbreak{} Wheeler,\allowbreak{} com o mesmo processo de antes.\allowbreak{}
\section*{Wheeler 2:\allowbreak{} o acidente}
\par{}Neste ponto a narrativa faz economia de si mesma:\allowbreak{} em vez de repropor iterativamente os mesmos eventos,\allowbreak{} suficientemente previsíveis,\allowbreak{} parte em direção ao final encadeando uma temporalidade por definição singularizante e uma aspectualidade de fato pontual:\allowbreak{} a do acidente.\allowbreak{} Tentando sair do estacionamento de forma ruidosa e desajeitada – uma batida à frente,\allowbreak{} outra atrás – Wheeler não percebe um carro em disparada ao longo da rua e é atingido em cheio.\allowbreak{} Seu carro fica compassivamente em pedaços,\allowbreak{} e enquanto ele continua insensível a buzinar,\allowbreak{} sentado no veículo,\allowbreak{} é rebocado por um caminhão guincho que parte piedosamente em direção ao cemitério [fig.\allowbreak{} 11].\allowbreak{}
\par
{
\centering{
\includegraphics[width=\maxwidth{0.5\textwidth}]{r1982-2553-gal-29-0028-gf11.tif}
}
\captionof{figure}{\textbf{Fig.\allowbreak{}11.\allowbreak{}:} \textit{Carro e caminhão guincho}} 
}
\par

\section*{Walker-\allowbreak{}Wheeler:\allowbreak{} final}
\par{}O final nos apresenta assim uma nova metamorfose paralela,\allowbreak{} dupla.\allowbreak{} Por um lado,\allowbreak{} não há mais o híbrido homem-\allowbreak{}automóvel,\allowbreak{} mas uma figura mais complexa que poderíamos esquematicamente representar como:\allowbreak{}\par{}(\allowbreak{}homem +\allowbreak{} automóvel)\allowbreak{} +\allowbreak{} caminhão guincho\par{}Por outro,\allowbreak{} em uma observação aprofundada,\allowbreak{} não há mais o Wheeler de sempre,\allowbreak{} sistematicamente oposto ao dócil Walker,\allowbreak{} mas uma espécie de termo complexo entre os dois personagens:\allowbreak{} um Walker-\allowbreak{}Wheeler que,\allowbreak{} tendo encontrado um inimigo em comum,\allowbreak{} se aliam inesperadamente entre si.\allowbreak{} De quem se trata? Obviamente do narrador,\allowbreak{} que enquanto expõe sua óbvia moral benfeitora – “que isso lhe sirva de lição,\allowbreak{} dirija com atenção,\allowbreak{} respeite as regras.\allowbreak{}.\allowbreak{}.\allowbreak{}” – recebe um conclusivo,\allowbreak{} e útil,\allowbreak{} “\textit{shut up!\allowbreak{}}” [fig.\allowbreak{} 12].\allowbreak{}
\par
{
\centering{
\includegraphics[width=\maxwidth{0.5\textwidth}]{r1982-2553-gal-29-0028-gf12.tif}
}
\captionof{figure}{\textbf{Fig.\allowbreak{}12.\allowbreak{}:} \textit{Olhar de Walker\fshyp{}Wheeler no carro}} 
}
\par

\section*{Uma outra moral}
\par{}Assim,\allowbreak{} ao invés de alinhar-\allowbreak{}nos com a ideologia do narrador,\allowbreak{} o texto parece sugerir outras conclusões teóricas diversas,\allowbreak{} que vão além da moral banal em relação ao devido respeito ao código de trânsito.\allowbreak{}\par{}Se considerarmos o texto em sua linearidade narrativa,\allowbreak{} não será difícil reconhecer ao longo de seu desenvolvimento sintagmático as célebres quatro etapas de um esquema narrativo profundo:\allowbreak{} haveria o momento da manipulação (\allowbreak{}em que o enunciador se constrói como Destinador que institui o sistema de valores e o usa para estipular um contrato com Walker\fshyp{}Sujeito do querer)\allowbreak{}; o momento da competência (\allowbreak{}em que o carro pode ser interpretado como falso adjuvante do Sujeito,\allowbreak{} tecnicamente um poder-\allowbreak{}fazer ilusório,\allowbreak{} ou seja,\allowbreak{} uma espécie de \textit{trickster} que leva o herói em direção a destinos e valores absolutamente diversos daqueles previstos no contrato)\allowbreak{}; o momento da performance (\allowbreak{}em que Walker encontra em si mesmo,\allowbreak{} transformado em Wheeler,\allowbreak{} o próprio Antissujeito interno)\allowbreak{}; o momento da sanção (\allowbreak{}em que o enunciador retorna sob a forma de Destinador Julgador que considera negativamente a ação do Sujeito,\allowbreak{} reforçando os valores corretos do cidadão modelo)\allowbreak{}.\allowbreak{}\par{}Todavia,\allowbreak{} conforme dito acima,\allowbreak{} o texto parece manifestar ao menos uma estrutura circular,\allowbreak{} em que a introdução já prevê o final da história,\allowbreak{} sugerindo uma reprodutibilidade ao infinito dos eventos narrados,\allowbreak{} uma constitutiva ciclicidade dos fatos urbanos e sociais.\allowbreak{} E a zombaria final de Walker\fshyp{}Wheeler aliados contra o presumido Destinador julgador – junto a um olhar,\allowbreak{} do carro,\allowbreak{} que convida o espectador a tomar posição,\allowbreak{} de fato calando-\allowbreak{}o – parece provar esta segunda hipótese hermenêutica.\allowbreak{} \textit{Motormania} não é uma história contra a geral e moderníssima mania do carro,\allowbreak{} mas uma teoria implícita sobre espaços urbanos e sobre aquilo que,\allowbreak{} nesses e com esses,\allowbreak{} sempre e todavia ocorre.\allowbreak{} Trata-\allowbreak{}se de um texto que mostra e motiva a multiplicação dos híbridos,\allowbreak{} que mal se esconde por trás de uma duplicidade -\allowbreak{} Walker\fshyp{}Jekill \textit{vs} Wheeler\fshyp{}Hide – de fato simplificadora.\allowbreak{} A cidade,\allowbreak{} como nos é ratificado,\allowbreak{} não é apenas seu centro,\allowbreak{} mas começa onde ainda não existe,\allowbreak{} fora dela mesma,\allowbreak{} e na relação que vive com suas periferias,\allowbreak{} com os bairros residenciais externos que,\allowbreak{} segmentando-\allowbreak{}a em partes,\allowbreak{} ao mesmo tempo são mantidos junto,\allowbreak{} articulados sensivelmente.\allowbreak{} Assim,\allowbreak{} ela é também e sobretudo a rede viária que conecta e desconecta suas zonas e bairros,\allowbreak{} com tudo aquilo que ocorre nesses locais:\allowbreak{} conflitos frequentes e contratos raros entre entidades de natureza aparentemente diversa – humana e não humana – que se conectam entre si dando lugar a seres que de híbrido têm,\allowbreak{} em uma observação aprofundada,\allowbreak{} o único lado da substância da expressão,\allowbreak{} enquanto pelo do conteúdo,\allowbreak{} são absolutamente sensíveis,\allowbreak{} claros,\allowbreak{} funcionais ao contexto e à situação que,\allowbreak{} construindo-\allowbreak{}os,\allowbreak{} os leva a agir e a sentir.\allowbreak{} Por trás da metamorfose de fundo da cidade moderna (\allowbreak{}abstrata)\allowbreak{},\allowbreak{} do pedestre em automobilista e vice-\allowbreak{}versa,\allowbreak{} pululam inúmeras outras (\allowbreak{}concretíssimas)\allowbreak{} transformações,\allowbreak{} em que tecnologias como o rádio do carro,\allowbreak{} a buzina,\allowbreak{} os semáforos,\allowbreak{} o ônibus,\allowbreak{} o patinete,\allowbreak{} mas também articulações topológicas como rua\fshyp{}estacionamento\fshyp{}calçada\fshyp{}sinalética (\allowbreak{}vertical e horizontal)\allowbreak{} etc.\allowbreak{},\allowbreak{} contribuem para a produção,\allowbreak{} algumas vezes,\allowbreak{} de figuras actoriais diferentes,\allowbreak{} cada uma com paixões próprias,\allowbreak{} programas próprios,\allowbreak{} e o próprio caráter.\allowbreak{}\par{}Como ulterior e implícita demonstração do fato que,\allowbreak{} do ponto de vista sociossemiótico,\allowbreak{} espaços,\allowbreak{} sujeitos,\allowbreak{} corpos,\allowbreak{} coletividades e tecnologias devem ser considerados de acordo com uma única perspectiva teórica e metodológica,\allowbreak{} realçando como esses não são entidades separadas,\allowbreak{} nitidamente distinguíveis entre si,\allowbreak{} se não por uma abstração filosófica banal que o senso comum – frequentemente subordinado a muitas ciências humanas – apoia.\allowbreak{} Consequentemente,\allowbreak{} nenhum desses elementos – um lugar,\allowbreak{} um instrumento,\allowbreak{} um corpo,\allowbreak{} um grupo.\allowbreak{}.\allowbreak{}.\allowbreak{} – possui valências e funções próprias,\allowbreak{} desejos ou paixões específicos,\allowbreak{} se não por conotações sociais mais ou menos estáveis e naturalizadas.\allowbreak{} Assim (\allowbreak{}como nosso desenho nos diz explicitamente)\allowbreak{},\allowbreak{} não existem sujeitos bons ou maus por si mesmos,\allowbreak{} uma vez que é em suas traduções com meios e lugares que são determinadas suas personalidades reais.\allowbreak{} O Pateta em casa,\allowbreak{} no jardim,\allowbreak{} na garagem,\allowbreak{} no carro percorrendo vias arteriais,\allowbreak{} no carro com o rádio ligado,\allowbreak{} na cidade,\allowbreak{} em busca da vaga para estacionar,\allowbreak{} passeando pelas calçadas,\allowbreak{} na tentativa de atravessar a rua,\allowbreak{} imerso na leitura do jornal,\allowbreak{} não é absolutamente a mesma pessoa,\allowbreak{} não tem os mesmos programas e os mesmos sentimentos.\allowbreak{} Além do que também o carro é um “personagem” diferente no cemitério,\allowbreak{} na garagem,\allowbreak{} estacionado,\allowbreak{} correndo com outros carros,\allowbreak{} preso no caminhão guincho,\allowbreak{} etc.\allowbreak{} Analogamente (\allowbreak{}como nosso desenho diz com os próprios meios audiovisuais)\allowbreak{},\allowbreak{} não existem meios de transporte bons ou ruins por si mesmos (\allowbreak{}o carro negativo e a bicicleta positiva,\allowbreak{} por exemplo)\allowbreak{},\allowbreak{} dado que é sempre o contexto narrativo que lhes fornece uma “alma” de qualquer tipo,\allowbreak{} em suas relações constitutivas e dinâmicas com os sujeitos humanos que os utilizam,\allowbreak{} com os outros meios de transporte,\allowbreak{} com os lugares que percorrem.\allowbreak{} Wheeler se torna mau no carro se em conflito com outros híbridos como ele; torna-\allowbreak{}se perversamente tranquilo quando liga o rádio,\allowbreak{} transformando a si e ao carro em suínos provisórios; torna-\allowbreak{}se triste e passivo quando suporta a gentileza dos passageiros do ônibus que usam seu carro destruído como uma conveniente passarela.\allowbreak{} Ainda (\allowbreak{}e nosso desenho claramente destaca)\allowbreak{} não há espaços positivos por si mesmos e outros negativos por si mesmos:\allowbreak{} a faixa de rodagem é um lugar perigosíssimo para os pedestres,\allowbreak{} a menos que mergulhados na leitura do cotidiano,\allowbreak{} que,\allowbreak{} isolando Walker do resto do mundo,\allowbreak{} como por magia o torna invulnerável; da mesma forma,\allowbreak{} a calçada é um território seguro para Walker,\allowbreak{} a menos que por acaso se encontre com o menino do patinete que,\allowbreak{} na verdade,\allowbreak{} o atropela sem nenhuma dificuldade.\allowbreak{} Além de tudo,\allowbreak{} rindo dissimuladamente:\allowbreak{} “mais um!\allowbreak{}”.\allowbreak{}

\end{multicols}
\medskip\par\noindent
\footnotesize{\textsuperscript{2}O desenho \textit{Motormania} é facilmente encontrado em \textit{www.\allowbreak{}youtube.\allowbreak{}com} na versão original em inglês (\allowbreak{}\textit{http:\allowbreak{}\fshyp{}\fshyp{}www.\allowbreak{}youtube.\allowbreak{}com\fshyp{}watch?v=0ZgiVicpZGk})\allowbreak{} ou dublado em português (\allowbreak{}\textit{https:\allowbreak{}\fshyp{}\fshyp{}www.\allowbreak{}youtube.\allowbreak{}com\fshyp{}watch?v=RMZ3bsrtJZ0})\allowbreak{}.\allowbreak{}\\Tradução do italiano para o português de Rafael G.\allowbreak{} Lenzi\\Este artigo pretende se inserir,\allowbreak{} sobre o panorama dos estudos sociológicos,\allowbreak{} antropológicos e urbanísticos acerca do ambiente da cidade e do território metropolitano,\allowbreak{} na já vasta bibliografia sobre a semiótica da espacialidade urbana (\allowbreak{}Barthes,\allowbreak{} Greimas,\allowbreak{} Lotman,\allowbreak{} Marin,\allowbreak{} De Certeau,\allowbreak{} Hammad,\allowbreak{} etc.\allowbreak{})\allowbreak{},\allowbreak{} pelos quais reenviamos a nosso trabalho recente:\allowbreak{} \textit{Figure di città.\allowbreak{} Discorsi sociali e spazi urbani},\allowbreak{} Milano,\allowbreak{} Mimesis 2013.\allowbreak{} A ideia é cruzar a reflexão sobre a significação urbana (\allowbreak{}que considera espaço e sujeito constituindo-\allowbreak{}se reciprocamente)\allowbreak{} com a observação sobre o sentido da tecnologia,\allowbreak{} conduzida sempre com a escolta da teoria semiótica,\allowbreak{} de Bruno Latour e de sua escola (\allowbreak{}cf.\allowbreak{} para todos,\allowbreak{} \textit{Il senso degli oggetti tecnici},\allowbreak{} organizado por Alvise Mattozzi,\allowbreak{} Meltemi,\allowbreak{} Roma 2006)\allowbreak{},\allowbreak{} em que a ideia do híbrido “humano-\allowbreak{}não humano” é constitutiva.\allowbreak{} Assim como com a pesquisa semiótica sobre o corpo (\allowbreak{}cf.\allowbreak{} Jacques Fontanille,\allowbreak{} \textit{Séma et soma:\allowbreak{} les figures du corps},\allowbreak{} Paris,\allowbreak{} Maisonneuve et Larose,\allowbreak{} 2004; Gianfranco Marrone,\allowbreak{} \textit{La Cura Ludovico},\allowbreak{} Torino,\allowbreak{} Einaudi 2005)\allowbreak{}.\allowbreak{}\\}

\medskip\par\noindent
\footnotesize{Este é um artigo publicado em acesso aberto (Open Access) sob a licença Creative Commons Attribution Non-Commercial, que permite uso, distribuição e reprodução em qualquer meio, sem restrições desde que sem fins comerciais e que o trabalho original seja corretamente citado.}
