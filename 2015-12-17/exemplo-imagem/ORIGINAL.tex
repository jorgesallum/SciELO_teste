\selectlanguage{brazilian}
\setlistdepth{8}
\noindent{}KOCH, A.K., 
MONTEIRO, R.F., 
ILKIU-BORGES, A.L. \textbf{Checklist de Bromeliaceae da região da Volta Grande do Xingu, Pará, Brasil} Rodriguésia. 66(2): 455-464. \url{http://dx.doi.org/10.1590/2175-7860201566213}\bigskip{}

{
\renewcommand{\abstractname}{Resumo}
\begin{abstract}
\begin{spacing}{0.93}
Apresenta-\allowbreak{}se neste trabalho um inventário das espécies de Bromeliaceae da Volta Grande do Xingu (\allowbreak{}Pará,\allowbreak{} Brasil)\allowbreak{},\allowbreak{} sendo também fornecida uma chave para identificação das espécies da região,\allowbreak{} e seus respectivos status de conservação quando existentes.\allowbreak{} O material utilizado foi obtido através das atividades do Projeto Salvamento e Aproveitamento Científico da Flora da Usina Hidrelétrica de Belo Monte,\allowbreak{} bem como na coleção do herbário MG.\allowbreak{} Na área de estudo,\allowbreak{} Bromeliaceae encontra-\allowbreak{}se representada por 20 espécies e sete gêneros.\allowbreak{} \textit{Tillandsia} e \textit{Aechmea} foram os mais representativos.\allowbreak{} 
\end{spacing}
\vspace*{1.9mm}
\fontsize{9}{10.8}\selectfont{\textit{Palavras-chave:} Amazônia brasileira, Belo Monte, Poales}
\end{abstract}
}

\noindent{}KOCH, A.K., 
MONTEIRO, R.F., 
ILKIU-BORGES, A.L. \textbf{\textit{Checklist} of Bromeliaceae in the region of the Volta Grande do Xingu,\allowbreak{} Pará,\allowbreak{} Brasil} Rodriguésia. 66(2): 455-464. \url{http://dx.doi.org/10.1590/2175-7860201566213}\bigskip{}

{
\renewcommand{\abstractname}{Abstract}
\begin{abstract}
\begin{spacing}{0.93}
In this paper is showed a checklist of the species of Bromeliaceae of the Volta Grande do Xingu (\allowbreak{}Pará,\allowbreak{} Brazil)\allowbreak{},\allowbreak{} where is provided a identification key for the region species,\allowbreak{} and your respective status of conservation when available.\allowbreak{} The material used in this study was obtained through the activities of the Project Rescue and Scientific Utilization of the Flora of the Belo Monte hydroelectric dam,\allowbreak{} executed by Norte Energia S.\allowbreak{}A and in the collection of herbarium MG.\allowbreak{} In the study area,\allowbreak{} Bromeliaceae is represented by 20 species and seven genera.\allowbreak{} \textit{Tillandsia} and \textit{Aechmea} were the most representative.\allowbreak{} 
\end{spacing}
\vspace*{1.9mm}
\fontsize{9}{10.8}\selectfont{\textit{Key words:} Brazilian Amazon, Belo Monte, Poales}
\end{abstract}
}

\vspace*{-1.6mm}
{\noindent\fontsize{9}{10.8}\selectfont{Recebido em: 18/09/2014;
Aceito em: 09/12/2014.}}
\begin{multicols}{2}
\section*{Introdução}
\par{}Bromeliaceae é uma família essencialmente neotropical composta por 58 gêneros e aproximadamente 3.\allowbreak{}199 espécies (\allowbreak{} \textsuperscript{Luther 2012}; World Checklist 2014)\allowbreak{} herbáceas de hábitos variados e,\allowbreak{} segundo análises filogenéticas baseadas em dados moleculares,\allowbreak{} é monofilética (\allowbreak{} \textsuperscript{Givnish \textit{et al }.\allowbreak{} 2004},\allowbreak{} \textsuperscript{2007},\allowbreak{} \textsuperscript{2011})\allowbreak{}.\allowbreak{} Atualmente,\allowbreak{} encontra-\allowbreak{}se representada por 44 gêneros e 1.\allowbreak{}323 espécies no Brasil,\allowbreak{} das quais 1.\allowbreak{}155 são endêmicas,\allowbreak{} ocorrendo em todos os ecossistemas desde o nível do mar até elevadas altitudes (\allowbreak{} \textsuperscript{Forzza \textit{et al.\allowbreak{}} 2014})\allowbreak{}.\allowbreak{}\par{}Para a Amazônia brasileira foram registradas 132 espécies de Bromeliaceae em 22 gêneros,\allowbreak{} números baixos quando comparados aos da Mata Atlântica (\allowbreak{}891 espécies e 30 gêneros)\allowbreak{},\allowbreak{} um dos centros de diversidade da família (\allowbreak{} \textsuperscript{Forzza \textit{et al }.\allowbreak{} 2014}; \textsuperscript{Martinelli \textit{et al.\allowbreak{} } 2008})\allowbreak{}.\allowbreak{} Na Amazônia,\allowbreak{} as espécies são encontradas com maior frequência em áreas de baixios,\allowbreak{} campinas,\allowbreak{} campinaranas e igapós,\allowbreak{} como apontado por \textsuperscript{Nogueira-\allowbreak{}Braga (\allowbreak{}1977)\allowbreak{}},\allowbreak{} \textsuperscript{Ribeiro \textit{et al.\allowbreak{}} (\allowbreak{}1999)\allowbreak{}} e \textsuperscript{Sousa \&\allowbreak{\allowbreak{}\allowbreak{}}\allowbreak{} Wanderley (\allowbreak{}2007)\allowbreak{}}.\allowbreak{} Das 132 espécies registradas para a Amazônia brasileira,\allowbreak{} 53 também foram registradas no estado do Pará (\allowbreak{} \textsuperscript{Forzza \textit{et al.\allowbreak{}} 2014})\allowbreak{}.\allowbreak{}\par{}Inventários ou listagens de espécies de Bromeliaceae nessa região são pouco comuns,\allowbreak{} destacando-\allowbreak{}se \textsuperscript{Dubs (\allowbreak{}1998)\allowbreak{}} com a flora da família para o Mato Grosso e \textsuperscript{Ribeiro \textit{et al}.\allowbreak{} (\allowbreak{}1999)\allowbreak{} } com o tratamento taxonômico da família para a Reserva Adolpho Ducke em Manaus.\allowbreak{} No estado do Pará,\allowbreak{} os estudos se focaram apenas nas espécies de hábito epifítico,\allowbreak{} a saber:\allowbreak{} \textsuperscript{Koch \textit{et al.\allowbreak{}} (\allowbreak{}2013)\allowbreak{}},\allowbreak{} realizaram estudo florístico taxonômico das Bromeliaceae epifíticas da Floresta Nacional de Caxiuanã (\allowbreak{}cinco espécies)\allowbreak{}; \textsuperscript{Quaresma \&\allowbreak{\allowbreak{}\allowbreak{}}\allowbreak{} Medeiros (\allowbreak{}2009)\allowbreak{} } inventariaram a família na APA Ilha do Combu (\allowbreak{}seis espécies)\allowbreak{}; e \textsuperscript{Quaresma \&\allowbreak{\allowbreak{}\allowbreak{}}\allowbreak{} Jardim (\allowbreak{}2012}; \textsuperscript{2013)\allowbreak{}},\allowbreak{} estudaram a diversidade,\allowbreak{} a fitossociologia e a distribuição espacial das bromélias epifíticas encontradas por \textsuperscript{Quaresma \&\allowbreak{\allowbreak{}\allowbreak{}}\allowbreak{} Medeiros (\allowbreak{}2009)\allowbreak{}} em ambiente de várzea.\allowbreak{}\par{}Vê-\allowbreak{}se que os estudos sobre Bromeliaceae no Pará são pontuais,\allowbreak{} revelando que ainda há necessidade de conhecer a flora local de diversas áreas do estado,\allowbreak{} o que provavelmente contribuirá para o aumento de táxons registrados,\allowbreak{} podendo gerar futuras ações de conservação das espécies de bromélias no Pará e consequentemente na Amazônia.\allowbreak{}\par{}Dentre as áreas do Pará que necessitam de maior atenção quanto ao conhecimento de sua flora,\allowbreak{} encontra-\allowbreak{}se a região da Volta Grande do Xingu.\allowbreak{} Segundo \textsuperscript{Salomão \textit{et al.\allowbreak{}} (\allowbreak{}2007)\allowbreak{}},\allowbreak{} essa região passou por considerável mudança na sua cobertura vegetal original,\allowbreak{} devido à forte ação antrópica presente na zona de influência da rodovia Transamazônica (\allowbreak{}BR-\allowbreak{}230)\allowbreak{} e suas transversais,\allowbreak{} o que provocou uma grande e desordenada ocupação humana como resultado de projetos de colonização agrária durante os últimos 30 anos e,\allowbreak{} atualmente,\allowbreak{} abriga a construção da Usina Hidrelétrica de Belo Monte.\allowbreak{}\par{}Embora essa região tenha sofrido por muito tempo a perda considerável da sua cobertura vegetal original,\allowbreak{} ainda detém diferentes unidades de paisagens,\allowbreak{} como florestas ombrófilas ou úmidas,\allowbreak{} vegetação aluvial e florestas secundárias recentes e antigas (\allowbreak{}MPEG 2008)\allowbreak{}.\allowbreak{} Entretanto,\allowbreak{} estas paisagens ainda são pouco conhecidas no que se refere a sua riqueza e composição.\allowbreak{} Para tanto,\allowbreak{} as únicas fontes de dados sobre a flora da região da Volta Grande do Xingu,\allowbreak{} são a caracterização das florestas de Belo Monte (\allowbreak{} \textsuperscript{Salomão \textit{et al.\allowbreak{}} 2007})\allowbreak{},\allowbreak{} o Relatório Final de Vegetação (\allowbreak{}MPEG 2008)\allowbreak{} para a implantação da UHE Belo Monte,\allowbreak{} no qual apenas sete espécies de Bromeliaceae foram citadas e,\allowbreak{} \textsuperscript{Mello \textit{et al}.\allowbreak{} (\allowbreak{}2012)\allowbreak{}},\allowbreak{} que oferece um guia fotográfico da família para a localidade,\allowbreak{} composto por 15 espécies.\allowbreak{}\par{}Diante da necessidade de incrementar o conhecimento atual da flora de Bromeliaceae na Amazônia brasileira e no Pará,\allowbreak{} objetiva-\allowbreak{}se aqui a apresentação de um \textit{checklist} das espécies na Volta Grande do Xingu (\allowbreak{}Pará,\allowbreak{} Brasil)\allowbreak{},\allowbreak{} bem como divulgar uma chave de identificação para as espécies da região e apontar seus respectivos status de conservação.\allowbreak{}
\section*{Material \&\allowbreak{\allowbreak{}\allowbreak{}}\allowbreak{} Métodos}
\par{}A região da Volta Grande do Xingu abrange três municípios:\allowbreak{} Altamira,\allowbreak{} Anapu e Vitória do Xingu (\allowbreak{} Fig.\allowbreak{} 1)\allowbreak{},\allowbreak{} das mesorregiões do baixo e médio Xingu,\allowbreak{} no estado do Pará (\allowbreak{}MPEG 2008)\allowbreak{}.\allowbreak{} O clima da região é do tipo Am segundo classificação de Köppen,\allowbreak{} com temperatura média de 26° C,\allowbreak{} precipitação anual de 2.\allowbreak{}289 mm e umidade relativa entre 78 e 88\%\allowbreak{\allowbreak{}\allowbreak{}}\allowbreak{} (\allowbreak{} \textsuperscript{Sousa Jr.\allowbreak{} \textit{et al}.\allowbreak{} 2006 })\allowbreak{}.\allowbreak{}
\par
{
\centering{
\includegraphics[width=\maxwidth{0.5\textwidth}]{not-found.png}
}
\captionof{figure}{\textbf{Figura 1:} \textit{Localização da Volta Grande do rio Xingu,\allowbreak{} Pará,\allowbreak{} Brasil.\allowbreak{} \textbf{Figure 1} – Location of the Volta Grande do rio Xingu,\allowbreak{} Pará,\allowbreak{} Brazil.\allowbreak{}}} 
}
\par
\par{}A cobertura vegetal é composta por quatro fitofisionomias principais,\allowbreak{} a saber:\allowbreak{} floresta ombrófila densa,\allowbreak{} floresta ombrófila aluvial,\allowbreak{} floresta ombrófila aberta com palmeiras e floresta ombrófila aberta com palmeiras e cipós,\allowbreak{} conforme descreve \textsuperscript{Salomão \textit{et al}.\allowbreak{} (\allowbreak{}2007)\allowbreak{}},\allowbreak{} além de capoeiras e áreas de pastagens.\allowbreak{} \par{}O material utilizado neste trabalho foi obtido através das atividades do Projeto Salvamento e Aproveitamento Científico da Flora da Usina Hidrelétrica de Belo Monte,\allowbreak{} como parte do Plano Básico Ambiental executado pela Norte Energia S.\allowbreak{}A.\allowbreak{},\allowbreak{} na área de instalação da UHE (\allowbreak{}03°22'00"S\fshyp{}51°56'00"W)\allowbreak{},\allowbreak{} nos municípios de Vitória do Xingu e de Altamira.\allowbreak{} As amostras botânicas foram coletadas entre junho de 2011 e dezembro de 2013 pelas equipes de salvamento do projeto,\allowbreak{} abrangendo todas as unidades de paisagens reconhecidas por \textsuperscript{Salomão \textit{et al.\allowbreak{}} (\allowbreak{}2007)\allowbreak{}}.\allowbreak{} Além disso,\allowbreak{} foram consultadas as coleções dos herbários do Museu Paraense Emílio Goeldi (\allowbreak{}MG)\allowbreak{},\allowbreak{} Embrapa Amazônia Oriental (\allowbreak{}IAN)\allowbreak{},\allowbreak{} Jardim Botânico do Rio de Janeiro (\allowbreak{}RB)\allowbreak{} e o Relatório Final de Vegetação (\allowbreak{}MPEG 2008)\allowbreak{}.\allowbreak{} \par{}Os espécimes resgatados passaram por triagem e posterior destinação (\allowbreak{}exsicatas,\allowbreak{} cultivo e\fshyp{}ou reintrodução)\allowbreak{} conforme necessário.\allowbreak{} Amostras férteis foram herborizadas de acordo com \textsuperscript{Fidalgo \&\allowbreak{\allowbreak{}\allowbreak{}}\allowbreak{} Bononi (\allowbreak{}1984)\allowbreak{}} e depositadas no herbário MG.\allowbreak{} Amostras estéreis foram coletadas e cultivadas na casa de vegetação do Centro de Estudos Ambientais (\allowbreak{}CEA)\allowbreak{} da Norte Energia S.\allowbreak{}A.\allowbreak{} para florescimento e posterior herborização.\allowbreak{} Para as espécies que que foram registradas apenas por fotografias,\allowbreak{} são citadas na Tabela 1 as devidas referências e autorias das imagens utilizadas no presente estudo e\fshyp{}ou ainda amostras do herbário MG utilizadas como material de referência.\allowbreak{}\end{multicols}
\ctable[
  caption = {\textbf{Tabela 1:} \textit{Lista dos táxons registrados na região da Volta Grande do Xingu.\allowbreak{} EP:\allowbreak{} epífita; TE:\allowbreak{} terrícolas.\allowbreak{}*\allowbreak{}Material adicional examinado.\allowbreak{} List of taxa recorded in the region of the Volta Grande do Xingu.\allowbreak{} EP:\allowbreak{} epiphytes; TE:\allowbreak{} terrestrial.\allowbreak{}*\allowbreak{}Additional material examined}}, 
  width=\textwidth, pos = ht, left, long
]
{p{0.47\textwidth}p{0.38\textwidth}p{0.11\textwidth}}
{}
{ \\\hline
\multicolumn{1}{L{0.47\textwidth}}{\textbf{Tâxon}}
 & \multicolumn{1}{L{0.38\textwidth}}{\textbf{Voucher}}
 & \multicolumn{1}{C{0.11\textwidth}}{\textbf{Hábito}} \\\hline 
\multicolumn{1}{L{0.47\textwidth}}{\textit{Aechmea bromeliifolia} (\allowbreak{}Rudge)\allowbreak{} Baker} & \multicolumn{1}{L{0.38\textwidth}}{\textit{Dias,\allowbreak{} A.\allowbreak{} et al.\allowbreak{} 601} (\allowbreak{}MG)\allowbreak{}} & \multicolumn{1}{C{0.11\textwidth}}{EP} \\\hline \multicolumn{1}{L{0.47\textwidth}}{\textit{Aechmea castelnavii} Baker} & \multicolumn{1}{L{0.38\textwidth}}{\textit{Gomes,\allowbreak{} D.\allowbreak{} PSACF 648} (\allowbreak{}MG)\allowbreak{}} & \multicolumn{1}{C{0.11\textwidth}}{EP} \\\hline \multicolumn{1}{L{0.47\textwidth}}{\textit{Aechmea mertensii} (\allowbreak{}G.\allowbreak{}Mey.\allowbreak{})\allowbreak{} Schult.\allowbreak{} \&\allowbreak{\allowbreak{}\allowbreak{}}\allowbreak{} Schult.\allowbreak{}f.\allowbreak{}} & \multicolumn{1}{L{0.38\textwidth}}{\textit{Raul,\allowbreak{} F.\allowbreak{} PSACF 300} (\allowbreak{}MG)\allowbreak{}} & \multicolumn{1}{C{0.11\textwidth}}{EP} \\\hline \multicolumn{1}{L{0.47\textwidth}}{\textit{Aechmea setigera} Mart.\allowbreak{} ex Schult.\allowbreak{} \&\allowbreak{\allowbreak{}\allowbreak{}}\allowbreak{} Schult.\allowbreak{}f.\allowbreak{}} & \multicolumn{1}{L{0.38\textwidth}}{imagem (\allowbreak{} \textsuperscript{Mello \textit{et al}.\allowbreak{} 2012})\allowbreak{}} & \multicolumn{1}{C{0.11\textwidth}}{EP} \\\hline \multicolumn{1}{L{0.47\textwidth}}{\textit{Aechmea tocantina} Baker} & \multicolumn{1}{L{0.38\textwidth}}{\textit{Raul,\allowbreak{} F.\allowbreak{} PSACF 389} (\allowbreak{}MG)\allowbreak{}} & \multicolumn{1}{C{0.11\textwidth}}{EP} \\\hline \multicolumn{1}{L{0.47\textwidth}}{\textit{Ananas ananassoides} (\allowbreak{}Baker)\allowbreak{} L.\allowbreak{}B.\allowbreak{}Sm.\allowbreak{}} & \multicolumn{1}{L{0.38\textwidth}}{imagem (\allowbreak{}Mello et al.\allowbreak{} 2012)\allowbreak{}} & \multicolumn{1}{C{0.11\textwidth}}{TE} \\\hline \multicolumn{1}{L{0.47\textwidth}}{\textit{Ananas nanus} (\allowbreak{}L.\allowbreak{}B.\allowbreak{}Smith)\allowbreak{} L.\allowbreak{}B.\allowbreak{}Smith} & \multicolumn{1}{L{0.38\textwidth}}{\textit{Salomão,\allowbreak{} R.\allowbreak{} 981} (\allowbreak{}MG)\allowbreak{}} & \multicolumn{1}{C{0.11\textwidth}}{TE} \\\hline \multicolumn{1}{L{0.47\textwidth}}{\textit{Araeococcus micranthus} Brongn.\allowbreak{}} & \multicolumn{1}{L{0.38\textwidth}}{\textit{Cavalcante,\allowbreak{} P.\allowbreak{} \&\allowbreak{\allowbreak{}\allowbreak{}}\allowbreak{} Silva,\allowbreak{} M.\allowbreak{} 2854} (\allowbreak{}MG)\allowbreak{}} & \multicolumn{1}{C{0.11\textwidth}}{EP} \\\hline \multicolumn{1}{L{0.47\textwidth}}{\textit{Bromelia grandiflora} Mez} & \multicolumn{1}{L{0.38\textwidth}}{imagem (\allowbreak{}Antonio,\allowbreak{} L.\allowbreak{}C.\allowbreak{})\allowbreak{}} & \multicolumn{1}{C{0.11\textwidth}}{TE} \\\hline \multicolumn{1}{L{0.47\textwidth}}{\textit{Bromelia goeldiana} L.\allowbreak{}B.\allowbreak{}Sm.\allowbreak{}} & \multicolumn{1}{L{0.38\textwidth}}{imagem (\allowbreak{} \textsuperscript{Mello \textit{et al}.\allowbreak{} 2012})\allowbreak{}} & \multicolumn{1}{C{0.11\textwidth}}{TE} \\\hline \multicolumn{1}{L{0.47\textwidth}}{\textit{Guzmania lingulata} (\allowbreak{}L.\allowbreak{})\allowbreak{} Mez} & \multicolumn{1}{L{0.38\textwidth}}{\textit{Barbacena,\allowbreak{} V.\allowbreak{} PSACF 140} (\allowbreak{}MG)\allowbreak{}} & \multicolumn{1}{C{0.11\textwidth}}{EP} \\\hline \multicolumn{1}{L{0.47\textwidth}}{\textit{Tillandsia adpressiflora} Mez*\allowbreak{}} & \multicolumn{1}{L{0.38\textwidth}}{\textit{Egler,\allowbreak{} W.\allowbreak{} 959} (\allowbreak{}MG)\allowbreak{}} & \multicolumn{1}{C{0.11\textwidth}}{EP} \\\hline \multicolumn{1}{L{0.47\textwidth}}{\textit{Tillandsia anceps} Lodd} & \multicolumn{1}{L{0.38\textwidth}}{\textit{Vasconcelos,\allowbreak{} R.\allowbreak{} et al.\allowbreak{} 25} (\allowbreak{}MG)\allowbreak{}} & \multicolumn{1}{C{0.11\textwidth}}{EP} \\\hline \multicolumn{1}{L{0.47\textwidth}}{\textit{Tillandsia bulbosa} Hook.\allowbreak{}f.\allowbreak{}} & \multicolumn{1}{L{0.38\textwidth}}{\textit{Abreu,\allowbreak{} J.\allowbreak{} PSACF 137} (\allowbreak{}MG)\allowbreak{}} & \multicolumn{1}{C{0.11\textwidth}}{EP} \\\hline \multicolumn{1}{L{0.47\textwidth}}{\textit{Tillandsia fasciculata} Sw.\allowbreak{}*\allowbreak{}} & \multicolumn{1}{L{0.38\textwidth}}{\textit{Rabelo,\allowbreak{} B.\allowbreak{} 1243} (\allowbreak{}MG)\allowbreak{}} & \multicolumn{1}{C{0.11\textwidth}}{EP} \\\hline \multicolumn{1}{L{0.47\textwidth}}{\textit{Tillandsia monadelpha} (\allowbreak{}E.\allowbreak{}Morren)\allowbreak{} Baker*\allowbreak{}} & \multicolumn{1}{L{0.38\textwidth}}{\textit{Costa Neto,\allowbreak{} S.\allowbreak{} et al.\allowbreak{} 130} (\allowbreak{}MG)\allowbreak{}} & \multicolumn{1}{C{0.11\textwidth}}{EP} \\\hline \multicolumn{1}{L{0.47\textwidth}}{\textit{Tillandsia paraensis} Mez} & \multicolumn{1}{L{0.38\textwidth}}{\textit{Dias,\allowbreak{} A.\allowbreak{} et al.\allowbreak{} 529} (\allowbreak{}MG)\allowbreak{}} & \multicolumn{1}{C{0.11\textwidth}}{EP} \\\hline \multicolumn{1}{L{0.47\textwidth}}{\textit{Tillandsia streptocarpa} Baker} & \multicolumn{1}{L{0.38\textwidth}}{\textit{Souza,\allowbreak{} S.\allowbreak{} et al.\allowbreak{} 1} (\allowbreak{}MG)\allowbreak{}} & \multicolumn{1}{C{0.11\textwidth}}{EP} \\\hline \multicolumn{1}{L{0.47\textwidth}}{\textit{Tillandsia tenuifolia} L.\allowbreak{}*\allowbreak{}} & \multicolumn{1}{L{0.38\textwidth}}{\textit{Paula,\allowbreak{} J.\allowbreak{} \&\allowbreak{\allowbreak{}\allowbreak{}}\allowbreak{} Mendonça,\allowbreak{} R.\allowbreak{}C.\allowbreak{} 1235} (\allowbreak{}MG)\allowbreak{}} & \multicolumn{1}{C{0.11\textwidth}}{EP} \\\hline \multicolumn{1}{L{0.47\textwidth}}{\textit{Werauhia gigantea} (\allowbreak{}Mart.\allowbreak{} ex Schult.\allowbreak{} \&\allowbreak{\allowbreak{}\allowbreak{}}\allowbreak{} Schult.\allowbreak{}f.\allowbreak{})\allowbreak{} J.\allowbreak{}R.\allowbreak{} Grant *\allowbreak{}} & \multicolumn{1}{L{0.38\textwidth}}{\textit{Pires,\allowbreak{} J.\allowbreak{} \&\allowbreak{\allowbreak{}\allowbreak{}}\allowbreak{} Silva,\allowbreak{} N.\allowbreak{} 11722} (\allowbreak{}IAN)\allowbreak{}} & \multicolumn{1}{C{0.11\textwidth}}{EP} \\\hline 
}
\begin{multicols}{2}
\par{}A identificação das espécies foi baseada em bibliografia específica \textsuperscript{Mez (\allowbreak{}1891)\allowbreak{}},\allowbreak{} \textsuperscript{Smith \&\allowbreak{\allowbreak{}\allowbreak{}}\allowbreak{} Downs (\allowbreak{}1974 },\allowbreak{} \textsuperscript{1977} e \textsuperscript{1979})\allowbreak{} e,\allowbreak{} quando necessário,\allowbreak{} comparação com amostras de herbários.\allowbreak{} Para as abreviações dos nomes dos autores das espécies utilizou-\allowbreak{}se \textsuperscript{Brummitt \&\allowbreak{\allowbreak{}\allowbreak{}}\allowbreak{} Powell (\allowbreak{}1992)\allowbreak{}}.\allowbreak{} Os dados de distribuição geográfica das espécies estão de acordo com \textsuperscript{Forzza \textit{et al.\allowbreak{}} (\allowbreak{}2014)\allowbreak{}}.\allowbreak{} Também foi confirmada a ocorrência de espécies em listas de espécies ameaçadas e a classificação das mesmas em diferentes categorias de grau de ameaça consultando \textsuperscript{Martinelli \textit{et al.\allowbreak{}} (\allowbreak{}2013)\allowbreak{}},\allowbreak{} \textsuperscript{Forzza \textit{et al.\allowbreak{}} (\allowbreak{}2013)\allowbreak{}},\allowbreak{} \textsuperscript{IBAMA (\allowbreak{}2014)\allowbreak{}}\textsuperscript{SEMA-\allowbreak{}PA (\allowbreak{}2014)\allowbreak{}} e \textsuperscript{IUCN (\allowbreak{}2014)\allowbreak{}}.\allowbreak{}\par{}A construção da chave de identificação das espécies foi baseada na análise de espécimes herborizados e em espécimes vivos cultivados no CEA quando necessário.\allowbreak{} Além disso,\allowbreak{} foram obtidos dados na literatura supracitada.\allowbreak{} A terminologia morfológica está de acordo com Scharf \&\allowbreak{\allowbreak{}\allowbreak{}}\allowbreak{} \textsuperscript{Gouda (\allowbreak{}2008)\allowbreak{}}.\allowbreak{}
\section*{Resultados e Discussão}
\par{}Na região da Volta Grande do Xingu,\allowbreak{} Bromeliaceae encontra-\allowbreak{}se representada por 20 espécies distribuídas em sete gêneros,\allowbreak{} correspondendo a cerca de 38\%\allowbreak{\allowbreak{}\allowbreak{}}\allowbreak{} das espécies registradas para o Pará (\allowbreak{}53 espécies)\allowbreak{},\allowbreak{} sendo 16 epífitas e quatro terrícolas (\allowbreak{} Tab.\allowbreak{} 1)\allowbreak{}.\allowbreak{} Os gêneros \textit{Tillandsia} e \textit{Aechmea} foram os mais representativos com sete e seis espécies com duas espécies e \textit{Araeococcus},\allowbreak{} \textit{Guzmania} e \textit{Werauhia},\allowbreak{} com uma espécie cada.\allowbreak{}\par{}Quando comparado com os trabalhos que trataram a família para o Pará,\allowbreak{} o presente estudo mostra uma maior biodiversidade,\allowbreak{} apresentando mais que o dobro das espécies reportadas por \textsuperscript{Quaresma \&\allowbreak{\allowbreak{}\allowbreak{}}\allowbreak{} Medeiros (\allowbreak{}2009)\allowbreak{}} e \textsuperscript{Koch \textit{et al.\allowbreak{}} (\allowbreak{}2013)\allowbreak{}},\allowbreak{} que registraram seis e cinco espécies,\allowbreak{} respectivamente.\allowbreak{} Mesmo considerando somente as espécies epífitas da Volta Grande do Xingu,\allowbreak{} o resultado ainda é bastante expressivo em relação a riqueza reportada para as outras áreas do Estado.\allowbreak{}\par{}Em relação à exclusividade das espécies quando comparada com os demais estudos sobre Bromeliaceae no Pará,\allowbreak{} 12 são exclusivas da Volta Grande do Xingu:\allowbreak{} \textit{Aechmea castelnavii },\allowbreak{} \textit{A.\allowbreak{} tocantina},\allowbreak{} \textit{Ananas ananassoides},\allowbreak{} \textit{A.\allowbreak{} nanus},\allowbreak{} \textit{Bromelia grandiflora},\allowbreak{} \textit{B.\allowbreak{} goeldiana },\allowbreak{} \textit{Tillandsia adpressiflora},\allowbreak{} \textit{T.\allowbreak{} fasciculata },\allowbreak{} \textit{T.\allowbreak{} monadelpha},\allowbreak{} \textit{T.\allowbreak{} paraensis},\allowbreak{} \textit{T.\allowbreak{} streptocarpa} e \textit{T.\allowbreak{} tenuifolia}.\allowbreak{} Na APA Ilha do Combu (\allowbreak{} \textsuperscript{Quaresma \&\allowbreak{\allowbreak{}\allowbreak{}}\allowbreak{} Medeiros 2009})\allowbreak{} e,\allowbreak{} na Floresta Nacional de Caxiuanã (\allowbreak{} \textsuperscript{Koch \textit{et al}.\allowbreak{} 2013 })\allowbreak{} não houve exclusividade de espécies.\allowbreak{}\par{}Entre as espécies inventariadas,\allowbreak{} \textit{Aechmea mertensii},\allowbreak{} \textit{Guzmania lingulata } e \textit{Tillandsia bulbosa} são comuns entre as três áreas comparadas,\allowbreak{} fato presumido pela boa representatividade destas nos herbários amazônicos,\allowbreak{} corroborado pela ampla distribuição das mesmas na Amazônia.\allowbreak{}\par{}Em \textsuperscript{MPEG (\allowbreak{}2008)\allowbreak{}} foram reportadas as seguintes espécies:\allowbreak{} \textit{Aechmea mertensii},\allowbreak{} \textit{A.\allowbreak{} setigera},\allowbreak{} \textit{Ananas nanus},\allowbreak{} \textit{Bromelia goeldiana },\allowbreak{} \textit{Tillandsia bulbosa},\allowbreak{} \textit{T.\allowbreak{} streptocarpa} e \textit{Werauhia gigantea}.\allowbreak{} Além dessas \textsuperscript{Mello \textit{et al.\allowbreak{}} (\allowbreak{}2012)\allowbreak{}} registraram mais dez para a região,\allowbreak{} sendo \textit{Tillandsia tenuifolia} registrada pela primeira vez para o Pará,\allowbreak{} ampliando sua distribuição no Brasil.\allowbreak{}\par{}Quanto à distribuição das espécies reportadas para a Volta Grande do Xingu,\allowbreak{} três ocorrem em mais da metade dos Estados brasileiros,\allowbreak{} a saber:\allowbreak{} \textit{Ananas anassoides} (\allowbreak{}22 estados)\allowbreak{},\allowbreak{} \textit{Aechmea bromeliifolia} (\allowbreak{}20 estados)\allowbreak{} e \textit{Tillandsia tenuifolia } (\allowbreak{}16 estados)\allowbreak{} (\allowbreak{} \textsuperscript{Forzza \textit{et al }.\allowbreak{} 2014})\allowbreak{}.\allowbreak{} Por outro lado,\allowbreak{} outras três espécies apresentaram distribuição mais restrita,\allowbreak{} como \textit{Tillandsia fasciculata} que ocorre apenas nos estados do Amapá,\allowbreak{} Amazonas e Pará,\allowbreak{} \textit{Bromelia goeldiana} no Amazonas e Pará e \textit{Werauhia gigantea} no Mato Grosso e Pará (\allowbreak{} \textsuperscript{Forzza \textit{et al.\allowbreak{}} 2014})\allowbreak{}.\allowbreak{} Na área de estudo,\allowbreak{} \textit{Aechmea castelnavii },\allowbreak{} \textit{A.\allowbreak{} mertensii},\allowbreak{} \textit{A.\allowbreak{} tocantina},\allowbreak{} \textit{Bromelia goeldiana},\allowbreak{} \textit{Guzmania lingulata},\allowbreak{} \textit{Tillandsia bulbosa},\allowbreak{} \textit{T.\allowbreak{} adpressiflora},\allowbreak{} \textit{T.\allowbreak{} paraensis } e \textit{T.\allowbreak{} streptocarpa} foram as mais comuns durante as atividades do resgate de flora.\allowbreak{} Destas,\allowbreak{} as espécies do gênero \textit{Tillandsia} foram frequentemente encontradas em floresta ombrófila aluvial,\allowbreak{} nas margens ou ilhas do rio Xingu.\allowbreak{} As espécies dos gêneros \textit{Aechmea} e \textit{Bromelia} foram encontradas em áreas de floresta ombrófila densa e floresta ombrófila abertas ou ainda,\allowbreak{} em áreas de vegetação secundária.\allowbreak{} \textit{Guzmania lingulata} teve populações encontradas tanto em floresta ombrófila aluvial,\allowbreak{} quanto em floresta ombrófila densa.\allowbreak{} Essa última espécie é típica de subosque ou margens de rios com formações florestais mais fechadas,\allowbreak{} sempre formando pequenas populações,\allowbreak{} assim como constatado por \textsuperscript{Koch \textit{et al}.\allowbreak{} (\allowbreak{}2013)\allowbreak{}} na FLONA de Caxiuanã.\allowbreak{} \par{}As espécies pouco frequentes na região da Volta Grande do Xingu são \textit{Tillandsia anceps },\allowbreak{} encontrada na coleção do herbário MG e que não havia sido citada nos trabalhos apresentados para a região,\allowbreak{} e \textit{T.\allowbreak{} monadelpha} e \textit{T.\allowbreak{} tenuifolia },\allowbreak{} que foram encontradas apenas uma vez e em floresta ombrófila aluvial,\allowbreak{} entretanto,\allowbreak{} representadas por muitos indivíduos.\allowbreak{}\par{}Considerando o status de conservação das espécies encontradas na região de estudo,\allowbreak{} nenhuma espécie se encontra sob categoria de ameaça ou incluída no Livro Vermelho da Flora do Brasil (\allowbreak{} \textsuperscript{Forzza \textit{et al.\allowbreak{}} 2013})\allowbreak{}.\allowbreak{} Todavia,\allowbreak{} esse estudo e novos esforços que visem ampliar o conhecimento sobre Bromeliaceae no domínio Amazônia poderão subsidiar as próximas avaliações sobre espécies ameaçadas.\allowbreak{}\par{}Com este \textit{checklist} foi possível confirmar uma maior diversidade da família Bromeliaceae em determinadas áreas do domínio Amazônia.\allowbreak{} Neste contexto,\allowbreak{} na região da Volta Grande do Xingu,\allowbreak{} pelo menos quatro fitofisionomias diferentes foram intensamente coletadas fato que também pode propiciar uma maior riqueza,\allowbreak{} enquanto que na APA Ilha do Combu e na FLONA de Caxiuanã,\allowbreak{} com vegetação predominante de floresta de várzea na primeira e de terra firme na segunda,\allowbreak{} foram reportadas menos que dez espécies de Bromeliaceae em cada (\allowbreak{} \textsuperscript{Quaresma \&\allowbreak{\allowbreak{}\allowbreak{}}\allowbreak{} Medeiros 2009}; \textsuperscript{Koch \textit{et al.\allowbreak{}} 2013})\allowbreak{}.\allowbreak{} Este estudo demonstra também a importância da continuidade dos trabalhos de flora no Brasil,\allowbreak{} principalmente na Amazônia brasileira,\allowbreak{} pois com o mesmo foi possível ampliar a área de ocorrência da maioria das espécies no estado do Pará.\allowbreak{}\par{}A seguir,\allowbreak{} é apresentada uma chave para identificação das espécies de Bromeliaceae da região da Volta Grande do Xingu.\allowbreak{}
\end{multicols}
\par{\centering{Chave para identificação das Bromeliaceae da região da Volta Grande do Xingu,\allowbreak{} Pará,\allowbreak{} Brasil }}
\begin{customList1}
\item \par{}1.\allowbreak{} Plantas terrícolas
\begin{customList1}
\item \par{}2.\allowbreak{} Inflorescência em espiga estrobiliforme,\allowbreak{} coma apical presente
\begin{customList1}
\item \par{}3.\allowbreak{} Coma apical 2 a 3 vezes maior que a inflorescência .\allowbreak{}.\allowbreak{}.\allowbreak{}.\allowbreak{}.\allowbreak{}.\allowbreak{}.\allowbreak{}.\allowbreak{}.\allowbreak{}.\allowbreak{}.\allowbreak{}.\allowbreak{}.\allowbreak{}.\allowbreak{}.\allowbreak{}.\allowbreak{}.\allowbreak{}.\allowbreak{}.\allowbreak{}.\allowbreak{}.\allowbreak{}.\allowbreak{}.\allowbreak{}.\allowbreak{}.\allowbreak{}.\allowbreak{}.\allowbreak{}.\allowbreak{}.\allowbreak{}.\allowbreak{}.\allowbreak{}.\allowbreak{}.\allowbreak{}.\allowbreak{}.\allowbreak{}.\allowbreak{}.\allowbreak{}.\allowbreak{}.\allowbreak{}.\allowbreak{} \textit{Ananas nanus }
\item \par{}3'.\allowbreak{} Coma apical igual ou pouco maior que a inflorescência (\allowbreak{}nunca até 2 vezes maior)\allowbreak{}.\allowbreak{}.\allowbreak{}.\allowbreak{}.\allowbreak{}.\allowbreak{}.\allowbreak{}.\allowbreak{}.\allowbreak{}.\allowbreak{}.\allowbreak{}.\allowbreak{}.\allowbreak{}.\allowbreak{}.\allowbreak{}.\allowbreak{}.\allowbreak{}.\allowbreak{}.\allowbreak{}.\allowbreak{}.\allowbreak{}.\allowbreak{}.\allowbreak{}.\allowbreak{}.\allowbreak{}.\allowbreak{}.\allowbreak{}.\allowbreak{}.\allowbreak{}.\allowbreak{}.\allowbreak{}.\allowbreak{}.\allowbreak{}.\allowbreak{}.\allowbreak{}.\allowbreak{}.\allowbreak{}.\allowbreak{}.\allowbreak{}.\allowbreak{}.\allowbreak{}.\allowbreak{}.\allowbreak{}.\allowbreak{}.\allowbreak{}.\allowbreak{}.\allowbreak{}.\allowbreak{}.\allowbreak{}.\allowbreak{}.\allowbreak{}.\allowbreak{}.\allowbreak{}.\allowbreak{}.\allowbreak{}.\allowbreak{}.\allowbreak{}.\allowbreak{}.\allowbreak{}.\allowbreak{}.\allowbreak{}.\allowbreak{}.\allowbreak{}.\allowbreak{}.\allowbreak{}.\allowbreak{}.\allowbreak{}.\allowbreak{}.\allowbreak{}.\allowbreak{}.\allowbreak{}.\allowbreak{}.\allowbreak{}.\allowbreak{}.\allowbreak{}.\allowbreak{}.\allowbreak{}.\allowbreak{}.\allowbreak{}.\allowbreak{}.\allowbreak{}.\allowbreak{}.\allowbreak{}.\allowbreak{}.\allowbreak{}.\allowbreak{}.\allowbreak{}.\allowbreak{}.\allowbreak{}.\allowbreak{}.\allowbreak{}.\allowbreak{}.\allowbreak{}.\allowbreak{}.\allowbreak{}.\allowbreak{}.\allowbreak{}.\allowbreak{}.\allowbreak{}.\allowbreak{}.\allowbreak{}.\allowbreak{}.\allowbreak{}.\allowbreak{}.\allowbreak{}.\allowbreak{}.\allowbreak{}.\allowbreak{}.\allowbreak{}.\allowbreak{}.\allowbreak{} \textit{Ananas ananassoides} (\allowbreak{} Fig.\allowbreak{} 3a)\allowbreak{}\par{}




\end{customList1}

\item \par{}2'.\allowbreak{} Inflorescência nunca estrobiliforme,\allowbreak{} coma apical ausente
\begin{customList1}
\item \par{}4.\allowbreak{} Pedúnculo conspícuo,\allowbreak{} até 30 cm alt.\allowbreak{},\allowbreak{} pétalas roxas .\allowbreak{}.\allowbreak{}.\allowbreak{}.\allowbreak{}.\allowbreak{}.\allowbreak{}.\allowbreak{}.\allowbreak{}.\allowbreak{}.\allowbreak{}.\allowbreak{}.\allowbreak{}.\allowbreak{}.\allowbreak{} \textit{Bromelia goeldiana } (\allowbreak{} Fig.\allowbreak{} 3c-\allowbreak{}d)\allowbreak{}
\item \par{}4'.\allowbreak{} Pedúnculo inconspícuo,\allowbreak{} imerso no centro da roseta,\allowbreak{} pétalas magenta.\allowbreak{}.\allowbreak{}.\allowbreak{}.\allowbreak{}.\allowbreak{}.\allowbreak{}.\allowbreak{}.\allowbreak{}.\allowbreak{}.\allowbreak{}.\allowbreak{}.\allowbreak{}.\allowbreak{}.\allowbreak{}.\allowbreak{}.\allowbreak{}.\allowbreak{}.\allowbreak{}.\allowbreak{}.\allowbreak{}.\allowbreak{}.\allowbreak{}.\allowbreak{}.\allowbreak{}.\allowbreak{}.\allowbreak{}.\allowbreak{}.\allowbreak{}.\allowbreak{}.\allowbreak{}.\allowbreak{}.\allowbreak{}.\allowbreak{}.\allowbreak{}.\allowbreak{}.\allowbreak{}.\allowbreak{}.\allowbreak{}.\allowbreak{}.\allowbreak{}.\allowbreak{}.\allowbreak{}.\allowbreak{}.\allowbreak{}.\allowbreak{}.\allowbreak{}.\allowbreak{}.\allowbreak{}.\allowbreak{}.\allowbreak{}.\allowbreak{}.\allowbreak{}.\allowbreak{}.\allowbreak{}.\allowbreak{}.\allowbreak{}.\allowbreak{}.\allowbreak{}.\allowbreak{}.\allowbreak{}.\allowbreak{}.\allowbreak{}.\allowbreak{}.\allowbreak{}.\allowbreak{}.\allowbreak{}.\allowbreak{}.\allowbreak{}.\allowbreak{}.\allowbreak{}.\allowbreak{}.\allowbreak{}.\allowbreak{}.\allowbreak{}.\allowbreak{}.\allowbreak{}.\allowbreak{}.\allowbreak{}.\allowbreak{}.\allowbreak{}.\allowbreak{}.\allowbreak{}.\allowbreak{}.\allowbreak{}.\allowbreak{}.\allowbreak{}.\allowbreak{}.\allowbreak{}.\allowbreak{}.\allowbreak{}.\allowbreak{}.\allowbreak{}.\allowbreak{}.\allowbreak{}.\allowbreak{}.\allowbreak{}.\allowbreak{}.\allowbreak{}.\allowbreak{}.\allowbreak{}.\allowbreak{}.\allowbreak{}.\allowbreak{}.\allowbreak{}.\allowbreak{}.\allowbreak{}.\allowbreak{}.\allowbreak{}.\allowbreak{}.\allowbreak{}.\allowbreak{}.\allowbreak{}.\allowbreak{}.\allowbreak{}.\allowbreak{}.\allowbreak{}.\allowbreak{}.\allowbreak{}.\allowbreak{}.\allowbreak{}.\allowbreak{}.\allowbreak{}.\allowbreak{}.\allowbreak{}.\allowbreak{}.\allowbreak{}.\allowbreak{}.\allowbreak{}.\allowbreak{}.\allowbreak{}.\allowbreak{}.\allowbreak{} Bromelia grandiflora (\allowbreak{} Fig.\allowbreak{} 3e)\allowbreak{}
\end{customList1}

\end{customList1}

\item \par{}1'.\allowbreak{} Plantas epífitas
\begin{customList1}
\item \par{}5.\allowbreak{} Apêndices petalíneos presentes
\begin{customList1}
\item \par{}6.\allowbreak{} Lâminas foliares de margem inteira .\allowbreak{}.\allowbreak{}.\allowbreak{}.\allowbreak{}.\allowbreak{}.\allowbreak{}.\allowbreak{}.\allowbreak{}.\allowbreak{}.\allowbreak{}.\allowbreak{}.\allowbreak{}.\allowbreak{}.\allowbreak{}.\allowbreak{}.\allowbreak{}.\allowbreak{}.\allowbreak{}.\allowbreak{}.\allowbreak{}.\allowbreak{}.\allowbreak{}.\allowbreak{}.\allowbreak{}.\allowbreak{}.\allowbreak{}.\allowbreak{}.\allowbreak{}.\allowbreak{}.\allowbreak{}.\allowbreak{}.\allowbreak{}.\allowbreak{}.\allowbreak{}.\allowbreak{}.\allowbreak{}.\allowbreak{}.\allowbreak{}.\allowbreak{}.\allowbreak{}.\allowbreak{}.\allowbreak{}.\allowbreak{}.\allowbreak{}.\allowbreak{}.\allowbreak{}.\allowbreak{}.\allowbreak{}.\allowbreak{} \textit{Werauhia gigantea }
\item \par{}6'.\allowbreak{} Lâminas foliares de margem espinescente
\begin{customList1}
\item \par{}7.\allowbreak{} Pedúnculo branco-\allowbreak{}lanoso; inflorescência em espiga estrobiliforme .\allowbreak{}.\allowbreak{}.\allowbreak{}.\allowbreak{}.\allowbreak{}.\allowbreak{}.\allowbreak{}.\allowbreak{}.\allowbreak{}.\allowbreak{}.\allowbreak{}.\allowbreak{}.\allowbreak{}.\allowbreak{}.\allowbreak{}.\allowbreak{}.\allowbreak{}.\allowbreak{}.\allowbreak{}.\allowbreak{}.\allowbreak{}.\allowbreak{}.\allowbreak{}.\allowbreak{}.\allowbreak{}.\allowbreak{}.\allowbreak{}.\allowbreak{}.\allowbreak{}.\allowbreak{}.\allowbreak{}.\allowbreak{}.\allowbreak{}.\allowbreak{}.\allowbreak{}.\allowbreak{}.\allowbreak{}.\allowbreak{}.\allowbreak{}.\allowbreak{}.\allowbreak{}.\allowbreak{}.\allowbreak{}.\allowbreak{}.\allowbreak{}.\allowbreak{}.\allowbreak{}.\allowbreak{}.\allowbreak{}.\allowbreak{}.\allowbreak{}.\allowbreak{}.\allowbreak{}.\allowbreak{}.\allowbreak{}.\allowbreak{}.\allowbreak{}.\allowbreak{}.\allowbreak{}.\allowbreak{}.\allowbreak{}.\allowbreak{}.\allowbreak{}.\allowbreak{}.\allowbreak{}.\allowbreak{}.\allowbreak{}.\allowbreak{}.\allowbreak{}.\allowbreak{}.\allowbreak{}.\allowbreak{}.\allowbreak{}.\allowbreak{}.\allowbreak{}.\allowbreak{}.\allowbreak{}.\allowbreak{}.\allowbreak{}.\allowbreak{}.\allowbreak{}.\allowbreak{}.\allowbreak{}.\allowbreak{}.\allowbreak{}.\allowbreak{}.\allowbreak{}.\allowbreak{}.\allowbreak{}.\allowbreak{}.\allowbreak{}.\allowbreak{}.\allowbreak{}.\allowbreak{}.\allowbreak{}.\allowbreak{}.\allowbreak{}.\allowbreak{}.\allowbreak{}.\allowbreak{}.\allowbreak{} \textit{Aechmea bromeliifolia} (\allowbreak{} Fig.\allowbreak{} 2a)\allowbreak{}
\item \par{}7'.\allowbreak{} Pedúnculo lepidoto; inflorescência em panícula ou fascículos
\begin{customList1}
\item \par{}8.\allowbreak{} Folhas até 2 m compr.\allowbreak{}; brácteas pedunculares com margem inteira .\allowbreak{}.\allowbreak{}.\allowbreak{}.\allowbreak{}.\allowbreak{}.\allowbreak{}.\allowbreak{}.\allowbreak{}.\allowbreak{}.\allowbreak{}.\allowbreak{}.\allowbreak{}.\allowbreak{}.\allowbreak{}.\allowbreak{}.\allowbreak{}.\allowbreak{}.\allowbreak{}.\allowbreak{}.\allowbreak{}.\allowbreak{}.\allowbreak{}.\allowbreak{}.\allowbreak{}.\allowbreak{}.\allowbreak{}.\allowbreak{}.\allowbreak{}.\allowbreak{}.\allowbreak{}.\allowbreak{}.\allowbreak{}.\allowbreak{}.\allowbreak{}.\allowbreak{}.\allowbreak{}.\allowbreak{}.\allowbreak{}.\allowbreak{}.\allowbreak{}.\allowbreak{}.\allowbreak{}.\allowbreak{}.\allowbreak{}.\allowbreak{}.\allowbreak{}.\allowbreak{}.\allowbreak{}.\allowbreak{}.\allowbreak{}.\allowbreak{}.\allowbreak{}.\allowbreak{}.\allowbreak{}.\allowbreak{}.\allowbreak{}.\allowbreak{}.\allowbreak{}.\allowbreak{}.\allowbreak{}.\allowbreak{}.\allowbreak{}.\allowbreak{}.\allowbreak{}.\allowbreak{}.\allowbreak{}.\allowbreak{}.\allowbreak{}.\allowbreak{}.\allowbreak{}.\allowbreak{}.\allowbreak{}.\allowbreak{}.\allowbreak{}.\allowbreak{}.\allowbreak{}.\allowbreak{}.\allowbreak{}.\allowbreak{}.\allowbreak{}.\allowbreak{}.\allowbreak{}.\allowbreak{}.\allowbreak{}.\allowbreak{}.\allowbreak{}.\allowbreak{}.\allowbreak{} \textit{Aechmea tocantina} (\allowbreak{} Fig.\allowbreak{} 2h-\allowbreak{}i)\allowbreak{}
\item \par{}8'.\allowbreak{} Folhas até 1 m compr.\allowbreak{}; brácteas pedunculares com margem espinescente
\begin{customList1}
\item \par{}9.\allowbreak{} Inflorescência recurvada; bráctea floral primária espiniforme .\allowbreak{}.\allowbreak{}.\allowbreak{}.\allowbreak{}.\allowbreak{}.\allowbreak{}.\allowbreak{}.\allowbreak{}.\allowbreak{}.\allowbreak{}.\allowbreak{}.\allowbreak{}.\allowbreak{}.\allowbreak{}.\allowbreak{}.\allowbreak{}.\allowbreak{}.\allowbreak{}.\allowbreak{}.\allowbreak{}.\allowbreak{}.\allowbreak{}.\allowbreak{}.\allowbreak{}.\allowbreak{}.\allowbreak{}.\allowbreak{}.\allowbreak{}.\allowbreak{}.\allowbreak{}.\allowbreak{}.\allowbreak{}.\allowbreak{}.\allowbreak{}.\allowbreak{}.\allowbreak{}.\allowbreak{}.\allowbreak{}.\allowbreak{}.\allowbreak{}.\allowbreak{}.\allowbreak{}.\allowbreak{}.\allowbreak{}.\allowbreak{}.\allowbreak{}.\allowbreak{}.\allowbreak{}.\allowbreak{}.\allowbreak{}.\allowbreak{}.\allowbreak{}.\allowbreak{}.\allowbreak{}.\allowbreak{}.\allowbreak{}.\allowbreak{}.\allowbreak{}.\allowbreak{}.\allowbreak{}.\allowbreak{}.\allowbreak{}.\allowbreak{}.\allowbreak{}.\allowbreak{}.\allowbreak{}.\allowbreak{}.\allowbreak{}.\allowbreak{}.\allowbreak{}.\allowbreak{}.\allowbreak{}.\allowbreak{}.\allowbreak{}.\allowbreak{}.\allowbreak{}.\allowbreak{}.\allowbreak{}.\allowbreak{}.\allowbreak{}.\allowbreak{}.\allowbreak{}.\allowbreak{} \textit{Aechmea setigera} (\allowbreak{} Fig.\allowbreak{} 2f-\allowbreak{}g)\allowbreak{}
\item \par{}9'.\allowbreak{} Inflorescência ereta; bráctea floral primária nunca espiniforme
\begin{customList1}
\item \par{}10.\allowbreak{} Cálice verde-\allowbreak{}azulado em direção ao ápice; corola rósea.\allowbreak{}.\allowbreak{}.\allowbreak{}.\allowbreak{}.\allowbreak{}.\allowbreak{}.\allowbreak{}.\allowbreak{}.\allowbreak{}.\allowbreak{}.\allowbreak{}.\allowbreak{}.\allowbreak{}.\allowbreak{}.\allowbreak{}.\allowbreak{}.\allowbreak{}.\allowbreak{}.\allowbreak{}.\allowbreak{}.\allowbreak{}.\allowbreak{}.\allowbreak{}.\allowbreak{}.\allowbreak{}.\allowbreak{}.\allowbreak{}.\allowbreak{}.\allowbreak{}.\allowbreak{}.\allowbreak{}.\allowbreak{}.\allowbreak{}.\allowbreak{}.\allowbreak{}.\allowbreak{}.\allowbreak{}.\allowbreak{}.\allowbreak{}.\allowbreak{}.\allowbreak{}.\allowbreak{}.\allowbreak{}.\allowbreak{}.\allowbreak{}.\allowbreak{}.\allowbreak{}.\allowbreak{}.\allowbreak{}.\allowbreak{}.\allowbreak{}.\allowbreak{}.\allowbreak{}.\allowbreak{}.\allowbreak{}.\allowbreak{}.\allowbreak{}.\allowbreak{}.\allowbreak{}.\allowbreak{}.\allowbreak{}.\allowbreak{}.\allowbreak{}.\allowbreak{}.\allowbreak{}.\allowbreak{}.\allowbreak{}.\allowbreak{}.\allowbreak{}.\allowbreak{}.\allowbreak{} \textit{Aechmea castelnavii} (\allowbreak{} Fig.\allowbreak{} 2b-\allowbreak{}c)\allowbreak{}
\item \par{}10'.\allowbreak{} Cálice totalmente verde; corola amarela .\allowbreak{}.\allowbreak{}.\allowbreak{}.\allowbreak{}.\allowbreak{}.\allowbreak{}.\allowbreak{}.\allowbreak{}.\allowbreak{}.\allowbreak{}.\allowbreak{}.\allowbreak{}.\allowbreak{}.\allowbreak{}.\allowbreak{}.\allowbreak{}.\allowbreak{}.\allowbreak{}.\allowbreak{}.\allowbreak{}.\allowbreak{}.\allowbreak{}.\allowbreak{}.\allowbreak{}.\allowbreak{}.\allowbreak{}.\allowbreak{}.\allowbreak{}.\allowbreak{}.\allowbreak{}.\allowbreak{}.\allowbreak{}.\allowbreak{}.\allowbreak{}.\allowbreak{}.\allowbreak{}.\allowbreak{}.\allowbreak{}.\allowbreak{}.\allowbreak{}.\allowbreak{}.\allowbreak{}.\allowbreak{}.\allowbreak{}.\allowbreak{}.\allowbreak{}.\allowbreak{}.\allowbreak{}.\allowbreak{}.\allowbreak{}.\allowbreak{}.\allowbreak{}.\allowbreak{}.\allowbreak{}.\allowbreak{}.\allowbreak{}.\allowbreak{}.\allowbreak{}.\allowbreak{}.\allowbreak{}.\allowbreak{}.\allowbreak{}.\allowbreak{}.\allowbreak{}.\allowbreak{}.\allowbreak{}.\allowbreak{}.\allowbreak{}.\allowbreak{}.\allowbreak{}.\allowbreak{}.\allowbreak{}.\allowbreak{}.\allowbreak{}.\allowbreak{}.\allowbreak{}.\allowbreak{}.\allowbreak{}.\allowbreak{}.\allowbreak{}.\allowbreak{}.\allowbreak{}.\allowbreak{}.\allowbreak{}.\allowbreak{}.\allowbreak{}.\allowbreak{}.\allowbreak{}.\allowbreak{}.\allowbreak{}.\allowbreak{}.\allowbreak{}.\allowbreak{}.\allowbreak{}.\allowbreak{}.\allowbreak{}.\allowbreak{} \textit{Aechmea mertensii} (\allowbreak{} Fig.\allowbreak{} 2d-\allowbreak{}e)\allowbreak{}
\end{customList1}

\end{customList1}

\end{customList1}

\end{customList1}

\end{customList1}

\item \par{}5'.\allowbreak{} Apêndices petalíneos ausentes
\begin{customList1}
\item \par{}11.\allowbreak{} Rosetas não acumuladoras de água
\begin{customList1}
\item \par{}12.\allowbreak{} Frutos do tipo baga e sementes sem apêndices .\allowbreak{}.\allowbreak{}.\allowbreak{}.\allowbreak{}.\allowbreak{}.\allowbreak{}.\allowbreak{}.\allowbreak{}.\allowbreak{}.\allowbreak{}.\allowbreak{}.\allowbreak{}.\allowbreak{}.\allowbreak{}.\allowbreak{}.\allowbreak{}.\allowbreak{}.\allowbreak{}.\allowbreak{}.\allowbreak{}.\allowbreak{}.\allowbreak{}.\allowbreak{}.\allowbreak{}.\allowbreak{}.\allowbreak{}.\allowbreak{}.\allowbreak{}.\allowbreak{}.\allowbreak{}.\allowbreak{}.\allowbreak{}.\allowbreak{}.\allowbreak{}.\allowbreak{}.\allowbreak{}.\allowbreak{}.\allowbreak{}.\allowbreak{}.\allowbreak{}.\allowbreak{}.\allowbreak{}.\allowbreak{}.\allowbreak{}.\allowbreak{}.\allowbreak{}.\allowbreak{}.\allowbreak{}.\allowbreak{}.\allowbreak{}.\allowbreak{}.\allowbreak{}.\allowbreak{}.\allowbreak{}.\allowbreak{}.\allowbreak{}.\allowbreak{}.\allowbreak{}.\allowbreak{}.\allowbreak{}.\allowbreak{}.\allowbreak{}.\allowbreak{}.\allowbreak{}.\allowbreak{}.\allowbreak{}.\allowbreak{}.\allowbreak{}.\allowbreak{}.\allowbreak{}.\allowbreak{}.\allowbreak{}.\allowbreak{}.\allowbreak{}.\allowbreak{}.\allowbreak{}.\allowbreak{}.\allowbreak{}.\allowbreak{}.\allowbreak{}.\allowbreak{}.\allowbreak{}.\allowbreak{}.\allowbreak{}.\allowbreak{}.\allowbreak{}.\allowbreak{}.\allowbreak{}.\allowbreak{}.\allowbreak{}.\allowbreak{}.\allowbreak{}.\allowbreak{}.\allowbreak{}.\allowbreak{}.\allowbreak{}.\allowbreak{}.\allowbreak{}.\allowbreak{}.\allowbreak{}.\allowbreak{}.\allowbreak{}.\allowbreak{}.\allowbreak{}.\allowbreak{}.\allowbreak{}.\allowbreak{}.\allowbreak{}.\allowbreak{}.\allowbreak{}.\allowbreak{}.\allowbreak{}.\allowbreak{}.\allowbreak{}.\allowbreak{}.\allowbreak{}.\allowbreak{}.\allowbreak{}.\allowbreak{}.\allowbreak{}.\allowbreak{}.\allowbreak{}.\allowbreak{}.\allowbreak{}.\allowbreak{}.\allowbreak{}.\allowbreak{}.\allowbreak{}.\allowbreak{} \textit{Araeococcus micranthus} (\allowbreak{} Fig.\allowbreak{} 3b)\allowbreak{}
\item \par{}12'.\allowbreak{} Frutos do tipo cápsula e sementes com apêndices plumosos
\begin{customList1}
\item \par{}13.\allowbreak{} Lâminas foliares velutino-\allowbreak{}cinéreas.\allowbreak{}.\allowbreak{}.\allowbreak{}.\allowbreak{}.\allowbreak{}.\allowbreak{}.\allowbreak{}.\allowbreak{}.\allowbreak{}.\allowbreak{}.\allowbreak{}.\allowbreak{}.\allowbreak{}.\allowbreak{}.\allowbreak{} \textit{Tillandsia streptocarpa} (\allowbreak{} Fig.\allowbreak{} 4i)\allowbreak{}\par{}


\item \par{}13'.\allowbreak{} Lâminas foliares nunca velutino-\allowbreak{}cinéreas
\begin{customList1}
\item \par{}14.\allowbreak{} Bainhas foliares infladas formando um aspecto bulboso; pétalas arroxeadas.\allowbreak{}.\allowbreak{}.\allowbreak{}.\allowbreak{}.\allowbreak{}.\allowbreak{}.\allowbreak{}.\allowbreak{}.\allowbreak{}.\allowbreak{}.\allowbreak{}.\allowbreak{}.\allowbreak{}.\allowbreak{}.\allowbreak{}.\allowbreak{}.\allowbreak{}.\allowbreak{}.\allowbreak{}.\allowbreak{} .\allowbreak{}.\allowbreak{}.\allowbreak{}.\allowbreak{}.\allowbreak{}.\allowbreak{}.\allowbreak{}.\allowbreak{}.\allowbreak{}.\allowbreak{}.\allowbreak{}.\allowbreak{}.\allowbreak{}.\allowbreak{}.\allowbreak{}.\allowbreak{}.\allowbreak{}.\allowbreak{}.\allowbreak{}.\allowbreak{}.\allowbreak{}.\allowbreak{}.\allowbreak{}.\allowbreak{}.\allowbreak{}.\allowbreak{}.\allowbreak{}.\allowbreak{}.\allowbreak{}.\allowbreak{}.\allowbreak{}.\allowbreak{}.\allowbreak{}.\allowbreak{}.\allowbreak{}.\allowbreak{}.\allowbreak{}.\allowbreak{}.\allowbreak{}.\allowbreak{}.\allowbreak{}.\allowbreak{}.\allowbreak{} \textit{Tillandsia bulbosa} (\allowbreak{} Fig.\allowbreak{} 3i-\allowbreak{} 4a)\allowbreak{}
\item \par{}14'.\allowbreak{} Bainhas foliares nunca infladas; pétalas róseas ou brancas
\begin{customList1}
\item \par{}15.\allowbreak{} Inflorescência complanada com flores dísticas
\begin{customList1}
\item \par{}16.\allowbreak{} Brácteas florais verdes; pétalas brancas .\allowbreak{}.\allowbreak{}.\allowbreak{}.\allowbreak{}.\allowbreak{}.\allowbreak{}.\allowbreak{}.\allowbreak{}.\allowbreak{}.\allowbreak{}.\allowbreak{}.\allowbreak{}.\allowbreak{}.\allowbreak{}.\allowbreak{}.\allowbreak{}.\allowbreak{}.\allowbreak{}.\allowbreak{}.\allowbreak{}.\allowbreak{}.\allowbreak{}.\allowbreak{}.\allowbreak{}.\allowbreak{}.\allowbreak{}.\allowbreak{}.\allowbreak{}.\allowbreak{}.\allowbreak{}.\allowbreak{}.\allowbreak{}.\allowbreak{}.\allowbreak{}.\allowbreak{}.\allowbreak{}.\allowbreak{}.\allowbreak{}.\allowbreak{}.\allowbreak{}.\allowbreak{}.\allowbreak{}.\allowbreak{}.\allowbreak{}.\allowbreak{}.\allowbreak{}.\allowbreak{}.\allowbreak{}.\allowbreak{}.\allowbreak{}.\allowbreak{}.\allowbreak{}.\allowbreak{}.\allowbreak{}.\allowbreak{}.\allowbreak{}.\allowbreak{}.\allowbreak{}.\allowbreak{}.\allowbreak{}.\allowbreak{}.\allowbreak{}.\allowbreak{}.\allowbreak{}.\allowbreak{}.\allowbreak{}.\allowbreak{}.\allowbreak{}.\allowbreak{}.\allowbreak{}.\allowbreak{}.\allowbreak{}.\allowbreak{}.\allowbreak{}.\allowbreak{}.\allowbreak{} \textit{Tillandsia monadelpha} (\allowbreak{} Fig.\allowbreak{} 4d-\allowbreak{}e)\allowbreak{}
\item \par{}16'.\allowbreak{} Brácteas florais róseas; pétalas arroxeadas .\allowbreak{}.\allowbreak{}.\allowbreak{}.\allowbreak{}.\allowbreak{}.\allowbreak{}.\allowbreak{}.\allowbreak{}.\allowbreak{}.\allowbreak{}.\allowbreak{}.\allowbreak{}.\allowbreak{}.\allowbreak{}.\allowbreak{}.\allowbreak{}.\allowbreak{}.\allowbreak{}.\allowbreak{}.\allowbreak{}.\allowbreak{}.\allowbreak{}.\allowbreak{}.\allowbreak{}.\allowbreak{}.\allowbreak{}.\allowbreak{}.\allowbreak{}.\allowbreak{}.\allowbreak{}.\allowbreak{}.\allowbreak{}.\allowbreak{}.\allowbreak{}.\allowbreak{}.\allowbreak{}.\allowbreak{}.\allowbreak{}.\allowbreak{}.\allowbreak{}.\allowbreak{}.\allowbreak{}.\allowbreak{}.\allowbreak{}.\allowbreak{}.\allowbreak{}.\allowbreak{}.\allowbreak{}.\allowbreak{}.\allowbreak{}.\allowbreak{}.\allowbreak{}.\allowbreak{}.\allowbreak{}.\allowbreak{}.\allowbreak{}.\allowbreak{}.\allowbreak{}.\allowbreak{}.\allowbreak{}.\allowbreak{}.\allowbreak{}.\allowbreak{}.\allowbreak{}.\allowbreak{}.\allowbreak{}.\allowbreak{}.\allowbreak{}.\allowbreak{}.\allowbreak{}.\allowbreak{}.\allowbreak{}.\allowbreak{}.\allowbreak{}.\allowbreak{}.\allowbreak{}.\allowbreak{}.\allowbreak{}.\allowbreak{}.\allowbreak{}.\allowbreak{}.\allowbreak{}.\allowbreak{}.\allowbreak{}.\allowbreak{}.\allowbreak{}.\allowbreak{}.\allowbreak{}.\allowbreak{}.\allowbreak{}.\allowbreak{}.\allowbreak{}.\allowbreak{}.\allowbreak{}.\allowbreak{}.\allowbreak{}.\allowbreak{}.\allowbreak{}.\allowbreak{} \textit{Tillandsia anceps}
\end{customList1}

\item \par{}15'.\allowbreak{} Inflorescência nunca complanada com flores polísticas
\begin{customList1}
\item \par{}17.\allowbreak{} Flores com pétalas róseas; estames excertos .\allowbreak{}.\allowbreak{}.\allowbreak{}.\allowbreak{}.\allowbreak{}.\allowbreak{}.\allowbreak{}.\allowbreak{}.\allowbreak{}.\allowbreak{}.\allowbreak{}.\allowbreak{}.\allowbreak{}.\allowbreak{}.\allowbreak{}.\allowbreak{}.\allowbreak{}.\allowbreak{}.\allowbreak{}.\allowbreak{}.\allowbreak{}.\allowbreak{}.\allowbreak{}.\allowbreak{}.\allowbreak{}.\allowbreak{}.\allowbreak{}.\allowbreak{}.\allowbreak{}.\allowbreak{}.\allowbreak{}.\allowbreak{}.\allowbreak{}.\allowbreak{}.\allowbreak{}.\allowbreak{}.\allowbreak{}.\allowbreak{}.\allowbreak{}.\allowbreak{}.\allowbreak{}.\allowbreak{}.\allowbreak{}.\allowbreak{}.\allowbreak{}.\allowbreak{}.\allowbreak{}.\allowbreak{}.\allowbreak{}.\allowbreak{}.\allowbreak{}.\allowbreak{}.\allowbreak{}.\allowbreak{}.\allowbreak{}.\allowbreak{}.\allowbreak{}.\allowbreak{}.\allowbreak{}.\allowbreak{}.\allowbreak{}.\allowbreak{}.\allowbreak{}.\allowbreak{}.\allowbreak{}.\allowbreak{}.\allowbreak{}.\allowbreak{}.\allowbreak{}.\allowbreak{}.\allowbreak{}.\allowbreak{}.\allowbreak{}.\allowbreak{} \textit{Tillandsia paraensis} (\allowbreak{} Fig.\allowbreak{} 4f-\allowbreak{}g)\allowbreak{}
\item \par{}17'.\allowbreak{} Flores com pétalas brancas; estames inclusos .\allowbreak{}.\allowbreak{}.\allowbreak{}.\allowbreak{}.\allowbreak{}.\allowbreak{}.\allowbreak{}.\allowbreak{}.\allowbreak{}.\allowbreak{}.\allowbreak{}.\allowbreak{}.\allowbreak{}.\allowbreak{}.\allowbreak{}.\allowbreak{}.\allowbreak{}.\allowbreak{}.\allowbreak{}.\allowbreak{}.\allowbreak{}.\allowbreak{}.\allowbreak{}.\allowbreak{}.\allowbreak{}.\allowbreak{}.\allowbreak{}.\allowbreak{}.\allowbreak{}.\allowbreak{}.\allowbreak{}.\allowbreak{}.\allowbreak{}.\allowbreak{}.\allowbreak{}.\allowbreak{}.\allowbreak{}.\allowbreak{}.\allowbreak{}.\allowbreak{}.\allowbreak{}.\allowbreak{}.\allowbreak{}.\allowbreak{}.\allowbreak{}.\allowbreak{}.\allowbreak{}.\allowbreak{}.\allowbreak{}.\allowbreak{}.\allowbreak{}.\allowbreak{}.\allowbreak{}.\allowbreak{}.\allowbreak{}.\allowbreak{}.\allowbreak{}.\allowbreak{}.\allowbreak{}.\allowbreak{}.\allowbreak{}.\allowbreak{}.\allowbreak{}.\allowbreak{}.\allowbreak{}.\allowbreak{}.\allowbreak{}.\allowbreak{}.\allowbreak{}.\allowbreak{}.\allowbreak{}.\allowbreak{}.\allowbreak{}.\allowbreak{}.\allowbreak{} \textit{Tillandsia tenuifolia} (\allowbreak{} Fig.\allowbreak{} 4h)\allowbreak{}
\end{customList1}

\end{customList1}

\end{customList1}

\end{customList1}

\end{customList1}

\item \par{}11'.\allowbreak{} Rosetas acumuladoras de água
\begin{customList1}
\item \par{}18.\allowbreak{} Folhas até 5 cm larg.\allowbreak{}; bainhas foliares internamente verdes .\allowbreak{}.\allowbreak{}.\allowbreak{}.\allowbreak{}.\allowbreak{}.\allowbreak{}.\allowbreak{}.\allowbreak{}.\allowbreak{}.\allowbreak{}.\allowbreak{}.\allowbreak{}.\allowbreak{}.\allowbreak{}.\allowbreak{}.\allowbreak{}.\allowbreak{}.\allowbreak{}.\allowbreak{}.\allowbreak{}.\allowbreak{}.\allowbreak{}.\allowbreak{}.\allowbreak{}.\allowbreak{}.\allowbreak{}.\allowbreak{}.\allowbreak{}.\allowbreak{}.\allowbreak{}.\allowbreak{}.\allowbreak{}.\allowbreak{}.\allowbreak{}.\allowbreak{}.\allowbreak{}.\allowbreak{}.\allowbreak{}.\allowbreak{}.\allowbreak{}.\allowbreak{}.\allowbreak{}.\allowbreak{}.\allowbreak{}.\allowbreak{}.\allowbreak{}.\allowbreak{}.\allowbreak{}.\allowbreak{}.\allowbreak{}.\allowbreak{}.\allowbreak{}.\allowbreak{}.\allowbreak{}.\allowbreak{}.\allowbreak{}.\allowbreak{}.\allowbreak{}.\allowbreak{}.\allowbreak{}.\allowbreak{}.\allowbreak{}.\allowbreak{}.\allowbreak{}.\allowbreak{}.\allowbreak{}.\allowbreak{}.\allowbreak{}.\allowbreak{}.\allowbreak{}.\allowbreak{}.\allowbreak{}.\allowbreak{}.\allowbreak{}.\allowbreak{}.\allowbreak{}.\allowbreak{}.\allowbreak{}.\allowbreak{}.\allowbreak{}.\allowbreak{}.\allowbreak{}.\allowbreak{}.\allowbreak{}.\allowbreak{}.\allowbreak{}.\allowbreak{}.\allowbreak{}.\allowbreak{}.\allowbreak{}.\allowbreak{}.\allowbreak{}.\allowbreak{}.\allowbreak{}.\allowbreak{}.\allowbreak{}.\allowbreak{}.\allowbreak{}.\allowbreak{}.\allowbreak{}.\allowbreak{}.\allowbreak{}.\allowbreak{}.\allowbreak{}.\allowbreak{}.\allowbreak{}.\allowbreak{}.\allowbreak{}.\allowbreak{}.\allowbreak{}.\allowbreak{} \textit{Tillandsia fasciculata} (\allowbreak{} Fig.\allowbreak{} 4b-\allowbreak{}c)\allowbreak{}
\item \par{}18'.\allowbreak{} Folhas 10 a 15 cm larg.\allowbreak{}; bainhas foliares internamente roxo-\allowbreak{}vináceas.\allowbreak{}.\allowbreak{}.\allowbreak{}.\allowbreak{}.\allowbreak{}.\allowbreak{}.\allowbreak{}.\allowbreak{}.\allowbreak{}.\allowbreak{}.\allowbreak{}.\allowbreak{}.\allowbreak{}.\allowbreak{}.\allowbreak{}.\allowbreak{}.\allowbreak{}.\allowbreak{}.\allowbreak{}.\allowbreak{}.\allowbreak{}.\allowbreak{}.\allowbreak{}.\allowbreak{}.\allowbreak{}.\allowbreak{}.\allowbreak{}.\allowbreak{}.\allowbreak{}.\allowbreak{}.\allowbreak{}.\allowbreak{}.\allowbreak{}.\allowbreak{}.\allowbreak{}.\allowbreak{}.\allowbreak{}.\allowbreak{}.\allowbreak{}.\allowbreak{}.\allowbreak{}.\allowbreak{}.\allowbreak{}.\allowbreak{}.\allowbreak{}.\allowbreak{}.\allowbreak{}.\allowbreak{}.\allowbreak{}.\allowbreak{}.\allowbreak{}.\allowbreak{}.\allowbreak{}.\allowbreak{}.\allowbreak{}.\allowbreak{}.\allowbreak{}.\allowbreak{}.\allowbreak{}.\allowbreak{}.\allowbreak{}.\allowbreak{}.\allowbreak{}.\allowbreak{}.\allowbreak{}.\allowbreak{}.\allowbreak{}.\allowbreak{}.\allowbreak{}.\allowbreak{}.\allowbreak{}.\allowbreak{}.\allowbreak{}.\allowbreak{}.\allowbreak{}.\allowbreak{}.\allowbreak{}.\allowbreak{}.\allowbreak{}.\allowbreak{}.\allowbreak{}.\allowbreak{}.\allowbreak{}.\allowbreak{}.\allowbreak{}.\allowbreak{}.\allowbreak{}.\allowbreak{}.\allowbreak{}.\allowbreak{}.\allowbreak{}.\allowbreak{}Tillandsia adpressiflora (\allowbreak{} Fig.\allowbreak{} 3g-\allowbreak{}h)\allowbreak{}
\end{customList1}

\end{customList1}

\end{customList1}

\end{customList1}
\begin{multicols}{2}


\par
{
\centering{
\includegraphics[width=\maxwidth{0.5\textwidth}]{not-found.png}
}
\captionof{figure}{\textbf{Figura 2:} \textit{a.\allowbreak{} \textit{Aechmea bromeliifolia} – detalhe da inflorescência; b-\allowbreak{}c.\allowbreak{} \textit{Aechmea castelnavii} – b.\allowbreak{} detalhe da inflores-\allowbreak{} cência; c.\allowbreak{} flores e frutos imaturos; d-\allowbreak{}e.\allowbreak{} \textit{Aechmea mertensii} – d.\allowbreak{} detalhe da inflorescência; e.\allowbreak{} detalhe das bráteas; f-\allowbreak{}g.\allowbreak{} \textit{Aechmea setigera} – f.\allowbreak{} detalhe das brácteas da base do pedúnculo; g.\allowbreak{} detalhe da inflorescência; h-\allowbreak{}i.\allowbreak{} \textit{Aechmea tocantina } – h.\allowbreak{} hábito fértil; i.\allowbreak{} detalhe da inflorescência.\allowbreak{} \textbf{Figure 2} – a.\allowbreak{} \textit{Aechmea bromeliifolia} – detail of inflorescence; b-\allowbreak{}c.\allowbreak{} \textit{Aechmea castelnavii} – b.\allowbreak{} detail of inflorescence; c.\allowbreak{} flowers and ima-\allowbreak{} ture fruits; d-\allowbreak{}e.\allowbreak{} \textit{Aechmea mertensii} – d.\allowbreak{} detail of inflorescence; e.\allowbreak{} detail of bracts; f-\allowbreak{}g.\allowbreak{} \textit{Aechmea setigera} – f.\allowbreak{} detail of bracts at peduncle base; g.\allowbreak{} detail of inflorescence; h-\allowbreak{}i.\allowbreak{} \textit{Aechmea tocantina } – fertile habit; i.\allowbreak{} detail of inflorescence.\allowbreak{}}} 
}
\par


\par
{
\centering{
\includegraphics[width=\maxwidth{0.5\textwidth}]{not-found.png}
}
\captionof{figure}{\textbf{Figura 3:} \textit{a.\allowbreak{} \textit{Ananas ananassoides} – detalhe da infrutescência; b.\allowbreak{} \textit{Araeococcus micranthus} – detalhe da inflorescência; c-\allowbreak{}d.\allowbreak{} \textit{Bromelia goeldiana} – c.\allowbreak{} hábito fértil; d.\allowbreak{} detalhe da inflorescência e flores; e.\allowbreak{} \textit{Bromelia grandiflora} – detalhe da inflorescência; f.\allowbreak{} \textit{Guzmania lingulata} – hábito fértil; g-\allowbreak{}h.\allowbreak{} \textit{Tillandsia adpressiflora } – g.\allowbreak{} hábito vegetativo; h.\allowbreak{} detalhe da flor; i.\allowbreak{} \textit{Tillandsia bulbosa} –hábito vegetativo.\allowbreak{} \textbf{Figure 3} – a.\allowbreak{} \textit{Ananas ananassoides} – detail of infrutescence; b.\allowbreak{} \textit{Araeococcus micranthus} – detail of inflorescence; c-\allowbreak{}d.\allowbreak{} \textit{Bromelia goeldiana} – c.\allowbreak{} fertile habit; d.\allowbreak{} detail of inflorescence and flowers; e.\allowbreak{} \textit{Bromelia grandiflora} – detail of inflorescence; f.\allowbreak{} \textit{Guzmania lingulata} – fertile habit; g-\allowbreak{}h.\allowbreak{} \textit{Tillandsia adpressiflora } – g.\allowbreak{} vegetative habit; h.\allowbreak{} detail of flower; i.\allowbreak{} \textit{Tillandsia bulbosa} – vegetative habit.\allowbreak{}}} 
}
\par


\par
{
\centering{
\includegraphics[width=\maxwidth{0.5\textwidth}]{not-found.png}
}
\captionof{figure}{\textbf{Figura 4:} \textit{a.\allowbreak{} \textit{Tillandsia bulbosa} – detalhe da inflorescência; b-\allowbreak{}c.\allowbreak{} \textit{Tillandsia fasciculata} – b.\allowbreak{} hábito fértil; c.\allowbreak{} flor; d-\allowbreak{}e.\allowbreak{} \textit{Tillandsia monadelpha} – d.\allowbreak{} hábito fértil; e.\allowbreak{} detalhe da flor; f-\allowbreak{}g.\allowbreak{} \textit{Tillandsia paraensis} – f.\allowbreak{} hábito fértil; g.\allowbreak{} detalhe da flor; h.\allowbreak{} \textit{Tillandsia tenuifolia} – detalhe da inflorescência; i.\allowbreak{} \textit{Tillandsia streptocarpa} – detalhe da flor.\allowbreak{} \textbf{Figure 4} – a.\allowbreak{} \textit{Tillandsia bulbosa} – detail of inflorescence; b-\allowbreak{}c.\allowbreak{} \textit{Tillandsia fasciculate} – b.\allowbreak{} fertile habit; c.\allowbreak{} flower; d-\allowbreak{}e.\allowbreak{} \textit{Tillandsia mona-\allowbreak{}delpha} – d.\allowbreak{} fertile habit; e.\allowbreak{} detail of flower; f-\allowbreak{}g.\allowbreak{} \textit{Tillandsia paraensis} – f.\allowbreak{} fertile habit; g.\allowbreak{} detail of flower; h.\allowbreak{} \textit{Tillandsia tenuifolia} – detail of inflorescence; i.\allowbreak{} \textit{Tillandsia streptocarpa} – detail of flower.\allowbreak{}}} 
}
\par




\section*{Agradecimentos}
\par{}As autoras agradecem à Norte Energia S.\allowbreak{}A.\allowbreak{},\allowbreak{} o acesso e a disponibilidade dos dados das espécies obtidos do Projeto Salvamento e Aproveitamento Científico da Flora da UHE Belo Monte; à Biota Projetos e Consultoria Ambiental Ltda.\allowbreak{},\allowbreak{} o apoio logístico durante a coleta de dados e a confecção do mapa; ao CNPq,\allowbreak{} a concessão da bolsa de doutorado da primeira autora; à CAPES,\allowbreak{} a concessão da bolsa de doutorado da segunda autora; aos curadores dos herbários Regina Célia Viana Martins (\allowbreak{}IAN)\allowbreak{},\allowbreak{} Rafaela Campostrini Forzza (\allowbreak{}RB)\allowbreak{} e Pedro Lage Viana (\allowbreak{}MG)\allowbreak{},\allowbreak{} por permitirem o acesso às coleções científicas; ao Engenheiro Florestal Willian de Lemos Guimarães,\allowbreak{} a confecção do mapa da área de estudo; e aos revisores anônimos.\allowbreak{}
\begin{biblio}[Referências]
\bibliotitle{Brummitt,\allowbreak{} R.\allowbreak{}K.\allowbreak{} \&\allowbreak{\allowbreak{}\allowbreak{}}\allowbreak{} Powell,\allowbreak{} C.\allowbreak{}E.\allowbreak{} 1992.\allowbreak{} Authors of plants names.\allowbreak{} Royal Botanic Gardens,\allowbreak{} Kew.\allowbreak{} 732p.\allowbreak{}}

\bibliotitle{Dubs,\allowbreak{} B.\allowbreak{} 1998.\allowbreak{} Prodromus Florae Matogrossensis.\allowbreak{} Part I:\allowbreak{} Checklist of Angiosperms.\allowbreak{} Betrona,\allowbreak{} Verlaga.\allowbreak{} 444p.\allowbreak{}}

\bibliotitle{Fidalgo,\allowbreak{} O.\allowbreak{} \&\allowbreak{\allowbreak{}\allowbreak{}}\allowbreak{} Bononi,\allowbreak{} V.\allowbreak{}L.\allowbreak{}R.\allowbreak{} 1984.\allowbreak{} Técnicas de coleta,\allowbreak{} preservação e herborização do material botânico.\allowbreak{} Instituto de Botânica (\allowbreak{}Manual,\allowbreak{} n.\allowbreak{} 4)\allowbreak{},\allowbreak{} São Paulo.\allowbreak{} 62p.\allowbreak{} }

\bibliotitle{Forzza,\allowbreak{} R.\allowbreak{}C.\allowbreak{}; Costa,\allowbreak{} A.\allowbreak{}F.\allowbreak{}; Leme,\allowbreak{} E.\allowbreak{}M.\allowbreak{}C.\allowbreak{}; Versieux,\allowbreak{} L.\allowbreak{}M.\allowbreak{}; Wanderley,\allowbreak{} M.\allowbreak{}G.\allowbreak{}L.\allowbreak{}; Louzada,\allowbreak{} R.\allowbreak{}B.\allowbreak{}; Monteiro,\allowbreak{} R.\allowbreak{}F.\allowbreak{}; Judice,\allowbreak{} D.\allowbreak{}M.\allowbreak{}; Fernandez,\allowbreak{} E.\allowbreak{}P.\allowbreak{}; Borges,\allowbreak{} R.\allowbreak{}A.\allowbreak{}X.\allowbreak{}; Penedo,\allowbreak{} T.\allowbreak{}S.\allowbreak{}A.\allowbreak{}; Monteiro,\allowbreak{} N.\allowbreak{}P.\allowbreak{} \&\allowbreak{\allowbreak{}\allowbreak{}}\allowbreak{} Moraes,\allowbreak{} M.\allowbreak{}A.\allowbreak{} 2013.\allowbreak{} Bromeliaceae.\allowbreak{} \textit{In}:\allowbreak{} Martinelli,\allowbreak{} G.\allowbreak{} \&\allowbreak{\allowbreak{}\allowbreak{}}\allowbreak{} Moraes,\allowbreak{} M.\allowbreak{}A.\allowbreak{} Livro vermelho da flora do Brasil.\allowbreak{} Andrea Jakobsson \&\allowbreak{\allowbreak{}\allowbreak{}}\allowbreak{} Instituto de Pesquisas do Jardim Botânico do Rio de Janeiro,\allowbreak{} Rio de Janeiro.\allowbreak{} Pp.\allowbreak{} 315-\allowbreak{}397.\allowbreak{} }

\bibliotitle{Forzza,\allowbreak{} R.\allowbreak{}C.\allowbreak{}; Costa,\allowbreak{} A.\allowbreak{}; Siqueira Filho,\allowbreak{} J.\allowbreak{}A.\allowbreak{}; Martinelli,\allowbreak{} G.\allowbreak{}; Monteiro,\allowbreak{} R.\allowbreak{}F.\allowbreak{}; Santos-\allowbreak{}Silva,\allowbreak{} F.\allowbreak{}; Saraiva,\allowbreak{} D.\allowbreak{}P.\allowbreak{}; Paixão-\allowbreak{}Souza,\allowbreak{} B.\allowbreak{}; Louzada,\allowbreak{} R.\allowbreak{}B.\allowbreak{} \&\allowbreak{\allowbreak{}\allowbreak{}}\allowbreak{} Versieux,\allowbreak{} L.\allowbreak{} 2014.\allowbreak{} Bromeliaceae.\allowbreak{} \textit{In}:\allowbreak{} Lista de espécies da flora do Brasil.\allowbreak{} Jardim Botânico do Rio de Janeiro.\allowbreak{} Disponível em <http:\allowbreak{}\fshyp{}\fshyp{}reflora.\allowbreak{}jbrj.\allowbreak{}gov\fshyp{}jabot\fshyp{}floradobrasil\fshyp{}FB66>.\allowbreak{} Acesso em 15 julho 2014.\allowbreak{}}

\bibliotitle{Givnish,\allowbreak{} T.\allowbreak{}J.\allowbreak{}; Millam,\allowbreak{} K.\allowbreak{}C.\allowbreak{}; Evans,\allowbreak{} T.\allowbreak{} M; Hall,\allowbreak{} J.\allowbreak{}C.\allowbreak{}; Pires,\allowbreak{} J.\allowbreak{}C.\allowbreak{}; Berry P.\allowbreak{}E.\allowbreak{} \&\allowbreak{\allowbreak{}\allowbreak{}}\allowbreak{} Sytsma,\allowbreak{} K.\allowbreak{}J.\allowbreak{} 2004.\allowbreak{} Ancient vicariance or recent long-\allowbreak{} distance dispersal? Inferences about phylogeny and South American – African disjunction in Rapateaceae and Bromeliaceae based on ndhF sequence data.\allowbreak{} International Journal of Plant Sciences 165:\allowbreak{} 35-\allowbreak{}54.\allowbreak{} }

\bibliotitle{Givnish,\allowbreak{} T.\allowbreak{}J.\allowbreak{}; Millan,\allowbreak{} K.\allowbreak{}C.\allowbreak{}; Berry,\allowbreak{} P.\allowbreak{}E.\allowbreak{} \&\allowbreak{\allowbreak{}\allowbreak{}}\allowbreak{} Sytsma K.\allowbreak{}J.\allowbreak{} 2007.\allowbreak{} Phylogeny,\allowbreak{} adaptive radiation,\allowbreak{} and historical biogeography of Bromeliaceae inferred from ndhF sequence data.\allowbreak{} \textit{In}:\allowbreak{} Columbus,\allowbreak{} J.\allowbreak{}T.\allowbreak{}; Friar,\allowbreak{} E.\allowbreak{}A.\allowbreak{}; Porter,\allowbreak{} J.\allowbreak{}M.\allowbreak{}; Prince,\allowbreak{} L.\allowbreak{}M.\allowbreak{} \&\allowbreak{\allowbreak{}\allowbreak{}}\allowbreak{} Simpson,\allowbreak{} M.\allowbreak{}G.\allowbreak{} Monocots:\allowbreak{} Comparative biology and evolution -\allowbreak{} Poales,\allowbreak{} Rancho Santa Ana Botanic Garden,\allowbreak{} Claremont.\allowbreak{} Pp.\allowbreak{} 3-\allowbreak{}26.\allowbreak{}}

\bibliotitle{Givnish,\allowbreak{} T.\allowbreak{}J.\allowbreak{}; Barffus,\allowbreak{} M.\allowbreak{}H.\allowbreak{}J.\allowbreak{}; Van E.\allowbreak{}B.\allowbreak{}; Riina,\allowbreak{} R.\allowbreak{}; Schulte,\allowbreak{} K.\allowbreak{}; Horres,\allowbreak{} R.\allowbreak{}; Gonsiska,\allowbreak{} P.\allowbreak{}A.\allowbreak{}; Jabaily,\allowbreak{} R.\allowbreak{}S.\allowbreak{}; Crayn,\allowbreak{} D.\allowbreak{}M.\allowbreak{}; Smith,\allowbreak{} J.\allowbreak{}A.\allowbreak{}C.\allowbreak{}; Inverno,\allowbreak{} K.\allowbreak{}; Brown,\allowbreak{} G.\allowbreak{}K.\allowbreak{}; Evans,\allowbreak{} T.\allowbreak{}M.\allowbreak{}; Holst,\allowbreak{} B.\allowbreak{}K.\allowbreak{}; Luther,\allowbreak{} H.\allowbreak{}; Till,\allowbreak{} W.\allowbreak{}; Zizka,\allowbreak{} G.\allowbreak{}; Barry,\allowbreak{} P.\allowbreak{}E.\allowbreak{} \&\allowbreak{\allowbreak{}\allowbreak{}}\allowbreak{} Sytsma K.\allowbreak{}J.\allowbreak{} 2011.\allowbreak{} Phylogeny,\allowbreak{} adaptative radiation,\allowbreak{} and historical biogeography in Bromeliaceae:\allowbreak{} insights from an eight-\allowbreak{}locus plastid phylogeny.\allowbreak{} American Journal of Botany 98:\allowbreak{} 872-\allowbreak{}895.\allowbreak{} }

\bibliotitle{Luther,\allowbreak{} H.\allowbreak{}E.\allowbreak{} 2012.\allowbreak{} An alphabetical list of bromeliad binomials.\allowbreak{} Marie Selby Botanical Gardens and Bromeliad Society International,\allowbreak{} Sarasota.\allowbreak{} 44p.\allowbreak{}}

\bibliotitle{IBAMA.\allowbreak{} 2014.\allowbreak{} Lista das espécies da flora brasileira ameaçadas de extinção no Pará e demais estados do Bioma Amazônia.\allowbreak{} Disponível em <http:\allowbreak{}\fshyp{}\fshyp{}www.\allowbreak{}ibama.\allowbreak{}gov.\allowbreak{}br\fshyp{}documentos.\allowbreak{}flora-\allowbreak{}ameaçada>.\allowbreak{} Acesso em 5 agosto 2014.\allowbreak{}}

\bibliotitle{IUCN.\allowbreak{} 2014.\allowbreak{} Red list of threatened species.\allowbreak{} Disponível em <http:\allowbreak{}\fshyp{}\fshyp{}www.\allowbreak{}iucnredlist.\allowbreak{}org\fshyp{}>.\allowbreak{} Acesso em 5 agosto 2014.\allowbreak{}}

\bibliotitle{Koch,\allowbreak{} A.\allowbreak{}K.\allowbreak{}; Santos,\allowbreak{} J.\allowbreak{}U.\allowbreak{}M.\allowbreak{} \&\allowbreak{\allowbreak{}\allowbreak{}}\allowbreak{} Ilkiu-\allowbreak{}Borges,\allowbreak{} A.\allowbreak{}L.\allowbreak{} 2013.\allowbreak{} Bromeliaceae epífitas de uma área de conservação da Amazônia brasileira.\allowbreak{} Rodriguésia 64:\allowbreak{} 419-\allowbreak{}425.\allowbreak{} }

\bibliotitle{Martinelli,\allowbreak{} G.\allowbreak{}; Vieira,\allowbreak{} C.\allowbreak{}M.\allowbreak{}; Gonzalez,\allowbreak{} M.\allowbreak{}; Leitman,\allowbreak{} P.\allowbreak{}; Piratininga,\allowbreak{} A.\allowbreak{}; Costa,\allowbreak{} A.\allowbreak{}F.\allowbreak{} da \&\allowbreak{\allowbreak{}\allowbreak{}}\allowbreak{} Forzza,\allowbreak{} R.\allowbreak{}C.\allowbreak{} 2008.\allowbreak{} Bromeliaceae da Mata Atlântica:\allowbreak{} lista de espécies,\allowbreak{} distribuição e conservação.\allowbreak{} Rodriguésia 59:\allowbreak{} 209-\allowbreak{}258.\allowbreak{}}

\bibliotitle{Martinelli,\allowbreak{} G.\allowbreak{} \&\allowbreak{\allowbreak{}\allowbreak{}}\allowbreak{} Moraes,\allowbreak{} M.\allowbreak{}A.\allowbreak{} 2013.\allowbreak{} Livro vermelho da flora do Brasil.\allowbreak{} Andrea Jakobsson \&\allowbreak{\allowbreak{}\allowbreak{}}\allowbreak{} Instituto de Pesquisas do Jardim Botânico do Rio de Janeiro,\allowbreak{} Rio de Janeiro.\allowbreak{} Pp.\allowbreak{} 315-\allowbreak{}397.\allowbreak{}}

\bibliotitle{Mello,\allowbreak{} B.\allowbreak{}M.\allowbreak{}; Abreu,\allowbreak{} L.\allowbreak{}L.\allowbreak{}; Koch,\allowbreak{} A.\allowbreak{}K.\allowbreak{}; Cardoso,\allowbreak{} A.\allowbreak{}L.\allowbreak{}R.\allowbreak{} \&\allowbreak{\allowbreak{}\allowbreak{}}\allowbreak{} Ilkiu-\allowbreak{}Borges,\allowbreak{} A.\allowbreak{}L.\allowbreak{} 2012.\allowbreak{} Bromeliaceae on the great curve of the Xingu River,\allowbreak{} Pará,\allowbreak{} Brazil.\allowbreak{} Rapid color guide 458,\allowbreak{} versão 1.\allowbreak{} Environnmental \&\allowbreak{\allowbreak{}\allowbreak{}}\allowbreak{} conservation programs.\allowbreak{} The Field Museum,\allowbreak{} Chicago.\allowbreak{} 2p.\allowbreak{} Disponível em <http:\allowbreak{}\fshyp{}\fshyp{}fm2.\allowbreak{}fieldmuseum.\allowbreak{}org\fshyp{}plantguides\fshyp{}guideimages.\allowbreak{}asp?ID=533>.\allowbreak{} Acesso em 15 julho 2014.\allowbreak{} }

\bibliotitle{Mez,\allowbreak{} C.\allowbreak{} 1891.\allowbreak{} Bromeliaceae.\allowbreak{} \textit{In}:\allowbreak{} Martius,\allowbreak{} C.\allowbreak{}F.\allowbreak{}P.\allowbreak{}; Eichler,\allowbreak{} A.\allowbreak{}G.\allowbreak{}; Urban,\allowbreak{} I.\allowbreak{} (\allowbreak{}eds.\allowbreak{})\allowbreak{},\allowbreak{} \textit{Flora Brasiliensis}.\allowbreak{} München,\allowbreak{} Wien,\allowbreak{} Leipzig.\allowbreak{} Vol 3,\allowbreak{} pars 3.\allowbreak{} Pp.\allowbreak{} 173-\allowbreak{}634.\allowbreak{}}

\bibliotitle{MPEG -\allowbreak{} Museu Paraense Emílio Goeldi.\allowbreak{} 2008.\allowbreak{} Descrição e análise da flora da região do médio-\allowbreak{}baixo rio Xingú.\allowbreak{} Convênio MCT\fshyp{}MPEG\fshyp{}Camargo Corrêa\fshyp{}Odebrecht\fshyp{}Andrade Gutierrez\fshyp{}Eletrobrás\fshyp{}Fidesa.\allowbreak{} Relatório Técnico,\allowbreak{} Belém.\allowbreak{} 384p.\allowbreak{}}

\bibliotitle{Nogueira-\allowbreak{}Braga,\allowbreak{} M.\allowbreak{}M.\allowbreak{} 1977.\allowbreak{} Anatomia foliar de Bromeliaceae da campina.\allowbreak{} Acta Amazonica 7:\allowbreak{} 1-\allowbreak{}74.\allowbreak{}}

\bibliotitle{Quaresma,\allowbreak{} A.\allowbreak{}C.\allowbreak{} \&\allowbreak{\allowbreak{}\allowbreak{}}\allowbreak{} Medeiros,\allowbreak{} T.\allowbreak{}D.\allowbreak{}S.\allowbreak{} 2009.\allowbreak{} As bromélias.\allowbreak{} \textit{In}:\allowbreak{} Jardim,\allowbreak{} M.\allowbreak{}A.\allowbreak{}G.\allowbreak{} Diversidade biológica das áreas de proteção ambiental Ilhas do Combu e Algodoal-\allowbreak{}Maiandeua,\allowbreak{} Pará,\allowbreak{} Brasil.\allowbreak{} Museu Paraense Emilio Goeldi,\allowbreak{} Belém.\allowbreak{} Pp.\allowbreak{} 71-\allowbreak{}78.\allowbreak{}}

\bibliotitle{Quaresma,\allowbreak{} A.\allowbreak{}C.\allowbreak{} \&\allowbreak{\allowbreak{}\allowbreak{}}\allowbreak{} Jardim,\allowbreak{} M.\allowbreak{}A.\allowbreak{}G.\allowbreak{} 2012.\allowbreak{} Diversidade de bromeliáceas epífitas na Área de Proteção Ambiental Ilha do Combu,\allowbreak{} Belém,\allowbreak{} Pará,\allowbreak{} Brasil.\allowbreak{} Acta Botanica Brasilica 26:\allowbreak{} 290-\allowbreak{}294.\allowbreak{}}

\bibliotitle{Quaresma,\allowbreak{} A.\allowbreak{}C.\allowbreak{} \&\allowbreak{\allowbreak{}\allowbreak{}}\allowbreak{} Jardim,\allowbreak{} M.\allowbreak{}A.\allowbreak{}G.\allowbreak{} 2013.\allowbreak{} Fitossociologia e distribuição espacial de Bromélias epifíticas em uma Floresta de Várzea Estuarina Amazônica.\allowbreak{} Revista Brasileira de Biociências 11:\allowbreak{} 1-\allowbreak{}6.\allowbreak{}}

\bibliotitle{Ribeiro,\allowbreak{} J.\allowbreak{}E.\allowbreak{}S.\allowbreak{}; Hopkins,\allowbreak{} M.\allowbreak{}J.\allowbreak{}G.\allowbreak{}; Vicentini,\allowbreak{} A.\allowbreak{}; Sothers,\allowbreak{} C.\allowbreak{}A.\allowbreak{}; Costa,\allowbreak{} M.\allowbreak{}A.\allowbreak{}S.\allowbreak{}; Brito,\allowbreak{} J.\allowbreak{}M.\allowbreak{}; Souza,\allowbreak{} M.\allowbreak{}A.\allowbreak{}D.\allowbreak{}; Martins,\allowbreak{} L.\allowbreak{}H.\allowbreak{}P.\allowbreak{}; Lohmann,\allowbreak{} L.\allowbreak{}G.\allowbreak{}; Assunção,\allowbreak{} P.\allowbreak{}A.\allowbreak{}C.\allowbreak{}L.\allowbreak{}; Pereira,\allowbreak{} E.\allowbreak{}C.\allowbreak{}; Silva,\allowbreak{} C.\allowbreak{}F.\allowbreak{}; Mesquita,\allowbreak{} M.\allowbreak{}R.\allowbreak{} \&\allowbreak{\allowbreak{}\allowbreak{}}\allowbreak{} Procópio,\allowbreak{} L.\allowbreak{}C.\allowbreak{} 1999.\allowbreak{} Flora da Reserva Ducke:\allowbreak{} guia de identificação das plantas vasculares de uma floresta de terra-\allowbreak{}firme na Amazônia central.\allowbreak{} INPA,\allowbreak{} Manaus.\allowbreak{} 816p.\allowbreak{}}

\bibliotitle{Salomão,\allowbreak{} R.\allowbreak{}P.\allowbreak{}; Vieira,\allowbreak{} I.\allowbreak{}C.\allowbreak{}G.\allowbreak{}; Suemitsu,\allowbreak{} C.\allowbreak{}; Rosa,\allowbreak{} N.\allowbreak{}A.\allowbreak{}; Almeida,\allowbreak{} S.\allowbreak{}S.\allowbreak{}; Amaral,\allowbreak{} D.\allowbreak{}D.\allowbreak{} \&\allowbreak{\allowbreak{}\allowbreak{}}\allowbreak{} Menezes,\allowbreak{} M.\allowbreak{}P.\allowbreak{}M.\allowbreak{} 2007.\allowbreak{} As florestas de Belo Monte na grande curva do rio Xingu,\allowbreak{} Amazonia Oriental.\allowbreak{} Boletim do Museu Paraense Emílio Goeldi,\allowbreak{} Ciências Naturais 2:\allowbreak{} 57-\allowbreak{}153.\allowbreak{} }

\bibliotitle{Scharf,\allowbreak{} U.\allowbreak{} \&\allowbreak{\allowbreak{}\allowbreak{}}\allowbreak{} Gouda,\allowbreak{} E.\allowbreak{}J.\allowbreak{} 2008.\allowbreak{} Bringing Bromeliaceae Back to Homeland Botany.\allowbreak{} Journal of the Bromeliad Society 58:\allowbreak{} 123-\allowbreak{}129.\allowbreak{}}

\bibliotitle{SEMA.\allowbreak{} 2014.\allowbreak{} Espécies Ameaçadas.\allowbreak{} Disponível em <http:\allowbreak{}\fshyp{}\fshyp{}www.\allowbreak{}sema.\allowbreak{}pa.\allowbreak{}gov.\allowbreak{}br\fshyp{}2009\fshyp{}03\fshyp{}27\fshyp{}9439>.\allowbreak{} Acesso em 5 agosto 2014.\allowbreak{} }

\bibliotitle{Sousa G.\allowbreak{}M.\allowbreak{} \&\allowbreak{\allowbreak{}\allowbreak{}}\allowbreak{} Wanderley,\allowbreak{} M.\allowbreak{}G.\allowbreak{}L.\allowbreak{} 2007.\allowbreak{} \textit{Aechmea rodriguesiana } (\allowbreak{}L.\allowbreak{}B.\allowbreak{}Sm.\allowbreak{})\allowbreak{} L.\allowbreak{}B.\allowbreak{}Sm.\allowbreak{} (\allowbreak{}Bromeliaceae)\allowbreak{} uma espécie endêmica da Amazonia brasileira.\allowbreak{} Acta Amazonica 37:\allowbreak{} 517-\allowbreak{}520.\allowbreak{}}

\bibliotitle{Sousa Junior,\allowbreak{} W.\allowbreak{}C.\allowbreak{}; Reid,\allowbreak{} J.\allowbreak{} \&\allowbreak{\allowbreak{}\allowbreak{}}\allowbreak{} Leitão,\allowbreak{} N.\allowbreak{}C.\allowbreak{}S.\allowbreak{} 2006.\allowbreak{} Custos e benefícios do complexo hidrelétrico Belo Monte:\allowbreak{} uma abordagem econômico-\allowbreak{}ambiental.\allowbreak{} Conservation Strategy Fund -\allowbreak{}CSF,\allowbreak{} Lagoa Santa.\allowbreak{} 90p.\allowbreak{}}

\bibliotitle{Smith,\allowbreak{} L.\allowbreak{}B.\allowbreak{} \&\allowbreak{\allowbreak{}\allowbreak{}}\allowbreak{} Downs,\allowbreak{} R.\allowbreak{}J.\allowbreak{} 1974.\allowbreak{} Pitcairnioideae (\allowbreak{}Bromeliaceae)\allowbreak{}.\allowbreak{} Flora Neotropica,\allowbreak{} Monograph 14:\allowbreak{} 1-\allowbreak{}662.\allowbreak{}}

\bibliotitle{Smith,\allowbreak{} L.\allowbreak{}B.\allowbreak{} \&\allowbreak{\allowbreak{}\allowbreak{}}\allowbreak{} Downs,\allowbreak{} R.\allowbreak{} J.\allowbreak{} 1977.\allowbreak{} Tillandsioideae (\allowbreak{}Bromeliaceae)\allowbreak{}.\allowbreak{} Flora Neotropica,\allowbreak{} Monograph 14:\allowbreak{} 663-\allowbreak{}1492.\allowbreak{}}

\bibliotitle{Smith,\allowbreak{} L.\allowbreak{}B.\allowbreak{} \&\allowbreak{\allowbreak{}\allowbreak{}}\allowbreak{} Downs,\allowbreak{} R.\allowbreak{} J.\allowbreak{} 1979.\allowbreak{} Bromelioideae (\allowbreak{}Bromeliaceae)\allowbreak{}.\allowbreak{} Flora Neotropica,\allowbreak{} Monograph 14:\allowbreak{} 1493-\allowbreak{}2141.\allowbreak{}}

\bibliotitle{World Checklist of Selected Plant Families.\allowbreak{} Bromeliaceae.\allowbreak{} Disponível em <http:\allowbreak{}\fshyp{}\fshyp{}apps.\allowbreak{}kew.\allowbreak{}org\fshyp{}wcsp\fshyp{}incfamilies.\allowbreak{}do>.\allowbreak{} Acesso em 23 outubro 2014.\allowbreak{} }

\end{biblio}

\end{multicols}
\medskip\par\noindent
\footnotesize{Este é um artigo publicado em acesso aberto (Open Access) sob a licença Creative Commons Attribution Non-Commercial, que permite uso, distribuição e reprodução em qualquer meio, sem restrições desde que sem fins comerciais e que o trabalho original seja corretamente citado.}
