\selectlanguage{english}
\begin{abstract}
\par Spermatogenesis is a process where an important contribution of genes involved in folate-mediated one-carbon metabolism is observed. The aim of the present study was to investigate the association between male infertility and the \textit{MTHFR} (677C > T; 1298A > C), \textit{MTR} (2756A > G) and \textit{MTRR} (66A > G) polymorphisms in a Polish population. No significant differences in genotype or allele frequencies were detected between the groups of 284 infertile men and of 352 fertile controls. These results demonstrate that common polymorphisms in folate pathway genes are not major risk factors for non-obstructive male infertility in the Polish population.

\par \noindent \\
\footnotesize{\textit{Keywords}: MTHFR, MTR, MTRR, polymorphism, infertility}
\par \noindent \\
\footnotesize{Received: 12/06/2014;} \footnotesize{Accepted: 10/10/2014.} 
\end{abstract}
\par \initial{S}permatogenesis is a multistep developmental process coordinated by sequential expression of various genes, with an important contribution of genes involved in folate-mediated one-carbon metabolism. This pathway is mandatory for thymidylate and purine biosynthesis, thus providing substrates for DNA synthesis in rapidly dividing male germ cells. Via involvement in homocysteine metabolism, folates participate in DNA, RNA and histone methylation reactions, taking part in regulation of transcription. The key enzymes implicated in the above mentioned metabolic pathways are: 5,10-methylenetetrahydrofolate reductase (MTHFR), methionine synthase (MTR) and methionine synthase reductase (MTRR). It was found that polymorphisms defined within the coding sequences of these genes may affect metabolic pathways controlled by the enzymes.\par Within the \textit{MTHFR} gene, two functional single nucleotide polymorphisms (SNPs) were characterized. The \textit{MTHFR} 677C > T variant (rs1801133) encodes a thermolabile protein variant with enzymatic activity decreased by 35\% in heterozygotes and by 70\% in the homozygous state. The \textit{MTHFR} 1298A > C polymorphism (rs1801131) is associated with a 30\% decrease in enzymatic activity. The \textit{MTHFR} 677C > T and \textit{MTHFR} 1298A > C SNPs were also shown to be associated with DNA hypomethylation (Weiner \textit{et al.}, 2014). In the \textit{MTR} gene, an adenine to guanine transition at position 2756 (A > G, rs1805087) results in substitution of aspartic acid with glycine in codon 919 of the protein and is related to alterations in the folate metabolic pathway. The Asp919Gly substitution in the MTR enzyme results in its higher activity, leading to more effective homocysteine remethylation and methionine production (Ravel \textit{et al.}, 2009). The \textit{MTRR} gene includes a polymorphic locus \textit{MTRR} 66A > G (rs1801394), which was shown to slightly reduce enzymatic activity, but was associated with decreased plasma homocysteine concentrations (Park \textit{et al.}, 2005).\par The available information on associations of the above mentioned SNPs in \textit{MTHFR}, \textit{MTR} and \textit{MTRR} genes with male infertility reported from various populations is not consistent, and mostly evaluate \textit{MTHFR} gene. Most of the studies available are from Asian populations (Lee \textit{et al.}, 2006; A \textit{et al.}, 2007; Park \textit{et al.}, 2009), with some data from Caucasians: Italian (Stuppia \textit{et al.}, 2003), Dutch (Ebisch \textit{et al.}, 2003), Swedish (Murphy \textit{et al.}, 2011), French (Ravel \textit{et al.}, 2009; Montjean \textit{et al.}, 2011), German (Bezold \textit{et al.}, 2001), Spanish (Camprubi \textit{et al.}, 2013) and only one from an East European (\textit{i.e.} Slavic) population, from Russia (Weiner \textit{et al.}, 2014). The results are still debatable, and the observed differences may not only depend on ethnic differences but also on environmental factors, \textit{i.e.} folate intake, which in turn can influence DNA methylation and semen quality. The present study aimed at definition of associations of the common \textit{MTHFR}, \textit{MTR} and \textit{MTRR} polymorphisms with male infertility in a Polish (\textit{i.e.} Slavic) population.\par The study was carried out in 284 consecutive, otherwise healthy male patients (aged 22–49 years, mean 32.7 ± 4.7) without any chromosomal abnormalities, undergoing semen analysis due to infertility workup. The inclusion criteria were as follow: no children from current or previous relations with at least a year history of at least a year of regular (2–3 weekly), unprotected sexual activity without conception; female partners aged up to 35 years with regular menstrual bleedings and/or progesterone levels in the luteal phase of the cycle > 10 ng/mL, normal transvaginal ultrasound examination, negative testing for \textit{Chlamydia trachomatis} infection, without history of pelvic inflammatory disease or abdominal operations. Subjects were excluded from the study if semen analysis and clinical picture suggested obstructive azoospermia or testicular, epididymal, or accessory gland infection. Also, subjects with known systemic disease, BMI ≥ 30 kg/m\textsuperscript{2}, varicocele, history of mumps, testicular torsio or maldescence, trauma, as well as occupational hazards (exposure to solvents, pesticides, painting materials, heavy metals or radiation) were not taken into consideration.\par The control group consisted of 352 healthy males (aged 21–56 years, mean 34.7 ± 8.7) recruited among consecutive men accompanying their female partners at term labor in the University Department of Feto-Maternal Medicine. Paternity was confirmed by women; however the possible paternal discrepancy was additionally checked based on blood group verification. Both the men undergoing infertility examination, as well as the fertile controls were Caucasians of Polish origin, recruited within the same geographical region. The study was approved by the local ethics committee and written informed consent was obtained from all subjects.\par Genomic DNA was extracted from blood samples using GeneMATRIX Blood DNA Purification Kit (EURx, Poland). Pre-validated allelic discrimination TaqMan real-time PCR assays (Life Technologies, USA) were used for detection of the respective SNPs in \textit{MTHFR} (rs1801131, rs1801133), \textit{MTR} (rs1805087) and \textit{MTRR} (rs1801394) genes. Amplification was performed in a 7500 Fast Real-Time PCR System with incorporated SDS software for SNP genotyping (Applied Biosystems, USA) using TaqMan GTXpress Master Mix (Life Technologies, USA). Fluorescence data was captured after 40 PCR cycles.\par Allele and genotype frequencies were determined by direct counting of alleles. Concordance of genotype distribution with Hardy-Weinberg equilibrium was calculated using χ\textsuperscript{2} test. Genotype and allele frequencies between the study groups were compared by means of Fisher’s exact test. The effect of each polymorphism was tested in both a dominant and recessive model. All genotypes were distributed in concordance with Hardy-Weinberg equilibrium, both in infertile patients and control subjects. No significant differences between the study groups were noted, neither in genotype distribution, nor in allele frequencies. All genotyping results are given in Table 1.\onecolumn \ctable[ caption = {Distribution of \textit{MTR}, \textit{MTRR} and \textit{Fluorescence data after 40 PCR cyclesMTHFR} gene variants in infertile patients and control group.}, width=\textwidth, pos = ht, left ] { XXXXXXXXXXXXXXXX} { \tnote[*]{ Calculated by means of Fisher’s exact test.} }{ \FL {} & \multicolumn{2}{c}{Fertile n = 352} & \multicolumn{2}{c}{Infertile n = 284} & {p\textsuperscript{*}} & {OR (95\%CI)} \ML \multicolumn{2}{c}{} & \multicolumn{2}{c}{} & {} & {} \ML {} & {n} & {\%} & {n} & {\%} & {} & {} \ML \multicolumn{2}{c}{\textit{MTR rs1805087} (2756A > G, Asp919Gly)} & {} & {} & {} & {} & {} \NN {Genotype} & {} & {} & {} & {} & {} & {} \NN {AA} & {218} & {61.9\%} & {178} & {62.7\%} & {-} & {} \NN {AG} & {125} & {35.5\%} & {93} & {32.7\%} & {0.610} & {0.91 (0.78–1.46)} \NN {GG} & {9} & {2.6\%} & {13} & {4.6\%} & {0.271} & {1.77 (0.74–4.24)} \NN {(AG+GG) \textit{vs.} AA} & {134} & {38.1\%} & {106} & {37.3\%} & {0.869} & {0.97 (0.70–1.34)} \NN {(AA+AG) \textit{vs.} GG} & {343} & {97.4\%} & {271} & {95.4\%} & {0.193} & {0.55(0.23–1.29)} \NN {Allele} & {} & {} & {} & {} & {} & {} \NN {A} & {561} & {79.7\%} & {449} & {79.0\%} & {} & {} \NN {G} & {143} & {20.3\%} & {119} & {21.0\%} & {0.780} & {} \NN \multicolumn{2}{c}{\textit{MTRR rs1801394} (66A > G, Ile22Met)} & {} & {} & {} & {} & {} \NN {Genotype} & {} & {} & {} & {} & {} & {} \NN {AA} & {70} & {19.9\%} & {51} & {18.0\%} & {-} & {} \NN {AG} & {171} & {48.6\%} & {139} & {48.9\%} & {0.666} & {1.12 (0.73–1.71)} \NN {GG} & {111} & {31.5\%} & {94} & {33.1\%} & {0.564} & {1.16 (0.74–1.83)} \NN {(AG+GG) \textit{vs.} AA} & {282} & {80.1\%} & {233} & {82.0\%} & {0.608} & {1.13 (0.76–1.69)} \NN {(AA+AG) \textit{vs.} GG} & {241} & {68.5\%} & {190} & {66.9\%} & {0.733} & {0.93 (0.77–1.50)} \NN {Allele} & {} & {} & {} & {} & {} & {} \NN {A} & {311} & {44.2\%} & {241} & {42.4\%} & {} & {} \NN {G} & {393} & {55.8\%} & {327} & {57.6\%} & {0.569} & {} \NN \multicolumn{2}{c}{\textit{MTHFR rs1801133} (677C > T, Ala222Val)} & {} & {} & {} & {} & {} \NN {Genotype} & {} & {} & {} & {} & {} & {} \NN {CC} & {166} & {47.2\%} & {143} & {50.4\%} & {-} & {} \NN {CT} & {150} & {42.6\%} & {113} & {39.8\%} & {0.448} & {0.87 (0.63–1.21)} \NN {TT} & {36} & {10.2\%} & {28} & {9.9\%} & {0.783} & {0.90 (0.52–1.55)} \NN {(CT+TT) \textit{vs.} CC} & {186} & {52.8\%} & {141} & {49.6\%} & {0.426} & {0.880 (0.64–1.20)} \NN {(CC+CT) \textit{vs.} TT} & {316} & {89.8\%} & {256} & {90.1\%} & {0.895} & {1.04 (0.62–4.75)} \NN {Allele} & {} & {} & {} & {} & {} & {} \NN {C} & {482} & {68.5\%} & {399} & {70.2\%} & {} & {} \NN {T} & {222} & {31.5\%} & {169} & {29.8\%} & {0.502} & {} \NN \multicolumn{2}{c}{\textit{MTHFR rs1801131} (1298A > C, Glu429Ala)} & {} & {} & {} & {} & {} \NN {Genotype} & {} & {} & {} & {} & {} & {} \NN {AA} & {156} & {44.3\%} & {128} & {45.1\%} & {-} & {} \NN {AC} & {156} & {44.3\%} & {130} & {45.8\%} & {0.933} & {1.02 (0.73–1.41)} \NN {CC} & {40} & {11.4\%} & {26} & {9.2\%} & {0.413} & {0.79 (0.46–1.37)} \NN {(AC+CC) \textit{vs.} AA} & {196} & {55.7\%} & {156} & {54.9\%} & {0.873} & {0.97 (0.71–1.32)} \NN {(AA+AC) \textit{vs.} CC} & {312} & {88.6\%} & {258} & {90.8\%} & {0.433} & {1.27 (0.76–2.14)} \NN {Allele} & {} & {} & {} & {} & {} & {} \NN {A} & {468} & {66.5\%} & {386} & {68.0\%} & {} & {} \NN {C} & {236} & {33.5\%} & {182} & {32.0\%} & {0.589} & {} \NN \LL} \twocolumn \par No significant impact of the studied polymorphisms on male infertility was revealed in the present study. The original concept of the impact of \textit{MTHFR} variants on male reproduction and initial positive association of the thermolabile 677T variant with infertility came from the German study of Bezold \textit{et al.} (2001), who reported significant overrepresentation of TT homozygotes among male patients seeking fertility evaluation compared with control group (18.8 \textit{vs.} 9.5\%). This preliminary report, without detailed characterization of neither male infertility nor control subjects, has been subsequently followed by several studies in Caucasian populations, \textit{i.e.} Dutch (Ebisch \textit{et al.}, 2003), Italian (Stuppia \textit{et al.}, 2003), Swedish (Murphy \textit{et al.}, 2011), and Spanish (Camprubi \textit{et al.}, 2013). Contrary to the original report, the results of all aforementioned studies were negative. It should be noted that most of them simply lacked sufficient power to verify the existence of the investigated association, as numbers of participants were low (Table 2). Nonetheless negative association results were accompanied by findings on a potential relationship of \textit{MTHFR} genotype and sperm counts, but only in some studies. Ravel \textit{et al.} (2009) did not find any association between \textit{MTHFR} (677C > T, 1298A > C and 215GA - rs2066472) genetic variants and sperm counts in French infertile men, which was later confirmed by Montjean \textit{et al.} (2011) in a larger cohort of mixed ethnicity. Similarly, none of the genotypes was associated with neither standard seminogram parameters nor presence of sperm DNA hypomethylation (Camprubi \textit{et al.}, 2013). Finally, in the recent report from an East European population in Russia, Weiner \textit{et al.} (2014) have observed the association of \textit{MTHFR} genotype with azooospermia, but found no general impact of \textit{MTHFR} 677C > T and \textit{MTHFR} 1298A > C polymorphisms on male infertility. Summarizing the observations from Caucasian studies, including the present Polish one, it seems that \textit{MTHFR} 677C > T and \textit{MTHFR} 1298A > C polymorphisms are not associated with male infertility.\onecolumn \ctable[ caption = {Previous studies on genes of folate-mediated one-carbon metabolism pathway in relation to male infertility.}, width=\textwidth, pos = ht, left ] { XXXX} { }{ \FL {Genetic polymorphism studied} & {Association reported} & {Study population} & {Reference} \ML {\textit{MTHFR} 677C > T} & {\textit{MTHFR} 677C > T (TT homozygotes) with infertility} & {255 infertile men and 200 controls, ethnicity not given, Germany} & {Bezold \textit{et al.}, 2001} \NN {\textit{MTHFR} 677C > T} & {no association with infertility} & {93 infertile and 105 controls, Italian Caucasians} & {Stuppia \textit{et al.}, 2003} \NN {\textit{MTHFR} 677C > T} & {no association with subfertility} & {113 fertile and 77 subfertile males, Dutch Caucasians} & {Ebisch \textit{et al.}, 2003} \NN {20 SNPs in 12 genes related to folate, homocysteine and B12 metabolism} & {no association of folate-related gene polymorphisms with infertility; \textit{PEMT} (phosphatidylethanolamine\\N-methyltransferase) rs7946 and CD320 (transcobalamin receptor) rs173665 with infertility} & {153 infertile men and 184 controls, ethnicity not given, Sweden} & {Murphy \textit{et al.}, 2011} \NN {\textit{MTHFR} 677C > T} & {no association with infertility or sperm counts} & {107 infertile men and 25 controls, ethnicity not given, Spain} & {Camprubi \textit{et al.}, 2013} \NN {\textit{MTHFR} 677C > T, 1298A > C; \textit{MTR} 2756A > G; \textit{MTRR} 66A>G; \textit{SHMT1} 1420C > T; \textit{MTHFD1} 1958G > A; \textit{CBS} 844ins68} & {\textit{MTHFD}1 1958G>Aand \textit{MTR} 2756A > G with infertility (without correction for multiple testing only); \textit{MTHFR} 677C > T with azoospermia;} & {275 infertile men and 349 controls, Russian Caucasians} & {Weiner \textit{et al.}, 2014} \NN {\textit{MTHFR} 677C > T, 1298A > C, 215GA; \textit{MTRR} 66A > G, 524C > T; \textit{CBS 919G} > \textit{A}} & {no association with reduced sperm counts} & {70 azoospermia and 182 oligozoospermia cases, 114 normospermic controls, “French ethnic origin” stated} & {Ravel \textit{et al.}, 2009} \NN {\textit{MTHFR} 677C > T, 1298A > C} & {\textit{MTHFR} 677C > T with infertility} & {373 infertile men + 396 controls, Korean Asians} & {Park \textit{et al.}, 2005} \NN {\textit{MTHFR} 677C > T, 1298A > C; \textit{MTR} 2756A > G; \textit{MTRR} 66A>G} & {\textit{MTHFR} 677C > T and \textit{MTRR} 66A>G with infertility; \textit{MTHFR} 677C > T and \textit{MTR} 2756A > G with azoospermia} & {360 infertile men and 325 controls, Korean Asians} & {Lee \textit{et al.}, 2006} \NN {\textit{MTHFR} 677C > T} & {\textit{MTHFR} 677C > T with infertility} & {355 infertile and 252 fertile Chinese Asians} & {A \textit{et al.}, 2007} \NN {\textit{MTHFR} 677C > T, 1298A > C, 215GA; \textit{MTRR} 66A>G,524C>T} & {\textit{MTHFR} 677C > T with hyperhomocysteinemia; no association with sperm counts} & {522 men, mixed ethnic origin, France} & {Montjean \textit{et al.} 2011} \NN {\textit{MTHFR} 677C > T, 1298A > C} & {\textit{MTHFR} 677C > T with non-obstructive azoospermia and severe oligozoospermia} & {156 infertile men and 233 controls, mixed ethnic origin, Brazil} & {Gava \textit{et al.}, 2011} \NN {\textit{MTHFR} 677C > T, 1298A > C; \textit{DNMT3B} 46359C > T} & {no association with infertility} & {179 infertile and 200 fertile men, India} & {Dhillon \textit{et al., 2007}} \NN {\textit{MTHFR} 677C > T} & {\textit{MTHFR} 677C > T with infertility, also confirmed by meta-analysis of data from available studies} & {522 infertile and 315 controls, India} & {Gupta \textit{et al.}, 2011} \NN {\textit{MTHFR} 677C > T, 1298A > C; \textit{MTRR} 66A>G} & {\textit{MTHFR} 677C > T with infertility} & {150 infertile and 150 controls, Arab Jordanian population} & {Mfady \textit{et al.}, 2014} \NN {\textit{MTHFR} 677C > T} & {\textit{MTHFR} 677C > T with infertility in Asians, but not I Caucasians} & {metaanalysis of previously published data} & {Wu \textit{et al}, 2012} \NN {\textit{MTHFR} 1298A > C} & {\textit{MTHFR} 1298A > C with infertility and azoospermia (metaanalysis included only one Caucasian study)} & {metaanalysis of previously published data} & {Shen \textit{et al.}, 2012} \NN {\textit{MTHFR} 677C > T, 1298A > C; \textit{MTR} 2756A > G; \textit{MTRR} 66A>G} & {no association with infertility} & {284 infertile and 352 fertile Polish Caucasians} & {present study} \NN \LL} \twocolumn \par In contrast to Caucasian studies, investigations on the association of \textit{MTHFR} polymorphism with male infertility conducted in populations of non-European descent gave several positive results. Two large studies from Korea presented a significant association of the \textit{MTHFR} 677C > T (but not 1298A > C) polymorphism with infertility (Park \textit{et al.}, 2005, Lee \textit{et al.}, 2006). Moreover, these observations were supported by a study in Chinese patients, where \textit{MTHFR} 677T status was found to be a risk factor for male infertility (A \textit{et al.}, 2007). Data from Asian studies were also confirmed by other studies, including several reports revealing an impact of \textit{MTHFR} 677C > T polymorphism on infertility: from an Arabic population, \textit{i.e.} Jordanians, by Mfady \textit{et al.} (2014), a Brazilian report on males of mixed ethnicity by Gava \textit{et al}. (2011), or by Gupta \textit{et al.} (2011) from India. However, negative data for Indians, for both \textit{MTHFR} 677C > T and \textit{MTHFR} 1298A > C, were also reported (Dhillon \textit{et al.}, 2007). Such impact of ethnic differences is also reflected in meta-analyses. A stratified analysis by Wu \textit{et al.} (2012) showed that a significant association between \textit{MTHFR} 677C > T polymorphism and male infertility was present only in Asians (OR = 1.79 for two copies of T allele and OR = 1.42 for T allele carriers), but not in Caucasians. A meta-analysis published by Shen \textit{et al.} (2012) on the \textit{MTHFR} 1298A > C variant gave similar results. However, the authors joined genetically distinct ethnic groups for the analysis (Korean and Indian) as “Asians”, which does not seem to be fully justified. These meta-analyses also are in accordance with the negative observations from the present study in a Polish-Caucasian population.\par There is scarce data on two other polymorphisms evaluated in the present study, \textit{i.e. MTR} and \textit{MTRR}. The \textit{MTR} 2756A > G polymorphism was not associated with male infertility in the aforementioned Korean (Lee \textit{et al.}, 2006), Russian (Weiner \textit{et al.}, 2014), as well as the Swedish studies (Murphy \textit{et al.}, 2011). Our study does support these observations, as no impact of the \textit{MTR} 2756A > G polymorphism on infertility in Polish males was found. However, the Korean study by Lee \textit{et al.} (2006) found an association between \textit{MTR} 2756GG genotype and an increased risk of azoospermia.\par Similarly to the \textit{MTHFR} polymorphism, the \textit{MTRR} 66A > G polymorphism was found to impact male infertility in the Asian population. Lee \textit{et al.} (2006) documented that the \textit{MTRR} 66GG genotype promoted development of male infertility. Contrary to this, the Russian (Weiner \textit{et al.}, 2014) and French (Ravel \textit{et al.}, 2009) studies did not support the findings from the Korean population. Likewise, data from a Middle Eastern Arabic population demonstrated that the \textit{MTRR} 66A > G genotype distribution was not different in fertile and infertile groups (Mfady \textit{et al.}, 2014). Our results from a non-Russian, Slavic population did not reveal an association between the \textit{MTRR} 66A > G polymorphism and male infertility. In conclusion, the present study did not reveal a significant association of the \textit{MTHFR}, \textit{MTR}, \textit{MTRR} gene polymorphisms with non-obstructive male infertility in a Polish population.\par Nonetheless, the observed discrepancy between the results of studies conducted in different populations may result from both genetic determinants and environmental factors, including differences in folate consumption in different regions. Reduced folate levels can result from mutations in folate pathway genes, as well as insufficient dietary intake. Folate deficiency affects spermatogenesis by producing DNA hypomethylation and resultant gene expression changes, as well as inducing uracil misincorporation in the course of DNA synthesis, and thus errors in DNA repair, strand breakage and chromosomal abnormalities (Ravel \textit{et al.}, 2009). Deficiency of folates is also related with hyperhomocysteinemia, a risk factor for male infertility (Lee \textit{et al.}, 2006). Hyperhomocysteinemia may not only result from low folate consumption, but also from genetic variants in genes of the folate pathway (Bialecka \textit{et al.}, 2012). It was also demonstrated that folate treatment improved semen parameters, such as an increase in spermatozoa number and motility, as well as total normal sperm count.
\section*{Acknowledgments}
\par The study was supported in part from National Science Centre grant (UMO-2011/01/B/NZ5/04235).

\pagebreak\onecolumn
\begin{biblio}[References]
\tit{Zhou-Cun A,} Yuan Y, Si-Zhong Z, Na L and Wei Z (2017) Single nucleotide polymorphism C677T in the methylenetetrahydrofolate reductase gene might be a genetic risk factor for infertility for Chinese men with azoospermia or severe oligozoospermia. Asian J Androl 9:57-62.
\tit{Bezold G,} Lange M and Peter RU (2001) Homozygous methylenetetrahydrofolate reductase C677T mutation and male infertility. N Engl J Med 344:1172–1173.
\tit{Bialecka M,} Kurzawski M, Roszmann A, Robowski P, Sitek EJ, Honczarenko K, Gorzkowska A, Budrewicz S, Mak M, Jarosz M, \textit{et al.} (2012) Association of COMT, MTHFR, and SLC19A1(RFC-1) polymorphisms with homocysteine blood levels and cognitive impairment in Parkinson’s disease. Pharmacogenet Genomics 22:716–724.
\tit{Camprubí C,} Pladevall M, Grossmann M, Garrido N, Pons MC and Blanco JJ (2013) Lack of association of MTHFR rs1801133 polymorphism and CTCFL mutations with sperm methylation errors in infertile patients. J Assist Reprod Genet 30:1125–1131.
\tit{Dhillon VS,} Shahid M and Husain SA (2007) Associations of MTHFR DNMT3b 4977 bp deletion in mtDNA and GSTM1 deletion, and aberrant CpG island hypermethylation of GSTM1 in non-obstructive infertility in Indian men. Mol Hum Reprod 13:213–222.
\tit{Ebisch IMW,} Van Heerde WL, Thomas CMG, Van der Put N, Wong WY and Steegers-Theunissen RPM (2003) C677T methylenetetrahydrofolate reductase polymorphism interferes with the effects of folic acid and zinc sulfate on sperm concentration. Fertil Steril 80:1190–1194.
\tit{Gava MM,} Chagas Ede O, Bianco B, Christofolini DM, Pompeo AC, Glina and Barbosa CP (2011) Methylenetetrahydrofolate reductase polymorphisms are related to male infertility in Brazilian men. Genet Test Mol Biomarkers 15:153–157.
\tit{Gupta N,} Gupta S, Dama M, David A, Khanna G, Khanna A and Rajender S (2011) Strong association of 677C>T substitution in the MTHFR gene with male infertility - A study on an Indian population and a meta-analysis. PLoS One 6:e22277.
\tit{Lee HC,} Jeong YM, Lee SH, Cha KY, Song SH, Kim NK, Lee KW and Lee S (2006) Association study of four polymorphisms in three folate-related enzyme genes with non-obstructive male infertility. Hum Reprod 21:3162–3170.
\tit{Mfady DS,} Sadiq MF, Khabour OF, Fararjeh AS, Abu-Awad A and Khader Y (2014) Associations of variants in MTHFR and MTRR genes with male infertility in the Jordanian population. Gene 536:40–44.
\tit{Montjean D,} Benkhalifa M, Dessolle L, Cohen-Bacrie P, Belloc S, Siffroi JP, Ravel C, Bashamboo A and McElreavey K (2011) Polymorphisms in MTHFR and MTRR genes associated with blood plasma homocysteine concentration and sperm counts. Fertil Steril 95:635–640.
\tit{Murphy LE,} Mills JL, Molloy AM, Qian C, Carter TC, Strevens H, Wide-Swensson D, Giwercman A and Levine RJ (2011) Folate and vitamin B12 in idiopathic male infertility. Asian J Androl 13:856–861.
\tit{Park JH,} Lee HC, Jeong YM, Chung TG, Kim HJ, Kim NK, Lee SH and Lee S (2005) MTHFR C677T polymorphism associates with unexplained infertile male factors. J Assist Reprod Genet 22:361–368.
\tit{Ravel C,} Chantot-Bastaraud S, Chalmey C, Barreiro L, Aknin-Seifer I, Pfeffer J, Berthaut I, Mathieu EE, Mandelbaum J, Siffroi JP, \textit{et al.} (2009) Lack of association between genetic polymorphisms in enzymes associated with folate metabolism and unexplained reduced sperm counts. PLoS One 4:e6540.
\tit{Shen O,} Liu R, Wu W, Yu L and Wang X (2012) Association of the methylenetetrahydrofolate reductase gene A1298C polymorphism with male infertility: A meta-analysis. Ann Hum Genet 76:25–32.
\tit{Stuppia L,} Gatta V, Scarciolla O, Colosimo A, Guanciali-Franchi P, Calabrese G and Palka G. (2003) The methylenetethrahydrofolate reductase (MTHFR) C677T polymorphism and male infertility in Italy. J Endocrinol Invest 26:620–622.
\tit{Weiner AS,} Boyarskikh UA, Voronina EN, Tupikin AE, Korolkova OV, Morozov IV and Filipenko ML (2014) Polymorphisms in folate-metabolizing genes and risk of idiopathic male infertility: a study on a Russian population and a meta-analysis. Fertil Steril 101:87–94.
\tit{Wu W,} Shen O, Qin Y, Lu J, Niu X, Zhou Z, Lu C, Xia Y, Wang S and Wang X (2012) Methylenetetrahydrofolate reductase C677T polymorphism and the risk of male infertility: A meta-analysis. Int JAndrol 35:18–24.
\end{biblio}

\medskip\par\noindent
\begin{flushright}\footnotesize{\par \textit{Associate Editor: Angela M. Vianna-Morgante}
}\end{flushright}

\medskip\par\noindent
\footnotesize{This is an open-access article distributed under the terms of the Creative Commons Attribution License, which permits unrestricted use, distribution, and reproduction in any medium, provided the original work is properly cited.}
