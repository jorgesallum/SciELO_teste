\documentclass[scielo,showtrims,times,12pt,trimframe,conselho,spreadimages]{memoir}
\usepackage[largepost]{hedra}
\usepackage{scielo}

%\newcommand{\revista}{a}

\begin{document}


%AJUSTAR
\hspace{-25mm}\noindent\includegraphics[width=.25\textwidth]{logo1.jpg}
\begin{minipage}{4cm}
\vspace{-1cm}
\tiny\revista\par\site\par
\vspace{-1ex}\rule{4cm}{1pt}
\includegraphics[width=15mm]{ccbynp.jpg}
\end{minipage}
\hspace{60mm}\includegraphics[width=.2\textwidth]{logo2.jpg}
\vspace{2cm}
%%%%%%%%%%%%%%%%%%%%%%%%%%%%%%%%%%%%%%%%%%%%%%%%%%%%%%%%%%%%%%

\begin{center}
\titulo

\medskip

\begin{bfseries}
\subtit
\end{bfseries}

\medskip

\autor
\end{center}
\medskip

\noindent 
\refbiblio
\medskip


%para mais mudanças ver página 70 do memman
\absleftindent=0pt 
\absrightindent=0pt

\begin{abstract}
A list of zooplankton species identified during ten years of studies in the lake system of
the middle Rio Doce basin is presented. This lake system integrates the Atlantic Forest biome, a
biodiversity hotspot. Three types of studies were achieved by the Brazilian Long Term Ecological
Research Program (Brasil-LTER/PELD-UFMG site 4): i) a temporal study (study 1) which sampled
four lakes monthly and three lakes twice a year during ten years; ii) a comparative study of limnetic
and littoral species composition (study 2) and iii) a spatial study (study 3) that evaluated the species
composition of eighteen lakes (eight lakes inside the Rio Doce State Park (RDSP) and ten lakes in
its surrounding area) during one year with quarterly sampling. A total of 354 taxa were identified out
of which 175 belong to the Rotifera, 95 to the Protozoa (Amoeba Testacea), 55 to Cladocera and 25
to Copepoda. Although many identified species where common in tropical environments, we present
new records for the Middle Rio Doce basin. The group of lakes outside the RDSP showed higher
exclusive species compared to lakes inside the RDSP. This pattern may be due to higher disturbance
intensity and frequency to which the lakes outside RDSP are subjected, being an important factor
affecting community structure. These aquatic ecosystems presents more than half of the zooplankton
species registered for the Minas Gerais State and is, undoubtedly, one of the Brazil’s priorities for
conservation, sustaining high diversity in a very small, limited and threatened region.
Keywords: Species list, Atlantic Forest, Freshwater.
\end{abstract}

\lipsum


\end{document}
