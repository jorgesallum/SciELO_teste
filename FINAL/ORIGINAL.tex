\selectlanguage{english}
\section*{INTRODUCTION}\par In addition to the prevention, control, and treatment of Chagas disease, quality of life
 and means of improving it is an important concern of health professionals involved in
 the care of patients affected by the disease. However, despite of this issue to be
 considered relevant in last 30 years, only in recent years that has it gained greater
 prominence in the literature and attracted more attention from researchers and those
 professionals dealing with control and treatment of Chagas disease(1)(2)(3)(4)(5)(6)(7)(8)(9)(10). \par Quality of life in any particular disease including Chagas disease is a difficult issue
 to address owing to its complexity. Quality of life can be defined as the perception of
 a state of welfare or wellbeing, which involves the interplay of several medical,
 biological, psychological, social, and cultural factors that affect directly or
 indirectly the individual in a specific society or society as a whole. \par The World Health Organization defines quality of life as, \textit{an individual's
 perception of their position in life in the context of culture and value systems in
 which they live and in relation to their personal goals, expectations, standards and
 concerns}. Accordingly to The World Health Organization Quality of Life
 (WHOQOL)-BREF(11) assessment instrument, four
 broad domains must be considered in evaluation of quality of life: physical health,
 psychological health, social relationships, and environment; although not included,
 cultural establishment can be also an important additional domain. The physical health
 domain includes aspects related to the presence of pain, discomfort, energy level,
 fatigue, sleep, and rest. The psychological domain includes issues related to positive
 or negative feelings, self-esteem, body image, and appearance. Meanwhile, the social
 domain concerns interpersonal relationships, social support, and sexual activity.
 Finally, the environmental domain considers the perception of physical safety and
 protection, home environment, financial resources, recreation, leisure, physical
 environment, transportation, and others.\par As such, an integrated approach for evaluating quality of life is a complicated task;
 usually only limited aspects of a certain domain or at most some domains are integrally
 evaluated. Within this complex conceptual framework, factors related to the demand for
 adequate healthcare are very important, and these are also factors that constitute the
 basis for the attainment of quality of life requirements in other domains, representing
 critical markers of individual good health and welfare. A review published in the
 \textit{Journal of the Brazilian Society of Tropical Medicine} addressed the
 question of health-related quality of life regarding to some clinical, functional, and
 therapeutic factors in patients with Chagas disease(10). This question about Chagas disease has sparsely been discussed in the
 literature and thus has great and broad interest. Owing to its unique and diverse
 nosological, psychological, social, cultural, and environmental peculiarities, improving
 the quality of life of Chagas disease patients is a substantial challenge, and future
 studies are expected. \section*{BACKGROUND AND CHALLENGES}\par In brief, Chagas disease is an important human infectious parasitic disease caused by
 the protozoan \textit{Trypanosoma cruzi} and transmitted by hematophagous
 triatomid bugs. It is chronically debilitating, physically and psychologically limiting,
 and incurs high morbidity and mortality. Despite the recent successful and extensive
 interruption of vectorial transmission in many countries in which the disease is
 endemic, which has greatly decreased the number of new cases, millions of previously
 infected people still chronically suffer from it in their more productive phases of
 life. \par Approximately 8-10 million people are chronically infected worldwide, the majority of
 whom live in many countries of Central and South America and have one of the chronic
 clinical forms of the disease. Most patients live in underdeveloped socioeconomic
 conditions in rural areas where healthcare availability is precarious or unavailable. In
 Brazil, Chagas disease mainly occurs in the northeastern, central-western, and southern
 regions and more recently has been noted in the Amazon region, with an estimated 2-3
 million people (approximately 2\% of the total population) having the chronic forms of
 Chagas disease(1)(2)(12)(13)(14)(15)(16).\par Furthermore, the disease is now spreading to urban areas and non-endemic regions because
 of migration, blood transfusions, organs transplantation, and the spread of insect
 vectors. Thousands infected immigrants people from Latin American countries, are living
 in the United States, Canada, Australia, Japan, and European countries. Indigenous
 chronic cases have also been reported in the southern and southwestern United States
 (16)(17), totalizing 300,000 cases in this country.
 In Europe, a total of 24,000-39,000 immigrants are estimated to be infected, 7-21\% of
 whom present with chronic Chagas cardiomyopathy(18). Therefore, Chagas disease represents a serious public health problem
 that incurs an enormous social burden. In some endemic areas, chronic Chagas heart
 disease is the most common form of heart disease and early expected death or sudden
 death. \par Chagas disease initially occurs in an acute form that can subsequently develop into one
 of chronic forms after many years. The indeterminate chronic form, which affects
 approximately 50-60\% of patients, is characterized by no clinical or functional
 manifestations detected at the primary diagnostic level; many infected people retain
 this form for life and have excellent long-term prognosis(12)(15), except for the personal and social stigma
 as well as the psychological burden of having the disease. Meanwhile, the others 40-50\%
 have the gross distinctive chronic forms characterized by evolving clinical and
 functional disturbances of the heart and/or digestive tract, resulting in consequences
 of varying severity and an estimated 6,000 deaths per year. These disturbances occur
 exclusively or in association with or without symptoms of varying degree in these forms.
 Common manifestations of the cardiac form are heart enlargement with progressive heart
 failure, many various hypokinetic and hyperkinetic arrhythmias, cerebral or pulmonary
 thromboembolism, cardiac autonomic dysfunction, and expected or unexpected sudden death.
 In the digestive form, the almost exclusive manifestations are megaesophagus and/or
 megacolon. These cardiac and digestive disturbances are those that typically
 characterize the chronic Chagas disease. The cardiac form is the responsible for the
 high morbidity and mortality, a high risk of short- and mid-term disability, and poor
 prognosis of the disease in approximately 30-40\% of affected patients. Many of these
 patients with heart disease exhibit severe progressive mechanical and/or electrical
 impairments of the heart, which can result in death from heart failure or
 life-threatening arrhythmias(12)(13)(15)(16)(19)(20). Patients with these cardiac and/or
 digestive chronic forms have more comprised quality of life in several aspects and
 require more attention. \par The peculiar disturbances of variable intensity of cardiac autonomic modulation in many
 patients are of particular interest in the context of a systemic or integrated view of
 the organism, considering the integrative action of the autonomic nervous system;
 however, the functional and clinical implications of the cardiac autonomic dysfunction
 in Chagas disease are incompletely understood(20). Indeed, considering the important roles of the autonomic nervous system in
 the control of many organic functions and in involvement in psychological and social
 tasks, dysfunction of this system may be a risk factor in predispose or trigger cardiac
 and other functional disturbances as well as psychological and social difficulties that
 can interact in a complex manner(20)(21)(22)(23). In this context, the integrity of
 parasympathetic homeostatic function is important for the individual to cope with social
 stress as well as control emotions, while the heart must provide functional adaptation
 (21)(22)(23). Therefore, when considering the quality of
 life of Chagas disease patients, it is important to be aware that they may be prone to
 psychological derangement and social maladjustment in conjunction with their autonomic
 cardiovascular adaptive disturbances(1)(4)(8)(24).\par Because of these peculiar debilitating and/or disabling characteristics that affect
 patients' social and psychological wellbeing and physical health, Chagas disease is a
 dreaded and stigmatizing disease that requires a multidisciplinary approach concerning
 the quality of life.\section*{OVERCOMING THE CHALLENGES}\par Like any other disease, the adequate control of Chagas disease requires patients to have
 access to an adequate and affordable private or public healthcare system. If so, the
 disease is likely to be eliminated or at least controlled, improving physical health and
 alleviating the related psychological and social stresses. These favorable medical and
 psychosocial changes instigate a positive feedback loop that reinforces these changes in
 direction to a progressive better global health state, breaking the vicious cycle of
 disease perpetuation; in turn, this improves the patient's acceptance of the disease,
 sense of wellbeing, social adjustment, security, and prognosis, all of which result in
 better quality of life. In contrast, if access to healthcare is difficult, tiring,
 stressful, and/or of poor quality, the opposite chain of events and consequences occurs;
 in this case, the disease tends to worsen, compromising quality of life. \par Regarding the complex biopsychosocial context of Chagas disease, which encompasses
 several aspects that can affect quality of life of patients, multidisciplinary attention
 and actions are warranted to improve it(1)(4)(6)(7)(10). For attaining this, it is firstly critical
 to deconstruct and dispel the personal, cultural, and social stigma of the disease,
 which will serve as a starting point for additional actions.\par Like other diseases, the overall control of Chagas disease is frequently affected by
 political contexts of low financial support and efficiency as well as humiliating and
 degrading public healthcare in environments where access to private care is habitually
 prohibitive to the majority of population. Most patients with Chagas disease are exposed
 to these settings. \par Despite attempts to provide equal access to healthcare services as well as expectations
 of efficiency and efficacy, our public healthcare system is also not an open system in
 which individuals can easily and freely meet their health demands; it is a restrictive
 system in which only a few fortunate individuals can receive adequate healthcare.
 However, even such fortunate people often do not receive adequate care because of the
 technical, operational, and administrative inefficiencies of the healthcare system and
 of the frequent lack of more extensive engagement by the professionals involved.
 Consequently, the current system is far from providing ideal or even acceptable
 health-related quality of life for Chagas disease patients with few exceptions. In
 places where public authority and health professionals are able to give adequate
 healthcare, diseased people can attain better quality of life as a result of prompt and
 satisfactory intervention. \par As mentioned above, eliminating the stigma of Chagas disease is important for improving
 patient quality of life. All healthcare professionals should be aware of dispelling the
 fear and stigma of disease in order to improve the quality of life of patients. Chagas
 disease, with its peculiar nosological, psychological, and social problems, poses a
 particular challenge for healthcare providers in improving the quality of life of
 patients. Since its discovery, people have believed the disease to be always fatal or
 disabling. These beliefs are not baseless, considering the large number of sudden deaths
 and common complications associated with the disease in endemic regions, particularly in
 poor rural areas with low socioeconomic status and education levels(12)(14)(15)(16)(19). Although these attitudes still exist, they
 have significantly decreased, improving patients' quality of life including clinical,
 psychological, and social aspects. Considering that patients with Chagas disease have
 distinctive medical, psychological, social and cultural characteristics, it would be
 interesting to compare the quality of life of patients regarding these differences. \par From a medical perspective, there has been great success in the control of all forms of
 parasite transmission. In addition, there have been advances in the pathophysiological
 understanding of different manifestations, diagnosis, and treatment. These advances have
 led to the interruption or delay of the disease course, reducing associated morbidity
 and mortality; in turn, this has significantly improved the prognosis and daily
 activities and quality of life of patients with Chagas disease. In particular, the
 control of arrhythmias by means of artificial pacemakers and implantable
 cardioverter-defibrillators, surgical procedures, and effective antiarrhythmic drugs
 (e.g., amiodarone and propafenone) should be highlighted(19)(25)(26). Diuretics, cardioactive and vasoactive
 therapeutic agents, renin-angiotensin-aldosterone system inhibitors, the
 alpha/beta-blocker carvedilol, spironolactone, and arterial and venous vasodilators can
 now better control heart failure(20)(27)(28). Drugs such as beta-blockers, which
 influence cardiovascular autonomic modulation by increasing parasympathetic activity and
 reducing sympathetic activity, can also benefit cardiovascular homeostasis by enhancing
 heart rate variability and hemodynamic status, resulting in marked clinical
 improvement(20)(27)(28). These therapeutic measures for the
 principal and more serious manifestations of Chagas disease-related heart disease can
 favorably influence the subjacent pathophysiological mechanisms and significantly
 relieve symptoms and enhance physical and mental disposition, work capacity,
 psychological wellbeing, and social engagement, thus improving quality of life and
 prolonging lifespan in a great number of patients. Accordingly, in the last 10-20 years,
 mortality and morbidity rates as well as the number of hospitalizations due to Chagas
 disease have decreased progressively and significantly(15)(16)(19)(20)(26). \section*{CONCLUSIONS}\par Understanding quality of life requires the detailed and fragmented analysis of its many
 domains(1)(2)(3)(4)(5)(6)(7)(8)(9)(10)(25). However, it is only possible to see the
 bigger picture of quality of life when these domains are grouped and integrated. This
 approach allows us to fully judge the state of the physical, social, and psychological
 wellbeing of Chagas disease patients from the perspective of integrated health
 expectancy. Although this is a difficult approach, it is nonetheless possible, provided
 that appropriate integrated multidisciplinary attention may be available by an adequate
 and efficient healthcare system. Existing examples of this appropriate condition include
 the praiseworthy and successful initiatives of actions in global healthcare services for
 patients with Chagas disease in conjunction with the great personal efforts of those
 engaged towards the bigger picture(1)(2)(3)(4)(6)(7)(8)(10).

{The author declare that there is no conflict of interest}
\balance



\pagebreak\onecolumn
\begin{biblio}[References]
\tit{Storino R. Aspectos Sociales,} Económicos, Políticos, Culturales y
 Psicológicos. In:, Storino R Miles J, editors. Enfermedad de Chagas. Buenos Aires:
 Mosby Doyma, Argentina; 1994. p. 541-546.
\tit{Gontijo ED,} Rocha MO, Torquato de Oliveira U. Perfil
 clínico-epidemiológico de chagásicos atendidos em ambulatório de referência e
 proposição de modelo de atenção ao chagásico na perspectiva do SUS. Rev Soc Bras Med
 Trop 1996; 29:101-108.
\tit{Guariento ME,} Camilo MV, Camargo AM. Situação trabalhista do portador de
 doença de Chagas crônica, em um grande centro urbano. Cad Saude Publica
 1999;15:381-386.
\tit{Hueb MFD,} Loureiro SR. Revisão: Aspectos cognitivos e psicossociais
 associados a doença de Chagas. Psicol Estud 2005; 10:137-142.
\tit{Dourado KC,} Bestetti RB, Cordeiro JA, Theodoropoulos TA. Assessment of
 quality of life in patients with chronic heart failure secondary to Chagas'
 cardiomyopathy. Int J Cardiol 2006; 108:412-413.
\tit{Gontijo ED,} Guimarães TN, Magnani C, Paixão GM, Dupin S, Paixão LM.
 Qualidade de vida dos portadores de doença de Chagas. Rev Med Minas Gerais
 2009;19:281-285.
\tit{Oliveira BG,} Abreu MN, Abreu CD, Rocha MO, Ribeiro AL. Health-related
 quality of life in patients with Chagas disease. Rev Soc Bras Med Trop2011;
 44:150-156.
\tit{Ozaki Y,} GuarientoME, Almeida EA. Quality of life and depressive
 symptoms in Chagas disease patients. Qual Life Res 2011; 20:133-138.
\tit{Pelegrino VM,} Dantas RA, Ciol MA, Clark AM, Rossi LA, Simoes MV.
 Health-related quality of life in Brazilian outpatients with Chagas and non-Chagas
 cardiomyopathy. Heart Lung 2011; 40:e25-e31.
\tit{Sousa GR,} Costa HS, Souza AC, Nunes MCP, Lima MMO, Rocha MOC.
 Health-related quality of life in patients with Chagas disease: a review of the
 evidence. Rev Soc Bras Med Trop2015; 48:121-128.
\tit{World }Health Organization (WHO). WHOQOL-BREF: Introduction,
 Administration, Scoring and Generic Version of the Assessment. Field Trial Version.
 Geneva: WHO; 1996. [Cited 2015 April 9]. Available at:
 http://www.who.int/en/.
\tit{Prata }A. Clinical and epidemiological aspects of Chagas' disease. Lancet
 Infect Dis 2001; 1:92-100.
\tit{World }Health Organization (WHO). On behalf of the Special Programme for
 Research and Training in Tropical Diseases. Reporte del grupo de trabajo cientifico
 sobre La enfermedad de Chagas. Geneva: WHO; 2007.
\tit{Dias JC,} Prata A, Correia D. Problems and perspectives for Chagas
 disease control: in search of a realistic analysis. Rev Soc Bras Med Trop2008;
 41:193-196.
\tit{Nunes MC,} Dones W, Morillo CA, Encina JJ,. Ribeiro AL Chagas disease: an
 overview of clinical and epidemiological aspects. J Am Coll Cardiol 2013;
 62:767-776.
\tit{Biolo A,} Ribeiro AL, Clausella N. Chagas cardiomyopathy - Where do we
 stand after a hundred years? Prog Cardiovasc Dis 2010; 52:300-316.
\tit{Bern C,} Kjos S, Yabsley MJ, Montgomery SP. Trypanosoma cruzi and Chagas'
 Disease in the United States. Clin Microbiol Rev 2011; 24:655-681.
\tit{Guerri-Guttenberg RA,} Grana DR, Ambrosio G, Milei J. Chagas
 cardiomyopathy: Europe is not spared! Eur Heart J 2008;
 29:2587-2591.
\tit{Rassi Jr A,} Rassi A, Marin-Neto JA. Chagas' heart disease:
 pathophysiologic mechanisms, prognostic factors and risk stratification. Mem Inst
 Oswaldo Cruz 2009; 104 (suppl I):152-158.
\tit{Junqueira }Jr LF. Insights into the clinical and functional significance
 of cardiac autonomic dysfunction in Chagas disease. Rev Soc Bras Med Trop2012;
 45:243-252.
\tit{Porges }SW. Cardiac vagal tone: a physiological index of stress. Neurosci
 Biobehav Rev 1995; 19:225-233.
\tit{Souza GG,} Mendonça-de-SouzaAC, Barros EM, Coutinho EF, Oliveira L,
 Mendlowicz MV, et al. Resilience and vagal tone predict cardiac recovery from acute
 social stress. Stress 2007; 10:368-374.
\tit{McEwen BS,} Gianaros PJ. Central role of the brain in stress and
 adaptation: Links to socioeconomic status, health, and disease. Ann N Y Acad Sci
 2010; 1186:190-222.
\tit{Lima-Costa MF,} Castro-Costa E, Uchôa E, Firmino J, Ribeiro AL, Ferri CP,
 et al. A population-based study of the association between Trypanosoma cruzi
 infection and cognitive impairment in old age (the Bambuí study). Neuroepidemiology
 2009; 32:122-128.
\tit{Oliveira BG,} Velasquez-Melendez G, Rincon LG, Ciconelli RM, Sousa LA,.
 Ribeiro AL Health-related quality of life in Brazilian pacemaker patients. Pacing
 Clin Electrophysiol 2008; 31:1178-1183.
\tit{Gali WL,} Sarabanda AV, Baggio JM, Ferreira LG, Gomes GG, Marin-Neto JA,
 et al. Implantable cardioverter-defibrillators for treatment of sustained ventricular
 arrhythmias in patients with Chagas' heart disease: comparison with a control group
 treated with amiodarone alone. Europace 2014; 16:674-680.
\tit{Botoni FA,} Poole-Wilson PA, Ribeiro AL, Okonko DO, Oliveira BM, Pinto
 AS, et al. Randomized trial of carvedilol after renin-angiotensin system inhibition
 in chronic Chagas cardiomyopathy. Am Heart J 2007; 153:544(e1-e8).
\tit{Bestetti RB,} Otaviano AP, Cardinally-Neto A, Rocha BF, Theodoropoulos
 TA,. Cordeiro JA Effects of B-Blockers on outcome of patients with Chagas'
 cardiomyopathy with chronic heart failure. Int J Cardiol2011;
 151:205-208.
\end{biblio}

\medskip\par\noindent
\footnotesize{This is an open-access article distributed under the terms of the Creative
 Commons Attribution License}
