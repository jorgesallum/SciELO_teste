\selectlanguage{english}
\begin{abstract}
\par Keen competition and increasingly demanding customers have forced companies to use
 their resources more efficiently and to integrate production and transportation
 planning. In the last few years more and more researchers have also focused on this
 challenging problem by trying to determine the complexity of the individual problems
 and then developing fast and robust algorithms to solve them. This paper reviews
 existing literature on integrated production and distribution decisions at the
 tactical and operational level, where the distribution part is modelled as some
 variation of the well-known Vehicle Routing Problem (VRP). The focus is thereby on
 problems that explicitly consider deliveries to multiple clients in a
 less-than-truckload fashion. In terms of the production decisions we distinguish in
 our review between tactical and operational production problems by considering
 lot-sizing/capacity allocation and scheduling models, respectively.

\end{abstract}
\section*{INTRODUCTION}1\par Since the mid-1980s integrated production-distribution systems (IPDS) have been the
 subject of an active research area in Operations Research (OR). According to Sarmiento
 \& Nagi [\textit{45}] IPDS are models that jointly
 optimize decision variables of different production and distribution functions in a
 single optimization model. The optimization is carried out simultaneously (cf. [\textit{45}]) and possibly involves additional functions,
 such as inventory (cf. [\textit{21}]). The practical needs
 and recent theoretical results regarding techniques for the integrated planning of
 production and distribution are highlighted in the specialized literature (cf. [\textit{46}], [\textit{21}]).\par One driving force behind supply chain integration is the fierce competition companies
 have to face in today's global market. As a consequence, they are forced to utilize
 their resources efficiently by, among other things, reducing lead times and safety
 stocks. This reduction of safety stocks is supposed to break up the traditional
 procedure of developing production and distribution plans separately (i.e., in different
 departments) and sequentially. Reduced inventory leads to a closer linkage between
 production and distribution operations, which makes the joint planning inevitable.\par Another reason for the necessity of an integrated consideration is the increased
 customers' pressure on companies to offer individual products quickly, as well as the
 rising number of companies which have adopted direct-sell e-business models as their way
 of doing business. Consequently, many of them have implemented make-to-order production,
 meaning that custom-made products are manufactured and delivered within very short lead
 times. Due to the fact that these companies can only start producing the products after
 they have received an order and supply them directly to the customer after their
 completion, they only have little or no inventory at all. In order to maintain a desired
 on-time delivery performance at minimum total cost the operations must be jointly
 scheduled.\par Apart from that, the integration is also indispensable in supply chains with
 time-sensitive products that have a very short life-cycle. Such products cannot be
 stored but must be supplied to the customers immediately after their production.
 Examples include perishable goods, industrial adhesive materials, or newspapers and
 mailing.\par As per Chen [\textit{22}] a joint consideration of
 production and delivery schedules is also advantageous at the operational level when
 taking higher level decisions in a supply chain. First of all, supply chain planning can
 benefit since the results of production-delivery scheduling can be used as estimates for
 input data that are needed in working out production-distribution plans. In addition,
 due date or lead time setting decisions, which have a direct effect on customer service,
 can be made more accurately if the interdependency of order due dates, production
 schedule, and delivery schedule is considered. In spite of this fact, most of the
 existing lead time setting models do not involve distribution scheduling decisions.\par Due to the closer linkage of production and distribution coordinated planning has become
 inevitable, which is supported by integrated optimization models in OR. A large body of
 literature on IPDS can be found at the strategic and tactical planning level. Strategic
 IPDS concern longterm decisions, such as facility allocation, outsourcing (cf. [\textit{32}]), plant capacities, and transport channels (cf.
 [\textit{21}]). For reviews of these models refer to
 [\textit{28}], [\textit{43}] and [\textit{49}], for obtaining an insight
 into the modeling and possible solution methods see, for instance, the works by [\textit{6}], [\textit{34}], or
 [\textit{35}]. Those at the tactical level relate to
 models dealing with production, shipping and inventory quantities as well as the
 duration of the production/distribution cycle and can be found, for instance, in [\textit{21}] and [\textit{45}].
 The recently published review by Mula et al. [\textit{42}]
 especially contains solving methods for tactical and/or operational problems as well as
 their combination with strategic ones (cf. [\textit{42}]).
 In another fairly recent review, Chen [\textit{22}] has
 dealt with integrated production and outbound distributions scheduling. Surprisingly,
 many of the reviewed approaches presented by [\textit{22}]
 use a rather simplified distribution process of direct shipments, whereas in real
 industrial applications less-than-truckload settings are far more common than
 full-truckload settings.\par The main aim of this paper is thus to provide a state-of-the-art review of those
 integrated production-distribution problems which include routing decisions. On the
 production side we will distinguish between tactical models of lot-sizing or capacity
 allocation and operational models of production scheduling. The paper's target is to
 offer an insight into the problems already considered, their interrelations, the
 solution methods used as well as blank spots identified in the research landscape, and
 directions for future research.\par The remainder of this paper is organized as follows. In the next section the methodology
 used in the literature research and the classification process is presented. Section 3
 gives an overview of the existing models. Concluding comments and directions for future
 research are given in the final section.\section*{METHODOLOGY}2\par In this chapter the methodology concerning the literature research as well as the
 classification of existing and relevant IPDS are introduced.\subsection*{Literature Research}2.1\par The search process was conducted in two steps. In the first step all the articles
 citing the reviews by Chen ([\textit{21}], [\textit{22}]) were retrieved. Next, scientific-technical
 bibliographic databases, including e-journal portals such as EBSCO, Emerald,
 ScienceDirect, and Springer Link, were searched. In the second step the references
 listed in the papers found served as a continuous search reference.\par The search terms included several versions of \textit{production-distribution
 system} , such as \textit{production distribution model, problem and
 planning,} as well as some alternatives to the term
 \textit{integrated} , for example \textit{synchronized,
 coordinated} and \textit{combined}. Moreover, the keywords
 \textit{routing, vehicle routing} and \textit{VRP} were used.
 Finally, wildcard characters and asterisks were employed to find additional
 variations. Further restrictions, such as published dates or science(s), were not
 made.\par The papers found during that search were then analyzed with respect to their content
 and papers not dealing with the actual problem solving of specific IPDS were filtered
 out. This approach led to 37 papers which are characterized by the fact that the
 problem studied is described, the optimization model as well as the solving method
 (i.e., in most of the cases the algorithms) are presented, the experiments introduced
 and the results evaluated. Each of these papers will be discussed in the main part of
 the paper below.\par Four of these papers were published either in conference proceedings or as a PhD
 thesis; for the remaining 33 papers, Table 1
 shows the distribution of references according to the journals used in the
 review.\onecolumn \noindent
\ctable[
 caption = {Distribution of References.}, 
 width=\textwidth, pos = ht, left, long
]
{p{0.33\textwidth}p{0.33\textwidth}p{0.33\textwidth}}
{
}{ \\\hline
{\textbf{Distribution of Journals}}
 & {\centering \textbf{Number}}
 & {\centering \textbf{[\%]}} \\\hline 
{Annals of Operations Research} & {\centering 1} & {\centering 3} \\\hline {Computers and Chemical Engineering} & {\centering 1} & {\centering 3} \\\hline {Computers and Industrial Engineering} & {\centering 2} & {\centering 6} \\\hline {Computers and Operations Research} & {\centering 4} & {\centering 12} \\\hline {European Journal of Operations Research} & {\centering 8} & {\centering 24} \\\hline {Flexible Services and Manufacturing} & {\centering 1} & {\centering 3} \\\hline {IIE Transactions} & {\centering 2} & {\centering 6} \\\hline {INFORMS Journal on Computing} & {\centering 1} & {\centering 3} \\\hline {International Journal of Production Economics} & {\centering 3} & {\centering 9} \\\hline {Journal of Business Logistics} & {\centering 1} & {\centering 3} \\\hline {Journal of Scheduling} & {\centering 1} & {\centering 3} \\\hline {Latin America Applied Research} & {\centering 1} & {\centering 3} \\\hline {Management Science} & {\centering 1} & {\centering 3} \\\hline {Mathematical and Computer Modeling} & {\centering 1} & {\centering 3} \\\hline {Naval Research Logistics} & {\centering 1} & {\centering 3} \\\hline {Operations Research Letters} & {\centering 1} & {\centering 3} \\\hline {Production Planning and Control} & {\centering 1} & {\centering 3} \\\hline {Transportation Science} & {\centering 2} & {\centering 6} \\\hline 
}
\twocolumn 
\par From Table 1 we observe the expected result,
 i.e., that there is a \textit{core} of journals including the
 \textit{European Journal of Operations Research} and \textit{Computers and
 Operations Research} , where a significant portion of this research is
 published. However, besides this core which does not even account for 50\% of the
 volume of research in this area there is a wide distribution of articles in journals
 ranging from engineering to computer science to business. This wide distribution
 reflects the heterogeneity and the interdisciplinarity of the research area very
 well. However, it also shows that the area has not yet matured and there is still
 much need to classify the existing approaches and the blank spots where more research
 is needed.\subsection*{Classification}2.2\par The classification process can be divided into two steps. First, the relevant 37
 publications were categorized according to the characteristics of their production
 processes. Thirteen papers deal with joint lot sizing/capacity allocation + vehicle
 routing problems, while the remaining 24 papers consider joint production scheduling
 + vehicle routing problems.\par The first problem class links medium-term production decisions (i.e., lasting from
 two weeks to six months) with the operational VRP, and comprises topics such as
 production/distribution planning, capacity and inventory allocation as well as safety
 stock planning (cf. [\textit{32}]). Typical questions
 involve the lot-sizing as well as the batching of product deliveries, the clustering
 of clients and the timing of deliveries over multiple periods (see [\textit{21}]). Hence, the main decisions of this category
 of problems can be translated as: what and how much (and where, in the case of
 multi-facility problems) must be produced, when to produce it, how much to deliver to
 the different customers and which routes to use for serving these customers. Figure 1 shows a generic structure of this
 category.
\begin{figure}[!ht]
\centering
\includegraphics[width=\maxwidth{0.5\textwidth}]{not-found.png}
\caption{A generic structure of a tactical IPDS problem.} 
\label{Figure 1} 
\end{figure} 

\par The second problem class focuses on short-term decisions (i.e., daily events) and
 considers detailed production and transportation scheduling problems (cf. [\textit{32}]). Typical questions involve when and on which
 machine a job should be processed, when and in which vehicle the products should be
 delivered and which route each vehicle should choose (see [\textit{21}]). A generic structure of this category of problems is shown
 in Figure 2.
\begin{figure}[!ht]
\centering
\includegraphics[width=\maxwidth{0.5\textwidth}]{not-found.png}
\caption{A generic structure of an operational IPDS problem.} 
\label{Figure 2} 
\end{figure} 

\par In the second step, within each class the papers found were then grouped according to
 the characteristics of the production process and the vehicles, as shown in Table 2. Vehicle characteristics correspond to
 the fleet size and its composition. On the production side different criteria are
 used for tactical and operational models. For the former, the number of products and
 the planning horizon are considered; single product models typically focus on
 multiple period settings and deal with lot-sizing decisions, while multiple product
 models relate to capacity allocation settings in a single or multiple periods. In
 terms of the latter, the scheduling environment can be characterized by a single
 machine, parallel machines or flowshop setting. Finally, Table 2 also lists the Modeling/Solution approaches that are
 utilized in the various papers.\onecolumn \noindent
\ctable[
 caption = {Classification Criteria within each Problem Class.}, 
 width=\textwidth, pos = ht, left, long
]
{p{0.33\textwidth}p{0.33\textwidth}p{0.33\textwidth}}
{
}{ \\\hline
{\textbf{Upper Criterion}}
 & {\textbf{Lower Criterion}}
 & {\textbf{Alternatives}} \\\hline 
{\centering Number of Products (Tactical models)} & {\centering } & {\centering Single; Multiple} \\\hline {\centering Time Period(s) (Tactical models)} & {\centering } & {\centering Single; Multiple} \\\hline {\centering Machine Configuration (Operational Models)} & {\centering } & {\centering Single; Parallel; Flowshop} \\\hline {\centering Vehicle Characteristics} & {\centering number of vehicles type of fleet} & {\centering One; Limited; Unlimited Homogeneous; Heterogeneous} \\\hline {\centering Modeling/Solution Approach} & {\centering } & {\centering (Mixed) Integer Linear
 Programming ((M)ILP);} \\\hline {\centering Integer Non Linear
 Programming (INLP);} \\\hline {\centering Lagrangean Relaxation
 (LR);} \\\hline {\centering Dynamic Programming
 (DP);} \\\hline {\centering Branch \& Bound
 (B\&B);} \\\hline {\centering (Meta- )Heuristics (HEU)} \\\hline 
}
\twocolumn 
\par During the analysis of each paper, the information about the problem instances used
 in each paper is also gathered. The results are summarized in Table 3 (to simplify the visualization, only papers with
 numerical experiments that provided the data used or indicated a publicly available
 data source are shown).\onecolumn \noindent
\ctable[
 caption = {Information about data sources used in IPDS research.}, 
 width=\textwidth, pos = ht, left, long
]
{p{0.50\textwidth}p{0.50\textwidth}}
{
}{ \\\hline
{\textbf{Reference}}
 & {\textbf{Data source}} \\\hline 
{Adulyasak et al. (2012) [\textit{1}]} & {} \\\hline {Boudia \& Prins (2009) [\textit{12}]} & {Modified version of: Solomon MM. Algorithms
 for the vehicle} \\\hline {Boudia et al. (2007) [\textit{13}]} & {routing and scheduling problems with time
 windows constraints.} \\\hline {Boudia et al. (2008) [\textit{14}]} & {\textit{Operations Research,} 35:
 254-265,1987.} \\\hline {Chern \& Hsieh (2007) [\textit{24}]} & {Source: https://sites.google.com/site/yossiriadulyasak/publications} \\\hline {Lei et al. (2006) [\textit{36}]} & {} \\\hline {Méndez et al. (2005) [\textit{41}]} & {Data provided by the paper} \\\hline 
}
\twocolumn 
\section*{REVIEW OF IPDS MODELS INCLUDING ROUTING DECISIONS}3\par The review consists of two parts and is organized as follows. In the first part tactical
 problems are treated. The second part of the survey is dedicated to operational problems
 dealing with joint production scheduling and vehicle routing. Overview tables
 summarizing the existing results are presented for each class.\subsection*{Tactical IPD problems}3.1\par The existing models involve the integration of lot sizing or capacity allocation
 problems with distribution problems to customers (i.e., end customers, distribution
 centers, retailers). When comparing the papers to be reviewed, the first group deals
 with an integrated production-inventory-distribution routing problem (PIDRP)
 determining the production quantities of a single product, the inventory and delivery
 quantities as well as the routes ([\textit{1}], [\textit{7}], [\textit{8}],
 [\textit{9}], [\textit{12}], [\textit{13}], [\textit{14}], and [\textit{36}]). In
 contrast to that multi-product capacity allocation problems are studied in the second
 group of papers ([\textit{5}], [\textit{15}], [\textit{17}], [\textit{20}], and [\textit{30}]). In this latter group the paper by Chen et al. [\textit{20}] stands out in that it is the only approach dealing with a
 stochastic setting.\par Strictly speaking, all of the models in the first group are based, at least to a
 certain extent, on the work by Lei et al. [\textit{36}]
 in dealing with a single-product capacitated lot-sizing problem (CLSP) in their
 production and an allocation and vehicle routing problem (VRP) in their distribution
 part. On the other hand, in the second group the model presented in [\textit{17}] describes a basic model setting which is also
 expanded in [\textit{30}]. Below, we show an exemplary
 formulation of a basic model for tactical IPD problems which combines the features of
 the models in [\textit{36}] and [\textit{17}].\par The following notation is used:\par Sets:\par \textit{H}... set of production facilities\par \textit{J}... set of customers/distribution centers (DC)\par \textit{K}... set of products\par \textit{P}... set of periods in the planning horizon\par \textit{V}... set of vehicles\par Indexes:\par \textit{h}... production facility index\par \textit{j, j \textsuperscript{' }}... customer/DC indexes\par \textit{k}... product index\par \textit{p}... period index\par ν ... vehicle index\par Parameters:\par \textit{d \textsubscript{k, j, p}}... demand of product \textit{k} at
 customer \textit{j} in period \textit{p}\par 
\documentclass[crop,varwidth]{standalone}
\usepackage{amsmath,amsfonts,amsthm} 
\usepackage{amssymb}
\usepackage{mathtools}
\usepackage{bm} % bold
\usepackage{pmml-new} % math template
\usepackage{unicode-math}
\setmathfont{XITS Math}
\begin{document}
\begin{equation*}
{{\mathop{\unicode{8899}}\limits\sb{{\minormal{{}}{\miitalic{j}}{\minormal{,}}{\miitalic{k}}{\minormal{}}}{\mo{\unicode{8712}}}{\miitalic{J}}{\mo{*}}{\mo{\unicode{215}}}{\miitalic{J}}{\mo{*}}}}{\hspace{0.33em}}{\mo{\unicode{960}}}{\minormal{(}}{\miitalic{j}}{\minormal{,}}{\hspace{0.33em}}{\miitalic{k}}{\minormal{)}}{\hspace{0.33em}}={\hspace{0.33em}}{\miitalic{I}}{\minormal{.}}}
\end{equation*}
\end{document}
 ... per-unit capacity consumption of product
 \textit{k} in production facility \textit{h}\par 

    \[\let\par\empty

    
{{{{\miitalic{a}}}\sb{{{\miitalic{h}}{\miitalic{,}}{\miitalic{k}}}}\sp{{\miitalic{setup}}}}}


    \]

   ... capacity consumption for setup of product
 \textit{k} in production facility \textit{h}\par 

    \[\let\par\empty

    
{{{{\miitalic{A}}}\sb{{\miitalic{h}}}\sp{{\miitalic{prod}}}}}


    \]

   ... total available capacity in production facility
 \textit{h}\par \textit{t \textsubscript{ν,j, j' }}... travel time of vehicle v between customer
 locations \textit{j} and \textit{j'}\par \textit{T \textsubscript{ν}}... maximum admissible travel time of vehicle ν\par \input{mathml/math4.tex} ... capacity of vehicle ν\par 

    \[\let\par\empty

    
{{{{\miitalic{I}}}\sb{{{\miitalic{k}}{\miitalic{,}}{\miitalic{h}}}}\sp{{\minormal{min}}}}}


    \]

   ... safety stock of product \textit{k} at
 production facility \textit{h}\par 

    \[\let\par\empty

    
{{{{\miitalic{I}}}\sb{{{\miitalic{k}}{\miitalic{,}}{\miitalic{j}}}}\sp{{\minormal{min}}}}}


    \]

   ... safety stock of product \textit{k} at
 customer \textit{j}\par \input{mathml/math7.tex} ... maximum inventory level of product \textit{k
 } at production facility \textit{h}\par \input{mathml/math8.tex} ... maximum inventory level of product \textit{k
 } at customer \textit{j}\par \textit{M}... a sufficiently large number\par 

    \[\let\par\empty

    
{{{{\miitalic{c}}}\sb{{{\miitalic{k}}{\miitalic{,}}{\miitalic{h}}}}\sp{{\miitalic{setup}}}}}


    \]

   ... setup cost for product \textit{k} at
 production facility \textit{h}\par \input{mathml/math10.tex} ... per-unit production cost for product \textit{k
 } at production facility \textit{h}\par 

    \[\let\par\empty

    
{{{{\miitalic{c}}}\sb{{{\miitalic{k}}{\miitalic{,}}{\miitalic{h}}}}\sp{{{\miitalic{inv}}{\mo{\unicode{8722}}}{\miitalic{prod}}}}}}


    \]

   ... per-unit holding cost for product \textit{k
 } at production facility \textit{h}\par 

    \[\let\par\empty

    
{{{{\miitalic{c}}}\sb{{{\miitalic{k}}{\miitalic{,}}{\miitalic{j}}}}\sp{{{\miitalic{inv}}{\mo{\unicode{8722}}}{\miitalic{cust}}}}}}


    \]

   ... per-unit holding cost for product \textit{k
 } at customer \textit{j}\par 

    \[\let\par\empty

    
{{{{\miitalic{c}}}\sb{{{\miitalic{v}}{\miitalic{,}}{\miitalic{h}}}}\sp{{\miitalic{vehicle}}}}}


    \]

   ... fixed cost for vehicle v located at production
 facility \textit{h}\par 

    \[\let\par\empty

    
{{{{\miitalic{c}}}\sb{{{\miitalic{h}}{\minormal{,}}{\miitalic{j}}{\minormal{,}}{\miitalic{k}}}}\sp{{\miitalic{out}}}}}


    \]

   ... per-unit cost for outsourced transportation of
 product \textit{k} delivered from production facility \textit{h} to
 customer \textit{j}\par 

    \[\let\par\empty

    
{{{{\miitalic{c}}}\sb{{{\miitalic{v}}{\minormal{,}}{\miitalic{j}}{\minormal{,}}{\miitalic{j}}{\minormal{'}}}}\sp{{\miitalic{travel}}}}}


    \]

   ... travel cost of vehicle v travelling directly from
 \textit{j} to \textit{j'}\par Variables:\par \textit{x \textsubscript{k, h, p}}... production quantity of product \textit{k
 } in production facility \textit{h} in period \textit{p}\par \textit{y \textsubscript{k, h, p}}... binary decision variable indicating whether
 or not a setup for product \textit{k} in production facility \textit{h
 } in period \textit{p} is necessary\par \textit{q \textsubscript{h, j, k, p}}... shipping quantity of product \textit{k
 } from production facility \textit{h} to customer \textit{j
 } in period \textit{p}\par \textit{Q \textsubscript{h, j, k, p}}... outsourced shipping quantity of product
 \textit{k} from production facility \textit{h} to customer
 \textit{j} in period \textit{p}\par \textit{Z \textsubscript{v,h, j, j', p}}... binary decision variable indicating
 whether or not vehicle ν, located at production facility \textit{h} travels
 directly from \textit{j} to \textit{j \textsuperscript{'}} in period
 \textit{p}\par \textit{L \textsubscript{ν,h, j, j', p}}... load of vehicle ν, located at
 production facility \textit{h} when traveling between \textit{j} and
 \textit{j'} in period \textit{p}

    \[\let\par\empty

    
{{\begin{matrix}
{\minormal{min}} \endcell {{{\mathop{\unicode{8721}}\limits\sb{{{\miitalic{k}}{\mo{\unicode{8712}}}{\miitalic{K}}}}}{{\mathop{\unicode{8721}}\limits\sb{{{\miitalic{h}}{\hspace{0.33em}}{\mo{\unicode{8712}}}{\hspace{0.33em}}{\miitalic{H}}}}}{{\mathop{\unicode{8721}}\limits\sb{{{\miitalic{p}}{\hspace{0.33em}}{\mo{\unicode{8712}}}{\hspace{0.33em}}{\miitalic{P}}}}}{{\minormal{(}}{{{\miitalic{c}}}\sb{{{\miitalic{h}}{\minormal{,}}{\hspace{0.33em}}{\miitalic{k}}}}\sp{{\miitalic{setup}}}}}}}}\unicode{183}{\hspace{0.33em}}{{{\miitalic{y}}}\sb{{{\miitalic{k}}{\minormal{,}}{\hspace{0.33em}}{\miitalic{h}}{\minormal{,}}{\hspace{0.33em}}{\miitalic{p}}{\hspace{0.33em}}{\mo{+}}{\hspace{0.33em}}{{{\miitalic{c}}}\sb{{{\miitalic{h}}{\minormal{,}}{\miitalic{k}}}}\sp{{\miitalic{prod}}}}}}}\unicode{183}{\hspace{0.33em}}{{{\miitalic{x}}}\sb{{{\miitalic{k}}{\minormal{,}}{\hspace{0.33em}}{\miitalic{h}}{\minormal{,}}{\hspace{0.33em}}{\miitalic{p}}{\hspace{0.33em}}{\mo{+}}{\hspace{0.33em}}{{{\miitalic{c}}}\sb{{{\miitalic{h}}{\minormal{,}}{\hspace{0.33em}}{\miitalic{k}}}}\sp{{{\miitalic{inv}}{\mo{\unicode{8722}}}{\miitalic{prod}}}}}}}}\unicode{183}{\hspace{0.33em}}{{{\miitalic{I}}}\sb{{{\miitalic{k}}{\minormal{,}}{\hspace{0.33em}}{\miitalic{h}}{\minormal{,}}{\hspace{0.33em}}{\miitalic{p}}}}}{\minormal{)}}}\\
\endcell {+{{\mathop{\unicode{8721}}\limits\sb{{{\miitalic{k}}{\hspace{0.33em}}{\mo{\unicode{8712}}}{\hspace{0.33em}}{\miitalic{K}}}}}{{\mathop{\unicode{8721}}\limits\sb{{{\miitalic{h}}{\hspace{0.33em}}{\mo{\unicode{8712}}}{\hspace{0.33em}}{\miitalic{H}}}}}{{\mathop{\unicode{8721}}\limits\sb{{{\miitalic{j}}{\hspace{0.33em}}{\mo{\unicode{8712}}}{\hspace{0.33em}}{\miitalic{J}}}}}{{\mathop{\unicode{8721}}\limits\sb{{{\miitalic{p}}{\hspace{0.33em}}{\mo{\unicode{8712}}}{\hspace{0.33em}}{\miitalic{P}}}}}{{{\miitalic{c}}}\sb{{{\miitalic{h}}{\minormal{,}}{\hspace{0.33em}}{\miitalic{j}}{\minormal{,}}{\hspace{0.33em}}{\miitalic{k}}}}\sp{{\miitalic{out}}}}}}}}\unicode{183}{\hspace{0.33em}}{{{\miitalic{Q}}}\sb{{{\miitalic{h}}{\minormal{,}}{\hspace{0.33em}}{\miitalic{j}}{\minormal{,}}{\hspace{0.33em}}{\miitalic{k}}{\minormal{,}}{\hspace{0.33em}}{\miitalic{p}}}}}}\\
\endcell {+{{\mathop{\unicode{8721}}\limits\sb{{{\miitalic{v}}{\hspace{0.33em}}{\mo{\unicode{8712}}}{\hspace{0.33em}}{\miitalic{V}}}}}{{\mathop{\unicode{8721}}\limits\sb{{{\miitalic{h}}{\hspace{0.33em}}{\mo{\unicode{8712}}}{\hspace{0.33em}}{\miitalic{H}}}}}{{\mathop{\unicode{8721}}\limits\sb{{{\miitalic{j}}{\hspace{0.33em}}{\mo{\unicode{8712}}}{\hspace{0.33em}}{\miitalic{J}}}}}{{\mathop{\unicode{8721}}\limits\sb{{{\miitalic{p}}{\hspace{0.33em}}{\mo{\unicode{8712}}}{\hspace{0.33em}}{\miitalic{P}}}}}{{{\miitalic{c}}}\sb{{{\miitalic{v}}{\minormal{,}}{\hspace{0.33em}}{\miitalic{h}}}}\sp{{\miitalic{vehicle}}}}}}}}\unicode{183}{\hspace{0.33em}}{{{\miitalic{z}}}\sb{{{\miitalic{v}}{\minormal{,}}{\hspace{0.33em}}{\miitalic{h}}{\hspace{0.33em}}{\minormal{,}}{\hspace{0.33em}}{\miitalic{h}}{\hspace{0.33em}}{\minormal{,}}{\hspace{0.33em}}{\miitalic{j}}{\minormal{,}}{\hspace{0.33em}}{\miitalic{p}}}}}}\\
\endcell {+{{\mathop{\unicode{8721}}\limits\sb{{{\miitalic{v}}{\hspace{0.33em}}{\mo{\unicode{8712}}}{\hspace{0.33em}}{\miitalic{V}}}}}{{\mathop{\unicode{8721}}\limits\sb{{{\miitalic{h}}{\hspace{0.33em}}{\mo{\unicode{8712}}}{\hspace{0.33em}}{\miitalic{H}}}}}{{\mathop{\unicode{8721}}\limits\sb{{{\miitalic{j}}{\hspace{0.33em}}{\mo{\unicode{8712}}}{\minormal{{}}{\miitalic{h}}{\minormal{}}}{\mo{\unicode{8899}}}{\miitalic{J}}}}}{{\mathop{\unicode{8721}}\limits\sb{{{\msup{{\miitalic{j}}}{{\mo{\unicode{8242}}}}}{\mo{\unicode{8712}}}{\minormal{{}}{\miitalic{h}}{\minormal{}}}{\mo{\unicode{8899}}}{\miitalic{J}}}}}{{\mathop{\unicode{8721}}\limits\sb{{{\miitalic{p}}{\hspace{0.33em}}{\mo{\unicode{8712}}}{\hspace{0.33em}}{\miitalic{P}}}}}{{{\miitalic{c}}}\sb{{{\miitalic{v}}{\minormal{,}}{\hspace{0.33em}}{\miitalic{j}}{\minormal{,}}{\hspace{0.33em}}{\msup{{\miitalic{j}}}{{\mo{\unicode{8242}}}}}}}\sp{{\miitalic{travel}}}}}}}}}\unicode{183}{\hspace{0.33em}}{{{\miitalic{z}}}\sb{{{\miitalic{v}}{\minormal{,}}{\hspace{0.33em}}{\miitalic{h}}{\hspace{0.33em}}{\minormal{,}}{\hspace{0.33em}}{\miitalic{j}}{\hspace{0.33em}}{\minormal{,}}{\msup{{\miitalic{j}}}{{\mo{\unicode{8242}}}}}{\hspace{0.33em}}{\minormal{,}}{\miitalic{p}}}}}}\\
\endcell {+{{\mathop{\unicode{8721}}\limits\sb{{{\miitalic{k}}{\hspace{0.33em}}{\mo{\unicode{8712}}}{\hspace{0.33em}}{\miitalic{K}}}}}{{\mathop{\unicode{8721}}\limits\sb{{{\miitalic{j}}{\hspace{0.33em}}{\mo{\unicode{8712}}}{\hspace{0.33em}}{\miitalic{J}}}}}{{\mathop{\unicode{8721}}\limits\sb{{{\miitalic{p}}{\hspace{0.33em}}{\mo{\unicode{8712}}}{\hspace{0.33em}}{\miitalic{P}}}}}{{{\miitalic{c}}}\sb{{{\miitalic{h}}{\minormal{,}}{\hspace{0.33em}}{\miitalic{k}}}}\sp{{{\miitalic{inv}}{\mo{\unicode{8722}}}{\miitalic{cust}}}}}}}}\unicode{183}{\hspace{0.33em}}{{{\miitalic{I}}}\sb{{{\miitalic{k}}{\minormal{,}}{\hspace{0.33em}}{\miitalic{j}}{\minormal{,}}{\hspace{0.33em}}{\miitalic{p}}}}}{\hspace{0.33em}}{\hspace{0.33em}}{\hspace{0.33em}}{\hspace{0.33em}}{\hspace{0.33em}}{\hspace{0.33em}}{\hspace{0.33em}}{\hspace{0.33em}}{\hspace{0.33em}}{\hspace{0.33em}}{\hspace{0.33em}}{\hspace{0.33em}}{\hspace{0.33em}}{\hspace{0.33em}}{\hspace{0.33em}}{\hspace{0.33em}}{\hspace{0.33em}}{\hspace{0.33em}}{\hspace{0.33em}}{\hspace{0.33em}}{\hspace{0.33em}}{\hspace{0.33em}}{\hspace{0.33em}}{\hspace{0.33em}}{\hspace{0.33em}}{\hspace{0.33em}}{\hspace{0.33em}}{\hspace{0.33em}}{\hspace{0.33em}}{\hspace{0.33em}}{\hspace{0.33em}}{\hspace{0.33em}}{\minormal{(}}{1}{\minormal{)}}}
\end{matrix}}}


    \]

  \par subject to

    \[\let\par\empty

    
{{{{\miitalic{I}}}\sb{{{\minormal{k}}{\minormal{,}}{\minormal{h}}{\minormal{,}}{\minormal{p}}}}}={{{\miitalic{I}}}\sb{{{\minormal{k}}{\minormal{,}}{\minormal{h}}{\minormal{,}}{\minormal{p}}{\mo{\unicode{8722}}}{1}}}}+{{{\miitalic{x}}}\sb{{{\minormal{k}}{\minormal{,}}{\minormal{h}}{\minormal{,}}{\minormal{p}}}}}\unicode{8722}{{\mathop{\unicode{8721}}\limits\sb{{{\minormal{j}}{\mo{\unicode{8714}}}{\minormal{J}}}}}{{{{\miitalic{q}}}\sb{{{\minormal{h}}{\minormal{,}}{\minormal{j}}{\minormal{,}}{\minormal{k}}{\minormal{,}}{\minormal{p}}}}}\unicode{8722}{{\mathop{\unicode{8721}}\limits\sb{{{\minormal{j}}{\mo{\unicode{8714}}}{\minormal{J}}}}}{{{\miitalic{Q}}}\sb{{{\minormal{h}}{\minormal{,}}{\minormal{j}}{\minormal{,}}{\minormal{k}}{\minormal{,}}{\minormal{p}}}}}}}}{\hspace{0.33em}}{\hspace{0.33em}}{\hspace{0.33em}}{\hspace{0.33em}}{\hspace{0.33em}}\unicode{8704}{\miitalic{h}}\unicode{8714}{\miitalic{H}}{\minormal{,}}{\miitalic{k}}\unicode{8714}{\miitalic{K}}{\minormal{,}}{\miitalic{p}}\unicode{8714}{\miitalic{P}}{\hspace{0.33em}}{\hspace{0.33em}}{\hspace{0.33em}}{\hspace{0.33em}}{\hspace{0.33em}}{{\left({2}\right)}}}


    \]

  

    \[\let\par\empty

    
{{{{\miitalic{I}}}\sb{{{\miitalic{k}}{\miitalic{,}}{\miitalic{h}}{\miitalic{,}}{\miitalic{p}}}}}{\mo{=}}{{{\miitalic{I}}}\sb{{{\miitalic{k}}{\miitalic{,}}{\miitalic{h}}{\miitalic{,}}{\miitalic{p}}{\mo{\unicode{8722}}}{1}}}}{\mo{+}}{{{\miitalic{x}}}\sb{{{\miitalic{k}}{\miitalic{,}}{\miitalic{h}}{\miitalic{,}}{\miitalic{p}}}}}+{{\mathop{\unicode{8721}}\limits\sb{{{\miitalic{h}}{\mo{\unicode{8714}}}{\miitalic{H}}}}}{{{{\miitalic{q}}}\sb{{{\miitalic{h}}{\miitalic{,}}{\miitalic{j}}{\miitalic{,}}{\miitalic{k}}{\miitalic{,}}{\miitalic{p}}}}}{\mo{+}}{{\mathop{\unicode{8721}}\limits\sb{{{\miitalic{h}}{\mo{\unicode{8714}}}{\miitalic{H}}}}}{{{\miitalic{Q}}}\sb{{{\miitalic{h}}{\miitalic{,}}{\miitalic{j}}{\miitalic{,}}{\miitalic{k}}{\miitalic{,}}{\miitalic{p}}}}}}}}{\hspace{0.33em}}{\hspace{0.33em}}{\hspace{0.33em}}{\hspace{0.33em}}{\hspace{0.33em}}{\mo{\unicode{8704}}}{\miitalic{h}}{\mo{\unicode{8714}}}{\miitalic{J}}{\miitalic{,}}{\miitalic{k}}{\mo{\unicode{8714}}}{\miitalic{K}}{\miitalic{,}}{\miitalic{p}}{\mo{\unicode{8714}}}{\miitalic{P}}{\hspace{0.33em}}{\hspace{0.33em}}{\hspace{0.33em}}{\hspace{0.33em}}{\hspace{0.33em}}{{\left({3}\right)}}}


    \]

  

    \[\let\par\empty

    
{{{{\miitalic{I}}}\sb{{{\miitalic{k}}{\minormal{,}}{\miitalic{h}}}}\sp{{\minormal{min}}}}\unicode{8804}{{{\miitalic{I}}}\sb{{{\miitalic{k}}{\minormal{,}}{\miitalic{h}}{\minormal{,}}{\miitalic{p}}}}}\unicode{8804}{{{\miitalic{I}}}\sb{{{\miitalic{k}}{\minormal{,}}{\miitalic{h}}}}\sp{{\minormal{max}}}}{\hspace{0.33em}}{\hspace{0.33em}}{\hspace{0.33em}}{\hspace{0.33em}}{\hspace{0.33em}}\unicode{8704}{\miitalic{h}}\unicode{8714}{\miitalic{H}}{\minormal{,}}{\miitalic{k}}\unicode{8714}{\miitalic{K}}{\minormal{,}}{\miitalic{p}}\unicode{8714}{\miitalic{P}}{\hspace{0.33em}}{\hspace{0.33em}}{\hspace{0.33em}}{\hspace{0.33em}}{\hspace{0.33em}}{\left({4}\right)}}


    \]

  

    \[\let\par\empty

    
{{{{\miitalic{I}}}\sb{{{\miitalic{k}}{\miitalic{,}}{\miitalic{j}}}}\sp{{\miitalic{min}}}}{\mo{\unicode{8804}}}{{{\miitalic{I}}}\sb{{{\miitalic{k}}{\miitalic{,}}{\miitalic{j}}{\miitalic{}}{\miitalic{p}}}}}{\mo{\unicode{8804}}}{{{\miitalic{I}}}\sb{{{\miitalic{k}}{\miitalic{,}}{\miitalic{j}}}}\sp{{\miitalic{max}}}}{\hspace{0.33em}}{\hspace{0.33em}}{\hspace{0.33em}}{\hspace{0.33em}}{\hspace{0.33em}}{\mo{\unicode{8704}}}{\miitalic{h}}{\mo{\unicode{8714}}}{\miitalic{H}}{\miitalic{,}}{\miitalic{k}}{\mo{\unicode{8714}}}{\miitalic{K}}{\miitalic{,}}{\miitalic{p}}{\mo{\unicode{8714}}}{\miitalic{P}}{\hspace{0.33em}}{\hspace{0.33em}}{\hspace{0.33em}}{\hspace{0.33em}}{\hspace{0.33em}}{{\left({5}\right)}}}


    \]

  

    \[\let\par\empty

    
{{{\mathop{\unicode{8721}}\limits\sb{{{\miitalic{k}}{\mo{\unicode{8714}}}{\miitalic{K}}}}}{{\left({{{{\miitalic{a}}}\sb{{{\miitalic{h}}{\minormal{,}}{\miitalic{k}}}}\sp{{\miitalic{setup}}}}\unicode{183}{{{\miitalic{y}}}\sb{{{\miitalic{k}}{\minormal{,}}{\miitalic{h}}{\minormal{,}}{\miitalic{p}}}}}+{{{\miitalic{a}}}\sb{{{\miitalic{h}}{\minormal{,}}{\miitalic{k}}}}\sp{{\miitalic{prod}}}}\unicode{183}{{{\miitalic{x}}}\sb{{{\miitalic{k}}{\minormal{,}}{\miitalic{h}}{\minormal{,}}{\miitalic{p}}}}}}\right)}\unicode{8804}{{{\miitalic{A}}}\sb{{\miitalic{h}}}\sp{{\miitalic{prod}}}}}}{\hspace{0.33em}}{\hspace{0.33em}}{\hspace{0.33em}}{\hspace{0.33em}}{\hspace{0.33em}}\unicode{8704}{\miitalic{h}}\unicode{8714}{\miitalic{H}}{\minormal{,}}{\miitalic{p}}\unicode{8714}{\miitalic{P}}{\hspace{0.33em}}{\hspace{0.33em}}{\hspace{0.33em}}{\hspace{0.33em}}{\hspace{0.33em}}{\left({6}\right)}}


    \]

  

    \[\let\par\empty

    
{{{{\miitalic{x}}}\sb{{{\miitalic{k}}{\minormal{,}}{\miitalic{h}}{\minormal{,}}{\miitalic{p}}}}}\unicode{8804}{\miitalic{M}}\unicode{183}{{{\miitalic{y}}}\sb{{{\miitalic{k}}{\minormal{,}}{\miitalic{h}}{\minormal{,}}{\miitalic{p}}}}}{\hspace{0.33em}}{\hspace{0.33em}}{\hspace{0.33em}}{\hspace{0.33em}}{\hspace{0.33em}}\unicode{8704}{\miitalic{h}}\unicode{8714}{\miitalic{H}}{\minormal{,}}{\miitalic{k}}\unicode{8714}{\miitalic{K}}{\minormal{,}}{\miitalic{p}}\unicode{8714}{\miitalic{P}}{\hspace{0.33em}}{\hspace{0.33em}}{\hspace{0.33em}}{\hspace{0.33em}}{\hspace{0.33em}}{\left({7}\right)}}


    \]

  

    \[\let\par\empty

    
{{{\mathop{\unicode{8721}}\limits\sb{{{\miitalic{j}}{\miitalic{'}}{\mo{\unicode{8714}}}{\left\{{\miitalic{h}}\right\}}{\mo{\unicode{8899}}}{\miitalic{J}}{\miitalic{,}}{\miitalic{j}}{\miitalic{'}}{\mo{\unicode{8800}}}{\miitalic{j}}}}}{{\hspace{0.33em}}{\hspace{0.33em}}{\hspace{0.33em}}{\hspace{0.33em}}{\hspace{0.33em}}{{{\miitalic{z}}}\sb{{{\miitalic{v}}{\miitalic{,}}{\miitalic{h}}{\miitalic{,}}{\miitalic{j}}{\miitalic{',}}{\miitalic{j}}{\miitalic{,}}{\miitalic{p}}}}}}}{\mo{=}}{{\mathop{\unicode{8721}}\limits\sb{{{\miitalic{j}}{\miitalic{''}}{\mo{\unicode{8714}}}{\left\{{\miitalic{h}}\right\}}{\mo{\unicode{8899}}}{\miitalic{J}}{\miitalic{,}}{\miitalic{j}}{\miitalic{''}}{\mo{\unicode{8800}}}{\miitalic{j}}}}}{{\hspace{0.33em}}{\hspace{0.33em}}{\hspace{0.33em}}{\hspace{0.33em}}{\hspace{0.33em}}{{{\miitalic{z}}}\sb{{{\miitalic{v}}{\miitalic{,}}{\miitalic{h}}{\miitalic{,}}{\miitalic{j}}{\miitalic{'',}}{\miitalic{j}}{\miitalic{,}}{\miitalic{p}}}}}}}{\hspace{0.33em}}{\hspace{0.33em}}{\hspace{0.33em}}{\hspace{0.33em}}{\hspace{0.33em}}{\mo{\unicode{8704}}}{\miitalic{h}}{\mo{\unicode{8714}}}{\miitalic{H}}{\miitalic{,}}{\miitalic{j}}{\mo{\unicode{8714}}}{\miitalic{J}}{\miitalic{,}}{\miitalic{v}}{\mo{\unicode{8714}}}{\miitalic{V}}{\miitalic{,}}{\miitalic{p}}{\mo{\unicode{8714}}}{\miitalic{P}}{\hspace{0.33em}}{\hspace{0.33em}}{\hspace{0.33em}}{\hspace{0.33em}}{\hspace{0.33em}}{\left({8}\right)}}


    \]

  

    \[\let\par\empty

    
{{{\mathop{\unicode{8721}}\limits\sb{{{\minormal{j}}{\mo{\unicode{8714}}}{\minormal{J}}}}}{{{{\minormal{z}}}\sb{{{\minormal{v}}{\minormal{,}}{\minormal{h}}{\minormal{,}}{\minormal{h}}{\minormal{,}}{\minormal{j}}{\minormal{,}}{\minormal{p}}}}}\unicode{8804}{1}}}{\hspace{0.33em}}{\hspace{0.33em}}{\hspace{0.33em}}{\hspace{0.33em}}{\hspace{0.33em}}{\mo{\unicode{8704}}}{\minormal{h}}{\mo{\unicode{8714}}}{\minormal{H}}{\minormal{,}}{\minormal{v}}{\mo{\unicode{8714}}}{\minormal{V}}{\minormal{,}}{\minormal{p}}{\mo{\unicode{8714}}}{\minormal{P}}{\hspace{0.33em}}{\hspace{0.33em}}{\hspace{0.33em}}{\hspace{0.33em}}{\hspace{0.33em}}{\left({9}\right)}}


    \]

  

    \[\let\par\empty

    
{{{\mathop{\unicode{8721}}\limits\sb{{{\miitalic{j}}{\mo{\unicode{8714}}}{\left\{{\miitalic{h}}\right\}}{\mo{\unicode{8899}}}{\miitalic{J}}}}}{{{\mathop{\unicode{8721}}\limits\sb{{{\miitalic{j}}{\miitalic{'}}{\mo{\unicode{8714}}}{\left\{{\miitalic{h}}\right\}}{\mo{\unicode{8899}}}{\miitalic{J}}{\miitalic{,}}{\miitalic{j}}{\miitalic{'}}{\mo{\unicode{8800}}}{\miitalic{j}}}}}}{{{\miitalic{t}}}\sb{{{\miitalic{v}}{\miitalic{,}}{\miitalic{j}}{\miitalic{,}}{\miitalic{j}}{\miitalic{'}}}}}{\mo{\unicode{183}}}{{{\miitalic{z}}}\sb{{{\miitalic{v}}{\miitalic{,}}{\miitalic{h}}{\miitalic{,}}{\miitalic{j}}{\miitalic{,}}{\miitalic{j}}{\miitalic{'}}{\miitalic{p}}}}}{\mo{\unicode{8804}}}{{{\miitalic{T}}}\sb{{\miitalic{v}}}}}}{\hspace{0.33em}}{\hspace{0.33em}}{\hspace{0.33em}}{\hspace{0.33em}}{\hspace{0.33em}}{\mo{\unicode{8704}}}{\miitalic{h}}{\mo{\unicode{8714}}}{\miitalic{V}}{\miitalic{,}}{\miitalic{p}}{\mo{\unicode{8714}}}{\miitalic{P}}{\hspace{0.33em}}{\hspace{0.33em}}{\hspace{0.33em}}{\hspace{0.33em}}{\hspace{0.33em}}{\left({10}\right)}}


    \]

  

    \[\let\par\empty

    
{{{{\miitalic{L}}}\sb{{{\miitalic{v}}{\miitalic{,}}{\miitalic{h}}{\miitalic{,}}{\miitalic{j}}{\miitalic{,}}{\miitalic{j}}{\miitalic{',}}{\miitalic{p}}}}}{\mo{\unicode{8804}}}{{{\miitalic{A}}}\sb{{\miitalic{v}}}\sp{{\miitalic{vehicle}}}}\unicode{183}{{{\miitalic{z}}}\sb{{{\miitalic{v}}{\miitalic{,}}{\miitalic{h}}{\miitalic{,}}{\miitalic{j}}{\miitalic{,}}{\miitalic{j}}{\miitalic{',}}{\miitalic{p}}}}}{\hspace{0.33em}}{\hspace{0.33em}}{\hspace{0.33em}}{\hspace{0.33em}}{\hspace{0.33em}}{\mo{\unicode{8704}}}{\miitalic{h}}{\mo{\unicode{8714}}}{\miitalic{H}}{\miitalic{,}}{\miitalic{j}}{\miitalic{,}}{\miitalic{j}}{\miitalic{'}}{\mo{\unicode{8714}}}{\miitalic{J}}{\miitalic{,}}{\miitalic{v}}{\mo{\unicode{8714}}}{\miitalic{V}}{\miitalic{,}}{\miitalic{p}}{\mo{\unicode{8714}}}{\miitalic{P}}{\hspace{0.33em}}{\hspace{0.33em}}{\hspace{0.33em}}{\hspace{0.33em}}{\hspace{0.33em}}{\left({11}\right)}}


    \]

  

    \[\let\par\empty

    
{={\hspace{0.33em}}{{\mathop{\unicode{8721}}\limits\sb{{{\miitalic{k}}{\mo{\unicode{8712}}}{\miitalic{K}}}}}{{{\miitalic{q}}}\sb{{{\miitalic{h}}{\minormal{,}}{\hspace{0.33em}}{\miitalic{j}}{\minormal{,}}{\hspace{0.33em}}{\miitalic{k}}{\minormal{,}}{\hspace{0.33em}}{\miitalic{p}}}}}}{\hspace{0.33em}}{\hspace{0.33em}}{\hspace{0.33em}}{\hspace{0.33em}}{\hspace{0.33em}}\unicode{8704}{\miitalic{h}}\unicode{8712}{\miitalic{H}}{\minormal{,}}{\hspace{0.33em}}{\miitalic{j}}{\hspace{0.33em}}\unicode{8712}{\hspace{0.33em}}{\miitalic{J}}{\minormal{,}}{\hspace{0.33em}}{\miitalic{v}}{\hspace{0.33em}}\unicode{8712}{\hspace{0.33em}}{\miitalic{V}}{\minormal{,}}{\hspace{0.33em}}{\miitalic{p}}{\hspace{0.33em}}\unicode{8712}{\miitalic{P}}{\hspace{0.33em}}{\hspace{0.33em}}{\hspace{0.33em}}{\hspace{0.33em}}{\hspace{0.33em}}{\minormal{(}}{12}{\minormal{)}}}


    \]

  

    \[\let\par\empty

    
{{{\mathop{\unicode{8721}}\limits\sb{{{\miitalic{j}}{\mo{\unicode{8712}}}{\hspace{0.33em}}{\miitalic{J}}}}}{{{{\miitalic{L}}}\sb{{{\miitalic{v}}{\minormal{,}}{\hspace{0.33em}}{\miitalic{h}}{\minormal{,}}{\hspace{0.33em}}{\miitalic{j}}{\minormal{,}}{\hspace{0.33em}}{\miitalic{h}}{\minormal{,}}{\hspace{0.33em}}{\miitalic{p}}}}}{\hspace{0.33em}}\unicode{8722}{\hspace{0.33em}}{{\mathop{\unicode{8721}}\limits\sb{{{\msup{{\miitalic{j}}}{{\mo{\unicode{8242}}}}}{\hspace{0.33em}}{\mo{\unicode{8712}}}{\miitalic{J}}}}}{{{\miitalic{L}}}\sb{{{\miitalic{v}}{\minormal{,}}{\hspace{0.33em}}{\miitalic{h}}{\minormal{,}}{\hspace{0.33em}}{\miitalic{h}}{\minormal{,}}{\hspace{0.33em}}{\msup{{\miitalic{j}}}{{\mo{\unicode{8242}}}}}{\minormal{,}}{\hspace{0.33em}}{\miitalic{p}}}}}}}}={{\mathop{\unicode{8721}}\limits\sb{{{\miitalic{k}}{\mo{\unicode{8712}}}{\miitalic{K}}}}}{{\mathop{\unicode{8721}}\limits\sb{{{\msup{{\miitalic{j}}}{{\mo{\unicode{8243}}}}}{\mo{\unicode{8712}}}{\miitalic{J}}}}}{{{\miitalic{q}}}\sb{{{\miitalic{h}}{\minormal{,}}{\hspace{0.33em}}{\msup{{\miitalic{j}}}{{\mo{\unicode{8243}}}}}{\minormal{,}}{\hspace{0.33em}}{\miitalic{k}}{\minormal{,}}{\hspace{0.33em}}{\miitalic{p}}}}}}}{\hspace{0.33em}}{\hspace{0.33em}}{\hspace{0.33em}}{\hspace{0.33em}}{\hspace{0.33em}}\unicode{8704}{\miitalic{h}}\unicode{8712}{\miitalic{H}}{\minormal{,}}{\hspace{0.33em}}{\miitalic{v}}\unicode{8712}{\miitalic{V}}{\minormal{,}}{\hspace{0.33em}}{\miitalic{p}}\unicode{8712}{\miitalic{P}}{\hspace{0.33em}}{\hspace{0.33em}}{\hspace{0.33em}}{\hspace{0.33em}}{\hspace{0.33em}}{\minormal{(}}{13}{\minormal{)}}}


    \]

  

    \[\let\par\empty

    
{{{{\miitalic{x}}}\sb{{{\miitalic{k}}{\minormal{,}}{\hspace{0.33em}}{\miitalic{h}}{\minormal{,}}{\hspace{0.33em}}{\miitalic{p}}}}}\unicode{8805}{\hspace{0.33em}}{0}{\minormal{,}}{\hspace{0.33em}}{{{\miitalic{y}}}\sb{{{\miitalic{k}}{\minormal{,}}{\hspace{0.33em}}{\miitalic{h}}{\minormal{,}}{\hspace{0.33em}}{\miitalic{p}}}}}\unicode{8722}\unicode{8712}{\minormal{{}}{0}{\minormal{,}}{1}{\minormal{}}}{\hspace{0.33em}}{\hspace{0.33em}}{\hspace{0.33em}}{\hspace{0.33em}}{\hspace{0.33em}}\unicode{8704}{\miitalic{h}}\unicode{8712}{\miitalic{H}}{\minormal{,}}{\hspace{0.33em}}{\miitalic{k}}\unicode{8712}{\miitalic{K}}{\minormal{,}}{\hspace{0.33em}}{\miitalic{p}}\unicode{8712}{\miitalic{P}}{\hspace{0.33em}}{\hspace{0.33em}}{\hspace{0.33em}}{\hspace{0.33em}}{\hspace{0.33em}}{\minormal{(}}{14}{\minormal{)}}}


    \]

  

    \[\let\par\empty

    
{{{{\miitalic{q}}}\sb{{{\miitalic{h}}{\minormal{,}}{\hspace{0.33em}}{\miitalic{j}}{\minormal{,}}{\hspace{0.33em}}{\miitalic{k}}{\minormal{,}}{\hspace{0.33em}}{\miitalic{p}}}}}\unicode{8805}{0}{\minormal{,}}{\hspace{0.33em}}{{{\miitalic{Q}}}\sb{{{\miitalic{h}}{\minormal{,}}{\hspace{0.33em}}{\miitalic{j}}{\minormal{,}}{\hspace{0.33em}}{\miitalic{k}}{\minormal{,}}{\hspace{0.33em}}{\miitalic{p}}}}}\unicode{8805}{0}{\hspace{0.33em}}{\hspace{0.33em}}{\hspace{0.33em}}{\hspace{0.33em}}{\hspace{0.33em}}\unicode{8704}{\miitalic{h}}\unicode{8712}{\miitalic{H}}{\minormal{,}}{\hspace{0.33em}}{\miitalic{j}}\unicode{8712}{\miitalic{J}}{\minormal{,}}{\hspace{0.33em}}{\miitalic{k}}\unicode{8712}{\miitalic{K}}{\minormal{,}}{\hspace{0.33em}}{\miitalic{p}}\unicode{8712}{\miitalic{P}}{\hspace{0.33em}}{\hspace{0.33em}}{\hspace{0.33em}}{\hspace{0.33em}}{\hspace{0.33em}}{\minormal{(}}{15}{\minormal{)}}}


    \]

  

    \[\let\par\empty

    
{{{{\miitalic{z}}}\sb{{{\miitalic{v}}{\minormal{,}}{\hspace{0.33em}}{\miitalic{h}}{\minormal{,}}{\hspace{0.33em}}{\miitalic{j}}{\minormal{,}}{\hspace{0.33em}}{\msup{{\miitalic{j}}}{{\mo{\unicode{8242}}}}}{\minormal{,}}{\hspace{0.33em}}{\miitalic{p}}}}}\unicode{8712}{\hspace{0.33em}}{\minormal{{}}{0}{\minormal{,}}{\hspace{0.33em}}{1}{\minormal{},}}{\hspace{0.33em}}{{{\miitalic{L}}}\sb{{{\miitalic{v}}{\minormal{,}}{\hspace{0.33em}}{\miitalic{h}}{\minormal{,}}{\hspace{0.33em}}{\miitalic{j}}{\minormal{,}}{\hspace{0.33em}}{\msup{{\miitalic{j}}}{{\mo{\unicode{8242}}}}}{\minormal{,}}{\hspace{0.33em}}{\miitalic{p}}}}}\unicode{8805}{0}{\hspace{0.33em}}{\hspace{0.33em}}{\hspace{0.33em}}{\hspace{0.33em}}{\hspace{0.33em}}\unicode{8704}{\miitalic{h}}\unicode{8712}{\miitalic{H}}{\minormal{,}}{\hspace{0.33em}}{\miitalic{j}}{\minormal{,}}{\hspace{0.33em}}{\msup{{\miitalic{j}}}{{\mo{\unicode{8242}}}}}{\hspace{0.33em}}\unicode{8712}{\hspace{0.33em}}{\miitalic{J}}{\minormal{,}}{\hspace{0.33em}}{\miitalic{v}}{\hspace{0.33em}}\unicode{8712}{\hspace{0.33em}}{\miitalic{V}}{\minormal{,}}{\hspace{0.33em}}{\miitalic{p}}{\hspace{0.33em}}\unicode{8712}{\miitalic{P}}{\hspace{0.33em}}{\hspace{0.33em}}{\hspace{0.33em}}{\hspace{0.33em}}{\hspace{0.33em}}{\minormal{(}}{16}{\minormal{)}}}


    \]

  \par Objective (1) minimizes the total
 cost consisting of setup, production and inventory cost at the production facilities
 (first line), cost for outsourced transportation (second line), fixed and variable
 cost for in-house operated transportation (third line) and inventory cost at the
 customers (fourth line).\par Constraints (2) and (3) are the inventory balance constraints
 at the production facilities and customers, respectively. Constraints (4) and (5) impose safety stock and maximum inventory levels at the
 production facilities and customers, respectively. Constraints (6) make sure that the available capacity
 at each production facility is not violated. The fact that production of a product
 can only take place when the facility is set up for this product is modelled by
 constraints (7). Constraints (8) are the flow conservation
 constraints, implying that a vehicle entering a customer node also needs to leave
 this customer node. Constraints (9)
 account for the utilization of vehicles by modelling the first trip of a vehicle
 leaving its production facility. Constraints (10) make sure that the maximum admissible travel time of any
 vehicle is not violated. Constraints (11)-(13) account for the
 vehicle capacity and impose that there can not be any subtours among customer nodes
 not connected to any production facility. Finally, constraints (14)-(16) set the domains of the decision variables.\par As mentioned above, this model combines features of the models presented in [\textit{36}] and [\textit{17}]. While all of the single product models in Table 4 are based on the model by Lei et al. [\textit{36}] they also feature some important differences.
 Multiple plants and heterogeneous vehicles are only considered by Lei et al. [\textit{36}], setup costs are included in all papers
 except for [\textit{36}], and different solution
 approaches are presented. In short, a two-phase methodology is proposed in [\textit{36}], which means that the routes are determined
 separately in the second phase of the approach. The model presented and solved in
 [\textit{13}] has also been used for the solution
 methodologies proposed in [\textit{1}] (adaptive large
 neighborhood search), [\textit{12}] (reactive greedy
 randomized adaptive search procedure) and [\textit{14}]
 (memetic algorithm with population management). On the basis of [\textit{13}] and the two-phase solution approach of [\textit{36}] Bard \& Nananukul developed a reactive
 tabu search procedure in [\textit{7}]. The idea of
 coupling heuristics with decomposition methods and branch-and-price respectively to
 efficiently solve the PIDRP is presented in [\textit{8}] and [\textit{9}]. More details about these
 paper are now presented.\onecolumn \noindent
\ctable[
 caption = {Tactical IPDS Problems.}, 
 width=\textwidth, pos = ht, left, long
]
{p{0.16\textwidth}p{0.16\textwidth}p{0.16\textwidth}p{0.16\textwidth}p{0.16\textwidth}p{0.16\textwidth}}
{
}{ \\\hline
{\textbf{Number of Products}}
 & {\textbf{Time Period(s)}}
 & \multicolumn{2}{p{0.16\textwidth}}{\textbf{Vehicle Characteristics}}
 & {\textbf{Modelling Approach}}
 & {\textbf{References}} \\\hline 
{\textbf{Number}}
 & {\textbf{Type}} \\\hline 
{\centering Single} & {\centering Multiple} & {\centering Limited} & {\centering Homogeneous} & {\centering MIP, HEU} & {\centering \textit{Bard \& Nananukul
 (2009a)}} \\\hline {\centering } & {\centering } & {\centering } & {\centering } & {\centering MIP, HEU} & {\centering \textit{Bard \& Nananukul
 (2009b)}} \\\hline {\centering } & {\centering } & {\centering } & {\centering } & {\centering MIP, HEU} & {\centering \textit{Bard \& Nananukul
 (2010)}} \\\hline {\centering } & {\centering } & {\centering } & {\centering } & {\centering ILP, HEU} & {\centering \textit{Boudia et al.
 (2007)}} \\\hline {\centering } & {\centering } & {\centering } & {\centering } & {\centering ILP, HEU} & {\centering \textit{Boudia et al.
 (2008)}} \\\hline {\centering } & {\centering } & {\centering } & {\centering } & {\centering ILP, HEU} & {\centering \textit{Adulyasak et al.
 (2012)}} \\\hline {\centering } & {\centering } & {\centering } & {\centering Heterogeneous} & {\centering MIP, HEU} & {\centering \textit{Lei et al.
 (2006)}} \\\hline {\centering } & {\centering } & {\centering } & {\centering } & {\centering HEU} & {\centering \textit{Boudia \& Prins
 (2009)}} \\\hline {\centering Multiple} & {\centering Single} & {\centering Limited} & {\centering Heterogeneous} & {\centering MIP} & {\centering \textit{Aydinel et al.
 (2008)}} \\\hline {\centering } & {\centering } & {\centering Limited} & {\centering Homogeneous} & {\centering INLP, HEU} & {\centering \textit{Chen et al.
 (2009)}} \\\hline {\centering } & {\centering Multiple} & {\centering Limited,} & {\centering } & {\centering MIP, HEU} & {\centering \textit{Bredström \& Ronnqvist
 (2002)}} \\\hline {\centering } & {\centering } & {\centering Unlimited} & {\centering } & {\centering } & {\centering } \\\hline {\centering } & {\centering } & {\centering Unlimited} & {\centering } & {\centering MIP, HEU} & {\centering \textit{Chandra \& Fisher
 (1994)}} \\\hline {\centering } & {\centering } & {\centering Limited} & {\centering } & {\centering MIP, LR} & {\centering \textit{Fumero \& Vercellis
 (1999)}} \\\hline 
}
\twocolumn 
\par Lei et al. [\textit{36}] consider a two-stage PIDRP
 involving multiple production plants and customer demand centers, which both have
 limited inventory capacities. The production part of the model is characterized by a
 single-item, single-level CLSP without backlogging and is solved in the first phase
 of a two-phase solution approach. By contrast, the dissolving of the distribution
 part comprises both phases. At first, a transportation problem formulated as
 mixed-integer programming (MIP) model is solved by determining the optimal delivery
 quantities and trips per transporter. In order to find optimal routes, a delivery
 consolidation problem formulated in a similar way to the CVRP with the multiple use
 of vehicles is solved in the second phase. The solution is found by a heuristic
 transporter routing algorithm based on an extended optimal partitioning
 procedure.\par Heterogeneous in-house as well as chartered transporters, which can be used for
 several trips per period, and their traveling times (independent of the quantities
 delivered) are also incorporated in the model. The objective function minimizes
 production costs, inventory holding costs of the plants and demand centers, and total
 transportation costs. The computational performance of the solution approach was
 tested using 49 randomly generated test problems and a real-life supply network
 problem of a chemical company shipping products by water transportation. The
 solutions of the proposed approach were compared to those obtained by the MIP CPLEX
 solver. It has been shown that in 34 out of 49 cases CPLEX solver either could not
 find a feasible solution within the given time limit or produced a solution of worse
 quality than the proposed approach. With respect to the application, 2 plants, 13
 demand centres, 3 types of vessel, and 12 time periods were considered. Interplant
 distribution for the shipment of raw materials and maintenance schedules were also
 included.\par Boudia et al. [\textit{13}] and Boudia \& Prins
 [\textit{12}] developed similar MIP models for the
 problem described above but considered a single-plant case. Also, setup costs are
 included in the production part of the models. The distribution parts deals with the
 CVRP assuming a homogeneous fleet. The aim of the models is the minimization of the
 sum of setup, inventory holding and transportation costs. They propose a reactive
 greedy randomized adaptive search procedure (GRASP) in [\textit{13}] and a memetic algorithm with population management and
 path-relinking in [\textit{12}]. In the former work the
 problem is formulated as an integer linear program and solved by a GRASP
 metaheuristic which was either improved by a reactive mechanism or a path-relinking
 process. The algorithm was tested on 90 randomly generated instances with 50, 100, or
 200 customers and 20 time periods. Four metaheuristics (basic GRASP, reactive
 version, interleaved path relinking, and path relinking at the end) were compared
 with two earlier heuristics, namely a two-phase decoupled approach and a weakly
 coupled approach (also considered in [\textit{14}]). It
 has been shown that the best results are obtained by using GRASP but higher savings
 are achievable when using the reactive and path relinking version. In [\textit{12}] several versions of a memetic algorithm with
 population management were considered and compared with the results obtained either
 by using GRASP or the sequential two-phase heuristic from [\textit{13}]. The tests have shown that large-scale problems can be solved
 within a reasonable time and significant savings are achievable.\par In [\textit{1}] a deterministic single product, single
 plant, multiple customer problem with limited storage capacity at both the plant and
 the customers is considered. An adaptive large neighborhood search algorithm (ALNS)
 is proposed which is based on a problem decomposition approach. For each 200-client
 instance, different setup schedules are generated for which production quantities and
 delivery schedules are determined. Each of the solutions (with a different setup
 schedule) is then improved using the ALNS which considers customer-period
 combinations and tries to reschedule the shipment to a customer and the routing
 associated with that shipment. Computational results are presented with regard to the
 performance relative to some competing algorithms while no analysis is done
 concerning the integration value of the decisions.\par The production part of the model presented in [\textit{7}] is characterized by a single-plant, single-item CLSP including setup
 costs and the distribution part comprises a transportation problem and a CVRP
 assuming a homogeneous fleet that deliveries to 50, 100 or 200 clients. A similar
 solution approach to the one by Lei et al. [\textit{36}] is presented and tested with three data sets provided by Boudia et al.
 [\textit{13}]. The objective function minimizes the
 sum of production setup costs, a surrogate for the routing costs (fixed and variable
 costs) as well as inventory holding costs at the plant and customer sites. The
 researchers propose a reactive tabu search algorithm, using a dynamic tabu list, that
 comprises two-phases, of which the first is divided into two parts. Part one deals
 with determining the production and delivery quantities on each day (formulated as a
 MIP) and part two with finding solutions to the routing problem. In order to achieve
 these objectives the values obtained in the first part are used as demand data for
 the CVRP subroutine (based on the tabu search technique) treated in the second part.
 A neighborhood search is performed in the second phase of the solution approach to
 improve the current results. For quality reasons, lower bounds are also determined by
 solving a modified version of the lot-sizing distribution model (phase one).
 Additionally, path-relinking is used in a post-processing phase to achieve marginal
 cost reductions. It has been shown that, compared to the results obtained in [\textit{13}] (by using GRASP), improvements could be
 achieved but the run time was three to five times longer. The researchers have also
 concluded that path relinking is not very effective.\par In [\textit{8}] Bard \& Nananukul present the
 results of coupling heuristics with decomposition methods to find solutions to the
 50, 100 and 200 clients-size PIDRP instances examined in [\textit{7}]. For the computations, a previously developed branch-and-price
 (B\&P) algorithm is used that requires the solution of multiple inventory routing
 problems to generate columns for the master problem in each period. In order to
 improve the results, the researchers developed three heuristics for solving the
 inventory-routing problem component along with a model for determining periods in
 which at least one customer requires a delivery. Several experiments were carried out
 to evaluate the performances of the proposed solution approaches and to find out the
 most effective algorithm configuration. Computational results obtained by the usage
 of heuristics with different column generation strategies (adding one or multiple
 columns in each iteration) at the root node of the B\&P search tree as well as
 incorporating them into the B\&P algorithm are provided. Results are also given
 for solving the problem using CPLEX and the tabu search proposed in [\textit{7}] or the exact B\&P algorithm.\par Bard \& Nananukul's recently published paper [\textit{9}] provides another similar solution approach for the problem studied in
 [\textit{8}]. Methodological contributions of the
 proposed decomposition algorithm based on B\&P comprise a new branching rule to
 deal with the degeneracy characteristics of the master problem and a new approach for
 handling symmetry. Apart from that, a column generation and rounding heuristic were
 combined to improve the algorithm's efficiency. Extensive testing was carried out to
 compare the results achieved by the exact B\&P algorithm with those when using the
 suggested B\&P heuristic with different features. It has been observed that for
 instances with up to 50 customers and eight time periods the latter approach obtained
 high quality solutions within one hour and outperfromed both CPLEX and the exact
 B\&P algorithm alone.\par Differing from the models described above, multi-product problems have been studied
 in [\textit{5}], [\textit{15}], [\textit{17}], [\textit{20}], and [\textit{30}]. In contrast
 to the last three papers, the first two papers deal with real-life problems, multiple
 production facilities, deterministic demands, and multi-period planning horizon.
 Besides, they both do not consider delivery time windows. However, they differ in the
 extent of the production part studied. In [\textit{5}]
 only the allocation of up to 500 orders in 16 shipments is considered, while
 sequencing considerations are included in [\textit{15}]. In contrast to the other models, the paper by Chen et al. [\textit{20}] is the only one that deals with stochastic
 assumptions, a single time period, and delivery time windows. The models and solution
 methodologies are described in more detail below.\par Chandra \& Fisher [\textit{17}] consider a 2-stage,
 multi-product problem with a single production facility and multiple customers.
 Demand for each product in each period is deterministic and has to be satisfied
 without backlog. There is a setup cost for producing a product in each period.
 Inventory is allowed at both the plant and the customers. Transportation cost
 consists of a fixed part and a variable part which is determined by the routes of the
 vehicles, and hence vehicle routing is one of the decisions of the problem. The
 problem is formulated as a MIP and the authors compare sequential (first production,
 then transportation) and integrated approaches to instance sets of up to 10 products
 and 50 clients. They make the following observations based on computational tests on
 randomly generated data sets on various parameters: (i) Value of
 production-distribution coordination (measured as average cost reduction achieved by
 the coordination) increases with production capacity, vehicle capacity, number of
 customers, number of products, and number of time periods; (ii) value of coordination
 increases with relatively high distribution costs (fixed and variable) compared to
 production cost; and (iii) cost reduction varies from 3\% to 20\%. In some cases (e.g.
 when vehicle capacity is small), there is no value in coordination because in this
 case all deliveries will be made as full truckloads and hence no consolidation is
 necessary.\par A similar problem is studied by Fumero \& Vercellis ([\textit{30}]). They additionally assume that there is a limited number of
 vehicles available for product delivery in each time period. Based on a different MIP
 formulation they solve the problem using Langrangean relaxation. Also, by comparing
 coordination with the sequential production-distribution approach they obtain similar
 results to [\textit{17}].\par Aydinel et al. [\textit{5}] consider a real-life
 multi-plant problem of a forest products company which receives orders from multiple
 customers and ships their products either by train or a truck-railcar combination.
 Two mixed integer models differing in the assumptions concerning the distribution
 part are presented. The production part is given by a multi-plant order allocation
 problem and does not include any lot-sizing in the strict sense. In both models the
 distribution part comprises the determination of the carriers, the shipment sizes,
 and the routes. However, the supplier strictly chooses one of the transportation
 modes (open-mode model). The models were solved by CPLEX and tested using real order
 files of the company comprising two weeks, of which 309 railcar and 16 truck
 shipments were considered in the first week followed by 208 railcar and 21 truck
 shipments in the second. The results were compared to those of the company's current
 approach and revealed that cost savings ranging from 1.9\% to 2.4\% could be
 achieved.\par In [\textit{15}] Bredström \& Rönnqvist study a
 real-life problem of a Swedish paper mill company consisting of a three-stage supply
 chain: suppliers of raw materials (logs and wood chips) or imported materials,
 production sites (mills), and customers (domestic and foreign ones). The researchers
 provide two mixed-integer-linear-programming (MILP) models, one for the production
 and one for the distribution problem, but do not actually solve them in an integrated
 manner. The production plan problem is divided into two subproblems determining the
 sequence and the products to be produced and the recipe to be used in each time
 period. The distribution plan model comprises a multi-period, multi-commodity
 transportation problem and a CVRP with heterogeneous vehicles. It is assumed that the
 shipments of foreign logs to the production mills are carried out by cargo-ships, the
 distribution to foreign customers either by cargo-ships hired on a long-term basis or
 a rented boat for short trips. Deliveries are made by train or truck. A time
 discretization (weeks, days, periods) is also included. The objective function of the
 production model includes transportation, storage, and production plan costs, whereas
 the linear objective function of the distribution covers the sum of flow, storage,
 and boat usage costs. Concerning the solution approach, column generation together
 with a heuristic including constrained branching is used in the production plan
 model, whereas the distribution model is divided into subproblems that are solved
 repeatedly using B\&B on the integer variables. For the integration of the
 production and ship scheduling, the researchers propose two methods, either the
 sequential resolving of the subproblems or a two-phase approach. The latter
 proposition is similar to the methodology used in [\textit{36}]. Computational results are not provided.\par The last tactical-operational model found in the literature is the one by Chen et
 al., also a multiproduct model but with stochastic demands (cf. [\textit{20}]). A two-stage, single-period problem
 involving a supplier who produces perishable goods and delivers them to multiple
 retailers within an allowed time window is considered in this paper. The production
 part of the model is characterized by a single-machine scheduling problem, which
 determines the production quantities of the products as well as the time of starting
 the production of the first commodity per vehicle. Shortage costs, delay penalties,
 and decay rates are also incorporated in the model. A CVRP with soft time windows is
 given in its distribution part. A mixed integer non linear programming model is
 formulated with the aim of maximizing the supplier's total profit (i.e., the sum of
 the expected revenue if the demand is less than the quantity supplied or the expected
 revenue minus goodwill loss if the demand is higher than the supplied quantity minus
 production and transportation costs as well as a penalty in case of exceeding the
 allowed time window for delivery). For the problem solving, the problems are
 decomposed into two subproblems and solved by the Nelder-Mead method with boundary
 constraints (first problem) and by a heuristic algorithm (second problem)
 respectively. Several tests were carried out with the result that the problem can be
 solved within ten minutes for a maximum of 75 retailers. As opposed to the former
 solution process, in most cases, it takes a few hours to find a local optimal
 solution when using LINGO 10.0. Additionally, various sensitivity analyses
 (determining the correlation between the rate of decay and the supply quantity ratio,
 between the different time windows and numbers of vehicles, and between the average
 loading ratio and the number of vehicles) were conducted. The results have shown that
 the fleet size is an important factor leading to a trade-off between deterioration
 and increased transportation costs.\subsection*{Operational IPD problems}3.2\par The problems described in this section involve detailed scheduling of both production
 and distribution operations. As already stated by Chen [\textit{22}] only a very limited number of papers in the literature deal
 with detailed integrated production-distribution scheduling problems. The
 characteristics of these papers are summarized in Table 5.\onecolumn \noindent
\ctable[
 caption = {Operational IPD Problems.}, 
 width=\textwidth, pos = ht, left, long
]
{p{0.20\textwidth}p{0.20\textwidth}p{0.20\textwidth}p{0.20\textwidth}p{0.20\textwidth}}
{
}{ \\\hline
{\textbf{Machine Configuration}}
 & \multicolumn{2}{p{0.20\textwidth}}{\textbf{Vehicle Characteristics}}
 & {\textbf{Modeling Approach}}
 & {\textbf{References}} \\\hline 
{\textbf{Number}}
 & {\textbf{Type}} \\\hline 
{\centering Single} & {\centering One} & {\centering -} & {\centering DP} & {\centering \textit{Li et al.
 (2005)}} \\\hline {\centering } & {\centering } & {\centering } & {\centering MIP, B\&B} & {\centering \textit{Armstrong et al.
 (2008)}} \\\hline {\centering } & {\centering } & {\centering } & {\centering HEU} & {\centering \textit{Averbakh \& Xue
 (2007)}} \\\hline {\centering } & {\centering } & {\centering } & {\centering } & {\centering \textit{Averbakh
 (2010)}} \\\hline {\centering } & {\centering } & {\centering } & {\centering HEU} & {\centering \textit{Geismar et al.
 (2008)}} \\\hline {\centering } & {\centering Two} & {\centering Homogeneous} & {\centering HEU} & {\centering \textit{Dondo et al.
 (2003)}} \\\hline {\centering } & {\centering } & {\centering Heterogeneous} & {\centering HEU+MIP} & {\centering \textit{Mendez et al.
 (2005)}} \\\hline {\centering } & {\centering } & {\centering } & {\centering HEU} & {\centering \textit{Bonfill et al.
 (2008)}} \\\hline {\centering } & {\centering Limited} & {\centering Homogeneous} & {\centering MIP, HEU} & {\centering \textit{Mantel \& Fontein
 (1993)}} \\\hline {\centering } & {\centering } & {\centering } & {\centering MIP} & {\centering \textit{Van Buer et al.
 (1999)}} \\\hline {\centering } & {\centering } & {\centering } & {\centering HEU} & {\centering \textit{Russell et al.
 (2008)}} \\\hline {\centering } & {\centering } & {\centering } & {\centering } & {\centering \textit{Leung \& Chen
 (2013)}} \\\hline {\centering } & {\centering Unlimited} & {\centering Homogeneous} & {\centering HEU} & {\centering \textit{Hurter \& Buer
 (1996)}} \\\hline {\centering } & {\centering } & {\centering Heterogeneous} & {\centering HEU} & {\centering \textit{Chen \& Lee
 (2008)}} \\\hline {\centering } & {\centering } & {\centering Heterogeneous} & {\centering HEU} & {\centering \textit{Wang \& Lee
 (2005)}} \\\hline {\centering Single, Flowshop} & {\centering Limited} & {\centering Homogeneous,} & {\centering MIP, HEU} & {\centering \textit{Bonfill et al.
 (2008)}} \\\hline {\centering } & {\centering } & {\centering Heterogeneous} & {\centering } & {\centering } \\\hline {\centering Single, Parallel} & {\centering One} & {\centering -} & {\centering HEU} & {\centering \textit{Chang \& Lee
 (2004)}} \\\hline {\centering } & {\centering Unlimited} & {\centering Homogeneous} & {\centering HEU} & {\centering \textit{Chen \& Vairaktarakis
 (2005)}} \\\hline {\centering } & {\centering } & {\centering } & {\centering HEU} & {\centering \textit{Devapriya
 (2008)}} \\\hline {\centering Parallel} & {\centering Limited} & {\centering Homogeneous} & {\centering HEU} & {\centering \textit{Böhnlein et al.
 (2011)}} \\\hline {\centering } & {\centering Unlimited} & {\centering Homogeneous} & {\centering HEU} & {\centering \textit{Farahani et al.
 (2012)}} \\\hline {\centering } & {\centering Limited} & {\centering Heterogeneous} & {\centering MIP, HEU} & {\centering \textit{Ulrich (2013)}} \\\hline {\centering Flowshop} & {\centering Limited} & {\centering Heterogeneous} & {\centering MIP, HEU} & {\centering \textit{Méndez et al.
 (2006)}} \\\hline {\centering } & {\centering Unlimited} & {\centering Homogeneous} & {\centering HEU} & {\centering \textit{Li \& Vairaktarakis
 (2007)}} \\\hline {\centering } & {\centering } & {\centering } & {\centering MIP, HEU} & {\centering \textit{Chiang et al.
 (2009)}} \\\hline 
}
\twocolumn 
\par While the explicit models studied in these papers differ significantly due to their
 operational nature and practical application there are some basic concepts that
 define integrated production distribution models on the operational level. A recent
 example nicely highlighting these concepts is the MIP model described by [\textit{47}]. In this problem, the goal is to minimize the
 total tardiness of all jobs in an IPD system, where the production site is modeled as
 a parallel machine environment and the distribution is performed by \textit{t
 } routes from ν vehicles. The model is based on the following notation:\par Sets:\par \textit{V}... set of vehicles\par \textit{J}... set of customers/jobs, tours\par \textit{K}... set of machines\par Indexes:\par 0 ... production facility index\par \textit{i, j}... customer/job indexes\par ν ... vehicle index\par \textit{k}... machine index\par \textit{t}... vehicle tour index\par Parameters:\par 

    \[\let\par\empty

    
{{{{\miitalic{A}}}\sb{{\miitalic{v}}}\sp{{\miitalic{vehicle}}}}}


    \]

   ... capacity of vehicle ν\par \textit{d \textsubscript{j}}... due date of job \textit{j}\par \textit{q \textsubscript{j}}... size of job \textit{j}\par \textit{p \textsubscript{j}}... processing time of job \textit{j}\par \textit{M}... a sufficiently large number\par \textit{r \textsubscript{k}}... ready time for machine \textit{k}\par 
{
\centering
\includegraphics[width=\maxwidth{0.5\textwidth}]{not-found.png}
} \textsubscript{ν} ... ready
 time for vehicle ν\par \textit{s \textsubscript{0}}... service time at the production facility\par \textit{s \textsubscript{j}}... service time at destination of job
 \textit{j}\par \textit{t \textsubscript{i, j}}... travel time between delivery sites for jobs
 \textit{i} and \textit{j}\par w\textit{\textsubscript{j}} , w \textit{\textsubscript{j }}... lower and upper bound of the delivery time window of job
 \textit{j}\par Variables:\par \textit{C \textsubscript{j}}... Completion time of job \textit{j}\par D\textit{\textsubscript{j }}... Delivery time of job \textit{j}\par \textit{S \textsubscript{ν,t}}... Start time of tour \textit{t} of
 vehicle ν\par T\textit{\textsubscript{j }}... Tardiness of job \textit{j}\par \textit{g \textsubscript{j,ν,t}}... binary variable that is 1 when job \textit{j
 } is delivered by tour \textit{t} of vehicle ν\par \textit{x \textsubscript{i,j}}... binary variable that is 1 when job \textit{j
 } is processed after job \textit{i}\par \textit{y \textsubscript{k,j}}... binary variable that is 1 when job \textit{j
 } is the first one to be processed on machine \textit{m}\par \textit{z \textsubscript{i,j,v,t}}... binary variable that is 1 when job
 \textit{j} is delivered after immediately after job \textit{i} on
 the tour \textit{t} of vehicle ν\par The first block of constraints (18)-(21) is related to the
 manufacturing process. The second one is related to the distribution site and shown
 in constraints (22)-(35). The objective function is presented
 in equation (17).

    \[\let\par\empty

    
{{\minormal{min}}{{\mathop{\unicode{8721}}\limits\sb{{{\miitalic{j}}{\mo{\unicode{8714}}}{\miitalic{J}}}}}{{\miitalic{Tj}}{\hspace{0.33em}}{\hspace{0.33em}}{\hspace{0.33em}}{\hspace{0.33em}}{\hspace{0.33em}}{\hspace{0.33em}}{\hspace{0.33em}}{\hspace{0.33em}}{\hspace{0.33em}}{\hspace{0.33em}}{\hspace{0.33em}}{\left({17}\right)}}}}


    \]

  

    \[\let\par\empty

    
{{{\mathop{\unicode{8721}}\limits\sb{{{\miitalic{j}}{\mo{\unicode{8714}}}{\miitalic{J}}}}}{{\miitalic{yk}}{\minormal{,}}{\miitalic{j}}\unicode{8804}{1}{\hspace{0.33em}}{\hspace{0.33em}}{\hspace{0.33em}}{\hspace{0.33em}}{\hspace{0.33em}}\unicode{8704}{\miitalic{k}}\unicode{8714}{\miitalic{K}}{\hspace{0.33em}}{\hspace{0.33em}}{\hspace{0.33em}}{\hspace{0.33em}}{\hspace{0.33em}}{\left({18}\right)}}}}


    \]

  

    \[\let\par\empty

    
{{{\mathop{\unicode{8721}}\limits\sb{{{\minormal{k}}{\mo{\unicode{8714}}}{\minormal{K}}}}}{{{{\miitalic{y}}}\sb{{{\miitalic{k}}{\minormal{,}}{\miitalic{j}}}}}+{{\mathop{\unicode{8721}}\limits\sb{{{\minormal{i}}{\mo{\unicode{8714}}}{\minormal{J}}{\mo{\unicode{8899}}}{{\left\{{0}\right\}}}{\minormal{,}}{\minormal{i}}{\mo{\unicode{8800}}}{\minormal{j}}}}}{{{\miitalic{x}}}\sb{{{\minormal{i}}{\minormal{,}}{\minormal{j}}}}}}={1}{\hspace{0.33em}}{\hspace{0.33em}}{\hspace{0.33em}}{\hspace{0.33em}}{\hspace{0.33em}}{{\unicode{8704}}\sb{{\miitalic{j}}}}\unicode{8714}{\miitalic{J}}{\hspace{0.33em}}{\hspace{0.33em}}{\hspace{0.33em}}{\hspace{0.33em}}{\hspace{0.33em}}{{\left({19}\right)}}}}}


    \]

  

    \[\let\par\empty

    
{{{{\miitalic{c}}}\sb{{\miitalic{j}}}}\unicode{8805}{{{\miitalic{c}}}\sb{{\miitalic{i}}}}+{{{\miitalic{p}}}\sb{{\miitalic{j}}}}\unicode{8722}{\miitalic{M}}\unicode{183}{\left({{1}\unicode{8722}{{{\miitalic{x}}}\sb{{{\miitalic{i}}{\minormal{,}}{\miitalic{j}}}}}}\right)}{\hspace{0.33em}}{\hspace{0.33em}}{\hspace{0.33em}}{\hspace{0.33em}}{\hspace{0.33em}}\unicode{8704}{\miitalic{i}}{\minormal{,}}{\miitalic{j}}\unicode{8714}{\miitalic{J}}{\hspace{0.33em}}{\hspace{0.33em}}{\hspace{0.33em}}{\hspace{0.33em}}{\hspace{0.33em}}{\left({20}\right)}}


    \]

  

    \[\let\par\empty

    
{{{{\miitalic{c}}}\sb{{\miitalic{j}}}}\unicode{8805}{{{\miitalic{y}}}\sb{{{\miitalic{k}}{\minormal{,}}{\miitalic{j}}}}}\unicode{183}{\left({{{{\miitalic{r}}}\sb{{\miitalic{k}}}}\unicode{8722}{{{\miitalic{p}}}\sb{{\miitalic{j}}}}}\right)}{\hspace{0.33em}}{\hspace{0.33em}}{\hspace{0.33em}}{\hspace{0.33em}}{\hspace{0.33em}}{{\unicode{8704}}\sb{{{\miitalic{i}}{\minormal{,}}{\miitalic{j}}}}}\unicode{8714}{\miitalic{J}}{\hspace{0.33em}}{\hspace{0.33em}}{\hspace{0.33em}}{\hspace{0.33em}}{\hspace{0.33em}}{\left({21}\right)}}


    \]

  

    \[\let\par\empty

    
{{{{\miitalic{g}}}\sb{{{0}{\minormal{,}}{\miitalic{v}}{\minormal{,}}{\miitalic{t}}}}}\unicode{8805}{{{\miitalic{g}}}\sb{{{\miitalic{j}}{\minormal{,}}{\miitalic{v}}{\minormal{,}}{\miitalic{t}}}}}{\hspace{0.33em}}{\hspace{0.33em}}{\hspace{0.33em}}{\hspace{0.33em}}{\hspace{0.33em}}{{\unicode{8704}}\sb{{\miitalic{j}}}}\unicode{8714}{\miitalic{J}}{\minormal{,}}{\miitalic{v}}\unicode{8714}{\miitalic{V}}{\minormal{,}}{\miitalic{t}}\unicode{8714}{\miitalic{J}}{\hspace{0.33em}}{\hspace{0.33em}}{\hspace{0.33em}}{\hspace{0.33em}}{\hspace{0.33em}}{\left({22}\right)}}


    \]

  

    \[\let\par\empty

    
{{{\mathop{\unicode{8721}}\limits\sb{{{\miitalic{v}}{\mo{\unicode{8714}}}{\miitalic{V}}}}}}{{\mathop{\unicode{8721}}\limits\sb{{{\miitalic{t}}{\mo{\unicode{8714}}}{\miitalic{T}}}}}}{{{\miitalic{g}}}\sb{{{\miitalic{j}}{\minormal{,}}{\miitalic{v}}{\minormal{,}}{\miitalic{t}}}}}={1}{\hspace{0.33em}}{\hspace{0.33em}}{\hspace{0.33em}}{\hspace{0.33em}}{\hspace{0.33em}}{{\unicode{8704}}\sb{{\miitalic{j}}}}\unicode{8714}{\miitalic{J}}{\hspace{0.33em}}{\hspace{0.33em}}{\hspace{0.33em}}{\hspace{0.33em}}{\hspace{0.33em}}{\left({23}\right)}}


    \]

  

    \[\let\par\empty

    
{{\miitalic{M}}\unicode{183}{{\mathop{\unicode{8721}}\limits\sb{{{\miitalic{j}}{\mo{=}}{1}}}\sp{{\left|{\miitalic{J}}\right|}}}{{{\miitalic{g}}}\sb{{{\miitalic{j}}{\minormal{,}}{\miitalic{v}}{\minormal{,}}{\miitalic{t}}}}}}\unicode{8805}{{\mathop{\unicode{8721}}\limits\sb{{{\miitalic{j}}{\mo{=}}{1}}}\sp{{\left|{\miitalic{J}}\right|}}}{{{\miitalic{g}}}\sb{{{\miitalic{j}}{\minormal{,}}{\miitalic{v}}{\minormal{,}}{\miitalic{t}}{\mo{+}}{1}}}}}{\hspace{0.33em}}{\hspace{0.33em}}{\hspace{0.33em}}{\hspace{0.33em}}{\hspace{0.33em}}{{\unicode{8704}}\sb{{\miitalic{v}}}}\unicode{8714}{\miitalic{V}}{\minormal{,}}{\miitalic{t}}={1}{\minormal{,...}}{\left|{\miitalic{J}}\right|}\unicode{8722}{1}{\hspace{0.33em}}{\hspace{0.33em}}{\hspace{0.33em}}{\hspace{0.33em}}{\hspace{0.33em}}{\left({24}\right)}}


    \]

  \input{mathml/math41.tex}

    \[\let\par\empty

    
{{{{\miitalic{g}}}\sb{{{\miitalic{j}}{\minormal{,}}{\miitalic{v}}{\minormal{,}}{\miitalic{t}}}}}={{\mathop{\unicode{8721}}\limits\sb{{{\miitalic{i}}{\mo{\unicode{8714}}}{\miitalic{J}}{\left\{{0}\right\}}{\minormal{,}}{\miitalic{i}}{\mo{\unicode{8800}}}{\miitalic{j}}}}}}{{{\miitalic{z}}}\sb{{{\miitalic{j}}{\minormal{,}}{\miitalic{i}}{\minormal{,}}{\miitalic{v}}{\minormal{,}}{\miitalic{t}}}}}{\hspace{0.33em}}{\hspace{0.33em}}{\hspace{0.33em}}{\hspace{0.33em}}{\hspace{0.33em}}{{\unicode{8704}}\sb{{\miitalic{j}}}}{{\unicode{8714}}\sb{{{\miitalic{J}}{\minormal{,}}{\miitalic{v}}}}}{{\unicode{8714}}\sb{{{\miitalic{V}}{\minormal{,}}{\miitalic{t}}}}}\unicode{8714}{\miitalic{J}}{\hspace{0.33em}}{\hspace{0.33em}}{\hspace{0.33em}}{\hspace{0.33em}}{\hspace{0.33em}}{\left({26}\right)}}


    \]

  

    \[\let\par\empty

    
{{{{\miitalic{A}}}\sb{{\miitalic{v}}}\sp{{\miitalic{vehicle}}}}{\hspace{0.33em}}\unicode{8805}{\hspace{0.33em}}{{\mathop{\unicode{8721}}\limits\sb{{{\miitalic{j}}{\hspace{0.33em}}{\mo{\unicode{8712}}}{\miitalic{J}}}}}{{{\miitalic{q}}}\sb{{\miitalic{j}}}}}{\hspace{0.33em}}+{\hspace{0.33em}}{{{\miitalic{g}}}\sb{{{\miitalic{j}}{\minormal{,}}{\hspace{0.33em}}{\miitalic{v}}{\minormal{,}}{\hspace{0.33em}}{\miitalic{t}}}}}{\hspace{0.33em}}{\hspace{0.33em}}{\hspace{0.33em}}{\hspace{0.33em}}{\hspace{0.33em}}\unicode{8704}{\miitalic{v}}{\hspace{0.33em}}\unicode{8712}{\hspace{0.33em}}{\miitalic{V}}{\minormal{,}}{\hspace{0.33em}}{\miitalic{t}}{\hspace{0.33em}}\unicode{8712}{\hspace{0.33em}}{\miitalic{J}}{\hspace{0.33em}}{\hspace{0.33em}}{\hspace{0.33em}}{\hspace{0.33em}}{\hspace{0.33em}}{\minormal{(}}{27}{\minormal{)}}}


    \]

  

    \[\let\par\empty

    
{{{{\miitalic{S}}}\sb{{{\miitalic{v}}{\minormal{,}}{\hspace{0.33em}}{1}}}}{\hspace{0.33em}}\unicode{8805}{\hspace{0.33em}}{{{\mathop{{\miitalic{r}}}\limits\sp{{\mo{\unicode{770}}}}}}\sb{{\miitalic{v}}}}{\hspace{0.33em}}+{\hspace{0.33em}}{{{\miitalic{s}}}\sb{{0}}}{\hspace{0.33em}}{\hspace{0.33em}}{\hspace{0.33em}}{\hspace{0.33em}}{\hspace{0.33em}}\unicode{8704}{\miitalic{v}}{\hspace{0.33em}}\unicode{8712}{\hspace{0.33em}}{\miitalic{V}}{\hspace{0.33em}}{\hspace{0.33em}}{\hspace{0.33em}}{\hspace{0.33em}}{\hspace{0.33em}}{\minormal{(}}{28}{\minormal{)}}}


    \]

  

    \[\let\par\empty

    
{{{{\miitalic{S}}}\sb{{{\miitalic{v}}{\minormal{,}}{\hspace{0.33em}}{\miitalic{t}}}}}{\hspace{0.33em}}\unicode{8805}{\hspace{0.33em}}{{{\miitalic{C}}}\sb{{\miitalic{j}}}}{\hspace{0.33em}}+{\hspace{0.33em}}{{{\miitalic{s}}}\sb{{0}}}{\hspace{0.33em}}\unicode{8722}{\hspace{0.33em}}{\miitalic{M}}\unicode{183}{\hspace{0.33em}}{\minormal{(}}{1}{\hspace{0.33em}}\unicode{8722}{\hspace{0.33em}}{{{\miitalic{g}}}\sb{{{\miitalic{j}}{\minormal{,}}{\miitalic{v}}{\minormal{,}}{\miitalic{t}}}}}{\minormal{)}}{\hspace{0.33em}}{\hspace{0.33em}}{\hspace{0.33em}}{\hspace{0.33em}}{\hspace{0.33em}}\unicode{8704}{\miitalic{j}}{\hspace{0.33em}}\unicode{8712}{\hspace{0.33em}}{\miitalic{J}}{\minormal{,}}{\hspace{0.33em}}{\miitalic{v}}{\hspace{0.33em}}\unicode{8712}{\hspace{0.33em}}{\miitalic{V}}{\minormal{,}}{\hspace{0.33em}}{\miitalic{t}}{\hspace{0.33em}}\unicode{8712}{\hspace{0.33em}}{\miitalic{J}}{\hspace{0.33em}}{\hspace{0.33em}}{\hspace{0.33em}}{\hspace{0.33em}}{\hspace{0.33em}}{\minormal{(}}{29}{\minormal{)}}}


    \]

  

    \[\let\par\empty

    
{{{{\miitalic{S}}}\sb{{{\miitalic{v}}{\minormal{,}}{\hspace{0.33em}}{\miitalic{t}}{\hspace{0.33em}}{\mo{+}}{\hspace{0.33em}}{1}}}}{\hspace{0.33em}}\unicode{8805}{\hspace{0.33em}}{{{\miitalic{D}}}\sb{{\miitalic{j}}}}{\hspace{0.33em}}+{\hspace{0.33em}}{{{\miitalic{s}}}\sb{{\miitalic{j}}}}{\hspace{0.33em}}+{\hspace{0.33em}}{{{\miitalic{t}}}\sb{{{\miitalic{j}}{\minormal{,}}{\hspace{0.33em}}{0}}}}{\hspace{0.33em}}+{\hspace{0.33em}}{{{\miitalic{s}}}\sb{{0}}}{\hspace{0.33em}}\unicode{8722}{\hspace{0.33em}}{\miitalic{M}}\unicode{183}{\hspace{0.33em}}{\minormal{(}}{1}{\hspace{0.33em}}\unicode{8722}{\hspace{0.33em}}{{{\miitalic{g}}}\sb{{{\miitalic{j}}{\minormal{,}}{\miitalic{v}}{\minormal{,}}{\miitalic{t}}}}}{\minormal{)}}{\hspace{0.33em}}{\hspace{0.33em}}{\hspace{0.33em}}{\hspace{0.33em}}{\hspace{0.33em}}\unicode{8704}{\miitalic{j}}{\hspace{0.33em}}\unicode{8712}{\hspace{0.33em}}{\miitalic{J}}{\minormal{,}}{\hspace{0.33em}}{\miitalic{v}}{\hspace{0.33em}}\unicode{8712}{\hspace{0.33em}}{\miitalic{V}}{\minormal{,}}{\hspace{0.33em}}{\miitalic{t}}{\hspace{0.33em}}={\hspace{0.33em}}{1}{\minormal{,}}{\hspace{0.33em}}{\minormal{...,}}{\hspace{0.33em}}{\left|{\miitalic{J}}\right|}{\hspace{0.33em}}\unicode{8722}{\hspace{0.33em}}{1}{\hspace{0.33em}}{\hspace{0.33em}}{\hspace{0.33em}}{\hspace{0.33em}}{\hspace{0.33em}}{\minormal{(}}{30}{\minormal{)}}}


    \]

  

    \[\let\par\empty

    
{{{{\miitalic{D}}}\sb{{\miitalic{j}}}}{\hspace{0.33em}}\unicode{8805}{\hspace{0.33em}}{{{\mathop{{\miitalic{w}}}\limits\sb{\unicode{9472}}}}\sb{{\miitalic{j}}}}{\hspace{0.33em}}{\hspace{0.33em}}{\hspace{0.33em}}{\hspace{0.33em}}{\hspace{0.33em}}\unicode{8704}{\miitalic{j}}{\hspace{0.33em}}\unicode{8712}{\hspace{0.33em}}{\miitalic{J}}{\hspace{0.33em}}{\hspace{0.33em}}{\hspace{0.33em}}{\hspace{0.33em}}{\hspace{0.33em}}{\minormal{(}}{31}{\minormal{)}}}


    \]

  

    \[\let\par\empty

    
{{{{\miitalic{D}}}\sb{{\miitalic{j}}}}{\hspace{0.33em}}\unicode{8805}{\hspace{0.33em}}{{{\miitalic{S}}}\sb{{{\miitalic{v}}{\minormal{,}}{\hspace{0.33em}}{\miitalic{t}}}}}{\hspace{0.33em}}+{\hspace{0.33em}}{{{\miitalic{t}}}\sb{{{0}{\miitalic{j}}}}}{\hspace{0.33em}}\unicode{8722}{\hspace{0.33em}}{\miitalic{M}}\unicode{183}{\hspace{0.33em}}{\minormal{(}}{1}{\hspace{0.33em}}\unicode{8722}{\hspace{0.33em}}{{{\miitalic{g}}}\sb{{{\miitalic{j}}{\minormal{,}}{\hspace{0.33em}}{\miitalic{v}}{\minormal{,}}{\hspace{0.33em}}{\miitalic{t}}}}}{\minormal{)}}{\hspace{0.33em}}{\hspace{0.33em}}{\hspace{0.33em}}{\hspace{0.33em}}{\hspace{0.33em}}\unicode{8704}{\miitalic{j}}{\hspace{0.33em}}\unicode{8712}{\hspace{0.33em}}{\miitalic{J}}{\minormal{,}}{\hspace{0.33em}}{\miitalic{v}}{\hspace{0.33em}}\unicode{8712}{\hspace{0.33em}}{\miitalic{V}}{\minormal{,}}{\hspace{0.33em}}{\miitalic{t}}{\hspace{0.33em}}\unicode{8712}{\hspace{0.33em}}{\miitalic{J}}{\hspace{0.33em}}{\hspace{0.33em}}{\hspace{0.33em}}{\hspace{0.33em}}{\hspace{0.33em}}{\minormal{(}}{32}{\minormal{)}}}


    \]

  

    \[\let\par\empty

    
{{{{\miitalic{D}}}\sb{{\miitalic{j}}}}{\hspace{0.33em}}\unicode{8805}{\hspace{0.33em}}{{{\miitalic{D}}}\sb{{\miitalic{i}}}}{\hspace{0.33em}}+{\hspace{0.33em}}{{{\miitalic{s}}}\sb{{\miitalic{i}}}}{\hspace{0.33em}}+{\hspace{0.33em}}{{{\miitalic{t}}}\sb{{{\miitalic{i}}{\minormal{,}}{\hspace{0.33em}}{\miitalic{j}}}}}{\hspace{0.33em}}\unicode{8722}{\hspace{0.33em}}{\miitalic{M}}\unicode{183}{\hspace{0.33em}}{\minormal{(}}{1}{\hspace{0.33em}}\unicode{8722}{\hspace{0.33em}}{{{\miitalic{z}}}\sb{{{\miitalic{i}}{\minormal{,}}{\hspace{0.33em}}{\miitalic{j}}{\hspace{0.33em}}{\minormal{,}}{\hspace{0.33em}}{\miitalic{v}}{\minormal{,}}{\hspace{0.33em}}{\miitalic{t}}}}}{\minormal{)}}{\hspace{0.33em}}{\hspace{0.33em}}{\hspace{0.33em}}{\hspace{0.33em}}{\hspace{0.33em}}\unicode{8704}{\miitalic{i}}{\minormal{,}}{\hspace{0.33em}}{\miitalic{j}}{\hspace{0.33em}}\unicode{8712}{\hspace{0.33em}}{\miitalic{J}}{\minormal{,}}{\hspace{0.33em}}{\miitalic{i}}{\hspace{0.33em}}\unicode{8800}{\hspace{0.33em}}{\miitalic{j}}{\minormal{,}}{\hspace{0.33em}}{\miitalic{v}}{\hspace{0.33em}}\unicode{8712}{\hspace{0.33em}}{\miitalic{V}}{\minormal{,}}{\hspace{0.33em}}{\miitalic{t}}{\hspace{0.33em}}\unicode{8712}{\hspace{0.33em}}{\miitalic{J}}{\hspace{0.33em}}{\hspace{0.33em}}{\hspace{0.33em}}{\hspace{0.33em}}{\hspace{0.33em}}{\minormal{(}}{33}{\minormal{)}}}


    \]

  

    \[\let\par\empty

    
{{{{\miitalic{T}}}\sb{{\miitalic{j}}}}\unicode{8805}{0}{\hspace{0.33em}}{\hspace{0.33em}}{\hspace{0.33em}}{\hspace{0.33em}}{\hspace{0.33em}}\unicode{8704}{\miitalic{j}}\unicode{8714}{\miitalic{J}}{\hspace{0.33em}}{\hspace{0.33em}}{\hspace{0.33em}}{\hspace{0.33em}}{\hspace{0.33em}}{\left({34}\right)}}


    \]

  

    \[\let\par\empty

    
{{{{\miitalic{T}}}\sb{{\miitalic{j}}}}{\hspace{0.33em}}\unicode{8805}{\hspace{0.33em}}{{{\miitalic{D}}}\sb{{\miitalic{j}}}}{\hspace{0.33em}}\unicode{8722}{\hspace{0.33em}}{{{\mathop{{\miitalic{w}}}\limits\sp{\unicode{9472}}}}\sb{{\miitalic{j}}}}{\hspace{0.33em}}{\hspace{0.33em}}{\hspace{0.33em}}{\hspace{0.33em}}{\hspace{0.33em}}\unicode{8704}{\miitalic{j}}{\hspace{0.33em}}\unicode{8712}{\hspace{0.33em}}{\miitalic{J}}{\hspace{0.33em}}{\hspace{0.33em}}{\hspace{0.33em}}{\hspace{0.33em}}{\hspace{0.33em}}{\minormal{(}}{35}{\minormal{)}}}


    \]

  \par Constraints (18) assure that only one
 job is the first one to be processed on each machine. Constraints (19) state that a
 job is sequenced either on the first position of a machine or after some already
 sequenced job. Constraints (20)-(21) state the
 boundaries for the completion times of each job on the manufacturing stage.\par Constraints (22) state that a tour
 will include the processing site ( \textit{j} = 0) if it performs any
 delivery. Constraints (23) guarantee
 that each job is assigned to a unique tour of a single vehicle. Constraints (24) assure that there are no empty tours
 before an active tour. Constraints (25)-(26) assure that a
 vehicle that delivers a job \textit{j} travels from another customer or from
 the processing site. Constraints (27)
 limit the vehicles capacity. Constraints (28)-(30) limit the minimum
 value of the starting time of each route. Constraints (31)-(33) limit
 the minimum value of the delivery time of each job \textit{j}. The tardiness
 is bounded by constraints (34)-(35).\par As mentioned above thismodel does not cover all practical characteristics studied in
 the different papers shown in Table 5. Rather
 it gives a flavor of some of the main issues that form the basis of operational IPD
 systems. Below we will now review the specifics of the existing papers in this
 area.\par A practical well known motivation for the study on joint production
 scheduling/vehicle routing problems is the newspaper production and distribution
 problem. This problem usually comprehends one or more facilities where different
 sections of the daily version of the newspaper are printed. Production has to take
 place within a very short time period (e.g., Russell et al. [\textit{44}] indicates a scenario where production occurs between midnight
 and 3 a.m.). Different geographic regions, containing different clients, can compose
 the daily versions with different sets of sections. The goal is to deliver the
 different newspaper compositions to the different clients respecting some time
 window-like constraints (e.g., in the case of the problem studied by [\textit{44}], the last delivery must occur by 4 a.m.),
 minimizing some fitness function (such as distribution cost). One possible expansion
 of this problem is to consider the loading docks as a scarce resource, allowing just
 a limited number of simultaneous vehicle loadings. This problem is described by
 Hurter and Buer [\textit{33}] as the problem of
 distributing highly perishable products under severe time constraints.\par To address the newspaper production and distribution problem, the literature offers
 some different approaches. The first approach found is the one proposed by Mantel
 \& Fontein [\textit{40}], which presents an MILP
 formulation for the problem, as well as heuristics based on Floyd's method and Clarke
 \& Wright's method to plan the newspaper deliveries to up to 60 delivery sites.
 Hurter \& Buer [\textit{33}] also approach this
 problem by presenting a two-stage algorithm. A single-vector representation of the
 problem is proposed by Van Buer et al. [\textit{48}].
 Combined with some definitions of different neighborhoods, a Simulated Annealing and
 a Tabu Search approach to the problem are proposed. The proposed methods were applied
 to a dataset generated based on real data originating from a medium-size newspaper
 company. A similar production system is studied by Chiang et al. [\textit{25}]. In their case, the problem is characterized
 by three different daily editions ("class of products"), to be produced in two
 identical production lines. There are seven distinct delivery zones. To solve this
 problem, the authors present a mathematical model and a two-phase heuristic approach,
 that incorporates a reactive tabu search to refine the results obtained by the
 application of an integer programming model in a generalized version of the original
 problem. To evaluate the effectiveness of the heuristic, Chiang et al. [\textit{25}] use a simulation model. This simulation model
 also allows the authors to analyze the effects of the variability on the results
 provided by the heuristic. A tabu search is also used by Russell et al. [\textit{44}] to optimize the existing company's operations
 of an American newspaper company. In this case, there are six different geographic
 zones supplied by 47 vehicles that deliver to 818 carriers. The carriers cover more
 than 200 zip codes. The application of the proposed heuristic generates savings of
 more than \$60,000 a year. Significant savings on practical scenarios are also
 presented by Böhnlein et al. [\textit{10}], who apply a
 5-type multi-agent system to real data from one of the largest German newspaper
 companies, and allow about 17\% of reduction on the variable costs on the planning of
 the utilization of a 61-vehicle fleet.\par Besides newspaper production scenarios, another practical motivation for integrating
 scheduling and distribution is presented by Geismar et al. [\textit{31}]. In this case, the authors present a problem encountered in
 chemical adhesive materials manufacturing. This problem is similar to the newspaper
 planning problem in the sense that both products (chemical adhesives and newspapers)
 are products with a relatively short lifespan (hours in the case of newspapers, seven
 days in the adhesive case presented by [\textit{31}]).
 In their approached problem, Geismar et al. present a production system composed of a
 single plant, a single capacitated vehicle and an uncorrelated set of up to 50
 clients dispersed randomly, aiming to minimize the total arrival date of the orders.
 Although the paper does not present data provided from a specific company, the
 random-generated instances allow the authors to present some interesting results: a
 lower bound for the problem considering makespan, as well as genetic and memetic
 algorithms to solve it. In both cases, the developed heuristic is a 2-stage algorithm
 designed as follows: in the first stage, a genetic/memetic algorithm generates a
 sequence of orders; the second stage uses a shortest path algorithm to allocate and
 sequence the orders in routes. The authors report statistical evidence that, with a
 ρ-value= 0.0005, the genetic algorithm performed better than the memetic
 algorithm.\par Another practical relevance of integrated production and transportation decisions in
 the operational level is presented by Farahani et al. [\textit{29}]. In their paper, the authors analyze how the quality of
 catering foods can be improved by minimizing the time that the produced orders wait
 to be shipped. Their claim is that, for this category of food products, the quality
 of a delivered order decreases according to the time spent in transit after
 production. The authors develop an iterative solution procedure based on a Large
 Neighborhood Search algorithm and successfully apply it to 50, 100 and 200-order
 instances. The results shows that the planning technique proposed allows a better
 quality of the delivered food without a considerable increase in total costs.\par Beyond direct practical applications, one can find in the literature studies that
 deal with different scenarios of integrated production-transportation problems. A
 single-machine production stage, now allowing preemption, is approached by Averbakh
 [\textit{3}] and Averbakh \& Xue [\textit{4}]. Starting with a single-client scenario and
 then moving to a multi-client scenario, both [\textit{3}] and [\textit{4}] present a set of
 different theoretical analyses and algorithms related to the problem of minimizing
 the sum of total weighted flowtime and delivery costs.\par Chang \& Lee [\textit{18}] analyze random-generated
 scenarios by considering a production system where a set of jobs is performed on a
 single machine or two identical machines, and then grouped in batches for
 transportation by a single capacitated vehicle to one or two client areas. Some
 properties of those problems are derived. For the single machine/single client
 problem, an NP-Hardness proof is derived. To minimize makespan in a single
 machine/single vehicle/single vehicle area problem, Chang \& Lee state a property
 of this problem, where there is an optimal schedule in which: (i) jobs are processed
 without idle time; (ii) all jobs that are allocated to a vehicle are processed
 sequentially in the production stage; (iii) there is no order precedence between each
 production order belonging to the same route; (iv) each distribution batch is
 delivered according to the well-known Short Processing Time (SPT) rule. This property
 allows Chang \& Lee to derive a heuristic that uses the well-known First Fit
 Decreasing (FDD) bin-packing rule for the stated problem. The previous problem is
 then expanded into a two identical parallel machines production environment, stated
 as being NP-hard in the strong sense. To solve this problem, a heuristic is also
 proposed. Another extension presented by the authors is regarding the one machine/one
 vehicle/two customers area, also stated as NP-Hard in the strong sense. Using results
 from the FDD rule and the well-known Johnson's rule (used to solve the
 \textit{F} 2//\textit{C \textsubscript{max}} problem), a heuristic is
 also proposed. All the proposed heuristics are followed by proofs of worst-case
 performance.\par As with Averbakh \& Xue [\textit{4}], Chen \& Lee
 [\textit{19}] are devoted to presenting theoretical
 analysis to an integrated production-transportation problem. They approach the
 problem of producing a set of orders on a single machine. Once produced, the orders
 must be delivered by a multi-mode (¿2) distribution system. Multiple client sites are
 considered. The goal is to minimize a weighted function between job delivery time and
 transportation costs. A multi-mode distribution system is also approached by Wang
 \& Lee [\textit{50}]. This latter paper deals with
 two transportation modes that can be used to transport orders produced by a single
 machine environment. The due dates are strongly considered in this paper, which
 presents strategies to: (i) minimize transportation costs in a zero-tardiness
 solution and (ii) minimize the weighted sum of total tardiness and transportation
 costs. A B\&B algorithm is presented and, in the instances presented (up to 20
 clients), solves the problem more efficiently.\par Following this line of research, Méndez et al. [\textit{41}] consider a comprehensive multi-stage production and transport
 scheduling problem. The production part of the proposed model deals with identifying
 the batches to be produced, their assignment to the lines, the sequencing, and the
 timing. The distribution part provides a delivery schedule comprising the loads, the
 assignment of orders to the capacitated vehicles, the routes, and the timing of the
 deliveries. Regarding the transport scheduling, it is assumed that the transport time
 includes the travel time, a discharge time depending on the amount delivered, and a
 fixed stop time. Delivery due dates, slack times, and average speeds of the vehicles
 are also considered. The objective is to minimize weighted total costs including
 weight values for the travel time, the number of routes, as well as the earliness and
 tardiness of the orders. In order to reduce the computational effort, different
 heuristic rules (earliest due date rule, minimum slack time rule, clustering of
 distribution centres) are combined and embedded in the proposed continuous time-based
 MILP model. A case study is provided assuming that two products are manufactured in a
 single batch production facility and distributed among eight retail outlets by two
 vehicles with different capacities. The production recipe for both products comprises
 three production stages and eight operations. Also, various weight values, ranging
 from 5 to 100, are included in the objective function. The computational results of
 the MILP are compared with those of eight hybrid MILP-heuristic approaches, stating
 that better solutions with modest computational effort can be obtained in the latter
 case.\par The work by Bonfill et al. [\textit{11}] is based on
 the paper by Méndez et al. [\textit{41}], and provides
 different heuristic-based solution approaches to the problems examined therein. The
 production scheduling part uses a rule-based heuristic algorithm suggested in [\textit{16}], whereas in the transport scheduling part an
 own rule-based heuristic algorithm for the order selection, vehicle assignment,
 loading, and timing is developed. A two-stage sequential approach (with the
 production part being solved first) and an integrated strategy (i.e., production
 orders and due dates are updated in accordance, containing temporal requirements
 implied by the transport schedule) are provided. Several versions of the proposed
 solving methods with different combinations of the priority rules are tested in two
 case studies. The first is very similar to the case study presented in [\textit{41}] and is solved by the integrated approach with
 three criteria (i.e., minimum summed lateness, minimum flow time, and minimum
 multiple cost). Due to its structure, the second problem, an adapted version of the
 problem treated in [\textit{27}], is tested by the
 sequential approach only. This study comprises a single-product, single-stage
 facility problem, in which the finished product has to be delivered to ten
 destinations by two homogeneous vehicles. The results have also been compared to
 those achieved by the approach used in [\textit{41}].
 The researchers have concluded that the application of the presented integrated
 algorithm leads to better performances in various aspects (e.g., material flow
 management).\par Chen \& Vairaktarakis [\textit{23}] also approached
 a similar type of problem. In this case, the manufacturing system is composed of a
 production phase - stated as a single machine or a set of identical parallel machines
 - and a distribution phase - composed of one or multiple capacitated vehicles that
 deliver products to multiple (up to 160) clients geographically dispersed. Inspired
 by real-world applications such as adhesive chemicals manufacturing, a zero-inventory
 policy between the production and the distribution phases is stated. The objective
 function is composed of a weighted sum of the service level and the distribution
 cost. The customer service level is measured by the maximum delivery time or by the
 average delivery time. Those definitions - two manufacturing environments, two fleet
 specifications (single and multiple vehicles) and two different definitions of the
 objective functions - generate a set of eight different problems. Exact algorithms
 are stated for the following cases: single machine, one or multiple vehicles,
 minimizing average distribution time and total distribution costs; parallel machine,
 one vehicle, minimizing average distribution time and total distribution costs. The
 remaining problems are solved by heuristics proposed in the paper. A comparison of
 the solutions obtained by the integrated approach and sequential approaches for the
 problems is then performed. The authors realize that, for problems with an objective
 function composed of the average delivery time, the integrated approach generates a
 gain of 5\% on the fitness value. It is also realized that when the number of clients
 increases, this difference tends to increase. When analyzing problems that minimize
 the maximum delivery time, the gain is more than 5\% for problems with more than three
 vehicles and where the weight factor of the objective function emphasizes the
 delivery time.\par Armstrong et al. [\textit{2}] consider an integrated
 problem composed of a single machine that produces a product with a limited lifespan.
 The production is delivered by a single capacitated vehicle under a fixed client
 delivery sequence subject to a time windows constraint. This problem is stated as
 NP-Hard. The authors present an MIP model, as well as a heuristic to determine a
 lower bound for the problem. Furthermore, a B\&B procedure is presented to obtain
 the optimal value. According to the results presented by Armstrong et al., the time
 required to solve the 100 clientsproblem using the B\&B approach was significantly
 lower than the time required by a commercial solver to solve the MIP model: in the
 best case, the time of the B\&B approach was close to 1.5\% of the commercial
 solver time; in larger instances - e.g., with 50 customers - the MIP model could not
 be solved, but the B\&B approach found the result in less than two minutes.\par Li et al. [\textit{39}], approach a single
 manufacturing facility with distribution accomplished by a set of one or infinite
 bounded capacity vehicles to a set of one or multiple clients. The goal is to
 minimize the total arrival time of the orders. The authors prove that this is an
 NP-Hard problem in the strong sense for the multiple clients case, and
 \textit{O} (\textit{n \textsuperscript{2}}) for the single client case. To
 solve this problem, the authors derive a dynamic programming approach that shows a
 complexity of a polynomial function of the number of clients when the number of
 clients is greater than one, and with lower complexity if the vehicle is
 uncapacitated.\par A similar problem is approached by Devapriya [\textit{26}]. The author approaches two different sets of problems: a single plant
 problem (solved by three different heuristics) and a multi-plant problem (solved by
 five heuristics). Sets of 40, 60 and 80 clients are considered. When several
 production facilities are considered, the vehicle fleet of each facility is
 independent, and a vehicle is not allowed to visit more than one production site. The
 developed heuristics are based on the concept of first applying a routing algorithm
 and then a cluster algorithm.\par Two different approaches for the transportation problem were used by Li \&
 Vairaktarakis [\textit{38}]. In their specific case,
 the production stage is represented as a 2-machine flowshop environment, that
 produces up to 80 orders. The distribution phase is carried by a third party carrier,
 with an unlimited number of vehicles. The orders are delivered to up to five
 different locations, where each location contains several customers. The objective is
 to minimize the sum of the client waiting costs and transportation costs. In this
 specific problem, the transportation is modelled as direct deliveries or as a
 milk-run. To solve those problems, the authors develop a polynomialtime approximation
 scheme for the problem. Further heuristics with guaranteed lower bounds are also
 developed.\par A bi-objective planning of a single-machine production site and a capacitated vehicle
 transportation system is presented by Leung \& Chen [\textit{37}]. In their paper, the problem of minimizing maximum lateness
 is extended to determine an algorithm that allows one, given a minimum/maximum
 lateness, to obtain the minimum fleet size. Moreover, an algorithm to minimize the
 weighted sum of the maximum lateness and the number of used vehicles is presented.
 Although no numerical examples are presented, proofs of optimality and complexity of
 the proposed algorithm are discussed in the paper.\section*{FINAL REMARKS AND POTENTIAL FOR FUTURE RESEARCH}4\par The objective of this paper was to present a review of integrated production and
 distribution planning models which include routing decisions. In order to provide a
 structured and clear survey, the research results have been classified according to
 their decision level, into tactical and operational problems.\par In conclusion, the integrated planning of production and distribution operations is
 critical in today's business. In order to achieve optimal performance, mathematical
 optimization models might deliver decisive information if the functions are integrated
 and jointly planned. Such models of IPDS including routing aspects have gained an
 increasing research interest especially in the last few years. However, the research
 results show that a wide range of problems remain open for future research. Concerning
 the transportation part of the model, aspects such as delivery time windows, split
 deliveries or backhauls could be considered. With respect to the production part, the
 study of multi-machine and multi-product problems or combinations of tactical and
 operational problems is suggested. Apart from that, models including multiple production
 sites and stochastic assumptions need to be investigated to a larger extent. With regard
 to the solution approaches faster and more robust algorithms need to be developed to
 treat complex IPDS arising in real-world applications.
\section*{ACKNOWLEDGMENTS}
\par Roberto F. Tavares Neto thanks FAPESP/Brazil for its financial support (process
 2011/14800-3).

\medskip\par\noindent
\footnotesize{\textit{Keywords}: integrated production and distribution problems, lot-sizing, routing, scheduling}
\medskip\par\noindent
\footnotesize{\textit{Received}: 31/12/2012.} \par \noindent
\footnotesize{\textit{Accepted}: 01/12/2013.} \medskip\par\noindent
\footnotesize{*
				Corresponding author.
}
\balance
\pagebreak\onecolumn
\begin{biblio}[REFERENCES]
\tit{ ADULYASAK Y,} CORDEAU J-F \& JANS R. 2012. Optimization-based
 adaptive large neighborhood search for the production routing problem.
 \textit{Transportation Science}.
\tit{ ARMSTRONG R,} GAO S \& LEI L. 2008. A zero-inventory production
 and distribution problem with a fixed customer sequence. \textit{Annals of Operations
 Research} , \textbf{159}(1):395-414.
\tit{ }AVERBAKH I. 2010. On-line integrated production-distribution
 scheduling problems with capacitated deliveries. \textit{European Journal of
 Operational Research} , \textbf{200}:377-384.
\tit{ }AVERBAKH I \& XUE Z. 2007. On-line supply chain scheduling
 problems with preemption. \textit{European Journal of Operational Research} ,
 \textbf{181}:500-504.
\tit{ AYDINEL M,} SOWLATI T, CERDA X, COPE E \& GERSCHMAN M. 2008.
 Optimization of production allocation and transportation of customer orders for a
 leading forest products company. \textit{Mathematical and Computer
 Modelling} , \textbf{48}:1158-1169.
\tit{ }BARBAROSOĞLU G \& ÖZGÜR D. 1999. Hierarchical design of an
 integrated production and 2- echelon distribution system. \textit{European Journal of
 Operational Research} , \textbf{118}(3):464-484.
\tit{ }BARD JF \& NANANUKUL N. 2009. The integrated
 production-inventory-distribution-routing problem. \textit{Journal of
 Scheduling} , \textbf{12}:257-280.
\tit{[8] }BARD JF \& NANANUKUL N. 2009. Heuristics for a multiperiod
 inventory routing problem with production decisions. \textit{Computers \&
 Industrial Engineering} , \textbf{57}:713-723.
\tit{ }BARD JF \& NANANUKUL N. 2010. A branch-and-price algorithm for an
 integrated production and inventory routing problem. \textit{Computers \&
 Operations Research} , \textbf{37}:2202-2217.
\tit{ BÖHNLEIN D,} SCHWEIGER K \& TUMA A. 2011. Multi-agent-based
 transport planning in the newspaper industry. \textit{International Journal of
 Production} Economics, \textbf{131}:146-157.
\tit{ BONFILL A,} ESPUÑA A \& PUIGJANER L. 2008. Decision support
 framework for coordinated production and transport scheduling in scm.
 \textit{Computers \& Chemical Engineering} ,
 \textbf{32}:1206-1224.
\tit{ }BOUDIA M \& PRINS C. 2009. A memetic algorithm with dynamic
 population management for an integrated production-distribution problem.
 \textit{European Journal of Operational Research} ,
 \textbf{195}:703-715.
\tit{ BOUDIA M,} LOULY MAO \& PRINS C. 2007. A reactive grasp and path
 relinking for a combined production-distribution problem. \textit{Computers \&
 Operations Research} , \textbf{34}:3402-3419.
\tit{ BOUDIA M,} LOULY MAO \& PRINS C. 2008. Fast heuristics for a
 combined production planning and vehicle routing problem. \textit{Production Planning
 \& Control} , \textbf{19}(2):85-96.
\tit{ }BREDSTRÖM D \& RÖNNQVIST M. 2002. Integrated production planning
 and route scheduling in pulp mill industry. In \textit{HICSS '02: Proceedings of the
 35th Annual Hawaii International Conference on System Sciences (HICSS'02)-Volume
 3} , Washington, DC, 2002. IEEE Computer Society.
\tit{ }CANTON J. 2003. \textit{Integrated support system for planning and
 scheduling of batch chemical plants}. PhD thesis, Universitat Politecnia
 de Catalunya.
\tit{ }CHANDRA P \& FISHER ML. 2004. Coordination of production and
 distribution planning. \textit{European Journal of Operational Research} ,
 \textbf{72}:503-517.
\tit{ }CHANG Y-C \& LEE C-Y. 2004. Machine scheduling with job delivery
 coordination. \textit{European Journal of Operational Research} ,
 \textbf{158}(1):470-487.
\tit{ }CHEN B \& LEE C-Y. 2008. Logistics scheduling with batching and
 transportation. \textit{European Journal of Operational Research} ,
 \textbf{189}:871-876.
\tit{ CHEN H-K,} HSUEH C-F \& CHANG M-S. 2009. Production scheduling
 and vehicle routing with time windows for perishable food products. \textit{Computers
 \& Operations Research} , \textbf{36}:2311-2319.
\tit{ }CHEN Z-L. 2004. Integrated production and distribution operations:
 Taxonomy, models, and review. In: D. Simchi-Levi, S.D. Wu, and Z.-J. Shen, editors,
 \textit{Handbook of Quantitative Supply Chain Analysis: Modeling in the E-Business
 Era}. Kluwer Academic Publishers.
\tit{ }CHEN Z-L. 2010. Integrated production and outbound distribution
 scheduling: Review and extensions. \textit{Operations Research} ,
 \textbf{58}(1):130-148.
\tit{ }CHEN Z-L \& VAIRAKTARAKIS GL. 2005. Integrated scheduling of
 production and distribution operations. \textit{Management Science} ,
 \textbf{51}(4):614-628.
\tit{ }CHERN C-C \& HSIEH J-S. 2007. A heuristic algorithm for master
 planning that satisfies multiple objectives. \textit{Computers \& Operations
 Research} , \textbf{34}:3491-3513.
\tit{ CHIANG W,} RUSSELL R, XU X \& ZEPEDA D. 2009. A
 simulation/metaheuristic approach to newspaper production and distribution supply
 chain problems. \textit{International Journal of Production Economics} ,
 \textbf{121}:752-767.
\tit{ }DEVAPRIYA P. 2008. \textit{Optimal feet size of an integrated
 production and distribution scheduling problem for a single perishable
 product}. PhD thesis, Clemson University.
\tit{ DONDO R,} MÉNDEZ CA \& CERDÁ J. 2003. An optimal approach to the
 multiple-depot heterogeneous vehicle routing problem with time window and capacity
 constraints. \textit{Latin American Applied Research} ,
 \textbf{33}:129-134.
\tit{ ERENGܸ SS,} SIMPSON NC \& VAKHARIA AJ. 1999. Integrated
 production/distribution planning in supply chains: An invited review.
 \textit{European Journal of Operational Research} ,
 \textbf{115}:219-236.
\tit{ FARAHANI P,} GRUNOW M \& GÜNTHER H. 2012. Integrated production
 and distribution planning for perishable food products. \textit{Flexible Services and
 Manufacturing} , \textbf{24}:28-51.
\tit{ }FUMERO F \& VERCELLIS C. 1999. Synchronized development of
 production, inventory, and distribution schedules. \textit{Transportation
 Science} , \textbf{33}:330-340.
\tit{ GEISMAR HN,} LAPORTE G, LEI L \& SRISKANDARAJAH C. 2008. The
 integrated production and transportation scheduling problem for a product with a
 short life span and non-instantaneous transportation time. \textit{INFORMS Journal on
 Computing} , \textbf{20}(1):21-33.
\tit{ HUANG GQ,} LAUJSK \& MAK KL. 2003. The impacts of sharing
 production information on supply chain dynamics: a review of the literature.
 \textit{International Journal of Production Research} ,
 \textbf{41}(7):1483-1517.
\tit{ }HURTER AP \& VAN BUER MG. 1996. The newspaper
 productiondistribution problem. \textit{Journal of Business Logistics} ,
 \textbf{17}(1):85-107.
\tit{ }JAYARAMAN V \& PIRKUL H. 2001. Planning and coordination of
 production and distribution facilities for multiple commodities. \textit{European
 Journal of Operational Research} ,
 \textbf{133}(2):394-408.
\tit{ }KESKIN BB \& ÜSTER H. 2007. Meta-heuristic approaches with
 memory and evolution for a multiproduct production/distribution system design
 problem. \textit{European Journal of Operational Research} ,
 \textbf{182}:663-682.
\tit{ LEI L,} LIU S, RUSZCZYNSKI A \& PARK S. 2006. On the integrated
 production, inventory, and distribution routing problem. \textit{IIE
 Transactions} , \textbf{38}:955-970.
\tit{ }LEUNG JYT \& CHEN Z-L. 2013. Integrated production and
 distribution with fixed delivery departure dates. \textit{Operations Research
 Letters} , \textbf{41}:290-293.
\tit{ }LI C-L \& VAIRAKTARAKISGL. 2007. Coordinating production and
 distribution of jobs with bundling operations. \textit{IIE Transactions} ,
 \textbf{39}(2):203-215.
\tit{ LI C-L,} VAIRAKTARAKIS GL \& LEE C-Y. 2005. Machine scheduling
 with deliveries to multiple customer locations. \textit{European Journal of
 Operational Research} , \textbf{164}(1):39-51.
\tit{ }MANTEL RJ \& FONTEIN M. 1993. A practical solution to a
 newspaper distribution problem. \textit{International Journal of Production
 Economics} , \textbf{30}:591-599.
\tit{ MÉNDEZ C,} BONFILL A, ESPUÑA A \& PUIGJANER L. 2006. A rigorous
 approach to coordinate production and transport scheduling in a multi-site system.
 In: MARQUARDT W \& PANTELIDES C, editors, \textit{16th European Symposium on
 Computer Aided Process Engineering and 9th International Symposium on Process
 Systems Engineering, volume 21 of Computer Aided Chemical Engineering} ,
 pages 2171-2176. Elsevier.
\tit{ MULA J,} PEIDROD, DÍAZ-MADROÑERO M \& VICENS E. 2010.
 Mathematical programming models for supply chain production and transport planning.
 \textit{European Journal of Operational Research} ,
 \textbf{204}:377-390.
\tit{ }OWEN SH \& DASKIN MS. 1998. Strategic facility location: A
 review. \textit{European Journal of Operational Research} ,
 \textbf{111}(3):423-447.
\tit{ RUSSELL R,} CHIANG W \& ZEPEDA D. 2008. Integrating multi-product
 production and distribution in newspaper logistics. \textit{Computers \&
 Operations Research} , \textbf{35}:1578-1588.
\tit{ }SARMIENTO AM \& NAGI R. 1999. A review of integrated analysis of
 production-distribution systems. Technical report, Department of Industrial
 Engineering, State University of New York and Buffalo, Buffalo, NY
 14260.
\tit{ }STECKE KE \& ZHAO X. 2007. Production and transportation
 integration for a make-to-order manufacturing company with a commit-to-delivery
 business mode. \textit{Manufacturing \& Service Operations Management} ,
 \textbf{9}:206-224.
\tit{ }ULRICH CA. 2013. Integrated machine scheduling and vehicle routing
 with time windows. \textit{European Journal of Operational Research} ,
 \textbf{227}:152-165.
\tit{ VAN BUER MG,} WOODRUFF DL \& OLSON RT. 1999. Solving the medium
 newspaper production/distribution problem. \textit{European Journal of Operational
 Research} , \textbf{115}:237-253.
\tit{ }VIDAL CJ \& GOETSCHALCKX M. 1997. Strategic
 production-distribution models: A critical review with emphasis on global supply
 chain models. \textit{European Journal of Operational Research} ,
 \textbf{98}:1-18.
\tit{ }WANG H \& LEE C-Y. 2005. Production and transport logistics
 scheduling with two transport mode choices. \textit{Naval Research
 Logistics} , \textbf{52}:796-809.
\end{biblio}

\medskip\par\noindent
\footnotesize{\#\par This work was initiated, while E. Bogendorfer was doing her MSc dissertation at the
 University of Graz.
}

\medskip\par\noindent
\footnotesize{This is an Open Access article distributed under the terms of the Creative
 Commons Attribution Non-Commercial License which permits unrestricted
 non-commercial use, distribution, and reproduction in any medium, provided the
 original work is properly cited. }
