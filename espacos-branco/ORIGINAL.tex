\selectlanguage{english}
\removeabstracttitle

{
\begin{abstract}
 \textbf{OBJECTIVE:}
 Identify spatial distribution patterns of the proportion of nonadherence to tuberculosis treatment and its associated factors.\allowbreak{}
\par{} \textbf{METHODS:}
 We conducted an ecological study based on secondary and primary data from municipalities of the metropolitan area of Buenos Aires,\allowbreak{} Argentina.\allowbreak{} An exploratory analysis of the characteristics of the area and the distributions of the cases included in the sample (\allowbreak{}proportion of nonadherence)\allowbreak{} was also carried out along with a multifactor analysis by linear regression.\allowbreak{} The variables related to the characteristics of the population,\allowbreak{} residences and families were analyzed.\allowbreak{}
\par{} \textbf{RESULTS:}
 Areas with higher proportion of the population without social security benefits (\allowbreak{}p = 0.\allowbreak{}007)\allowbreak{} and of households with unsatisfied basic needs had a higher risk of nonadherence (\allowbreak{}p = 0.\allowbreak{}032)\allowbreak{}.\allowbreak{} In addition,\allowbreak{} the proportion of nonadherence was higher in areas with the highest proportion of households with no public transportation within 300 meters (\allowbreak{}p = 0.\allowbreak{}070)\allowbreak{}.\allowbreak{}
\par{} \textbf{CONCLUSIONS:}
 We found a risk area for the nonadherence to treatment characterized by a population living in poverty,\allowbreak{} with precarious jobs and difficult access to public transportation.\allowbreak{}


\vspace*{4mm}
\fontsize{9}{10.8}\selectfont{\textbf{Keywords:} \textit{Tuberculosis, drug therapy, Medication Adherence, Socioeconomic Factors, Health Inequalities, Ecological Studies}}
\end{abstract}
}


{
\begin{abstract}
 \textbf{OBJETIVO:}
 Identificar patrones de distribución espacial de la proporción de la no-\allowbreak{}adherencia al tratamiento de la tuberculosis y sus factores asociados.\allowbreak{}
\par{} \textbf{METODOS:}
 Estudio ecológico con datos secundarios y primarios en municipios seleccionados del Área Metropolitana de Buenos Aires.\allowbreak{} Se realizó un análisis exploratorio de las características del área y de las distribuciones de los casos incluidos en la muestra (\allowbreak{}proporción de no-\allowbreak{}adherencia)\allowbreak{} y un análisis de múltiples factores por regresión lineal.\allowbreak{} Se analizaron variables referidas a las características de la población,\allowbreak{} las viviendas y los hogares.\allowbreak{}
\par{} \textbf{RESULTADOS:}
 Las áreas con mayor proporción de población que no realizaba aportes jubilatorios (\allowbreak{}p = 0,\allowbreak{}007)\allowbreak{} y con mayor proporción de hogares con necesidades básicas insatisfechas según capacidad de subsistencia presentaron mayor riesgo de no-\allowbreak{}adherencia (\allowbreak{}p = 0,\allowbreak{}032)\allowbreak{}.\allowbreak{} La proporción de no-\allowbreak{}adherencia fue más elevada en las áreas con mayor proporción de viviendas sin servicio de transporte público a menos de 300 m (\allowbreak{}p = 0,\allowbreak{}070)\allowbreak{}.\allowbreak{}
\par{} \textbf{CONCLUSIONES:}
 Existe un área de riesgo para la no-\allowbreak{}adherencia al tratamiento,\allowbreak{} caracterizada por tener una población que vive en condiciones de pobreza y precariedad laboral,\allowbreak{} con dificultades de acceso al servicio de transporte público.\allowbreak{}


\vspace*{4mm}
\fontsize{9}{10.8}\selectfont{\textbf{Keywords:} \textit{Tuberculosis,\allowbreak{} quimioterapia, Cumplimiento de la Medicación, Factores Socioeconómicos, Desigualdades en la Salud, Estudios Ecológicos}}
\end{abstract}
}

\begin{multicols}{2}
\section*{INTRODUCTION}
\par{}Although tuberculosis (\allowbreak{}TB)\allowbreak{} is a curable disease that can be prevented,\allowbreak{} it is an important public health issue in Argentina.\allowbreak{} Each year,\allowbreak{} more than 10,\allowbreak{}000 new cases and more than 800 deaths caused by this disease are reported.\allowbreak{} The geographic distribution of TB in the country is not uniform as in the rest of the world.\allowbreak{}\protect\footnote{ Instituto Nacional de Enfermedades Respiratorias “Dr.\allowbreak{} Emilio Coni”.\allowbreak{} Notificación de casos de tuberculosis en la República Argentina.\allowbreak{} Período 1980-\allowbreak{}2011.\allowbreak{} Buenos Aires:\allowbreak{} Ministerio de Salud; 2012.\allowbreak{}} The nonadherence to treatment is considered one of the main obstacles for the control of the disease due to the consequences of its discontinuation,\allowbreak{} associated with the social vulnerability of patients.\allowbreak{}\textsuperscript{\textsuperscript{7}}\par{}TB persists as a public health problem,\allowbreak{} despite the low cost of its diagnosis and treatment.\allowbreak{} These measures are part of the strategy of the directly observed treatment,\allowbreak{} short-\allowbreak{}course (\allowbreak{}DOTS)\allowbreak{} recommended by the World Health Organization (\allowbreak{}WHO)\allowbreak{} to reduce the nonadherence to the treatment,\allowbreak{}\textsuperscript{\textsuperscript{25}} which were adopted in Argentina and implemented with the \textit{Programa Nacion}\textit{al de Control de la }\textit{Tuberculosis} (\allowbreak{}PNCTB – National Tuberculosis Control Program)\allowbreak{}.\allowbreak{}\protect\footnote{ Zerbini EV,\allowbreak{} Darnaud RMH,\allowbreak{} Prieto VG.\allowbreak{} Programa Nacional de Control de la Tuberculosis:\allowbreak{} Normas Técnicas 2008.\allowbreak{} 3.\allowbreak{} ed.\allowbreak{} Santa Fé:\allowbreak{} Instituto Nacional de Enfermedades Respiratorias Dr.\allowbreak{} Emilio Coni; 2008.\allowbreak{}} Although the implementation of the DOTS strategy has been carried out in the country for the last 10 years,\allowbreak{} the proportion of cases who have gave up treatment was 12.\allowbreak{}0\%\allowbreak{\allowbreak{}\allowbreak{}}\allowbreak{} in 2010,\allowbreak{} one of the highest in recent years.\allowbreak{}\protect\footnote{ Instituto Nacional de Enfermedades Respiratorias “Dr.\allowbreak{} Emilio Coni”.\allowbreak{} Resultado del tratamiento de la tuberculosis pulmonar ED(\allowbreak{}+\allowbreak{})\allowbreak{} en la República Argentina.\allowbreak{} Período 1980-\allowbreak{}2010.\allowbreak{} Buenos Aires:\allowbreak{} Ministerio de Salud; 2012.\allowbreak{}}\par{}Studies address the treatment adherence from a focus based on environmental factors\textsuperscript{\textsuperscript{12}}\textsuperscript{,\allowbreak{}}\textsuperscript{\textsuperscript{24}} and the individual factors related to the patient.\allowbreak{}\textsuperscript{\textsuperscript{1}}\textsuperscript{,\allowbreak{}}\textsuperscript{\textsuperscript{4}}\textsuperscript{,\allowbreak{}}\textsuperscript{\textsuperscript{5}}\textsuperscript{,\allowbreak{}}\textsuperscript{\textsuperscript{13}}\textsuperscript{,\allowbreak{}}\textsuperscript{\textsuperscript{16}}\textsuperscript{,\allowbreak{}}\textsuperscript{\textsuperscript{18}}\textsuperscript{,\allowbreak{}}\textsuperscript{\textsuperscript{22}}\par{}The occurrence of TB and its consequences to health are related to the social conditions.\allowbreak{}\textsuperscript{\textsuperscript{20}} To understand its behavior in a territory and its determinants it is essential to establish equitable actions that aim at reducing inequalities and improve adherence to the treatment.\allowbreak{} \protect\footnote{ Acosta LSW.\allowbreak{} O mapa de Porto Alegre e a tuberculose:\allowbreak{} distribuição espacial e determinantes sociais [dissertation].\allowbreak{} Porto Alegre (\allowbreak{}RS)\allowbreak{}:\allowbreak{} Faculdade de Medicina da UFRGS; 2008.\allowbreak{}} The ecological studies aims to identify,\allowbreak{} based on social characteristics and on the territory,\allowbreak{} relations with the distribution of diseases and health outcomes,\allowbreak{} considering the different hierarchical levels of the determinants.\allowbreak{}\textsuperscript{\textsuperscript{3}}\textsuperscript{,\allowbreak{}}\textsuperscript{\textsuperscript{9}}\textsuperscript{,\allowbreak{}}\textsuperscript{\textsuperscript{14}}\par{}Despite the importance of such studies,\allowbreak{} in Argentina no studies can be found about the characteristics of social groups and the area where they live and the relationship with the nonadherence to the TB treatment.\allowbreak{}\par{}The objective of this study was to identify patterns of spatial distribution of the proportion of nonadherence to the tuberculosis treatment and its associated factors.\allowbreak{}
\section*{METHODS}
\par{}This spatial-\allowbreak{}ecological study was conducted in seven municipalities of the Sixth Health Region (\allowbreak{}6\textsuperscript{th} HR)\allowbreak{} in the Buenos Aires metropolitan area (\allowbreak{}BAMA)\allowbreak{} (\allowbreak{}where there are 116 census fractions – Figure 1)\allowbreak{}:\allowbreak{} Almirante Brown,\allowbreak{} Avellaneda,\allowbreak{} Berazategui,\allowbreak{} Esteban Echeverría,\allowbreak{} Ezeiza,\allowbreak{} Lomas de Zamora and Quilmes.\allowbreak{} The two other municipalities that are also part of the 6\textsuperscript{th} HR (\allowbreak{}Lanus and Florencio Varela)\allowbreak{} could not be included because the locations did not have an Ethics Committee to evaluate the cross-\allowbreak{}sectional protocol of the study that provides the georeferenced cases (\allowbreak{}adherence and nonadherence)\allowbreak{}.\allowbreak{}\protect\footnote{ Arrossi S,\allowbreak{} Herrero MB,\allowbreak{} Faccia K,\allowbreak{} Greco A,\allowbreak{} Ramirez Lijó S,\allowbreak{} Aizemberg L et al.\allowbreak{} Evaluación de los factores predictivos de la no-\allowbreak{}adherencia al tratamiento de la tuberculosis en municipios seleccionados del área metropolitana de Buenos Aires:\allowbreak{} estudio colaborativo multicéntrico.\allowbreak{} Buenos Aires:\allowbreak{} Ministerio de Salud de la Nación; 2008 (\allowbreak{}ECM 2008)\allowbreak{}.\allowbreak{}}\par{}
\par
{
\centering{
\includegraphics[width=\maxwidth{0.5\textwidth}]{not-found.png}
}
\captionof{figure}{\textbf{Figure 1:} \textit{Study area:\allowbreak{} selected municipalities from Sixth Health Region (\allowbreak{}6th HR)\allowbreak{} and census fractions.\allowbreak{} Buenos Aires,\allowbreak{} Argentina,\allowbreak{} 2001.\allowbreak{}}} 
}
\par
\par{}The 6\textsuperscript{th} HR has about 3,\allowbreak{}653,\allowbreak{}000 inhabitants,\allowbreak{} and it is the most populated region of Buenos Aires.\allowbreak{}\protect\footnote{ Ministerio de Salud de la Provincia de Buenos Aires (\allowbreak{}ARG)\allowbreak{}.\allowbreak{} Diagnóstico de las Regiones Sanitarias 2007-\allowbreak{}2008,\allowbreak{} La Plata,\allowbreak{} Buenos Aires; 2008.\allowbreak{}} It also concentrates 13.\allowbreak{}0\%\allowbreak{\allowbreak{}\allowbreak{}}\allowbreak{} of all reported cases of TB in the country and it is the sanitary region that has the largest number of TB cases in the province every year,\allowbreak{} with the highest dropout index (\allowbreak{}25.\allowbreak{}0\%\allowbreak{\allowbreak{}\allowbreak{}}\allowbreak{})\allowbreak{} and the lowest DOTS coverage (\allowbreak{}12.\allowbreak{}0\%\allowbreak{\allowbreak{}\allowbreak{}}\allowbreak{})\allowbreak{}.\allowbreak{}\protect\footnote{ Instituto Nacional de Enfermedades Respiratorias “Dr.\allowbreak{} Emilio Coni”.\allowbreak{} Resultado del tratamiento de la tuberculosis pulmonar ED(\allowbreak{}+\allowbreak{})\allowbreak{} en la República Argentina.\allowbreak{} Período 1980-\allowbreak{}2010.\allowbreak{} Buenos Aires:\allowbreak{} Ministerio de Salud; 2012.\allowbreak{}}\par{}The database and mapping of the National Census of Population and Households (\allowbreak{}2001)\allowbreak{},\allowbreak{} of the \textit{Insti}\textit{tuto Nacional de Est}\textit{adísticas y Censos} (\allowbreak{}INDEC – National Institute of Statistics and Censuses)\allowbreak{} were used as a secondary data source.\allowbreak{}\protect\footnote{ INDEC.\allowbreak{} Instituto Nacional de Estadísticas y Censos [Argentina].\allowbreak{} Buenos Aires; 2001 [cited 2015 May 21].\allowbreak{} Available from:\allowbreak{} \textit{http:\allowbreak{}\fshyp{}\fshyp{}www.\allowbreak{}indec.\allowbreak{}gov.\allowbreak{}ar}} All cases reported,\allowbreak{} from households in the municipal districts selected by the 6\textsuperscript{th} HR and treated at health services located in the region in 2007,\allowbreak{} were referenced.\allowbreak{} This was possible because these individuals participated in a study that aims to identify the foreknowledge of the nonadherence to tuberculosis treatment in these municipalities.\allowbreak{}\textsuperscript{\textsuperscript{1}} We also calculated the proportion of nonadherence to the TB treatment for the census fractions (\allowbreak{}analysis units in this study)\allowbreak{} of the municipalities of 6\textsuperscript{th} HR.\allowbreak{}\par{}The information has been grouped into three types of indicators according to the census classification.\allowbreak{}\protect\footnote{ INDEC.\allowbreak{} Instituto Nacional de Estadísticas y Censos [Argentina].\allowbreak{} Buenos Aires; 2001 [cited 2015 May 21].\allowbreak{} Available from:\allowbreak{} \textit{http:\allowbreak{}\fshyp{}\fshyp{}www.\allowbreak{}indec.\allowbreak{}gov.\allowbreak{}ar}} The characteristics of the area were considered according to the presence of wastewater treatment; electricity per household; gas network; at least one block paved; regular waste collection service at least twice a week; public transportation within 300 m.\allowbreak{} The proportion of households was considered according to the type of the pavement’s predominant material,\allowbreak{} water supply system,\allowbreak{} presence or absence of public water network,\allowbreak{} and type of health service.\allowbreak{}\par{}We considered the following:\allowbreak{} proportion of households grouped according to overcrowding (\allowbreak{}three or more persons per room)\allowbreak{}; lack of basic needs (\allowbreak{}overcrowding,\allowbreak{} housing,\allowbreak{} sanitation,\allowbreak{} education,\allowbreak{} and subsistence capacity)\allowbreak{}; index of household material privation (\allowbreak{}IHMP)\allowbreak{};\protect\footnote{ According to the National Institute of Statistics and Censuses (\allowbreak{}INDEC)\allowbreak{},\allowbreak{} and the Index of Household Material Privation (\allowbreak{}IHMP)\allowbreak{} it is a variable that identifies the residences according to their material deprivation in two dimensions – material and patrimonial resources.\allowbreak{} In relation to the Unsatisfied Basic Needs (\allowbreak{}UBN)\allowbreak{},\allowbreak{} the households with this characteristic have at least one of the following indicators of deprivation:\allowbreak{} overcrowding (\allowbreak{}more than 3 persons per room)\allowbreak{}; housing (\allowbreak{}living in an improper location [leased place,\allowbreak{} hotel or pension room,\allowbreak{} shack,\allowbreak{} place without rooms],\allowbreak{} not considering house,\allowbreak{} apartment and farm)\allowbreak{}; health conditions (\allowbreak{}without water-\allowbreak{}closet)\allowbreak{}; school attendance (\allowbreak{}with at least one school-\allowbreak{}age child [6-\allowbreak{}12 years] who do not attend school)\allowbreak{}; subsistence capacity (\allowbreak{}with four or more individuals per family unit,\allowbreak{} whose responsible has not concluded the third grade of Elementary School)\allowbreak{}.\allowbreak{}} economic situation of the family; and the presence of refrigerator,\allowbreak{} freezer,\allowbreak{} landline or cell phone,\allowbreak{} microwave,\allowbreak{} computer with Internet connection,\allowbreak{} kitchen with sink and piped water in the residence.\allowbreak{}\par{}We considered the proportion of population according to sex,\allowbreak{} age,\allowbreak{} health plan,\allowbreak{} marital status and literacy.\allowbreak{} Moreover,\allowbreak{} we considered the ratio of individuals according to educational level,\allowbreak{} state of activity and of retirement contribution (\allowbreak{}contributes and is discounted; does not contribute nor receive discount; no remuneration)\allowbreak{}.\allowbreak{}\par{}The statistical software package Stata 10.\allowbreak{}0 and two geographic information systems,\allowbreak{} ArcView 3.\allowbreak{}2 and GeoDA 8,\allowbreak{} were used to elaborate maps and perform the spatial analysis.\allowbreak{} The dropout rate was calculated by dividing the number of cases of nonadherence by the total number of patients who have started the treatment in each unit of analysis (\allowbreak{}census fraction)\allowbreak{}.\allowbreak{} The Bayesian and Freeman-\allowbreak{}Tukey square-\allowbreak{}root transformations,\allowbreak{} empirical for these measures,\allowbreak{} were calculated having as a reference the set of fractions of the municipal census.\allowbreak{} The thematic maps with these proportions were elaborated to choose the most appropriate way to present the spatial distribution patterns.\allowbreak{}\par{}The exploratory analysis of the area characteristics and the nonadherence ratio distribution were also performed.\allowbreak{} The multifactorial analysis was performed with linear regression.\allowbreak{} In this model,\allowbreak{} the independent variables were the sociodemographic and socioeconomic characteristics of the groups and the areas related to dropout cases.\allowbreak{} The dependent variable was the “nonadherence”.\allowbreak{} The variables applied in the multiple linear regression model were those that had a significant association (\allowbreak{}p < 0,\allowbreak{}20)\allowbreak{} in the bivariate analysis.\allowbreak{} The final model included variables with a significance level of p = 0.\allowbreak{}05 and those considered essential for the explanatory model.\allowbreak{}\par{}The study protocol was approved by the Ethics Committee of each hospital included.\allowbreak{}
\section*{RESULTS}
\par{}The city of Avellaneda had residences with better overall conditions and available basic services.\allowbreak{} It was also the city with the lowest variations for each indicator of the census fractions.\allowbreak{} On the other hand,\allowbreak{} the city of Ezeiza,\allowbreak{} with the worst situation concerning most part of the analyzed indicators,\allowbreak{} had high variations in the census fractions.\allowbreak{} The distribution of the population was more homogeneous among municipalities,\allowbreak{} although the city of Avellaneda generally has the best situation regarding indicators (\allowbreak{}Table 1)\allowbreak{}.\allowbreak{}\par{}\end{multicols}
\begin{landscape}
\ctable[
  caption = {\textbf{Table 1:} \textit{Characteristics of the area,\allowbreak{} residences,\allowbreak{} households and population related to the cases of tuberculosis.\allowbreak{} Sixth Health Region (\allowbreak{}6th HR)\allowbreak{},\allowbreak{} Buenos Aires,\allowbreak{} Argentina,\allowbreak{} 2001.\allowbreak{}}}, 
  width=\linewidth, pos = ht, left, long
]
{p{0.28\linewidth}p{0.08\linewidth}p{0.08\linewidth}p{0.08\linewidth}p{0.08\linewidth}p{0.08\linewidth}p{0.08\linewidth}p{0.09\linewidth}p{0.08\linewidth}}
{}
{ \\\hline
\multicolumn{1}{L{0.28\linewidth}}{\textbf{Characteristic}}
 & \multicolumn{8}{L{0.67\linewidth}}{\textbf{Existence (\allowbreak{}\%\allowbreak{\allowbreak{}\allowbreak{}}\allowbreak{})\allowbreak{} }} \\\hline 
\multicolumn{1}{L{0.28\linewidth}}{\textbf{AB}}
 & \multicolumn{1}{L{0.08\linewidth}}{\textbf{AV}}
 & \multicolumn{1}{L{0.08\linewidth}}{\textbf{BZ}}
 & \multicolumn{1}{L{0.08\linewidth}}{\textbf{EE}}
 & \multicolumn{1}{L{0.08\linewidth}}{\textbf{EZ}}
 & \multicolumn{1}{L{0.08\linewidth}}{\textbf{LZ}}
 & \multicolumn{1}{L{0.08\linewidth}}{\textbf{QM}}
 & \multicolumn{1}{L{0.09\linewidth}}{\textbf{Total}} \\\hline 
\multicolumn{8}{L{0.87\linewidth}}{\textbf{}} \\\hline 
\multicolumn{1}{L{0.28\linewidth}}{\textbf{\%\allowbreak{\allowbreak{}\allowbreak{}}\allowbreak{}}}
 & \multicolumn{1}{L{0.08\linewidth}}{\textbf{\%\allowbreak{\allowbreak{}\allowbreak{}}\allowbreak{}}}
 & \multicolumn{1}{L{0.08\linewidth}}{\textbf{\%\allowbreak{\allowbreak{}\allowbreak{}}\allowbreak{}}}
 & \multicolumn{1}{L{0.08\linewidth}}{\textbf{\%\allowbreak{\allowbreak{}\allowbreak{}}\allowbreak{}}}
 & \multicolumn{1}{L{0.08\linewidth}}{\textbf{\%\allowbreak{\allowbreak{}\allowbreak{}}\allowbreak{}}}
 & \multicolumn{1}{L{0.08\linewidth}}{\textbf{\%\allowbreak{\allowbreak{}\allowbreak{}}\allowbreak{}}}
 & \multicolumn{1}{L{0.08\linewidth}}{\textbf{\%\allowbreak{\allowbreak{}\allowbreak{}}\allowbreak{}}}
 & \multicolumn{1}{L{0.09\linewidth}}{\textbf{\%\allowbreak{\allowbreak{}\allowbreak{}}\allowbreak{}}} \\\hline 
\multicolumn{9}{L{0.95\linewidth}}{Area and residences characteristics} \\\hline \multicolumn{1}{L{0.28\linewidth}}{ Electricity per household} & \multicolumn{1}{C{0.08\linewidth}}{96.\allowbreak{}0} & \multicolumn{1}{C{0.08\linewidth}}{100} & \multicolumn{1}{C{0.08\linewidth}}{98.\allowbreak{}3} & \multicolumn{1}{C{0.08\linewidth}}{98.\allowbreak{}9} & \multicolumn{1}{C{0.08\linewidth}}{97.\allowbreak{}5} & \multicolumn{1}{C{0.08\linewidth}}{98.\allowbreak{}7} & \multicolumn{1}{C{0.09\linewidth}}{95.\allowbreak{}3} & \multicolumn{1}{C{0.08\linewidth}}{97.\allowbreak{}2} \\\hline \multicolumn{1}{L{0.28\linewidth}}{ Paved Street} & \multicolumn{1}{C{0.08\linewidth}}{73.\allowbreak{}0} & \multicolumn{1}{C{0.08\linewidth}}{100} & \multicolumn{1}{C{0.08\linewidth}}{75.\allowbreak{}4} & \multicolumn{1}{C{0.08\linewidth}}{72.\allowbreak{}4} & \multicolumn{1}{C{0.08\linewidth}}{78.\allowbreak{}2} & \multicolumn{1}{C{0.08\linewidth}}{87.\allowbreak{}2} & \multicolumn{1}{C{0.09\linewidth}}{72.\allowbreak{}8} & \multicolumn{1}{C{0.08\linewidth}}{76.\allowbreak{}1} \\\hline \multicolumn{1}{L{0.28\linewidth}}{ Waterwaste and sewage treatment system} & \multicolumn{1}{C{0.08\linewidth}}{13.\allowbreak{}0} & \multicolumn{1}{C{0.08\linewidth}}{85.\allowbreak{}3} & \multicolumn{1}{C{0.08\linewidth}}{73.\allowbreak{}8} & \multicolumn{1}{C{0.08\linewidth}}{4.\allowbreak{}9} & \multicolumn{1}{C{0.08\linewidth}}{2.\allowbreak{}0} & \multicolumn{1}{C{0.08\linewidth}}{6.\allowbreak{}7} & \multicolumn{1}{C{0.09\linewidth}}{59.\allowbreak{}0} & \multicolumn{1}{C{0.08\linewidth}}{37.\allowbreak{}5} \\\hline \multicolumn{1}{L{0.28\linewidth}}{ Garbage collection service} & \multicolumn{1}{C{0.08\linewidth}}{87.\allowbreak{}3} & \multicolumn{1}{C{0.08\linewidth}}{100} & \multicolumn{1}{C{0.08\linewidth}}{95.\allowbreak{}3} & \multicolumn{1}{C{0.08\linewidth}}{89.\allowbreak{}8} & \multicolumn{1}{C{0.08\linewidth}}{97.\allowbreak{}2} & \multicolumn{1}{C{0.08\linewidth}}{92.\allowbreak{}9} & \multicolumn{1}{C{0.09\linewidth}}{88.\allowbreak{}7} & \multicolumn{1}{C{0.08\linewidth}}{91.\allowbreak{}5} \\\hline \multicolumn{1}{L{0.28\linewidth}}{ Gas pipeline network} & \multicolumn{1}{C{0.08\linewidth}}{63.\allowbreak{}6} & \multicolumn{1}{C{0.08\linewidth}}{97.\allowbreak{}8} & \multicolumn{1}{C{0.08\linewidth}}{87.\allowbreak{}9} & \multicolumn{1}{C{0.08\linewidth}}{64.\allowbreak{}0} & \multicolumn{1}{C{0.08\linewidth}}{62.\allowbreak{}0} & \multicolumn{1}{C{0.08\linewidth}}{75.\allowbreak{}3} & \multicolumn{1}{C{0.09\linewidth}}{69.\allowbreak{}7} & \multicolumn{1}{C{0.08\linewidth}}{72.\allowbreak{}2} \\\hline \multicolumn{1}{L{0.28\linewidth}}{ Electrical power installation} & \multicolumn{1}{C{0.08\linewidth}}{44.\allowbreak{}5} & \multicolumn{1}{C{0.08\linewidth}}{100} & \multicolumn{1}{C{0.08\linewidth}}{99.\allowbreak{}7} & \multicolumn{1}{C{0.08\linewidth}}{41.\allowbreak{}3} & \multicolumn{1}{C{0.08\linewidth}}{30.\allowbreak{}1} & \multicolumn{1}{C{0.08\linewidth}}{89.\allowbreak{}3} & \multicolumn{1}{C{0.09\linewidth}}{92.\allowbreak{}7} & \multicolumn{1}{C{0.08\linewidth}}{74.\allowbreak{}4} \\\hline \multicolumn{1}{L{0.28\linewidth}}{ Deprived dwelling} & \multicolumn{1}{C{0.08\linewidth}}{34.\allowbreak{}8} & \multicolumn{1}{C{0.08\linewidth}}{2.\allowbreak{}4} & \multicolumn{1}{C{0.08\linewidth}}{23.\allowbreak{}1} & \multicolumn{1}{C{0.08\linewidth}}{38.\allowbreak{}0} & \multicolumn{1}{C{0.08\linewidth}}{39.\allowbreak{}3} & \multicolumn{1}{C{0.08\linewidth}}{31.\allowbreak{}0} & \multicolumn{1}{C{0.09\linewidth}}{30.\allowbreak{}0} & \multicolumn{1}{C{0.08\linewidth}}{30.\allowbreak{}7} \\\hline \multicolumn{1}{L{0.28\linewidth}}{ Ceramic floors,\allowbreak{} flag stone or mosaic} & \multicolumn{1}{C{0.08\linewidth}}{48.\allowbreak{}1} & \multicolumn{1}{C{0.08\linewidth}}{75.\allowbreak{}8} & \multicolumn{1}{C{0.08\linewidth}}{56.\allowbreak{}2} & \multicolumn{1}{C{0.08\linewidth}}{41.\allowbreak{}3} & \multicolumn{1}{C{0.08\linewidth}}{38.\allowbreak{}4} & \multicolumn{1}{C{0.08\linewidth}}{53.\allowbreak{}0} & \multicolumn{1}{C{0.09\linewidth}}{52.\allowbreak{}5} & \multicolumn{1}{C{0.08\linewidth}}{50.\allowbreak{}4} \\\hline \multicolumn{1}{L{0.28\linewidth}}{ Water supply in the residence} & \multicolumn{1}{C{0.08\linewidth}}{64.\allowbreak{}3} & \multicolumn{1}{C{0.08\linewidth}}{76.\allowbreak{}7} & \multicolumn{1}{C{0.08\linewidth}}{78.\allowbreak{}7} & \multicolumn{1}{C{0.08\linewidth}}{58.\allowbreak{}7} & \multicolumn{1}{C{0.08\linewidth}}{52.\allowbreak{}3} & \multicolumn{1}{C{0.08\linewidth}}{69.\allowbreak{}1} & \multicolumn{1}{C{0.09\linewidth}}{74.\allowbreak{}2} & \multicolumn{1}{C{0.08\linewidth}}{69.\allowbreak{}0} \\\hline \multicolumn{1}{L{0.28\linewidth}}{ Public transportation within 300 m} & \multicolumn{1}{C{0.08\linewidth}}{90.\allowbreak{}2} & \multicolumn{1}{C{0.08\linewidth}}{100} & \multicolumn{1}{C{0.08\linewidth}}{89.\allowbreak{}9} & \multicolumn{1}{C{0.08\linewidth}}{85.\allowbreak{}2} & \multicolumn{1}{C{0.08\linewidth}}{81.\allowbreak{}8} & \multicolumn{1}{C{0.08\linewidth}}{89.\allowbreak{}9} & \multicolumn{1}{C{0.09\linewidth}}{84.\allowbreak{}9} & \multicolumn{1}{C{0.08\linewidth}}{87.\allowbreak{}2} \\\hline \multicolumn{9}{L{0.95\linewidth}}{Characteristics of households and population} \\\hline \multicolumn{1}{L{0.28\linewidth}}{ Kitchen with piped water} & \multicolumn{1}{C{0.08\linewidth}}{62.\allowbreak{}0} & \multicolumn{1}{C{0.08\linewidth}}{77.\allowbreak{}5} & \multicolumn{1}{C{0.08\linewidth}}{75.\allowbreak{}2} & \multicolumn{1}{C{0.08\linewidth}}{56.\allowbreak{}0} & \multicolumn{1}{C{0.08\linewidth}}{48.\allowbreak{}7} & \multicolumn{1}{C{0.08\linewidth}}{68.\allowbreak{}4} & \multicolumn{1}{C{0.09\linewidth}}{71.\allowbreak{}2} & \multicolumn{1}{C{0.08\linewidth}}{66.\allowbreak{}1} \\\hline \multicolumn{1}{L{0.28\linewidth}}{ Microwave oven} & \multicolumn{1}{C{0.08\linewidth}}{9.\allowbreak{}8} & \multicolumn{1}{C{0.08\linewidth}}{27.\allowbreak{}7} & \multicolumn{1}{C{0.08\linewidth}}{14.\allowbreak{}1} & \multicolumn{1}{C{0.08\linewidth}}{10.\allowbreak{}6} & \multicolumn{1}{C{0.08\linewidth}}{9.\allowbreak{}1} & \multicolumn{1}{C{0.08\linewidth}}{10.\allowbreak{}1} & \multicolumn{1}{C{0.09\linewidth}}{14.\allowbreak{}2} & \multicolumn{1}{C{0.08\linewidth}}{12.\allowbreak{}4} \\\hline \multicolumn{1}{L{0.28\linewidth}}{ Overcrowding} & \multicolumn{1}{C{0.08\linewidth}}{7.\allowbreak{}1} & \multicolumn{1}{C{0.08\linewidth}}{10.\allowbreak{}2} & \multicolumn{1}{C{0.08\linewidth}}{5.\allowbreak{}8} & \multicolumn{1}{C{0.08\linewidth}}{7.\allowbreak{}2} & \multicolumn{1}{C{0.08\linewidth}}{8.\allowbreak{}7} & \multicolumn{1}{C{0.08\linewidth}}{6.\allowbreak{}3} & \multicolumn{1}{C{0.09\linewidth}}{5.\allowbreak{}9} & \multicolumn{1}{C{0.08\linewidth}}{6.\allowbreak{}4} \\\hline \multicolumn{1}{L{0.28\linewidth}}{ Water-\allowbreak{}closet without flushing system or without water-\allowbreak{}closet} & \multicolumn{1}{C{0.08\linewidth}}{28.\allowbreak{}6} & \multicolumn{1}{C{0.08\linewidth}}{0.\allowbreak{}7} & \multicolumn{1}{C{0.08\linewidth}}{17.\allowbreak{}6} & \multicolumn{1}{C{0.08\linewidth}}{31.\allowbreak{}1} & \multicolumn{1}{C{0.08\linewidth}}{32.\allowbreak{}1} & \multicolumn{1}{C{0.08\linewidth}}{25.\allowbreak{}0} & \multicolumn{1}{C{0.09\linewidth}}{23.\allowbreak{}1} & \multicolumn{1}{C{0.08\linewidth}}{24.\allowbreak{}4} \\\hline \multicolumn{1}{L{0.28\linewidth}}{ Refrigerator (\allowbreak{}with or without freezer)\allowbreak{}} & \multicolumn{1}{C{0.08\linewidth}}{81.\allowbreak{}8} & \multicolumn{1}{C{0.08\linewidth}}{77.\allowbreak{}6} & \multicolumn{1}{C{0.08\linewidth}}{85.\allowbreak{}1} & \multicolumn{1}{C{0.08\linewidth}}{81.\allowbreak{}0} & \multicolumn{1}{C{0.08\linewidth}}{76.\allowbreak{}5} & \multicolumn{1}{C{0.08\linewidth}}{82.\allowbreak{}0} & \multicolumn{1}{C{0.09\linewidth}}{82.\allowbreak{}4} & \multicolumn{1}{C{0.08\linewidth}}{81.\allowbreak{}9} \\\hline \multicolumn{1}{L{0.28\linewidth}}{ Computer (\allowbreak{}with or without Internet)\allowbreak{}} & \multicolumn{1}{C{0.08\linewidth}}{8.\allowbreak{}9} & \multicolumn{1}{C{0.08\linewidth}}{23.\allowbreak{}6} & \multicolumn{1}{C{0.08\linewidth}}{13.\allowbreak{}7} & \multicolumn{1}{C{0.08\linewidth}}{10.\allowbreak{}1} & \multicolumn{1}{C{0.08\linewidth}}{9.\allowbreak{}2} & \multicolumn{1}{C{0.08\linewidth}}{9.\allowbreak{}7} & \multicolumn{1}{C{0.09\linewidth}}{12.\allowbreak{}5} & \multicolumn{1}{C{0.08\linewidth}}{11.\allowbreak{}6} \\\hline \multicolumn{1}{L{0.28\linewidth}}{ Literate population} & \multicolumn{1}{C{0.08\linewidth}}{84.\allowbreak{}1} & \multicolumn{1}{C{0.08\linewidth}}{91.\allowbreak{}7} & \multicolumn{1}{C{0.08\linewidth}}{85.\allowbreak{}3} & \multicolumn{1}{C{0.08\linewidth}}{83.\allowbreak{}9} & \multicolumn{1}{C{0.08\linewidth}}{83.\allowbreak{}3} & \multicolumn{1}{C{0.08\linewidth}}{86.\allowbreak{}7} & \multicolumn{1}{C{0.09\linewidth}}{85.\allowbreak{}8} & \multicolumn{1}{C{0.08\linewidth}}{85.\allowbreak{}2} \\\hline \multicolumn{1}{L{0.28\linewidth}}{ Population without health insurance} & \multicolumn{1}{C{0.08\linewidth}}{63.\allowbreak{}7} & \multicolumn{1}{C{0.08\linewidth}}{33.\allowbreak{}0} & \multicolumn{1}{C{0.08\linewidth}}{57.\allowbreak{}8} & \multicolumn{1}{C{0.08\linewidth}}{63.\allowbreak{}2} & \multicolumn{1}{C{0.08\linewidth}}{62.\allowbreak{}7} & \multicolumn{1}{C{0.08\linewidth}}{63.\allowbreak{}7} & \multicolumn{1}{C{0.09\linewidth}}{54.\allowbreak{}8} & \multicolumn{1}{C{0.08\linewidth}}{59.\allowbreak{}0} \\\hline \multicolumn{1}{L{0.28\linewidth}}{ Telephone (\allowbreak{}landline,\allowbreak{} mobile or both)\allowbreak{}} & \multicolumn{1}{C{0.08\linewidth}}{53.\allowbreak{}1} & \multicolumn{1}{C{0.08\linewidth}}{69.\allowbreak{}5} & \multicolumn{1}{C{0.08\linewidth}}{58.\allowbreak{}3} & \multicolumn{1}{C{0.08\linewidth}}{53.\allowbreak{}9} & \multicolumn{1}{C{0.08\linewidth}}{49.\allowbreak{}0} & \multicolumn{1}{C{0.08\linewidth}}{54.\allowbreak{}8} & \multicolumn{1}{C{0.09\linewidth}}{54.\allowbreak{}4} & \multicolumn{1}{C{0.08\linewidth}}{54.\allowbreak{}8} \\\hline \multicolumn{1}{L{0.28\linewidth}}{ No material deprivation in the residences} & \multicolumn{1}{C{0.08\linewidth}}{34.\allowbreak{}7} & \multicolumn{1}{C{0.08\linewidth}}{65.\allowbreak{}7} & \multicolumn{1}{C{0.08\linewidth}}{43.\allowbreak{}6} & \multicolumn{1}{C{0.08\linewidth}}{33.\allowbreak{}9} & \multicolumn{1}{C{0.08\linewidth}}{31.\allowbreak{}6} & \multicolumn{1}{C{0.08\linewidth}}{38.\allowbreak{}7} & \multicolumn{1}{C{0.09\linewidth}}{41.\allowbreak{}5} & \multicolumn{1}{C{0.08\linewidth}}{39.\allowbreak{}3} \\\hline \multicolumn{1}{L{0.28\linewidth}}{ With material deprivation in the residences} & \multicolumn{1}{C{0.08\linewidth}}{21.\allowbreak{}4} & \multicolumn{1}{C{0.08\linewidth}}{0.\allowbreak{}2} & \multicolumn{1}{C{0.08\linewidth}}{16.\allowbreak{}1} & \multicolumn{1}{C{0.08\linewidth}}{21.\allowbreak{}7} & \multicolumn{1}{C{0.08\linewidth}}{23.\allowbreak{}2} & \multicolumn{1}{C{0.08\linewidth}}{19.\allowbreak{}7} & \multicolumn{1}{C{0.09\linewidth}}{19.\allowbreak{}0} & \multicolumn{1}{C{0.08\linewidth}}{19.\allowbreak{}2} \\\hline \multicolumn{1}{L{0.28\linewidth}}{ With at least a UBN indicator} & \multicolumn{1}{C{0.08\linewidth}}{20.\allowbreak{}7} & \multicolumn{1}{C{0.08\linewidth}}{4.\allowbreak{}4} & \multicolumn{1}{C{0.08\linewidth}}{17.\allowbreak{}2} & \multicolumn{1}{C{0.08\linewidth}}{19.\allowbreak{}3} & \multicolumn{1}{C{0.08\linewidth}}{20.\allowbreak{}2} & \multicolumn{1}{C{0.08\linewidth}}{18.\allowbreak{}8} & \multicolumn{1}{C{0.09\linewidth}}{23.\allowbreak{}0} & \multicolumn{1}{C{0.08\linewidth}}{19.\allowbreak{}9} \\\hline \multicolumn{1}{L{0.28\linewidth}}{ With UBN per subsistence condition} & \multicolumn{1}{C{0.08\linewidth}}{7.\allowbreak{}0} & \multicolumn{1}{C{0.08\linewidth}}{1.\allowbreak{}4} & \multicolumn{1}{C{0.08\linewidth}}{6.\allowbreak{}2} & \multicolumn{1}{C{0.08\linewidth}}{4.\allowbreak{}9} & \multicolumn{1}{C{0.08\linewidth}}{5.\allowbreak{}8} & \multicolumn{1}{C{0.08\linewidth}}{18.\allowbreak{}3} & \multicolumn{1}{C{0.09\linewidth}}{21.\allowbreak{}1} & \multicolumn{1}{C{0.08\linewidth}}{11.\allowbreak{}8} \\\hline \multicolumn{1}{L{0.28\linewidth}}{ With one spouse unemployed or inactive} & \multicolumn{1}{C{0.08\linewidth}}{50.\allowbreak{}8} & \multicolumn{1}{C{0.08\linewidth}}{50.\allowbreak{}1} & \multicolumn{1}{C{0.08\linewidth}}{48.\allowbreak{}2} & \multicolumn{1}{C{0.08\linewidth}}{53.\allowbreak{}0} & \multicolumn{1}{C{0.08\linewidth}}{52.\allowbreak{}7} & \multicolumn{1}{C{0.08\linewidth}}{48.\allowbreak{}7} & \multicolumn{1}{C{0.09\linewidth}}{49.\allowbreak{}9} & \multicolumn{1}{C{0.08\linewidth}}{50.\allowbreak{}3} \\\hline \multicolumn{1}{L{0.28\linewidth}}{ Population that contributes to or receives retirement funds} & \multicolumn{1}{C{0.08\linewidth}}{12.\allowbreak{}6} & \multicolumn{1}{C{0.08\linewidth}}{27.\allowbreak{}7} & \multicolumn{1}{C{0.08\linewidth}}{15.\allowbreak{}2} & \multicolumn{1}{C{0.08\linewidth}}{13.\allowbreak{}0} & \multicolumn{1}{C{0.08\linewidth}}{13.\allowbreak{}6} & \multicolumn{1}{C{0.08\linewidth}}{13.\allowbreak{}2} & \multicolumn{1}{C{0.09\linewidth}}{17.\allowbreak{}3} & \multicolumn{1}{C{0.08\linewidth}}{15.\allowbreak{}1} \\\hline \multicolumn{1}{L{0.28\linewidth}}{ Unemployed population} & \multicolumn{1}{C{0.08\linewidth}}{17.\allowbreak{}5} & \multicolumn{1}{C{0.08\linewidth}}{12.\allowbreak{}3} & \multicolumn{1}{C{0.08\linewidth}}{18.\allowbreak{}7} & \multicolumn{1}{C{0.08\linewidth}}{16.\allowbreak{}0} & \multicolumn{1}{C{0.08\linewidth}}{14.\allowbreak{}5} & \multicolumn{1}{C{0.08\linewidth}}{19.\allowbreak{}8} & \multicolumn{1}{C{0.09\linewidth}}{17.\allowbreak{}3} & \multicolumn{1}{C{0.08\linewidth}}{17.\allowbreak{}3} \\\hline \multicolumn{9}{p\linewidth}{\textsuperscript{ } \footnotesize{Sources:\allowbreak{} Cross-\allowbreak{}sectional study (\allowbreak{}ECM,\allowbreak{} 2008)\allowbreak{}e and National Census of Population and Households,\allowbreak{} INDEC,\allowbreak{} 2001.\allowbreak{}\textsuperscript{g}}}\\\multicolumn{9}{p\linewidth}{\textsuperscript{ } \footnotesize{AB:\allowbreak{} Almirante Brown; AV:\allowbreak{} Avellaneda; BZ:\allowbreak{} Berazategui; EE:\allowbreak{} Esteban Echeverría; EZ:\allowbreak{} Ezeiza; LZ:\allowbreak{} Lomas de Zamora; QM:\allowbreak{} Quilmes; UBN:\allowbreak{} unsatisfied basic needs}}
}
\end{landscape}
\begin{multicols}{2}
\par{}The risk of nonadherence was higher in areas with highest proportion of households without public transportation within 300 m (\allowbreak{}$\allowbreak{\allowbreak{}\allowbreak{}}\allowbreak{}ρ$\allowbreak{\allowbreak{}\allowbreak{}}\allowbreak{} = 0.\allowbreak{}21)\allowbreak{},\allowbreak{} as well as in areas with the highest proportion of residences that do not have refrigerator (\allowbreak{}with or without freezer)\allowbreak{} (\allowbreak{}$\allowbreak{\allowbreak{}\allowbreak{}}\allowbreak{}ρ$\allowbreak{\allowbreak{}\allowbreak{}}\allowbreak{} = 0.\allowbreak{}17)\allowbreak{} and those that have water-\allowbreak{}closet without flushing system or without water-\allowbreak{}closet (\allowbreak{}$\allowbreak{\allowbreak{}\allowbreak{}}\allowbreak{}ρ$\allowbreak{\allowbreak{}\allowbreak{}}\allowbreak{} = 0.\allowbreak{}17)\allowbreak{}.\allowbreak{} The risk of nonadherence to the treatment was higher in areas with the highest proportion of households with UBN related to subsistence (\allowbreak{}$\allowbreak{\allowbreak{}\allowbreak{}}\allowbreak{}ρ$\allowbreak{\allowbreak{}\allowbreak{}}\allowbreak{} = 0.\allowbreak{}26)\allowbreak{} and with a greater proportion of active population composed of workers that do not receive nor make social security contributions (\allowbreak{}$\allowbreak{\allowbreak{}\allowbreak{}}\allowbreak{}ρ$\allowbreak{\allowbreak{}\allowbreak{}}\allowbreak{} = 0.\allowbreak{}21)\allowbreak{} (\allowbreak{}Table 2)\allowbreak{}.\allowbreak{}\par{}\end{multicols}
\ctable[
  caption = {\textbf{Table 2:} \textit{Relationship between sociodemographic and socioeconomic characteristics and the proportion of nonadherence in the selected municipalities.\allowbreak{} Sixth Health Region (\allowbreak{}6th HR)\allowbreak{},\allowbreak{} Buenos Aires,\allowbreak{} Argentina,\allowbreak{} 2001.\allowbreak{}}}, 
  width=\textwidth, pos = ht, left, long
]
{p{0.47\textwidth}p{0.34\textwidth}p{0.16\textwidth}}
{}
{ \\\hline
\multicolumn{1}{L{0.47\textwidth}}{\textbf{Characteristic}}
 & \multicolumn{1}{L{0.34\textwidth}}{\textbf{Correlation coefficient (\allowbreak{}$\allowbreak{\allowbreak{}\allowbreak{}}\allowbreak{}ρ$\allowbreak{\allowbreak{}\allowbreak{}}\allowbreak{} =)\allowbreak{}}}
 & \multicolumn{1}{L{0.16\textwidth}}{\textbf{p}} \\\hline 
\multicolumn{1}{L{0.47\textwidth}}{Refrigerator or freezer} & \multicolumn{1}{C{0.34\textwidth}}{0.\allowbreak{}17} & \multicolumn{1}{C{0.16\textwidth}}{0.\allowbreak{}194} \\\hline \multicolumn{1}{L{0.47\textwidth}}{Contribution to retirement:\allowbreak{} do not contribute or do not receive salary} & \multicolumn{1}{C{0.34\textwidth}}{0.\allowbreak{}21} & \multicolumn{1}{C{0.16\textwidth}}{0.\allowbreak{}104} \\\hline \multicolumn{1}{L{0.47\textwidth}}{UBN indicator of subsistence conditions} & \multicolumn{1}{C{0.34\textwidth}}{0.\allowbreak{}26} & \multicolumn{1}{C{0.16\textwidth}}{0.\allowbreak{}048} \\\hline \multicolumn{1}{L{0.47\textwidth}}{Public transportation within 300 m (\allowbreak{}3 blocks)\allowbreak{}} & \multicolumn{1}{C{0.34\textwidth}}{0.\allowbreak{}21} & \multicolumn{1}{C{0.16\textwidth}}{0.\allowbreak{}104} \\\hline \multicolumn{1}{L{0.47\textwidth}}{Water-\allowbreak{}closet without flushing system or without water-\allowbreak{}closet} & \multicolumn{1}{C{0.34\textwidth}}{0.\allowbreak{}17} & \multicolumn{1}{C{0.16\textwidth}}{0.\allowbreak{}200} \\\hline \multicolumn{1}{L{0.47\textwidth}}{Population between 15 and 64 years} & \multicolumn{1}{C{0.34\textwidth}}{0.\allowbreak{}15} & \multicolumn{1}{C{0.16\textwidth}}{> 0.\allowbreak{}20} \\\hline \multicolumn{1}{L{0.47\textwidth}}{Illiterate population} & \multicolumn{1}{C{0.34\textwidth}}{0.\allowbreak{}06} & \multicolumn{1}{C{0.16\textwidth}}{> 0.\allowbreak{}20} \\\hline \multicolumn{1}{L{0.47\textwidth}}{Piped water} & \multicolumn{1}{C{0.34\textwidth}}{0.\allowbreak{}15} & \multicolumn{1}{C{0.16\textwidth}}{> 0.\allowbreak{}20} \\\hline \multicolumn{1}{L{0.47\textwidth}}{Condition:\allowbreak{} unemployed} & \multicolumn{1}{C{0.34\textwidth}}{0.\allowbreak{}05} & \multicolumn{1}{C{0.16\textwidth}}{> 0.\allowbreak{}20} \\\hline \multicolumn{1}{L{0.47\textwidth}}{Population without health insurance} & \multicolumn{1}{C{0.34\textwidth}}{0.\allowbreak{}05} & \multicolumn{1}{C{0.16\textwidth}}{> 0.\allowbreak{}20} \\\hline \multicolumn{1}{L{0.47\textwidth}}{Water supply in the residence} & \multicolumn{1}{C{0.34\textwidth}}{0.\allowbreak{}14} & \multicolumn{1}{C{0.16\textwidth}}{> 0.\allowbreak{}20} \\\hline \multicolumn{1}{L{0.47\textwidth}}{Washing machine or washing tank} & \multicolumn{1}{C{0.34\textwidth}}{0.\allowbreak{}15} & \multicolumn{1}{C{0.16\textwidth}}{> 0.\allowbreak{}20} \\\hline \multicolumn{1}{L{0.47\textwidth}}{With at least a UBN indicator} & \multicolumn{1}{C{0.34\textwidth}}{0.\allowbreak{}07} & \multicolumn{1}{C{0.16\textwidth}}{> 0.\allowbreak{}20} \\\hline \multicolumn{1}{L{0.47\textwidth}}{Level of education:\allowbreak{} Incomplete Elementary School} & \multicolumn{1}{C{0.34\textwidth}}{0.\allowbreak{}13} & \multicolumn{1}{C{0.16\textwidth}}{> 0.\allowbreak{}20} \\\hline \multicolumn{3}{p\textwidth}{\textsuperscript{ } \footnotesize{Source:\allowbreak{} Original compilation.\allowbreak{} Cross-\allowbreak{}sectional study (\allowbreak{}ECM,\allowbreak{} 2008)\allowbreak{}e and National Census of Population and Households,\allowbreak{} INDEC,\allowbreak{} 2001.\allowbreak{}\textsuperscript{g }}}
}
\begin{multicols}{2}
\par{}The employed population groups – those that do not receive from nor contribute to the system of Social Security Retirement – were more prone to nonadherence (\allowbreak{}p = 0.\allowbreak{}007)\allowbreak{}.\allowbreak{} Those who had the subsistence capacity as a deprivation of a basic need also had higher risk of not adhering to the treatment (\allowbreak{}p = 0.\allowbreak{}032)\allowbreak{}.\allowbreak{} The probability of nonadherence increased for households that do not have public transportation network in area within 300 m (\allowbreak{}p = 0.\allowbreak{}070)\allowbreak{}.\allowbreak{} However these results were not statistically significant (\allowbreak{}Table 3)\allowbreak{}.\allowbreak{}\par{}\end{multicols}
\ctable[
  caption = {\textbf{Table 3:} \textit{Multiple regression model for sociodemographic and socioeconomic characteristics related to the dropout proportion in the selected municipalities.\allowbreak{} Sixth Health Region (\allowbreak{}6th HR)\allowbreak{},\allowbreak{} Buenos Aires,\allowbreak{} Argentina,\allowbreak{} 2001.\allowbreak{}}}, 
  width=\textwidth, pos = ht, left, long
]
{p{0.46\textwidth}p{0.18\textwidth}p{0.21\textwidth}p{0.11\textwidth}}
{}
{ \\\hline
\multicolumn{1}{L{0.46\textwidth}}{\textbf{Characteristic}}
 & \multicolumn{1}{L{0.18\textwidth}}{\textbf{Regression coefficient adj}}
 & \multicolumn{1}{L{0.21\textwidth}}{\textbf{95\%\allowbreak{\allowbreak{}\allowbreak{}}\allowbreak{}CI}}
 & \multicolumn{1}{L{0.11\textwidth}}{\textbf{p}} \\\hline 
\multicolumn{1}{L{0.46\textwidth}}{Contribution to retirement:\allowbreak{} do not contribute nor receive it} & \multicolumn{1}{C{0.18\textwidth}}{1,\allowbreak{}068.\allowbreak{}21} & \multicolumn{1}{C{0.21\textwidth}}{300.\allowbreak{}35;1836.\allowbreak{}07} & \multicolumn{1}{C{0.11\textwidth}}{0.\allowbreak{}007} \\\hline \multicolumn{1}{L{0.46\textwidth}}{UBN indicator of subsistence conditions} & \multicolumn{1}{C{0.18\textwidth}}{145.\allowbreak{}18} & \multicolumn{1}{C{0.21\textwidth}}{12.\allowbreak{}56;277.\allowbreak{}79} & \multicolumn{1}{C{0.11\textwidth}}{0.\allowbreak{}032} \\\hline \multicolumn{1}{L{0.46\textwidth}}{Public transportation within 300 m} & \multicolumn{1}{C{0.18\textwidth}}{103.\allowbreak{}06} & \multicolumn{1}{C{0.21\textwidth}}{-\allowbreak{}8.\allowbreak{}70;214.\allowbreak{}81} & \multicolumn{1}{C{0.11\textwidth}}{0.\allowbreak{}070} \\\hline \multicolumn{4}{p\textwidth}{\textsuperscript{ } \footnotesize{Source:\allowbreak{} Original compilation.\allowbreak{} Cross-\allowbreak{}sectional study (\allowbreak{}ECM,\allowbreak{} 2008)\allowbreak{}e and National Census of Population and Households,\allowbreak{} INDEC,\allowbreak{} 2001.\allowbreak{}\textsuperscript{g }}}
}
\begin{multicols}{2}
\par{}Regarding the proportion of the active population that make no contributions to retirement funds,\allowbreak{} we could observe two stripes with lighter areas and a peripheral area demarcated out of it (\allowbreak{}dark areas)\allowbreak{} (\allowbreak{}Figure 2,\allowbreak{} A)\allowbreak{}.\allowbreak{} We also observed a concentration of census fractions of the population that do not make retirement contributions nor receive them.\allowbreak{} The same pattern of distribution was observed in all the region,\allowbreak{} that is,\allowbreak{} dark areas at the peripheral region and lighter color fractions at central areas where is the largest proportion of dropouts.\allowbreak{} Regarding UBN related to subsistence capacity (\allowbreak{}Figure 2,\allowbreak{} B)\allowbreak{},\allowbreak{} the highest percentage of residences with this deprivation is concentrated mainly in two uniform places:\allowbreak{} Lomas de Zamora and Quilmes.\allowbreak{} About the availability of public transportation within 300 m (\allowbreak{}Figure 2,\allowbreak{} C)\allowbreak{},\allowbreak{} we observed three stripes with lighter areas and a periphery demarcated beyond these areas,\allowbreak{} with lower percentages of availability of this service,\allowbreak{} where the proportion of nonadherence was higher,\allowbreak{} predominantly in Lomas de Zamora,\allowbreak{} the border with the capital,\allowbreak{} in the city of Quilmes,\allowbreak{} mainly between Lomas de Zamora and Almirante Brown border,\allowbreak{} and most part of Ezeiza.\allowbreak{}\par{}
\par
{
\centering{
\includegraphics[width=\maxwidth{0.5\textwidth}]{not-found.png}
}
\captionof{figure}{\textbf{Figure 2:} \textit{Distribution of the indicators of the multiple regression model and the dropout proportion.\allowbreak{} Selected municipalities from Sixth Health Region (\allowbreak{}6th HR)\allowbreak{} by census fraction.\allowbreak{} Buenos Aires,\allowbreak{} Argentina,\allowbreak{} 2001.\allowbreak{}}} 
}
\par

\section*{DISCUSSION}
\par{}The areas with the highest proportion of population that do not receive nor contribute to retirement funds had a larger proportion of nonadherence to the treatment.\allowbreak{} That situation was also observed in areas with larger amounts of UBN households according to their subsistence capacity,\allowbreak{} and in the areas with the highest proportion of residences that had no public transport service within 300 m.\allowbreak{} The latter variable was included in the final model,\allowbreak{} albeit its statistical significance were close to the significance level (\allowbreak{}p = 0.\allowbreak{}05)\allowbreak{},\allowbreak{} whereas it is the only variable related to accessibility barriers.\allowbreak{} On the other hand,\allowbreak{} the model that includes only the first two indicators (\allowbreak{}“do not receive benefits from nor make contributions to retirement funds” and “UBN related to subsistence capacity”)\allowbreak{} do not differ on quality of the adjustment to the model,\allowbreak{} which also includes the transportation variable.\allowbreak{}\par{}The analyzed studies of ecological type are important for diagnosing the population health,\allowbreak{} especially when the territory is analyzed in an exploratory way for the verification of the spatial distribution pattern of a particular health event.\allowbreak{}\textsuperscript{\textsuperscript{15}}\textsuperscript{-\allowbreak{}}\textsuperscript{\textsuperscript{23}}\par{}Rose\textsuperscript{\textsuperscript{19}} has claimed that two aspects should be considered in the etiology of health problems:\allowbreak{} the causes of individual cases and the determinants of disease rates among populations.\allowbreak{} In this sense,\allowbreak{} although strategies are essential for the individual risk prevention and protection of individuals susceptible to treatment dropout,\allowbreak{} identifying the dropout determinants among populations is particularly relevant to the control of the disease.\allowbreak{}\textsuperscript{\textsuperscript{23}} The analysis of risk variability at the ecological level is essential for understanding the social determinants of health and diseases and allows for the investigation of the hypothesis that the distribution of nonadherence in an area is related to the living conditions.\allowbreak{}\textsuperscript{\textsuperscript{23}}\par{}The ability to finish the treatment is influenced by the living conditions of the area in which the TB patients live.\allowbreak{} The proportion of nonadherence was higher in areas with the highest proportion of residences that do not have a public transport network in within 300 m,\allowbreak{} indicating difficulties related to access and mobility of the population.\allowbreak{} This indicator may be a proxy for other features of the area related to the availability of resources and services.\allowbreak{} The highest dropout rate was observed in areas with low levels of piped water and paved streets.\allowbreak{} The reason why these indicators have not been included in the final model may be due to the small number of cases that have or not joined the study.\allowbreak{}\par{}The results indicate higher dropout proportion in areas with the highest proportion of households with poor conditions and lower level of resources,\allowbreak{} as is the case of residences that have no refrigerator,\allowbreak{} or have water-\allowbreak{}closet without flushing system,\allowbreak{} or do not have water-\allowbreak{}closet.\allowbreak{} These results indicate a lower socioeconomic level in these areas.\allowbreak{}\par{}The association between socioeconomic level and nonadherence to TB treatment has been analyzed in different countries and regions.\allowbreak{}\textsuperscript{\textsuperscript{6}}\textsuperscript{,\allowbreak{}}\textsuperscript{\textsuperscript{8}}\textsuperscript{,\allowbreak{}}\textsuperscript{\textsuperscript{10}}\textsuperscript{,\allowbreak{}}\textsuperscript{\textsuperscript{16}}\textsuperscript{,\allowbreak{}}\textsuperscript{\textsuperscript{22}} In our study,\allowbreak{} the fact that tuberculosis has free treatment suggests that different factors related to treatment costs determine the nonadherence of patients of low socioeconomic level.\allowbreak{}\par{}There are other characteristics of the population associated with the largest dropout proportion,\allowbreak{} which are related to an increase in socioeconomic status of vulnerability in households with higher job insecurity,\allowbreak{} with lower levels of formal education of the head of household,\allowbreak{} and a lower number of individuals employed per household.\allowbreak{} The areas that had a higher percentage of households with four or more people per family member employed and which head of household has not completed the third grade of Elementary School (\allowbreak{}UNB per subsistence capacity)\allowbreak{} had a higher proportion of nonadherence.\allowbreak{} This indicator is a proxy for the level of household income according to the quantity of family members employed in relation to all members that are part of the family unit.\allowbreak{} For its turn,\allowbreak{} this indicator also measures the lack of goods and services that are necessary to live and for an individual to feel part of the society based on a conception of poverty as “deprivation”.\allowbreak{}\protect\footnote{ Feres JC,\allowbreak{} Mancero X.\allowbreak{} Enfoques para la medición de la pobreza.\allowbreak{} Breve revisión de la literatura.\allowbreak{} Santiago:\allowbreak{} CEPAL; 2001.\allowbreak{}} These results are consistent with other studies that indicate that low income households are associated with the worst health outcomes.\allowbreak{}\protect\footnote{ Acosta LSW.\allowbreak{} O mapa de Porto Alegre e a tuberculose:\allowbreak{} distribuição espacial e determinantes sociais [dissertation].\allowbreak{} Porto Alegre (\allowbreak{}RS)\allowbreak{}:\allowbreak{} Faculdade de Medicina da UFRGS; 2008.\allowbreak{}} On the other hand,\allowbreak{} this UBN indicator includes the education level of the head of household.\allowbreak{} The studies that include the educational level in their analysis found a statistically significant association with adherence in the areas with a population with lower educational level.\allowbreak{}\protect\footnote{ Acosta LSW.\allowbreak{} O mapa de Porto Alegre e a tuberculose:\allowbreak{} distribuição espacial e determinantes sociais [dissertation].\allowbreak{} Porto Alegre (\allowbreak{}RS)\allowbreak{}:\allowbreak{} Faculdade de Medicina da UFRGS; 2008.\allowbreak{}} We have observed an increase in the proportion of nonadherence in areas where the population has Incomplete Elementary School.\allowbreak{} Several studies have shown that education can influence health practices of a population in its association with income level,\allowbreak{} employment conditions,\allowbreak{} as well as in its association with the level of knowledge that people have about these practices.\allowbreak{}\textsuperscript{\textsuperscript{21}}\par{}The areas with the highest proportion of individuals employed without social protection have a higher percentage of nonadherence.\allowbreak{} Several studies have pointed out the influence of patients working conditions on the nonadherence to treatment.\allowbreak{}\textsuperscript{\textsuperscript{17}} Thus,\allowbreak{} employment reduces the ability to follow the treatment in the context of high rates of informal employment and low-\allowbreak{}income without social protection,\allowbreak{} since that means for patients losing working days and income,\allowbreak{} i.\allowbreak{}e.\allowbreak{},\allowbreak{} basic income,\allowbreak{} as shown by Balasubramanian et al.\allowbreak{}\textsuperscript{\textsuperscript{2}} In the study by Galiano \&\allowbreak{\allowbreak{}\allowbreak{}}\allowbreak{} Montesinos,\allowbreak{}\textsuperscript{\textsuperscript{11}} the highest rate of dropouts was also associated with the condition of being male,\allowbreak{} employed and without social protection.\allowbreak{}\par{}A limitation of this study is the use of the data from the 2001 National Census of Population and Households,\allowbreak{}\protect\footnote{ According to the National Institute of Statistics and Censuses (\allowbreak{}INDEC)\allowbreak{},\allowbreak{} and the Index of Household Material Privation (\allowbreak{}IHMP)\allowbreak{} it is a variable that identifies the residences according to their material deprivation in two dimensions – material and patrimonial resources.\allowbreak{} In relation to the Unsatisfied Basic Needs (\allowbreak{}UBN)\allowbreak{},\allowbreak{} the households with this characteristic have at least one of the following indicators of deprivation:\allowbreak{} overcrowding (\allowbreak{}more than 3 persons per room)\allowbreak{}; housing (\allowbreak{}living in an improper location [leased place,\allowbreak{} hotel or pension room,\allowbreak{} shack,\allowbreak{} place without rooms],\allowbreak{} not considering house,\allowbreak{} apartment and farm)\allowbreak{}; health conditions (\allowbreak{}without water-\allowbreak{}closet)\allowbreak{}; school attendance (\allowbreak{}with at least one school-\allowbreak{}age child [6-\allowbreak{}12 years] who do not attend school)\allowbreak{}; subsistence capacity (\allowbreak{}with four or more individuals per family unit,\allowbreak{} whose responsible has not concluded the third grade of Elementary School)\allowbreak{}.\allowbreak{}} since the up-\allowbreak{}to-\allowbreak{}date socioeconomic indicators were not available in the period in which the nonadherence cases were evaluated.\allowbreak{} This is the first study conducted in Argentina with the purpose of detecting spatial distribution patterns of demographic and socioeconomic characteristics of the region and its relation with treatment dropout.\allowbreak{} The technological advances of geographic information systems (\allowbreak{}GIS)\allowbreak{} have allowed the incorporation of spatial structure of variables,\allowbreak{} with the advantage of analyzing the problem as a particularized social phenomenon in socioeconomic and sociodemographic conditions.\allowbreak{}\textsuperscript{\textsuperscript{15}}\textsuperscript{,\allowbreak{}}\textsuperscript{\textsuperscript{23}} The GIS and analysis of secondary data (\allowbreak{}census)\allowbreak{} are useful tools to identify the areas and conditions relevant to the process of nonadherence to TB treatment and for the monitoring based on territory to identify the preferential population groups.\allowbreak{}\par{}The increase in the proportion of nonadherence to TB treatment in 6\textsuperscript{th} HR is related to residences in areas with the highest proportion of active people that do not contribute to retirement funds,\allowbreak{} higher proportion of households deprived of basic needs (\allowbreak{}per subsistence capacity)\allowbreak{},\allowbreak{} and higher proportion of households that do not have public transportation within 300 m.\allowbreak{} Our results lead us to establish a risk area to the nonadherence to treatment,\allowbreak{} characterized by a greater proportion of population living in poverty,\allowbreak{} poor working conditions,\allowbreak{} and with difficult access to public transportation.\allowbreak{} This information may result in the adoption of appropriate measures for the treatment of individuals that live in these areas and to reduce the risk of dropout.\allowbreak{}

\medskip\par\noindent
{}The authors declare no conflict of interest.\allowbreak{}
\begin{biblio}[REFERENCES]
\bibliotitle{.\allowbreak{}Arrossi S,}\allowbreak{} Herrero MB,\allowbreak{} Greco A,\allowbreak{} Ramos S.\allowbreak{} Factores predictivos de la no-\allowbreak{}adherencia al tratamiento de la tuberculosis en municipios del área metropolitana de Buenos Aires,\allowbreak{} Argentina.\allowbreak{} \textit{Salud Colectiva.\allowbreak{} }2012;8(\allowbreak{}1)\allowbreak{}:\allowbreak{}65-\allowbreak{}76.\allowbreak{} DOI:\allowbreak{}10.\allowbreak{}1590\fshyp{}S1851-\allowbreak{}82652012000300012
\bibliotitle{.\allowbreak{}Balasubramanian VN,}\allowbreak{} Oommen K,\allowbreak{} Samuel R.\allowbreak{} DOT or not? Direct observation of anti-\allowbreak{}tuberculosis treatment and patient outcomes,\allowbreak{} Kerala State,\allowbreak{} India.\allowbreak{} \textit{The Int J Tuberc Lung Dis.\allowbreak{}} 2000;4(\allowbreak{}5)\allowbreak{}:\allowbreak{}409-\allowbreak{}13.\allowbreak{}
\bibliotitle{.\allowbreak{}Barata }RB.\allowbreak{} Epidemiologia social.\allowbreak{} \textit{Rev Bras Epidemiol.\allowbreak{} }2005;8(\allowbreak{}1)\allowbreak{}:\allowbreak{}7-\allowbreak{}17.\allowbreak{} DOI:\allowbreak{}10.\allowbreak{}1590\fshyp{}S1415-\allowbreak{}790X2005000100002
\bibliotitle{.\allowbreak{}Braga JU,}\allowbreak{} Pinheiro JS,\allowbreak{} Matsuda JS,\allowbreak{} Barreto JAP,\allowbreak{} Feijão AMM.\allowbreak{} Fatores associados ao abandono do tratamento da tuberculose nos serviços de atenção básica em dois municípios brasileiros,\allowbreak{} Manaus e Fortaleza,\allowbreak{} 2006 a 2008.\allowbreak{} \textit{Cad Saude Colet.\allowbreak{}} 2012;20(\allowbreak{}2)\allowbreak{}:\allowbreak{}225-\allowbreak{}33.\allowbreak{}
\bibliotitle{.\allowbreak{}Brasil PE,}\allowbreak{} Braga JU.\allowbreak{} Meta-\allowbreak{}analysis of factors related to health services that predict treatment default by tuberculosis patients.\allowbreak{} \textit{Cad Saude Publica.\allowbreak{}} 2008;24(\allowbreak{}4)\allowbreak{}:\allowbreak{}485-\allowbreak{}502.\allowbreak{} DOI:\allowbreak{}10.\allowbreak{}1590\fshyp{}S0102-\allowbreak{}311X2008001600003
\bibliotitle{.\allowbreak{}Cáceres FM,}\allowbreak{} Orozco LC.\allowbreak{} Incidencia y factores asociados al abandono del tratamiento antituberculoso.\allowbreak{} \textit{Rev Biomedica}.\allowbreak{} 2007;27(\allowbreak{}4)\allowbreak{}:\allowbreak{}498-\allowbreak{}504.\allowbreak{} DOI:\allowbreak{}10.\allowbreak{}7705\fshyp{}biomedica.\allowbreak{}v27i4.\allowbreak{}170
\bibliotitle{.\allowbreak{}Center }for Diseases Control.\allowbreak{} Core curriculum on tuberculosis:\allowbreak{} What the clinician should know.\allowbreak{} 4.\allowbreak{} ed.\allowbreak{} Atlanta:\allowbreak{} U.\allowbreak{}S.\allowbreak{} Department of Health \&\allowbreak{\allowbreak{}\allowbreak{}}\allowbreak{} Human Services; 2000.\allowbreak{}
\bibliotitle{.\allowbreak{}Culqui DR,}\allowbreak{} Grijalva CG,\allowbreak{} Reategui SR,\allowbreak{} Cajo JM,\allowbreak{} Suárez LA.\allowbreak{} Factores pronósticos del abandono del tratamiento antituberculoso en una región endémica del Perú.\allowbreak{} \textit{Rev Panam Salud Publica.\allowbreak{} }2005;18(\allowbreak{}1)\allowbreak{}:\allowbreak{}14-\allowbreak{}20.\allowbreak{} DOI:\allowbreak{}10.\allowbreak{}1590\fshyp{}S1020-\allowbreak{}49892005000600003
\bibliotitle{.\allowbreak{}Diez }Roux AV.\allowbreak{} A glossary for multilevel analysis.\allowbreak{} \textit{J Epidemiol Community Health.\allowbreak{} }2002;56(\allowbreak{}8)\allowbreak{}:\allowbreak{}588-\allowbreak{}94.\allowbreak{} DOI:\allowbreak{}10.\allowbreak{}1136\fshyp{}jech.\allowbreak{}56.\allowbreak{}8.\allowbreak{}588
\bibliotitle{.\allowbreak{}Galdós-\allowbreak{}Tangüis H,}\allowbreak{} Caylá JÁ,\allowbreak{} García de Olalla P,\allowbreak{} Jansá JM,\allowbreak{} Brugal MT.\allowbreak{} Factors predicting non-\allowbreak{}completion of tuberculosis treatment among HIV-\allowbreak{}infected patients in Barcelona (\allowbreak{}1987-\allowbreak{}1996)\allowbreak{}.\allowbreak{} \textit{Int J Tuberc Lung Dis.\allowbreak{} }2000;4(\allowbreak{}1)\allowbreak{}:\allowbreak{}55-\allowbreak{}60.\allowbreak{}
\bibliotitle{.\allowbreak{}Galiano M,}\allowbreak{} Montesinos N.\allowbreak{} Modelo predictivo de abandono del tratamiento antituberculoso para la región Metropolitana de Chile.\allowbreak{} \textit{Enferm Clin.\allowbreak{} }2005;15(\allowbreak{}4)\allowbreak{}:\allowbreak{}192-\allowbreak{}8.\allowbreak{} DOI:\allowbreak{}10.\allowbreak{}1016\fshyp{}S1130-\allowbreak{}8621(\allowbreak{}05)\allowbreak{}71111-\allowbreak{}6
\bibliotitle{.\allowbreak{}Gonçalves MJ,}\allowbreak{} Leon AC,\allowbreak{} Penna ML.\allowbreak{} A multilevel analysis of tuberculosis associated factors.\allowbreak{} \textit{Rev Salud Publica (\allowbreak{}Bogota)\allowbreak{}.\allowbreak{}} 2009;11(\allowbreak{}6)\allowbreak{}:\allowbreak{}918-\allowbreak{}30.\allowbreak{}
\bibliotitle{.\allowbreak{}Herrero MB,}\allowbreak{} Greco A,\allowbreak{} Ramos S,\allowbreak{} Arrossi S.\allowbreak{} Del riesgo individual a la vulnerabilidad social:\allowbreak{} factores asociados a la no-\allowbreak{}adherencia al tratamiento de la tuberculosis.\allowbreak{} Revisión de la literatura.\allowbreak{} \textit{Rev Argent Salud Publica.\allowbreak{} }2011;2(\allowbreak{}8)\allowbreak{}:\allowbreak{}36-\allowbreak{}42.\allowbreak{}
\bibliotitle{.\allowbreak{}Krieger }N.\allowbreak{} Glosario de epidemiología social.\allowbreak{} \textit{Rev Panam Salud Publica.\allowbreak{} }2002;11(\allowbreak{}5-\allowbreak{}6)\allowbreak{}:\allowbreak{}480-\allowbreak{}90.\allowbreak{} DOI:\allowbreak{}10.\allowbreak{}1590\fshyp{}S1020-\allowbreak{}49892002000500028
\bibliotitle{.\allowbreak{}Lima MLC,}\allowbreak{} Ximenes RA,\allowbreak{} Souza ER,\allowbreak{} Luna CF,\allowbreak{} Albuquerque MFPM.\allowbreak{} Análise espacial dos determinantes socioeconômicos dos homicídios no estado no Pernambuco.\allowbreak{} \textit{Rev Saude Publica.\allowbreak{} }2005;39(\allowbreak{}2)\allowbreak{}:\allowbreak{}176-\allowbreak{}82.\allowbreak{} DOI:\allowbreak{}10.\allowbreak{}1590\fshyp{}S0034-\allowbreak{}89102005000200006
\bibliotitle{.\allowbreak{}Mishra P,}\allowbreak{} Hansen E,\allowbreak{} Sabroe S,\allowbreak{} Kafle K.\allowbreak{} Socio-\allowbreak{}economic status and adherence to tuberculosis treatment:\allowbreak{} a case-\allowbreak{}control study in a district of Nepal.\allowbreak{} \textit{Int J Tuberc Lung Dis.\allowbreak{} }2005;9(\allowbreak{}10)\allowbreak{}:\allowbreak{}1134-\allowbreak{}9.\allowbreak{}
\bibliotitle{.\allowbreak{}Nene B,}\allowbreak{} Jayant K,\allowbreak{} Arrossi S,\allowbreak{} Shastri S,\allowbreak{} Budukh A,\allowbreak{} Hingmire S,\allowbreak{} et al.\allowbreak{} Determinants of women’s participation in cervical cancer screening trial,\allowbreak{} Maharashtra,\allowbreak{} India.\allowbreak{} \textit{Bull World Health Organ}.\allowbreak{} 2007;85(\allowbreak{}4)\allowbreak{}:\allowbreak{}264-\allowbreak{}72.\allowbreak{} DOI:\allowbreak{}10.\allowbreak{}2471\fshyp{}BLT.\allowbreak{}06.\allowbreak{}031195
\bibliotitle{.\allowbreak{}O’Boyle S,}\allowbreak{} Power J,\allowbreak{} Ibrahim M,\allowbreak{} Watson J.\allowbreak{} Factors affecting patient compliance with anti-\allowbreak{}tuberculosis chemotherapy using the directly observed treatment short-\allowbreak{}course strategy (\allowbreak{}DOTS)\allowbreak{}.\allowbreak{} \textit{Int J Tuberc Lung Dis.\allowbreak{} }2002;6(\allowbreak{}4)\allowbreak{}:\allowbreak{}307-\allowbreak{}12.\allowbreak{}
\bibliotitle{.\allowbreak{}Rose }G.\allowbreak{} Sick individuals and sick populations.\allowbreak{} \textit{Int J Epidemiol.\allowbreak{} }2001;30(\allowbreak{}3)\allowbreak{}:\allowbreak{} 427-\allowbreak{}32.\allowbreak{} DOI:\allowbreak{}10.\allowbreak{}1093\fshyp{}ije\fshyp{}30.\allowbreak{}3.\allowbreak{}427
\bibliotitle{.\allowbreak{}Sabaté }E.\allowbreak{} Adherencia a los tratamientos a largo plazo:\allowbreak{} pruebas para la acción.\allowbreak{} Geneva:\allowbreak{} Organización Mundial de la Salud; Organización Panamericana de la Salud; 2004.\allowbreak{}
\bibliotitle{.\allowbreak{}Sabates R,}\allowbreak{} Feinstein L.\allowbreak{} The role of education in the uptake of preventative health care:\allowbreak{} the case of cervical screening in Britain.\allowbreak{} \textit{Soc Sci Med.\allowbreak{} }2006;62(\allowbreak{}12)\allowbreak{}:\allowbreak{} 2998-\allowbreak{}3010.\allowbreak{} DOI:\allowbreak{}10.\allowbreak{}1016\fshyp{}j.\allowbreak{}socscimed.\allowbreak{}2005.\allowbreak{}11.\allowbreak{}032
\bibliotitle{.\allowbreak{}Sosa Pineda N,}\allowbreak{} Pereira S,\allowbreak{} Barreto M.\allowbreak{} Abandono del tratamiento de la tuberculosis en Nicaragua:\allowbreak{} resultados de un estudio comparativo.\allowbreak{} \textit{Rev Panam Salud Publica.\allowbreak{} }2005;17(\allowbreak{}4)\allowbreak{}:\allowbreak{}271-\allowbreak{}8.\allowbreak{} DOI:\allowbreak{}10.\allowbreak{}1590\fshyp{}S1020-\allowbreak{}49892005000400008
\bibliotitle{.\allowbreak{}Souza WV,}\allowbreak{} Albuquerque MFPM,\allowbreak{} Barcelos CC,\allowbreak{} Ximenes RA,\allowbreak{} Carvalho MS.\allowbreak{} Tuberculose no Brasil:\allowbreak{} construção de um sistema de vigilância de base territorial.\allowbreak{} \textit{Rev Saude Publica}.\allowbreak{} 2005;39(\allowbreak{}1)\allowbreak{}:\allowbreak{}82-\allowbreak{}89.\allowbreak{} DOI:\allowbreak{}10.\allowbreak{}1590\fshyp{}S0034-\allowbreak{}89102005000100011
\bibliotitle{.\allowbreak{}Ximenes RAA,}\allowbreak{} Albuquerque MFPM,\allowbreak{} Souza WV,\allowbreak{} Montarroyos UR,\allowbreak{} Diniz GT,\allowbreak{} Luna CF.\allowbreak{} Is it better to be rich in a poor area or poor in a rich area? A multilevel analysis of a case-\allowbreak{}control study of social determinants of tuberculosis.\allowbreak{} \textit{Int J Epidemiol.\allowbreak{}} 2009;38(\allowbreak{}5)\allowbreak{}:\allowbreak{}1285-\allowbreak{}96.\allowbreak{}
\bibliotitle{.\allowbreak{}World }Health Organization.\allowbreak{} Treatment of tuberculosis:\allowbreak{} guidelines for national programmes.\allowbreak{} 3.\allowbreak{} ed.\allowbreak{} Geneva; 2003.\allowbreak{}
\end{biblio}

\medskip\par\noindent
\footnotesize{This is an Open Access article distributed under the terms of the Creative Commons Attribution Non-Commercial License, which permits unrestricted non-commercial use, distribution, and reproduction in any medium, provided the original work is properly cited.}
\end{multicols}
