\selectlanguage{spanish}
\removeabstracttitle

{
\begin{abstract}
 \textbf{OBJETIVO:}
 Identificar patrones de distribución espacial de la proporción de la no-\allowbreak{}adherencia al tratamiento de la tuberculosis y sus factores asociados.\allowbreak{}
\par{} \textbf{METODOS:}
 Estudio ecológico con datos secundarios y primarios en municipios seleccionados del Área Metropolitana de Buenos Aires.\allowbreak{} Se realizó un análisis exploratorio de las características del área y de las distribuciones de los casos incluidos en la muestra (\allowbreak{}proporción de no-\allowbreak{}adherencia)\allowbreak{} y un análisis de múltiples factores por regresión lineal.\allowbreak{} Se analizaron variables referidas a las características de la población,\allowbreak{} las viviendas y los hogares.\allowbreak{}
\par{} \textbf{RESULTADOS:}
 Las áreas con mayor proporción de población que no realizaba aportes jubilatorios (\allowbreak{}p = 0,\allowbreak{}007)\allowbreak{} y con mayor proporción de hogares con necesidades básicas insatisfechas según capacidad de subsistencia presentaron mayor riesgo de no-\allowbreak{}adherencia (\allowbreak{}p = 0,\allowbreak{}032)\allowbreak{}.\allowbreak{} La proporción de no-\allowbreak{}adherencia fue más elevada en las áreas con mayor proporción de viviendas sin servicio de transporte público a menos de 300 m (\allowbreak{}p = 0,\allowbreak{}070)\allowbreak{}.\allowbreak{}
\par{} \textbf{CONCLUSIONES:}
 Existe un área de riesgo para la no-\allowbreak{}adherencia al tratamiento,\allowbreak{} caracterizada por tener una población que vive en condiciones de pobreza y precariedad laboral,\allowbreak{} con dificultades de acceso al servicio de transporte público.\allowbreak{}


\vspace*{4mm}
\fontsize{9}{10.8}\selectfont{\textbf{Keywords:} \textit{Tuberculosis,\allowbreak{} quimioterapia, Cumplimiento de la Medicación, Factores Socioeconómicos, Desigualdades en la Salud, Estudios Ecológicos}}
\end{abstract}
}


{
\begin{abstract}
 \textbf{OBJECTIVE:}
 Identify spatial distribution patterns of the proportion of nonadherence to tuberculosis treatment and its associated factors.\allowbreak{}
\par{} \textbf{METHODS:}
 We conducted an ecological study based on secondary and primary data from municipalities of the metropolitan area of Buenos Aires,\allowbreak{} Argentina.\allowbreak{} An exploratory analysis of the characteristics of the area and the distributions of the cases included in the sample (\allowbreak{}proportion of nonadherence)\allowbreak{} was also carried out along with a multifactor analysis by linear regression.\allowbreak{} The variables related to the characteristics of the population,\allowbreak{} residences and families were analyzed.\allowbreak{}
\par{} \textbf{RESULTS:}
 Areas with higher proportion of the population without social security benefits (\allowbreak{}p = 0.\allowbreak{}007)\allowbreak{} and of households with unsatisfied basic needs had a higher risk of nonadherence (\allowbreak{}p = 0.\allowbreak{}032)\allowbreak{}.\allowbreak{} In addition,\allowbreak{} the proportion of nonadherence was higher in areas with the highest proportion of households with no public transportation within 300 meters (\allowbreak{}p = 0.\allowbreak{}070)\allowbreak{}.\allowbreak{}
\par{} \textbf{CONCLUSIONS:}
 We found a risk area for the nonadherence to treatment characterized by a population living in poverty,\allowbreak{} with precarious jobs and difficult access to public transportation.\allowbreak{}


\vspace*{4mm}
\fontsize{9}{10.8}\selectfont{\textbf{Keywords:} \textit{Tuberculosis, drug therapy, Medication Adherence, Socioeconomic Factors, Health Inequalities, Ecological Studies}}
\end{abstract}
}

\begin{multicols}{2}
\section*{INTRODUCCIÓN}
\par{}Si bien la tuberculosis (\allowbreak{}TB)\allowbreak{} es una enfermedad curable y prevenible,\allowbreak{} es un importante problema de salud pública en la Argentina.\allowbreak{} Ocurren más de 10.\allowbreak{}000 nuevos casos y más de 800 muertes por esta enfermedad cada año.\allowbreak{} La distribución geográfica de la TB en el país no es uniforme como sucede mundialmente.\allowbreak{}\protect\footnote{ Instituto Nacional de Enfermedades Respiratorias “Dr.\allowbreak{} Emilio Coni”.\allowbreak{} Notificación de casos de tuberculosis en la República Argentina.\allowbreak{} Período 1980-\allowbreak{}2011.\allowbreak{} Buenos Aires:\allowbreak{} Ministerio de Salud; 2012.\allowbreak{}} La no-\allowbreak{}adherencia al tratamiento es considerada una de las principales barreras para el control de la enfermedad,\allowbreak{} debido a las consecuencias de su interrupción,\allowbreak{} y está asociada a la vulnerabilidad social de los pacientes.\allowbreak{}\textsuperscript{\textsuperscript{7}}\par{}La TB persiste como problema de salud pública a pesar del bajo costo del diagnóstico y tratamiento.\allowbreak{} Estas medidas integran la estrategia de tratamiento directamente observado de corta duración (\allowbreak{}DOTS)\allowbreak{},\allowbreak{} recomendada por la Organización Mundial de la Salud (\allowbreak{}OMS)\allowbreak{} para reducir la no-\allowbreak{}adherencia al tratamiento.\allowbreak{}\textsuperscript{\textsuperscript{25}} Estas fueron adoptadas en Argentina e implementadas por medio del Programa Nacional de Control de la Tuberculosis (\allowbreak{}PNCTB)\allowbreak{}.\allowbreak{}\protect\footnote{ Zerbini EV,\allowbreak{} Darnaud RMH,\allowbreak{} Prieto VG.\allowbreak{} Programa Nacional de Control de la Tuberculosis:\allowbreak{} Normas Técnicas 2008.\allowbreak{} 3.\allowbreak{} ed.\allowbreak{} Santa Fé:\allowbreak{} Instituto Nacional de Enfermedades Respiratorias Dr.\allowbreak{} Emilio Coni; 2008.\allowbreak{}} Aunque la implementación de la estrategia DOTS en el país se lleva a cabo hace 10 años,\allowbreak{} la proporción de casos que abandonó el tratamiento fue del 12,\allowbreak{}0\%\allowbreak{\allowbreak{}\allowbreak{}}\allowbreak{} en 2010,\allowbreak{} una de las más elevadas en los últimos años.\allowbreak{}\protect\footnote{ Instituto Nacional de Enfermedades Respiratorias “Dr.\allowbreak{} Emilio Coni”.\allowbreak{} Resultado del tratamiento de la tuberculosis pulmonar ED(\allowbreak{}+\allowbreak{})\allowbreak{} en la República Argentina.\allowbreak{} Período 1980-\allowbreak{}2010.\allowbreak{} Buenos Aires:\allowbreak{} Ministerio de Salud; 2012.\allowbreak{}}\par{}Estudios abordan la adherencia al tratamiento desde un enfoque basado en los factores ambientales\textsuperscript{\textsuperscript{12}}\textsuperscript{,\allowbreak{}}\textsuperscript{\textsuperscript{24}} y los individuales relacionados con el paciente.\allowbreak{}\textsuperscript{\textsuperscript{1}}\textsuperscript{,\allowbreak{}}\textsuperscript{\textsuperscript{4}}\textsuperscript{,\allowbreak{}}\textsuperscript{\textsuperscript{5}}\textsuperscript{,\allowbreak{}}\textsuperscript{\textsuperscript{13}}\textsuperscript{,\allowbreak{}}\textsuperscript{\textsuperscript{16}}\textsuperscript{,\allowbreak{}}\textsuperscript{\textsuperscript{18}}\textsuperscript{,\allowbreak{}}\textsuperscript{\textsuperscript{22}}\par{}Tanto la ocurrencia de TB como sus consecuencias sobre la salud están relacionados con las condiciones sociales de vida.\allowbreak{}\textsuperscript{\textsuperscript{2}}\textsuperscript{0} Comprender su comportamiento en un territorio y sus determinantes es esencial para el establecimiento de acciones equitativas tendientes a disminuir las inequidades y mejorar la adherencia al tratamiento.\allowbreak{}\protect\footnote{ Acosta LSW.\allowbreak{} O mapa de Porto Alegre e a tuberculose:\allowbreak{} distribuição espacial e determinantes sociais [dissertação de mestrado].\allowbreak{} Porto Alegre (\allowbreak{}RS)\allowbreak{}:\allowbreak{} Faculdade de Medicina da UFRGS; 2008.\allowbreak{}} Estudios ecológicos buscan identificar,\allowbreak{} en las características sociales y del área,\allowbreak{} las relaciones con la distribución de las enfermedades y los resultados en salud,\allowbreak{} respetando los diferentes niveles jerárquicos de los determinantes.\allowbreak{}\textsuperscript{\textsuperscript{3}}\textsuperscript{,\allowbreak{}}\textsuperscript{\textsuperscript{9}}\textsuperscript{,\allowbreak{}}\textsuperscript{\textsuperscript{14}}\par{}A pesar de la importancia de este tipo de estudios,\allowbreak{} no se han encontrado en Argentina estudios sobre las características de los grupos sociales y del área donde viven y su relación con la no-\allowbreak{}adherencia al tratamiento de la TB.\allowbreak{}\par{}Este estudio tuvo como objetivo identificar patrones de distribución espacial de la proporción de la no-\allowbreak{}adherencia al tratamiento de la tuberculosis y sus factores asociados.\allowbreak{}
\section*{METODOS}
\par{}Estudio ecológico espacial en siete municipios de la Región Sanitaria Sexta (\allowbreak{}RSVI)\allowbreak{} del Área Metropolitana de Buenos Aires (\allowbreak{}AMBA)\allowbreak{} (\allowbreak{}donde existen 116 fracciones censales – Figura 1)\allowbreak{}:\allowbreak{} Almirante Brown,\allowbreak{} Avellaneda,\allowbreak{} Berazategui,\allowbreak{} Esteban Echeverría,\allowbreak{} Ezeiza,\allowbreak{} Lomas de Zamora y Quilmes.\allowbreak{} Los dos municipios restantes que integran la RSVI (\allowbreak{}Lanús y Florencio Varela)\allowbreak{} no pudieron ser incluidos puesto que no disponían de Comité de Ética para la evaluación del protocolo del estudio de corte transversal de donde provienen los casos (\allowbreak{}adherentes y no adherentes)\allowbreak{} georreferenciados.\allowbreak{}\protect\footnote{ Arrossi S,\allowbreak{} Herrero MB,\allowbreak{} Faccia K,\allowbreak{} Greco A,\allowbreak{} Ramirez Lijó S,\allowbreak{} Aizemberg L et al.\allowbreak{} Evaluación de los factores predictivos de la no-\allowbreak{}adherencia al tratamiento de la tuberculosis en municipios seleccionados del área metropolitana de Buenos Aires:\allowbreak{} estudio colaborativo multicéntrico.\allowbreak{} Buenos Aires:\allowbreak{} Ministerio de Salud de la Nación; 2008 (\allowbreak{}ECM 2008)\allowbreak{}.\allowbreak{}}\par{}
\par
{
\centering{
\includegraphics[width=\maxwidth{0.5\textwidth}]{not-found.png}
}
\captionof{figure}{\textbf{Figura 1:} \textit{Área de estudio:\allowbreak{} municipios seleccionados de la Región Sanitaria VI (\allowbreak{}RSVI)\allowbreak{} y fracciones censales.\allowbreak{} Buenos Aires,\allowbreak{} Argentina,\allowbreak{} 2001.\allowbreak{}}} 
}
\par
\par{}La RSVI tiene aproximadamente 3.\allowbreak{}653.\allowbreak{}000 habitantes,\allowbreak{} la más poblada de la provincia de Buenos Aires.\allowbreak{}\protect\footnote{ Ministerio de Salud de la Provincia de Buenos Aires (\allowbreak{}ARG)\allowbreak{}.\allowbreak{} Diagnóstico de las Regiones Sanitarias 2007-\allowbreak{}2008,\allowbreak{} La Plata,\allowbreak{} Buenos Aires; 2008.\allowbreak{}} Concentra el 13,\allowbreak{}0\%\allowbreak{\allowbreak{}\allowbreak{}}\allowbreak{} del total de casos de TB notificados en el país.\allowbreak{} Es la región sanitaria que notifica más casos de TB en la provincia anualmente,\allowbreak{} con la más elevada tasa de abandono (\allowbreak{}25,\allowbreak{}0\%\allowbreak{\allowbreak{}\allowbreak{}}\allowbreak{})\allowbreak{} y la más baja cobertura de DOTS (\allowbreak{}12,\allowbreak{}0\%\allowbreak{\allowbreak{}\allowbreak{}}\allowbreak{})\allowbreak{}.\allowbreak{}\protect\footnote{ Instituto Nacional de Enfermedades Respiratorias “Dr.\allowbreak{} Emilio Coni”.\allowbreak{} Resultado del tratamiento de la tuberculosis pulmonar ED(\allowbreak{}+\allowbreak{})\allowbreak{} en la República Argentina.\allowbreak{} Período 1980-\allowbreak{}2010.\allowbreak{} Buenos Aires:\allowbreak{} Ministerio de Salud; 2012.\allowbreak{}}\par{}Se utilizó el banco de datos y la cartografía del Censo Nacional de Población y Viviendas 2001 del Instituto Nacional de Estadísticas y Censos (\allowbreak{}INDEC)\allowbreak{} como fuente de datos secundarios.\allowbreak{}\protect\footnote{ INDEC.\allowbreak{} Instituto Nacional de Estadísticas y Censos [Argentina].\allowbreak{} Buenos Aires; 2001 [citado 2015 mai 21].\allowbreak{} Disponible en:\allowbreak{} \textit{http:\allowbreak{}\fshyp{}\fshyp{}www.\allowbreak{}indec.\allowbreak{}gov.\allowbreak{}ar}} Se georreferenciaron todos los casos notificados,\allowbreak{} con residencia en los municipios seleccionados en la RSVI y tratados en los servicios de salud de dichos municipios en el año 2007.\allowbreak{} Esto fue posible dado que estos individuos participaron de estudio realizado para identificar los factores predictivos de la no-\allowbreak{}adherencia al tratamiento de tuberculosis en dichos municipios.\allowbreak{}\textsuperscript{\textsuperscript{1}} Se realizó el cálculo de la proporción de no-\allowbreak{}adherencia al tratamiento de la TB para las fracciones censales (\allowbreak{}unidades de análisis de este estudio)\allowbreak{} de los municipios de la RSVI.\allowbreak{}\par{}Las informaciones fueron agrupadas en tres tipos de indicadores de acuerdo a la clasificación del Censo.\allowbreak{}\protect\footnote{ INDEC.\allowbreak{} Instituto Nacional de Estadísticas y Censos [Argentina].\allowbreak{} Buenos Aires; 2001 [citado 2015 mai 21].\allowbreak{} Disponible en:\allowbreak{} \textit{http:\allowbreak{}\fshyp{}\fshyp{}www.\allowbreak{}indec.\allowbreak{}gov.\allowbreak{}ar}} Las características del área fueron consideradas según la existencia de proporción de:\allowbreak{} cloacas; energía eléctrica por red domiciliaria; gas de red; al menos una cuadra pavimentada; servicio regular de recolección de residuos al menos dos veces por semana; transporte público a menos de 300 m.\allowbreak{} Se consideró la proporción de viviendas según tipo de material predominante de los pisos; tipo de sistema de provisión de agua; tenencia o no de agua procedente de red pública; y tipo de servicio sanitario.\allowbreak{}\par{}Se consideró:\allowbreak{} la proporción de hogares según hacinamiento agrupado (\allowbreak{}tres personas o mas por cuarto)\allowbreak{}; necesidades básicas insatisfechas (\allowbreak{}de hacinamiento,\allowbreak{} vivienda,\allowbreak{} instalaciones sanitarias,\allowbreak{} escolaridad o capacidad de subsistencia)\allowbreak{}; índice de privación material de los hogares (\allowbreak{}IPMH)\allowbreak{};\protect\footnote{ Según el Instituto Nacional de Estadísticas y Censos (\allowbreak{}INDEC)\allowbreak{} el Índice de Privación Material de Hogares,\allowbreak{} es una variable que identifica a los hogares según su situación respecto a la privación material en cuanto a dos dimensiones,\allowbreak{} recursos corrientes y patrimoniales.\allowbreak{} Respecto del índice de Necesidades Básicas Insatisfechas,\allowbreak{} los hogares con NBI son los hogares que presentan al menos uno de los siguientes indicadores de privación:\allowbreak{} hacinamiento (\allowbreak{}más de tres personas por cuarto)\allowbreak{}; vivienda (\allowbreak{}habitan en una vivienda de tipo inconveniente [pieza de inquilinato,\allowbreak{} pieza de hotel o pensión,\allowbreak{} casilla,\allowbreak{} local no construido para habitación o vivienda móvil],\allowbreak{} excluyendo casa,\allowbreak{} departamento y rancho)\allowbreak{}; condiciones sanitarias (\allowbreak{}no tienen ningún tipo de retrete)\allowbreak{}; asistencia escolar (\allowbreak{}tienen al menos un niño en edad escolar [6 a 12 años] que no asiste a la escuela)\allowbreak{}; capacidad de subsistencia (\allowbreak{}tienen cuatro o más personas por miembro ocupado,\allowbreak{} cuyo jefe no haya completado el tercer grado de escolaridad primaria)\allowbreak{}.\allowbreak{}} condición de actividad económica del núcleo; y tenencia de nevera con o sin congelador,\allowbreak{} teléfono ya sea fijo o celular,\allowbreak{} horno a microondas,\allowbreak{} computadora con conexión a internet,\allowbreak{} cocina con pileta e instalación de agua.\allowbreak{}\par{}Se consideró la proporción de población según sexo,\allowbreak{} edad,\allowbreak{} cobertura de salud,\allowbreak{} convivencia en pareja y condición de alfabetismo.\allowbreak{} Asimismo,\allowbreak{} se consideró la proporción de individuos según situación educacional,\allowbreak{} condición de actividad y aporte jubilatorio (\allowbreak{}aporta y le descuentan; no aporta ni le descuentan; no recibe sueldo)\allowbreak{}.\allowbreak{}\par{}Se utilizó el paquete estadístico Stata 10.\allowbreak{}0 y dos sistemas de información geográfica para la confección de los mapas y el análisis espacial:\allowbreak{} ArcView 3.\allowbreak{}2 y GeoDA 8.\allowbreak{} Se calculó la proporción de abandono dividiendo el número de casos no-\allowbreak{}adherentes por el número total de pacientes que iniciaron el tratamiento en cada unidad de análisis (\allowbreak{}fracción censal)\allowbreak{}.\allowbreak{} Fueron calculadas las transformaciones de tipo raíces cuadráticas Freeman-\allowbreak{}Tukey y bayesianas empíricas para estas medidas,\allowbreak{} teniendo como referencia el conjunto de fracciones censales de los municipios.\allowbreak{} Se construyeron mapas temáticos con esas proporciones para elegir la manera más adecuada de presentar los padrones de distribución espaciales.\allowbreak{}\par{}Se realizó un análisis exploratorio de las características del área y la distribución de la proporción de no-\allowbreak{}adherencia.\allowbreak{} Se realizó un análisis de múltiples factores por regresión lineal.\allowbreak{} En este modelo,\allowbreak{} las variables independientes fueron las características sociodemográficas y socioeconómicas de los grupos y las áreas de donde provienen los casos de abandono.\allowbreak{} La variable dependiente fue la “proporción de no-\allowbreak{}adherencia”.\allowbreak{} Las variables introducidas en el modelo de análisis por regresión lineal múltiple fueron aquellas que mostraron asociación significativa (\allowbreak{}p < 0,\allowbreak{}20)\allowbreak{} en el análisis bivariado.\allowbreak{} El modelo final incluyó las variables con nivel de significancia p = 0,\allowbreak{}05 y aquellas que eran consideradas esenciales para el modelo explicativo.\allowbreak{}\par{}El protocolo del estudio fue aprobado por el Comité de Ética de cada hospital participante.\allowbreak{}
\section*{RESULTADOS}
\par{}El partido de Avellaneda tuvo viviendas con mejores condiciones generales y disponibilidad de servicios básicos.\allowbreak{} Asimismo,\allowbreak{} fue el partido que presentó menores variaciones para cada uno de los indicadores al interior de las fracciones censales.\allowbreak{} En el otro extremo,\allowbreak{} estuvo el partido de Ezeiza,\allowbreak{} con peor situación con relación a la mayor parte de los indicadores analizados,\allowbreak{} con grandes variaciones al interior de las fracciones censales.\allowbreak{} La distribución de la población fue más homogénea entre los municipios,\allowbreak{} aún cuando en general,\allowbreak{} Avellaneda continuó siendo el municipio con mejor situación de estos indicadores (\allowbreak{}Tabla 1)\allowbreak{}.\allowbreak{}\par{}\end{multicols}
\begin{landscape}
\ctable[
  caption = {\textbf{Tabla 1:} \textit{Características del área,\allowbreak{} las viviendas,\allowbreak{} los hogares y la población de donde provienen los casos de tuberculosis.\allowbreak{} Región Sanitaria VI,\allowbreak{} Buenos Aires,\allowbreak{} Argentina,\allowbreak{} 2001.\allowbreak{}}}, 
  width=\linewidth, pos = ht, left, long
]
{p{0.28\linewidth}p{0.08\linewidth}p{0.08\linewidth}p{0.08\linewidth}p{0.08\linewidth}p{0.08\linewidth}p{0.08\linewidth}p{0.09\linewidth}p{0.08\linewidth}}
{}
{ \\\hline
\multicolumn{1}{L{0.28\linewidth}}{\textbf{Característica}}
 & \multicolumn{8}{L{0.67\linewidth}}{\textbf{Existencia de (\allowbreak{}en \%\allowbreak{\allowbreak{}\allowbreak{}}\allowbreak{})\allowbreak{} }} \\\hline 
\multicolumn{1}{L{0.28\linewidth}}{\textbf{AB}}
 & \multicolumn{1}{L{0.08\linewidth}}{\textbf{AV}}
 & \multicolumn{1}{L{0.08\linewidth}}{\textbf{BZ}}
 & \multicolumn{1}{L{0.08\linewidth}}{\textbf{EE}}
 & \multicolumn{1}{L{0.08\linewidth}}{\textbf{EZ}}
 & \multicolumn{1}{L{0.08\linewidth}}{\textbf{LZ}}
 & \multicolumn{1}{L{0.08\linewidth}}{\textbf{QM}}
 & \multicolumn{1}{L{0.09\linewidth}}{\textbf{Total}} \\\hline 
\multicolumn{8}{L{0.87\linewidth}}{\textbf{}} \\\hline 
\multicolumn{1}{L{0.28\linewidth}}{\textbf{\%\allowbreak{\allowbreak{}\allowbreak{}}\allowbreak{}}}
 & \multicolumn{1}{L{0.08\linewidth}}{\textbf{\%\allowbreak{\allowbreak{}\allowbreak{}}\allowbreak{}}}
 & \multicolumn{1}{L{0.08\linewidth}}{\textbf{\%\allowbreak{\allowbreak{}\allowbreak{}}\allowbreak{}}}
 & \multicolumn{1}{L{0.08\linewidth}}{\textbf{\%\allowbreak{\allowbreak{}\allowbreak{}}\allowbreak{}}}
 & \multicolumn{1}{L{0.08\linewidth}}{\textbf{\%\allowbreak{\allowbreak{}\allowbreak{}}\allowbreak{}}}
 & \multicolumn{1}{L{0.08\linewidth}}{\textbf{\%\allowbreak{\allowbreak{}\allowbreak{}}\allowbreak{}}}
 & \multicolumn{1}{L{0.08\linewidth}}{\textbf{\%\allowbreak{\allowbreak{}\allowbreak{}}\allowbreak{}}}
 & \multicolumn{1}{L{0.09\linewidth}}{\textbf{\%\allowbreak{\allowbreak{}\allowbreak{}}\allowbreak{}}} \\\hline 
\multicolumn{9}{L{0.95\linewidth}}{Características del área y las viviendas} \\\hline \multicolumn{1}{L{0.28\linewidth}}{ Energía eléctrica por red domiciliaria} & \multicolumn{1}{C{0.08\linewidth}}{96,\allowbreak{}0} & \multicolumn{1}{C{0.08\linewidth}}{100} & \multicolumn{1}{C{0.08\linewidth}}{98,\allowbreak{}3} & \multicolumn{1}{C{0.08\linewidth}}{98,\allowbreak{}9} & \multicolumn{1}{C{0.08\linewidth}}{97,\allowbreak{}5} & \multicolumn{1}{C{0.08\linewidth}}{98,\allowbreak{}7} & \multicolumn{1}{C{0.09\linewidth}}{95,\allowbreak{}3} & \multicolumn{1}{C{0.08\linewidth}}{97,\allowbreak{}2} \\\hline \multicolumn{1}{L{0.28\linewidth}}{ Calle pavimentada} & \multicolumn{1}{C{0.08\linewidth}}{73,\allowbreak{}0} & \multicolumn{1}{C{0.08\linewidth}}{100} & \multicolumn{1}{C{0.08\linewidth}}{75,\allowbreak{}4} & \multicolumn{1}{C{0.08\linewidth}}{72,\allowbreak{}4} & \multicolumn{1}{C{0.08\linewidth}}{78,\allowbreak{}2} & \multicolumn{1}{C{0.08\linewidth}}{87,\allowbreak{}2} & \multicolumn{1}{C{0.09\linewidth}}{72,\allowbreak{}8} & \multicolumn{1}{C{0.08\linewidth}}{76,\allowbreak{}1} \\\hline \multicolumn{1}{L{0.28\linewidth}}{ Sistema de cloacas} & \multicolumn{1}{C{0.08\linewidth}}{13,\allowbreak{}0} & \multicolumn{1}{C{0.08\linewidth}}{85,\allowbreak{}3} & \multicolumn{1}{C{0.08\linewidth}}{73,\allowbreak{}8} & \multicolumn{1}{C{0.08\linewidth}}{4,\allowbreak{}9} & \multicolumn{1}{C{0.08\linewidth}}{2,\allowbreak{}0} & \multicolumn{1}{C{0.08\linewidth}}{6,\allowbreak{}7} & \multicolumn{1}{C{0.09\linewidth}}{59,\allowbreak{}0} & \multicolumn{1}{C{0.08\linewidth}}{37,\allowbreak{}5} \\\hline \multicolumn{1}{L{0.28\linewidth}}{ Servicio de recolección de residuos} & \multicolumn{1}{C{0.08\linewidth}}{87,\allowbreak{}3} & \multicolumn{1}{C{0.08\linewidth}}{100} & \multicolumn{1}{C{0.08\linewidth}}{95,\allowbreak{}3} & \multicolumn{1}{C{0.08\linewidth}}{89,\allowbreak{}8} & \multicolumn{1}{C{0.08\linewidth}}{97,\allowbreak{}2} & \multicolumn{1}{C{0.08\linewidth}}{92,\allowbreak{}9} & \multicolumn{1}{C{0.09\linewidth}}{88,\allowbreak{}7} & \multicolumn{1}{C{0.08\linewidth}}{91,\allowbreak{}5} \\\hline \multicolumn{1}{L{0.28\linewidth}}{ Gas natural de red} & \multicolumn{1}{C{0.08\linewidth}}{63,\allowbreak{}6} & \multicolumn{1}{C{0.08\linewidth}}{97,\allowbreak{}8} & \multicolumn{1}{C{0.08\linewidth}}{87,\allowbreak{}9} & \multicolumn{1}{C{0.08\linewidth}}{64,\allowbreak{}0} & \multicolumn{1}{C{0.08\linewidth}}{62,\allowbreak{}0} & \multicolumn{1}{C{0.08\linewidth}}{75,\allowbreak{}3} & \multicolumn{1}{C{0.09\linewidth}}{69,\allowbreak{}7} & \multicolumn{1}{C{0.08\linewidth}}{72,\allowbreak{}2} \\\hline \multicolumn{1}{L{0.28\linewidth}}{ Instalación de energía eléctrica} & \multicolumn{1}{C{0.08\linewidth}}{44,\allowbreak{}5} & \multicolumn{1}{C{0.08\linewidth}}{100} & \multicolumn{1}{C{0.08\linewidth}}{99,\allowbreak{}7} & \multicolumn{1}{C{0.08\linewidth}}{41,\allowbreak{}3} & \multicolumn{1}{C{0.08\linewidth}}{30,\allowbreak{}1} & \multicolumn{1}{C{0.08\linewidth}}{89,\allowbreak{}3} & \multicolumn{1}{C{0.09\linewidth}}{92,\allowbreak{}7} & \multicolumn{1}{C{0.08\linewidth}}{74,\allowbreak{}4} \\\hline \multicolumn{1}{L{0.28\linewidth}}{ Vivienda deficitaria} & \multicolumn{1}{C{0.08\linewidth}}{34,\allowbreak{}8} & \multicolumn{1}{C{0.08\linewidth}}{2,\allowbreak{}4} & \multicolumn{1}{C{0.08\linewidth}}{23,\allowbreak{}1} & \multicolumn{1}{C{0.08\linewidth}}{38,\allowbreak{}0} & \multicolumn{1}{C{0.08\linewidth}}{39,\allowbreak{}3} & \multicolumn{1}{C{0.08\linewidth}}{31,\allowbreak{}0} & \multicolumn{1}{C{0.09\linewidth}}{30,\allowbreak{}0} & \multicolumn{1}{C{0.08\linewidth}}{30,\allowbreak{}7} \\\hline \multicolumn{1}{L{0.28\linewidth}}{ Pisos de cerámica,\allowbreak{} baldosa o mosaico} & \multicolumn{1}{C{0.08\linewidth}}{48,\allowbreak{}1} & \multicolumn{1}{C{0.08\linewidth}}{75,\allowbreak{}8} & \multicolumn{1}{C{0.08\linewidth}}{56,\allowbreak{}2} & \multicolumn{1}{C{0.08\linewidth}}{41,\allowbreak{}3} & \multicolumn{1}{C{0.08\linewidth}}{38,\allowbreak{}4} & \multicolumn{1}{C{0.08\linewidth}}{53,\allowbreak{}0} & \multicolumn{1}{C{0.09\linewidth}}{52,\allowbreak{}5} & \multicolumn{1}{C{0.08\linewidth}}{50,\allowbreak{}4} \\\hline \multicolumn{1}{L{0.28\linewidth}}{ Provisión de agua dentro de la vivienda} & \multicolumn{1}{C{0.08\linewidth}}{64,\allowbreak{}3} & \multicolumn{1}{C{0.08\linewidth}}{76,\allowbreak{}7} & \multicolumn{1}{C{0.08\linewidth}}{78,\allowbreak{}7} & \multicolumn{1}{C{0.08\linewidth}}{58,\allowbreak{}7} & \multicolumn{1}{C{0.08\linewidth}}{52,\allowbreak{}3} & \multicolumn{1}{C{0.08\linewidth}}{69,\allowbreak{}1} & \multicolumn{1}{C{0.09\linewidth}}{74,\allowbreak{}2} & \multicolumn{1}{C{0.08\linewidth}}{69,\allowbreak{}0} \\\hline \multicolumn{1}{L{0.28\linewidth}}{ Transporte público a menos de 300 m de la vivienda} & \multicolumn{1}{C{0.08\linewidth}}{90,\allowbreak{}2} & \multicolumn{1}{C{0.08\linewidth}}{100} & \multicolumn{1}{C{0.08\linewidth}}{89,\allowbreak{}9} & \multicolumn{1}{C{0.08\linewidth}}{85,\allowbreak{}2} & \multicolumn{1}{C{0.08\linewidth}}{81,\allowbreak{}8} & \multicolumn{1}{C{0.08\linewidth}}{89,\allowbreak{}9} & \multicolumn{1}{C{0.09\linewidth}}{84,\allowbreak{}9} & \multicolumn{1}{C{0.08\linewidth}}{87,\allowbreak{}2} \\\hline \multicolumn{9}{L{0.95\linewidth}}{Características de los hogares y la población} \\\hline \multicolumn{1}{L{0.28\linewidth}}{ Con cocina con instalación de agua} & \multicolumn{1}{C{0.08\linewidth}}{62,\allowbreak{}0} & \multicolumn{1}{C{0.08\linewidth}}{77,\allowbreak{}5} & \multicolumn{1}{C{0.08\linewidth}}{75,\allowbreak{}2} & \multicolumn{1}{C{0.08\linewidth}}{56,\allowbreak{}0} & \multicolumn{1}{C{0.08\linewidth}}{48,\allowbreak{}7} & \multicolumn{1}{C{0.08\linewidth}}{68,\allowbreak{}4} & \multicolumn{1}{C{0.09\linewidth}}{71,\allowbreak{}2} & \multicolumn{1}{C{0.08\linewidth}}{66,\allowbreak{}1} \\\hline \multicolumn{1}{L{0.28\linewidth}}{ Con horno a microondas} & \multicolumn{1}{C{0.08\linewidth}}{9,\allowbreak{}8} & \multicolumn{1}{C{0.08\linewidth}}{27,\allowbreak{}7} & \multicolumn{1}{C{0.08\linewidth}}{14,\allowbreak{}1} & \multicolumn{1}{C{0.08\linewidth}}{10,\allowbreak{}6} & \multicolumn{1}{C{0.08\linewidth}}{9,\allowbreak{}1} & \multicolumn{1}{C{0.08\linewidth}}{10,\allowbreak{}1} & \multicolumn{1}{C{0.09\linewidth}}{14,\allowbreak{}2} & \multicolumn{1}{C{0.08\linewidth}}{12,\allowbreak{}4} \\\hline \multicolumn{1}{L{0.28\linewidth}}{ Con hacinamiento agrupado} & \multicolumn{1}{C{0.08\linewidth}}{7,\allowbreak{}1} & \multicolumn{1}{C{0.08\linewidth}}{10,\allowbreak{}2} & \multicolumn{1}{C{0.08\linewidth}}{5,\allowbreak{}8} & \multicolumn{1}{C{0.08\linewidth}}{7,\allowbreak{}2} & \multicolumn{1}{C{0.08\linewidth}}{8,\allowbreak{}7} & \multicolumn{1}{C{0.08\linewidth}}{6,\allowbreak{}3} & \multicolumn{1}{C{0.09\linewidth}}{5,\allowbreak{}9} & \multicolumn{1}{C{0.08\linewidth}}{6,\allowbreak{}4} \\\hline \multicolumn{1}{L{0.28\linewidth}}{ Con inodoro sin descarga o sin inodoro} & \multicolumn{1}{C{0.08\linewidth}}{28,\allowbreak{}6} & \multicolumn{1}{C{0.08\linewidth}}{0,\allowbreak{}7} & \multicolumn{1}{C{0.08\linewidth}}{17,\allowbreak{}6} & \multicolumn{1}{C{0.08\linewidth}}{31,\allowbreak{}1} & \multicolumn{1}{C{0.08\linewidth}}{32,\allowbreak{}1} & \multicolumn{1}{C{0.08\linewidth}}{25,\allowbreak{}0} & \multicolumn{1}{C{0.09\linewidth}}{23,\allowbreak{}1} & \multicolumn{1}{C{0.08\linewidth}}{24,\allowbreak{}4} \\\hline \multicolumn{1}{L{0.28\linewidth}}{ Con nevera (\allowbreak{}con o sin congelador)\allowbreak{}} & \multicolumn{1}{C{0.08\linewidth}}{81,\allowbreak{}8} & \multicolumn{1}{C{0.08\linewidth}}{77,\allowbreak{}6} & \multicolumn{1}{C{0.08\linewidth}}{85,\allowbreak{}1} & \multicolumn{1}{C{0.08\linewidth}}{81,\allowbreak{}0} & \multicolumn{1}{C{0.08\linewidth}}{76,\allowbreak{}5} & \multicolumn{1}{C{0.08\linewidth}}{82,\allowbreak{}0} & \multicolumn{1}{C{0.09\linewidth}}{82,\allowbreak{}4} & \multicolumn{1}{C{0.08\linewidth}}{81,\allowbreak{}9} \\\hline \multicolumn{1}{L{0.28\linewidth}}{ Con computadora (\allowbreak{}con o sin Internet)\allowbreak{}} & \multicolumn{1}{C{0.08\linewidth}}{8,\allowbreak{}9} & \multicolumn{1}{C{0.08\linewidth}}{23,\allowbreak{}6} & \multicolumn{1}{C{0.08\linewidth}}{13,\allowbreak{}7} & \multicolumn{1}{C{0.08\linewidth}}{10,\allowbreak{}1} & \multicolumn{1}{C{0.08\linewidth}}{9,\allowbreak{}2} & \multicolumn{1}{C{0.08\linewidth}}{9,\allowbreak{}7} & \multicolumn{1}{C{0.09\linewidth}}{12,\allowbreak{}5} & \multicolumn{1}{C{0.08\linewidth}}{11,\allowbreak{}6} \\\hline \multicolumn{1}{L{0.28\linewidth}}{ Población alfabetizada} & \multicolumn{1}{C{0.08\linewidth}}{84,\allowbreak{}1} & \multicolumn{1}{C{0.08\linewidth}}{91,\allowbreak{}7} & \multicolumn{1}{C{0.08\linewidth}}{85,\allowbreak{}3} & \multicolumn{1}{C{0.08\linewidth}}{83,\allowbreak{}9} & \multicolumn{1}{C{0.08\linewidth}}{83,\allowbreak{}3} & \multicolumn{1}{C{0.08\linewidth}}{86,\allowbreak{}7} & \multicolumn{1}{C{0.09\linewidth}}{85,\allowbreak{}8} & \multicolumn{1}{C{0.08\linewidth}}{85,\allowbreak{}2} \\\hline \multicolumn{1}{L{0.28\linewidth}}{ Población sin cobertura de salud} & \multicolumn{1}{C{0.08\linewidth}}{63,\allowbreak{}7} & \multicolumn{1}{C{0.08\linewidth}}{33,\allowbreak{}0} & \multicolumn{1}{C{0.08\linewidth}}{57,\allowbreak{}8} & \multicolumn{1}{C{0.08\linewidth}}{63,\allowbreak{}2} & \multicolumn{1}{C{0.08\linewidth}}{62,\allowbreak{}7} & \multicolumn{1}{C{0.08\linewidth}}{63,\allowbreak{}7} & \multicolumn{1}{C{0.09\linewidth}}{54,\allowbreak{}8} & \multicolumn{1}{C{0.08\linewidth}}{59,\allowbreak{}0} \\\hline \multicolumn{1}{L{0.28\linewidth}}{ Con teléfono (\allowbreak{}fijo,\allowbreak{} celular o ambas cosas)\allowbreak{}} & \multicolumn{1}{C{0.08\linewidth}}{53,\allowbreak{}1} & \multicolumn{1}{C{0.08\linewidth}}{69,\allowbreak{}5} & \multicolumn{1}{C{0.08\linewidth}}{58,\allowbreak{}3} & \multicolumn{1}{C{0.08\linewidth}}{53,\allowbreak{}9} & \multicolumn{1}{C{0.08\linewidth}}{49,\allowbreak{}0} & \multicolumn{1}{C{0.08\linewidth}}{54,\allowbreak{}8} & \multicolumn{1}{C{0.09\linewidth}}{54,\allowbreak{}4} & \multicolumn{1}{C{0.08\linewidth}}{54,\allowbreak{}8} \\\hline \multicolumn{1}{L{0.28\linewidth}}{ Ningún tipo de Privación Material del Hogar} & \multicolumn{1}{C{0.08\linewidth}}{34,\allowbreak{}7} & \multicolumn{1}{C{0.08\linewidth}}{65,\allowbreak{}7} & \multicolumn{1}{C{0.08\linewidth}}{43,\allowbreak{}6} & \multicolumn{1}{C{0.08\linewidth}}{33,\allowbreak{}9} & \multicolumn{1}{C{0.08\linewidth}}{31,\allowbreak{}6} & \multicolumn{1}{C{0.08\linewidth}}{38,\allowbreak{}7} & \multicolumn{1}{C{0.09\linewidth}}{41,\allowbreak{}5} & \multicolumn{1}{C{0.08\linewidth}}{39,\allowbreak{}3} \\\hline \multicolumn{1}{L{0.28\linewidth}}{ Con Privación Material de los Hogares Convergente} & \multicolumn{1}{C{0.08\linewidth}}{21,\allowbreak{}4} & \multicolumn{1}{C{0.08\linewidth}}{0,\allowbreak{}2} & \multicolumn{1}{C{0.08\linewidth}}{16,\allowbreak{}1} & \multicolumn{1}{C{0.08\linewidth}}{21,\allowbreak{}7} & \multicolumn{1}{C{0.08\linewidth}}{23,\allowbreak{}2} & \multicolumn{1}{C{0.08\linewidth}}{19,\allowbreak{}7} & \multicolumn{1}{C{0.09\linewidth}}{19,\allowbreak{}0} & \multicolumn{1}{C{0.08\linewidth}}{19,\allowbreak{}2} \\\hline \multicolumn{1}{L{0.28\linewidth}}{ Con al menos un indicador de NBI} & \multicolumn{1}{C{0.08\linewidth}}{20,\allowbreak{}7} & \multicolumn{1}{C{0.08\linewidth}}{4,\allowbreak{}4} & \multicolumn{1}{C{0.08\linewidth}}{17,\allowbreak{}2} & \multicolumn{1}{C{0.08\linewidth}}{19,\allowbreak{}3} & \multicolumn{1}{C{0.08\linewidth}}{20,\allowbreak{}2} & \multicolumn{1}{C{0.08\linewidth}}{18,\allowbreak{}8} & \multicolumn{1}{C{0.09\linewidth}}{23,\allowbreak{}0} & \multicolumn{1}{C{0.08\linewidth}}{19,\allowbreak{}9} \\\hline \multicolumn{1}{L{0.28\linewidth}}{ Con NBI por condición de subsistencia} & \multicolumn{1}{C{0.08\linewidth}}{7,\allowbreak{}0} & \multicolumn{1}{C{0.08\linewidth}}{1,\allowbreak{}4} & \multicolumn{1}{C{0.08\linewidth}}{6,\allowbreak{}2} & \multicolumn{1}{C{0.08\linewidth}}{4,\allowbreak{}9} & \multicolumn{1}{C{0.08\linewidth}}{5,\allowbreak{}8} & \multicolumn{1}{C{0.08\linewidth}}{18,\allowbreak{}3} & \multicolumn{1}{C{0.09\linewidth}}{21,\allowbreak{}1} & \multicolumn{1}{C{0.08\linewidth}}{11,\allowbreak{}8} \\\hline \multicolumn{1}{L{0.28\linewidth}}{ Con uno de los cónyuges desocupado o inactivo} & \multicolumn{1}{C{0.08\linewidth}}{50,\allowbreak{}8} & \multicolumn{1}{C{0.08\linewidth}}{50,\allowbreak{}1} & \multicolumn{1}{C{0.08\linewidth}}{48,\allowbreak{}2} & \multicolumn{1}{C{0.08\linewidth}}{53,\allowbreak{}0} & \multicolumn{1}{C{0.08\linewidth}}{52,\allowbreak{}7} & \multicolumn{1}{C{0.08\linewidth}}{48,\allowbreak{}7} & \multicolumn{1}{C{0.09\linewidth}}{49,\allowbreak{}9} & \multicolumn{1}{C{0.08\linewidth}}{50,\allowbreak{}3} \\\hline \multicolumn{1}{L{0.28\linewidth}}{ Población que aporta o le descuentan jubilación} & \multicolumn{1}{C{0.08\linewidth}}{12,\allowbreak{}6} & \multicolumn{1}{C{0.08\linewidth}}{27,\allowbreak{}7} & \multicolumn{1}{C{0.08\linewidth}}{15,\allowbreak{}2} & \multicolumn{1}{C{0.08\linewidth}}{13,\allowbreak{}0} & \multicolumn{1}{C{0.08\linewidth}}{13,\allowbreak{}6} & \multicolumn{1}{C{0.08\linewidth}}{13,\allowbreak{}2} & \multicolumn{1}{C{0.09\linewidth}}{17,\allowbreak{}3} & \multicolumn{1}{C{0.08\linewidth}}{15,\allowbreak{}1} \\\hline \multicolumn{1}{L{0.28\linewidth}}{ Población desocupada} & \multicolumn{1}{C{0.08\linewidth}}{17,\allowbreak{}5} & \multicolumn{1}{C{0.08\linewidth}}{12,\allowbreak{}3} & \multicolumn{1}{C{0.08\linewidth}}{18,\allowbreak{}7} & \multicolumn{1}{C{0.08\linewidth}}{16,\allowbreak{}0} & \multicolumn{1}{C{0.08\linewidth}}{14,\allowbreak{}5} & \multicolumn{1}{C{0.08\linewidth}}{19,\allowbreak{}8} & \multicolumn{1}{C{0.09\linewidth}}{17,\allowbreak{}3} & \multicolumn{1}{C{0.08\linewidth}}{17,\allowbreak{}3} \\\hline \multicolumn{9}{p\linewidth}{\textsuperscript{ } \footnotesize{Fuentes:\allowbreak{} Estudio de corte transversal (\allowbreak{}ECM 2008)\allowbreak{}\textsuperscript{e} y Censo Nacional de Población y Viviendas,\allowbreak{} INDEC,\allowbreak{} 2001.\allowbreak{}\textsuperscript{ g}}}\\\multicolumn{9}{p\linewidth}{\textsuperscript{ } \footnotesize{AB:\allowbreak{} Almirante Brown; AV:\allowbreak{} Avellaneda; BZ:\allowbreak{} Berazategui; EE:\allowbreak{} Esteban Echeverría; EZ:\allowbreak{} Ezeiza; LZ:\allowbreak{} Lomas de Zamora; QM:\allowbreak{} Quilmes; NBI:\allowbreak{} necesidades básicas insatisfechas}}
}
\end{landscape}
\begin{multicols}{2}
\par{}El riesgo de no-\allowbreak{}adherencia fue mayor en aquellas áreas con mayor proporción de viviendas sin la existencia de un transporte público a menos de 300 m (\allowbreak{}$\allowbreak{\allowbreak{}\allowbreak{}}\allowbreak{}ρ$\allowbreak{\allowbreak{}\allowbreak{}}\allowbreak{} = 0,\allowbreak{}21)\allowbreak{},\allowbreak{} como también en aquellas áreas con mayor proporción de viviendas que no poseían nevera,\allowbreak{} con o sin congelador (\allowbreak{}$\allowbreak{\allowbreak{}\allowbreak{}}\allowbreak{}ρ$\allowbreak{\allowbreak{}\allowbreak{}}\allowbreak{} = 0,\allowbreak{}17)\allowbreak{},\allowbreak{} y en las que tenían inodoro sin descarga o que no contaban con inodoro (\allowbreak{}$\allowbreak{\allowbreak{}\allowbreak{}}\allowbreak{}ρ$\allowbreak{\allowbreak{}\allowbreak{}}\allowbreak{} = 0,\allowbreak{}17)\allowbreak{}.\allowbreak{} El riesgo de no-\allowbreak{}adherencia al tratamiento fue mayor en aquellas áreas con mayor proporción de hogares con NBI por la capacidad de subsistencia (\allowbreak{}$\allowbreak{\allowbreak{}\allowbreak{}}\allowbreak{}ρ$\allowbreak{\allowbreak{}\allowbreak{}}\allowbreak{} = 0,\allowbreak{}26)\allowbreak{} y con proporción más elevada de población ocupada pero que no realizaban ni le descontaban aportes jubilatorios (\allowbreak{}$\allowbreak{\allowbreak{}\allowbreak{}}\allowbreak{}ρ$\allowbreak{\allowbreak{}\allowbreak{}}\allowbreak{} = 0,\allowbreak{}21)\allowbreak{} (\allowbreak{}Tabla 2)\allowbreak{}.\allowbreak{}\par{}\end{multicols}
\ctable[
  caption = {\textbf{Tabla 2:} \textit{Relaciones entre características sociodemográficas y socioeconómicas y la proporción de no adherencia en los municipios seleccionados.\allowbreak{} Región Sanitaria VI,\allowbreak{} Buenos Aires,\allowbreak{} Argentina,\allowbreak{} 2001.\allowbreak{}}}, 
  width=\textwidth, pos = ht, left, long
]
{p{0.41\textwidth}p{0.40\textwidth}p{0.16\textwidth}}
{}
{ \\\hline
\multicolumn{1}{L{0.41\textwidth}}{\textbf{Característica}}
 & \multicolumn{1}{L{0.40\textwidth}}{\textbf{Coeficiente de correlación (\allowbreak{}$\allowbreak{\allowbreak{}\allowbreak{}}\allowbreak{}ρ$\allowbreak{\allowbreak{}\allowbreak{}}\allowbreak{} =)\allowbreak{}}}
 & \multicolumn{1}{L{0.16\textwidth}}{\textbf{p}} \\\hline 
\multicolumn{1}{L{0.41\textwidth}}{Tenencia de nevera o congelador (\allowbreak{}con o sin congelador)\allowbreak{}} & \multicolumn{1}{C{0.40\textwidth}}{0,\allowbreak{}17} & \multicolumn{1}{C{0.16\textwidth}}{0,\allowbreak{}194} \\\hline \multicolumn{1}{L{0.41\textwidth}}{Aporte jubilatorio:\allowbreak{} no aportan o no recibe sueldo} & \multicolumn{1}{C{0.40\textwidth}}{0,\allowbreak{}21} & \multicolumn{1}{C{0.16\textwidth}}{0,\allowbreak{}104} \\\hline \multicolumn{1}{L{0.41\textwidth}}{Indicador NBI condiciones de subsistencia} & \multicolumn{1}{C{0.40\textwidth}}{0,\allowbreak{}26} & \multicolumn{1}{C{0.16\textwidth}}{0,\allowbreak{}048} \\\hline \multicolumn{1}{L{0.41\textwidth}}{Existencia de transporte público a menos de 300 metros (\allowbreak{}3 cuadras)\allowbreak{}} & \multicolumn{1}{C{0.40\textwidth}}{0,\allowbreak{}21} & \multicolumn{1}{C{0.16\textwidth}}{0,\allowbreak{}104} \\\hline \multicolumn{1}{L{0.41\textwidth}}{Inodoro sin descarga o sin inodoro} & \multicolumn{1}{C{0.40\textwidth}}{0,\allowbreak{}17} & \multicolumn{1}{C{0.16\textwidth}}{0,\allowbreak{}200} \\\hline \multicolumn{1}{L{0.41\textwidth}}{Población de 15 a 64 años} & \multicolumn{1}{C{0.40\textwidth}}{0,\allowbreak{}15} & \multicolumn{1}{C{0.16\textwidth}}{> 0,\allowbreak{}20} \\\hline \multicolumn{1}{L{0.41\textwidth}}{Población analfabeta} & \multicolumn{1}{C{0.40\textwidth}}{0,\allowbreak{}06} & \multicolumn{1}{C{0.16\textwidth}}{> 0,\allowbreak{}20} \\\hline \multicolumn{1}{L{0.41\textwidth}}{Existencia de red de agua corriente} & \multicolumn{1}{C{0.40\textwidth}}{0,\allowbreak{}15} & \multicolumn{1}{C{0.16\textwidth}}{> 0,\allowbreak{}20} \\\hline \multicolumn{1}{L{0.41\textwidth}}{Condición de actividad desagregada:\allowbreak{} desocupados} & \multicolumn{1}{C{0.40\textwidth}}{0,\allowbreak{}05} & \multicolumn{1}{C{0.16\textwidth}}{> 0,\allowbreak{}20} \\\hline \multicolumn{1}{L{0.41\textwidth}}{Población sin cobertura de salud} & \multicolumn{1}{C{0.40\textwidth}}{0,\allowbreak{}05} & \multicolumn{1}{C{0.16\textwidth}}{> 0,\allowbreak{}20} \\\hline \multicolumn{1}{L{0.41\textwidth}}{Provisión de agua en la vivienda} & \multicolumn{1}{C{0.40\textwidth}}{0,\allowbreak{}14} & \multicolumn{1}{C{0.16\textwidth}}{> 0,\allowbreak{}20} \\\hline \multicolumn{1}{L{0.41\textwidth}}{Tenencia de lavarropas automático o común} & \multicolumn{1}{C{0.40\textwidth}}{0,\allowbreak{}15} & \multicolumn{1}{C{0.16\textwidth}}{> 0,\allowbreak{}20} \\\hline \multicolumn{1}{L{0.41\textwidth}}{Al menos un indicador de NBI} & \multicolumn{1}{C{0.40\textwidth}}{0,\allowbreak{}07} & \multicolumn{1}{C{0.16\textwidth}}{> 0,\allowbreak{}20} \\\hline \multicolumn{1}{L{0.41\textwidth}}{Situación educativa:\allowbreak{} Primaria incompleta} & \multicolumn{1}{C{0.40\textwidth}}{0,\allowbreak{}13} & \multicolumn{1}{C{0.16\textwidth}}{> 0,\allowbreak{}20} \\\hline \multicolumn{3}{p\textwidth}{\textsuperscript{ } \footnotesize{Fuente:\allowbreak{} Elaboración propia.\allowbreak{} Estudio de corte transversal (\allowbreak{}ECM 2008)\allowbreak{}\textsuperscript{e} y Censo Nacional de Población y Viviendas,\allowbreak{} INDEC,\allowbreak{} 2001.\allowbreak{}\textsuperscript{g}}}
}
\begin{multicols}{2}
\par{}Los grupos poblacionales con empleo pero que no le descontaban los aportes jubilatorios,\allowbreak{} ni realizaban dichos aportes tuvieron más probabilidad de no-\allowbreak{}adherencia (\allowbreak{}p = 0,\allowbreak{}007)\allowbreak{}.\allowbreak{} Aquellos hogares que tuvieron la capacidad de subsistencia como necesidad básica insatisfecha también tuvieron más riesgo de no adherir al tratamiento (\allowbreak{}p = 0,\allowbreak{}032)\allowbreak{}.\allowbreak{} La probabilidad de no-\allowbreak{}adherencia aumentó en aquellas viviendas que no contaban con un servicio de transporte público a menos de 300 m de la vivienda (\allowbreak{}p = 0,\allowbreak{}070)\allowbreak{},\allowbreak{} aún cuando estos resultados no tuvieron significancia estadística (\allowbreak{}Tabla 3)\allowbreak{}.\allowbreak{}\par{}\end{multicols}
\ctable[
  caption = {\textbf{Tabla 3:} \textit{Modelo de regresión múltiple para las características sociodemográficas y socioeconómicas relacionadas a la proporción de abandono en municipios seleccionados.\allowbreak{} Región Sanitaria VI,\allowbreak{} Buenos Aires,\allowbreak{} Argentina,\allowbreak{} 2001.\allowbreak{}}}, 
  width=\textwidth, pos = ht, left, long
]
{p{0.48\textwidth}p{0.17\textwidth}p{0.20\textwidth}p{0.10\textwidth}}
{}
{ \\\hline
\multicolumn{1}{L{0.48\textwidth}}{\textbf{Característica}}
 & \multicolumn{1}{L{0.17\textwidth}}{\textbf{Coeficiente de regresión adj}}
 & \multicolumn{1}{L{0.20\textwidth}}{\textbf{IC95\%\allowbreak{\allowbreak{}\allowbreak{}}\allowbreak{}}}
 & \multicolumn{1}{L{0.10\textwidth}}{\textbf{p}} \\\hline 
\multicolumn{1}{L{0.48\textwidth}}{Aporte jubilatorio:\allowbreak{} no aporta ni le descuentan} & \multicolumn{1}{C{0.17\textwidth}}{1.\allowbreak{}068,\allowbreak{}21} & \multicolumn{1}{C{0.20\textwidth}}{300,\allowbreak{}35;1836,\allowbreak{}07} & \multicolumn{1}{C{0.10\textwidth}}{0,\allowbreak{}007} \\\hline \multicolumn{1}{L{0.48\textwidth}}{Indicador NBI condiciones de subsistência} & \multicolumn{1}{C{0.17\textwidth}}{145,\allowbreak{}18} & \multicolumn{1}{C{0.20\textwidth}}{12,\allowbreak{}56;277,\allowbreak{}79} & \multicolumn{1}{C{0.10\textwidth}}{0,\allowbreak{}032} \\\hline \multicolumn{1}{L{0.48\textwidth}}{Transporte público a menos de 300} & \multicolumn{1}{C{0.17\textwidth}}{103,\allowbreak{}06} & \multicolumn{1}{C{0.20\textwidth}}{-\allowbreak{}8,\allowbreak{}70;214,\allowbreak{}81} & \multicolumn{1}{C{0.10\textwidth}}{0,\allowbreak{}070} \\\hline \multicolumn{4}{p\textwidth}{\textsuperscript{ } \footnotesize{Fuente:\allowbreak{} Elaboración propia.\allowbreak{} Estudio de corte transversal (\allowbreak{}ECM 2008)\allowbreak{}\textsuperscript{e} y Censo Nacional de Población y Viviendas,\allowbreak{} INDEC,\allowbreak{} 2001.\allowbreak{}\textsuperscript{g}}}
}
\begin{multicols}{2}
\par{}Con relación a la proporción de la población ocupada que no realiza aportes jubilatorios,\allowbreak{} fue posible observar dos corredores con áreas más claras y una periferia que queda delimitada fuera de estos,\allowbreak{} de zonas oscuras (\allowbreak{}Figura 2,\allowbreak{} A)\allowbreak{}.\allowbreak{} Existió una concentración de fracciones censales cuya población no realiza ni le descuentan aportes jubilatorios.\allowbreak{} Se observó el mismo patrón de distribución en todos los partidos de la región,\allowbreak{} i.\allowbreak{}e.\allowbreak{},\allowbreak{} áreas oscuras en las periferias de los partidos y fracciones de colores más claros en las áreas centrales donde,\allowbreak{} a su vez,\allowbreak{} se encuentra mayor proporción del abandono.\allowbreak{} En cuanto a NBI por capacidad de subsistencia (\allowbreak{}Figura 2,\allowbreak{} B)\allowbreak{},\allowbreak{} los porcentajes más elevados de hogares con esta carencia se concentraron en dos partidos de manera casi uniforme:\allowbreak{} Lomas de Zamora y Quilmes.\allowbreak{} Con respecto a la disponibilidad de transporte público a menos de 300 m de la vivienda (\allowbreak{}Figura 2,\allowbreak{} C)\allowbreak{},\allowbreak{} se observaron tres corredores con áreas más claras y una periferia que quedó delimitada fuera de estos,\allowbreak{} de zonas con menores porcentajes de disponibilidad de este servicio y donde la proporción de no-\allowbreak{}adherencia fue más elevada,\allowbreak{} predominantemente en el área de Lomas de Zamora que limita con la Capital Federal,\allowbreak{} en el partido de Quilmes,\allowbreak{} principalmente en la zona que limita con Lomas de Zamora y Almirante Brown y en la mayor parte del partido de Ezeiza.\allowbreak{}\par{}
\par
{
\centering{
\includegraphics[width=\maxwidth{0.5\textwidth}]{not-found.png}
}
\captionof{figure}{\textbf{Figura 2:} \textit{Distribución de los indicadores del modelo de regresión múltiple,\allowbreak{} y la proporción de abandono.\allowbreak{} Municipios seleccionados de la Región Sanitaria VI,\allowbreak{} por fracción censal.\allowbreak{} Buenos Aires,\allowbreak{} Argentina,\allowbreak{} 2001.\allowbreak{}}} 
}
\par

\section*{DISCUSION}
\par{}Las áreas con mayor proporción de población que no realizaba aportes jubilatorios,\allowbreak{} ni le descontaban aportes jubilatorios,\allowbreak{} presentaron proporción más elevada de no-\allowbreak{}adherencia al tratamiento.\allowbreak{} Esta situación también se observó en las áreas con mayor proporción de hogares con NBI según capacidad de subsistencia y en las áreas con mayor proporción de viviendas que no contaban con un servicio de transporte público a menos de 300 m de la vivienda.\allowbreak{} El modelo final incluyó esta última variable,\allowbreak{} aún cuando la misma tuvo significancia estadística próxima al nivel de significancia (\allowbreak{}p = 0,\allowbreak{}05)\allowbreak{} dado que es la única variable vinculada a las barreras de accesibilidad.\allowbreak{} Por otra parte,\allowbreak{} el modelo que incluye solo los dos primeros indicadores (\allowbreak{}no realiza ni le descuentan aportes jubilatorios y NBI según capacidad de subsistencia)\allowbreak{} no difiere en cuanto a la bondad de ajuste del modelo que incluye también la variable relativa al transporte.\allowbreak{}\par{}Los estudios observacionales del tipo ecológico son importantes en el diagnóstico de salud de una población,\allowbreak{} en especial cuando el territorio es analizado de forma exploratoria en la verificación de un padrón de distribución espacial de determinado evento en salud.\allowbreak{}\textsuperscript{\textsuperscript{15}}\textsuperscript{-\allowbreak{}}\textsuperscript{\textsuperscript{23}}\par{}Rose\textsuperscript{\textsuperscript{19}} declaró que dos aspectos deben ser considerados en cuanto a la etiología de los problemas de salud:\allowbreak{} las causas de los casos entre los individuos y los determinantes de las tasas de enfermedad entre las poblaciones.\allowbreak{} En este sentido,\allowbreak{} si bien son fundamentales las estrategias para la prevención del riesgo individual y la protección de los individuos susceptibles de abandonar el tratamiento,\allowbreak{} la identificación de los factores determinantes del abandono entre las poblaciones adquiere particular relevancia para el control de la enfermedad.\allowbreak{}\textsuperscript{\textsuperscript{23}} El análisis de la variabilidad de los riesgos a nivel ecológico es fundamental para la comprensión de los determinantes sociales del proceso salud-\allowbreak{}enfermedad a la vez que posibilita investigar la hipótesis de que la distribución de la no-\allowbreak{}adherencia en un área se relaciona a las condiciones de vida.\allowbreak{}\textsuperscript{\textsuperscript{23}}\par{}La capacidad de finalizar o no el tratamiento es influenciada por las condiciones de vida del territorio donde viven los pacientes con TB.\allowbreak{} La proporción de no-\allowbreak{}adherencia fue más elevada en las áreas con mayor proporción de viviendas que no contaban con un servicio de transporte público a menos de 300 m de las mismas,\allowbreak{} lo que indica dificultades vinculadas al acceso y la movilidad por de la población.\allowbreak{} Este indicador,\allowbreak{} podría constituir un \textit{proxy} de otras características del área vinculadas a la disponibilidad de recursos y servicios.\allowbreak{} Se observaron mayor proporción de abandono en áreas con bajos niveles de existencia de agua de red y de calles pavimentadas.\allowbreak{} La razón por la cual estos indicadores no fueron incluidos en el modelo final puede deberse al número reducido de los casos adherentes y no adherentes del estudio.\allowbreak{}\par{}Los resultados indican mayor proporción de abandono en áreas con mayor proporción de viviendas con condiciones precarias y con menor nivel de recursos como es el caso de las viviendas que no poseen nevera o que tienen inodoro sin descarga o que no cuentan con inodoro.\allowbreak{} Estos resultados indican un nivel socioeconómico más bajo en estas áreas.\allowbreak{}\par{}La asociación entre nivel socioeconómico y la no-\allowbreak{}adherencia al tratamiento de la TB ha sido analizada en distintos países y regiones.\allowbreak{}\textsuperscript{\textsuperscript{6}}\textsuperscript{,\allowbreak{}}\textsuperscript{\textsuperscript{8}}\textsuperscript{,\allowbreak{}}\textsuperscript{\textsuperscript{10}}\textsuperscript{,\allowbreak{}}\textsuperscript{\textsuperscript{16}}\textsuperscript{,\allowbreak{}}\textsuperscript{\textsuperscript{22}} El hecho de que el tratamiento para TB sea gratuito en nuestro estudio sugiere que factores distintos al costo del tratamiento están determinando la no-\allowbreak{}adherencia de los pacientes con bajo nivel socioeconómico.\allowbreak{}\par{}Hay otras características de la población asociadas a una mayor proporción del abandono.\allowbreak{} Estas se relacionan con mayor situación de vulnerabilidad socioeconómica en los hogares,\allowbreak{} con una mayor precariedad laboral,\allowbreak{} con menores niveles de educación formal del jefe de hogar y con menor proporción de ocupados por hogar.\allowbreak{} Aquellas áreas que presentaban mayor porcentaje de hogares con cuatro o más personas por miembro ocupado y cuyo jefe no haya completado tercer grado de escolaridad primaria (\allowbreak{}NBI según capacidad de subsistencia)\allowbreak{} tuvieron mayor proporción de no-\allowbreak{}adherencia.\allowbreak{} Este indicador constituye un \textit{proxy} del nivel de ingresos del hogar según la cantidad de miembros ocupados con relación a la totalidad de los miembros que lo integran.\allowbreak{} A su vez,\allowbreak{} este indicador también mide la carencia de los bienes y servicios materiales requeridos para vivir y funcionar como miembro de la sociedad basado en una concepción de la pobreza como “necesidad”.\allowbreak{}\protect\footnote{ Feres JC,\allowbreak{} Mancero X.\allowbreak{} Enfoques para la medición de la pobreza.\allowbreak{} Breve revisión de la literatura.\allowbreak{} Santiago:\allowbreak{} CEPAL; 2001.\allowbreak{}} Estos resultados coinciden con otros estudios que apuntan que un bajo nivel de ingresos en los hogares se relaciona con peores resultados en salud.\allowbreak{}\protect\footnote{ Acosta LSW.\allowbreak{} O mapa de Porto Alegre e a tuberculose:\allowbreak{} distribuição espacial e determinantes sociais [dissertação de mestrado].\allowbreak{} Porto Alegre (\allowbreak{}RS)\allowbreak{}:\allowbreak{} Faculdade de Medicina da UFRGS; 2008.\allowbreak{}} Por otra parte,\allowbreak{} este indicador de NBI contempla el nivel educativo del jefe del hogar.\allowbreak{} Estudios que incluyen el nivel educativo en su análisis han hallado una asociación estadísticamente significativa con la adherencia en las áreas con una población con menor nivel de educación.\allowbreak{}\protect\footnote{ Acosta LSW.\allowbreak{} O mapa de Porto Alegre e a tuberculose:\allowbreak{} distribuição espacial e determinantes sociais [dissertação de mestrado].\allowbreak{} Porto Alegre (\allowbreak{}RS)\allowbreak{}:\allowbreak{} Faculdade de Medicina da UFRGS; 2008.\allowbreak{}} Se observó un aumento de la proporción de no-\allowbreak{}adherencia en áreas cuya población tiene primaria incompleta.\allowbreak{} Diversos estudios han demostrado que la educación puede influenciar las prácticas de salud de una población tanto a través de su asociación con el nivel ingreso,\allowbreak{} las condiciones de empleo,\allowbreak{} como a través de su asociación con el nivel de conocimiento que la población tiene sobre dichas prácticas.\allowbreak{}\textsuperscript{\textsuperscript{21}}\par{}Las áreas con mayor proporción de individuos con empleos sin protección social tienen mayor porcentaje de no-\allowbreak{}adherencia.\allowbreak{} Diversos estudios abordan la influencia de las condiciones laborales de los pacientes con la no-\allowbreak{}adherencia a los tratamientos.\allowbreak{}\textsuperscript{\textsuperscript{17}} Así,\allowbreak{} el empleo reduce la capacidad de cumplir con el tratamiento en contextos de alta informalidad y de bajo nivel de ingresos sin protección social,\allowbreak{} ya que para los pacientes implica la pérdida de días laborales y por ende de ingresos básicos,\allowbreak{} como muestra la investigación de Balasubramanian et al.\allowbreak{}\textsuperscript{\textsuperscript{2}} En el estudio de Galiano \&\allowbreak{\allowbreak{}\allowbreak{}}\allowbreak{} Montesinos,\allowbreak{}\textsuperscript{\textsuperscript{11}} el mayor abandono del tratamiento también estuvo asociado a la condición de ser varón,\allowbreak{} empleado y sin protección social.\allowbreak{}\par{}Una limitación de este estudio es que se utilizaron datos del Censo Nacional de Población y Viviendas del año 2001,\allowbreak{}\protect\footnote{ Según el Instituto Nacional de Estadísticas y Censos (\allowbreak{}INDEC)\allowbreak{} el Índice de Privación Material de Hogares,\allowbreak{} es una variable que identifica a los hogares según su situación respecto a la privación material en cuanto a dos dimensiones,\allowbreak{} recursos corrientes y patrimoniales.\allowbreak{} Respecto del índice de Necesidades Básicas Insatisfechas,\allowbreak{} los hogares con NBI son los hogares que presentan al menos uno de los siguientes indicadores de privación:\allowbreak{} hacinamiento (\allowbreak{}más de tres personas por cuarto)\allowbreak{}; vivienda (\allowbreak{}habitan en una vivienda de tipo inconveniente [pieza de inquilinato,\allowbreak{} pieza de hotel o pensión,\allowbreak{} casilla,\allowbreak{} local no construido para habitación o vivienda móvil],\allowbreak{} excluyendo casa,\allowbreak{} departamento y rancho)\allowbreak{}; condiciones sanitarias (\allowbreak{}no tienen ningún tipo de retrete)\allowbreak{}; asistencia escolar (\allowbreak{}tienen al menos un niño en edad escolar [6 a 12 años] que no asiste a la escuela)\allowbreak{}; capacidad de subsistencia (\allowbreak{}tienen cuatro o más personas por miembro ocupado,\allowbreak{} cuyo jefe no haya completado el tercer grado de escolaridad primaria)\allowbreak{}.\allowbreak{}} puesto que no estaban disponibles los indicadores socioeconómicos actualizados para el período en que se evaluó la no-\allowbreak{}adherencia de los casos.\allowbreak{} Este es el primer estudio que se lleva a cabo en Argentina con el propósito de detectar patrones espaciales de distribución de las características sociodemográficas y socioeconómicas de la región y su relación con el abandono del tratamiento.\allowbreak{} Los avances tecnológicos en el ámbito de los sistemas de información geográfica (\allowbreak{}SIG)\allowbreak{} permiten la incorporación de la estructura espacial de las variables,\allowbreak{} lo que añade la ventaja de analizar el problema como un fenómeno social particularizado en las condiciones socioeconómicas y sociodemográficas.\allowbreak{}\textsuperscript{\textsuperscript{15}}\textsuperscript{,\allowbreak{}}\textsuperscript{\textsuperscript{23}} El uso de SIG y el análisis de datos secundarios (\allowbreak{}censales)\allowbreak{},\allowbreak{} para la identificación de áreas y las condiciones relevantes para el proceso de no-\allowbreak{}adherencia al tratamiento de la tuberculosis,\allowbreak{} constituyen un instrumento útil para la vigilancia basado en el territorio,\allowbreak{} identificando los grupos de población prioritarios.\allowbreak{}\par{}El aumento de la proporción de no-\allowbreak{}adherencia al tratamiento de la TB en la RSVI está relacionada con vivir en áreas con una mayor proporción de población ocupada que no realiza aportes jubilatorios,\allowbreak{} mayor proporción de hogares con necesidades básicas insatisfechas según capacidad de subsistencia y con mayor proporción de viviendas que no cuentan con servicio de transporte público a menos de 300 m de la vivienda.\allowbreak{} Nuestros resultados llevan a delinear un área de riesgo para la no-\allowbreak{}adherencia al tratamiento,\allowbreak{} caracterizada por tener mayor proporción de población que vive en condiciones de pobreza y precariedad laboral,\allowbreak{} con dificultades de acceso al servicio de transporte público.\allowbreak{} Esta información puede resultar en la adopción de medidas más adecuadas para el tratamiento de los casos que viven en estas áreas y reducir el riesgo de abandono.\allowbreak{}

\medskip\par\noindent
{}Los autores declaran no haber conflicto de intereses.\allowbreak{}
\begin{biblio}[REFERENCES]
\bibliotitle{.\allowbreak{}Arrossi S,}\allowbreak{} Herrero MB,\allowbreak{} Greco A,\allowbreak{} Ramos S.\allowbreak{} Factores predictivos de la no-\allowbreak{}adherencia al tratamiento de la tuberculosis en municipios del área metropolitana de Buenos Aires,\allowbreak{} Argentina.\allowbreak{} \textit{Salud Colectiva.\allowbreak{} }2012;8(\allowbreak{}1)\allowbreak{}:\allowbreak{}65-\allowbreak{}76.\allowbreak{} DOI:\allowbreak{}10.\allowbreak{}1590\fshyp{}S1851-\allowbreak{}82652012000300012
\bibliotitle{.\allowbreak{}Balasubramanian VN,}\allowbreak{} Oommen K,\allowbreak{} Samuel R.\allowbreak{} DOT or not? Direct observation of anti-\allowbreak{}tuberculosis treatment and patient outcomes,\allowbreak{} Kerala State,\allowbreak{} India.\allowbreak{} \textit{The Int J Tuberc Lung Dis.\allowbreak{}} 2000;4(\allowbreak{}5)\allowbreak{}:\allowbreak{}409-\allowbreak{}13.\allowbreak{}
\bibliotitle{.\allowbreak{}Barata }RB.\allowbreak{} Epidemiologia social.\allowbreak{} \textit{Rev Bras Epidemiol.\allowbreak{} }2005;8(\allowbreak{}1)\allowbreak{}:\allowbreak{}7-\allowbreak{}17.\allowbreak{} DOI:\allowbreak{}10.\allowbreak{}1590\fshyp{}S1415-\allowbreak{}790X2005000100002
\bibliotitle{.\allowbreak{}Braga JU,}\allowbreak{} Pinheiro JS,\allowbreak{} Matsuda JS,\allowbreak{} Barreto JAP,\allowbreak{} Feijão AMM.\allowbreak{} Fatores associados ao abandono do tratamento da tuberculose nos serviços de atenção básica em dois municípios brasileiros,\allowbreak{} Manaus e Fortaleza,\allowbreak{} 2006 a 2008.\allowbreak{} \textit{Cad Saude Colet.\allowbreak{}} 2012;20(\allowbreak{}2)\allowbreak{}:\allowbreak{}225-\allowbreak{}33.\allowbreak{}
\bibliotitle{.\allowbreak{}Brasil PE,}\allowbreak{} Braga JU.\allowbreak{} Meta-\allowbreak{}analysis of factors related to health services that predict treatment default by tuberculosis patients.\allowbreak{} \textit{Cad Saude Publica.\allowbreak{}} 2008;24(\allowbreak{}4)\allowbreak{}:\allowbreak{}485-\allowbreak{}502.\allowbreak{} DOI:\allowbreak{}10.\allowbreak{}1590\fshyp{}S0102-\allowbreak{}311X2008001600003
\bibliotitle{.\allowbreak{}Cáceres FM,}\allowbreak{} Orozco LC.\allowbreak{} Incidencia y factores asociados al abandono del tratamiento antituberculoso.\allowbreak{} \textit{Rev Biomedica}.\allowbreak{} 2007;27(\allowbreak{}4)\allowbreak{}:\allowbreak{}498-\allowbreak{}504.\allowbreak{} DOI:\allowbreak{}10.\allowbreak{}7705\fshyp{}biomedica.\allowbreak{}v27i4.\allowbreak{}170
\bibliotitle{.\allowbreak{}Center }for Diseases Control.\allowbreak{} Core curriculum on tuberculosis:\allowbreak{} What the clinician should know.\allowbreak{} 4.\allowbreak{} ed.\allowbreak{} Atlanta:\allowbreak{} U.\allowbreak{}S.\allowbreak{} Department of Health \&\allowbreak{\allowbreak{}\allowbreak{}}\allowbreak{} Human Services; 2000.\allowbreak{}
\bibliotitle{.\allowbreak{}Culqui DR,}\allowbreak{} Grijalva CG,\allowbreak{} Reategui SR,\allowbreak{} Cajo JM,\allowbreak{} Suárez LA.\allowbreak{} Factores pronósticos del abandono del tratamiento antituberculoso en una región endémica del Perú.\allowbreak{} \textit{Rev Panam Salud Publica.\allowbreak{} }2005;18(\allowbreak{}1)\allowbreak{}:\allowbreak{}14-\allowbreak{}20.\allowbreak{} DOI:\allowbreak{}10.\allowbreak{}1590\fshyp{}S1020-\allowbreak{}49892005000600003
\bibliotitle{.\allowbreak{}Diez }Roux AV.\allowbreak{} A glossary for multilevel analysis.\allowbreak{} \textit{J Epidemiol Community Health.\allowbreak{} }2002;56(\allowbreak{}8)\allowbreak{}:\allowbreak{}588-\allowbreak{}94.\allowbreak{} DOI:\allowbreak{}10.\allowbreak{}1136\fshyp{}jech.\allowbreak{}56.\allowbreak{}8.\allowbreak{}588
\bibliotitle{.\allowbreak{}Galdós-\allowbreak{}Tangüis H,}\allowbreak{} Caylá JÁ,\allowbreak{} García de Olalla P,\allowbreak{} Jansá JM,\allowbreak{} Brugal MT.\allowbreak{} Factors predicting non-\allowbreak{}completion of tuberculosis treatment among HIV-\allowbreak{}infected patients in Barcelona (\allowbreak{}1987-\allowbreak{}1996)\allowbreak{}.\allowbreak{} \textit{Int J Tuberc Lung Dis.\allowbreak{} }2000;4(\allowbreak{}1)\allowbreak{}:\allowbreak{}55-\allowbreak{}60.\allowbreak{}
\bibliotitle{.\allowbreak{}Galiano M,}\allowbreak{} Montesinos N.\allowbreak{} Modelo predictivo de abandono del tratamiento antituberculoso para la región Metropolitana de Chile.\allowbreak{} \textit{Enferm Clin.\allowbreak{} }2005;15(\allowbreak{}4)\allowbreak{}:\allowbreak{}192-\allowbreak{}8.\allowbreak{} DOI:\allowbreak{}10.\allowbreak{}1016\fshyp{}S1130-\allowbreak{}8621(\allowbreak{}05)\allowbreak{}71111-\allowbreak{}6
\bibliotitle{.\allowbreak{}Gonçalves MJ,}\allowbreak{} Leon AC,\allowbreak{} Penna ML.\allowbreak{} A multilevel analysis of tuberculosis associated factors.\allowbreak{} \textit{Rev Salud Publica (\allowbreak{}Bogota)\allowbreak{}.\allowbreak{}} 2009;11(\allowbreak{}6)\allowbreak{}:\allowbreak{}918-\allowbreak{}30.\allowbreak{}
\bibliotitle{.\allowbreak{}Herrero MB,}\allowbreak{} Greco A,\allowbreak{} Ramos S,\allowbreak{} Arrossi S.\allowbreak{} Del riesgo individual a la vulnerabilidad social:\allowbreak{} factores asociados a la no-\allowbreak{}adherencia al tratamiento de la tuberculosis.\allowbreak{} Revisión de la literatura.\allowbreak{} \textit{Rev Argent Salud Publica.\allowbreak{} }2011;2(\allowbreak{}8)\allowbreak{}:\allowbreak{}36-\allowbreak{}42.\allowbreak{}
\bibliotitle{.\allowbreak{}Krieger }N.\allowbreak{} Glosario de epidemiología social.\allowbreak{} \textit{Rev Panam Salud Publica.\allowbreak{} }2002;11(\allowbreak{}5-\allowbreak{}6)\allowbreak{}:\allowbreak{}480-\allowbreak{}90.\allowbreak{} DOI:\allowbreak{}10.\allowbreak{}1590\fshyp{}S1020-\allowbreak{}49892002000500028
\bibliotitle{.\allowbreak{}Lima MLC,}\allowbreak{} Ximenes RA,\allowbreak{} Souza ER,\allowbreak{} Luna CF,\allowbreak{} Albuquerque MFPM.\allowbreak{} Análise espacial dos determinantes socioeconômicos dos homicídios no estado no Pernambuco.\allowbreak{} \textit{Rev Saude Publica.\allowbreak{} }2005;39(\allowbreak{}2)\allowbreak{}:\allowbreak{}176-\allowbreak{}82.\allowbreak{} DOI:\allowbreak{}10.\allowbreak{}1590\fshyp{}S0034-\allowbreak{}89102005000200006
\bibliotitle{.\allowbreak{}Mishra P,}\allowbreak{} Hansen E,\allowbreak{} Sabroe S,\allowbreak{} Kafle K.\allowbreak{} Socio-\allowbreak{}economic status and adherence to tuberculosis treatment:\allowbreak{} a case-\allowbreak{}control study in a district of Nepal.\allowbreak{} \textit{Int J Tuberc Lung Dis.\allowbreak{} }2005;9(\allowbreak{}10)\allowbreak{}:\allowbreak{}1134-\allowbreak{}9.\allowbreak{}
\bibliotitle{.\allowbreak{}Nene B,}\allowbreak{} Jayant K,\allowbreak{} Arrossi S,\allowbreak{} Shastri S,\allowbreak{} Budukh A,\allowbreak{} Hingmire S,\allowbreak{} et al.\allowbreak{} Determinants of women’s participation in cervical cancer screening trial,\allowbreak{} Maharashtra,\allowbreak{} India.\allowbreak{} \textit{Bull World Health Organ}.\allowbreak{} 2007;85(\allowbreak{}4)\allowbreak{}:\allowbreak{}264-\allowbreak{}72.\allowbreak{} DOI:\allowbreak{}10.\allowbreak{}2471\fshyp{}BLT.\allowbreak{}06.\allowbreak{}031195
\bibliotitle{.\allowbreak{}O’Boyle S,}\allowbreak{} Power J,\allowbreak{} Ibrahim M,\allowbreak{} Watson J.\allowbreak{} Factors affecting patient compliance with anti-\allowbreak{}tuberculosis chemotherapy using the directly observed treatment short-\allowbreak{}course strategy (\allowbreak{}DOTS)\allowbreak{}.\allowbreak{} \textit{Int J Tuberc Lung Dis.\allowbreak{} }2002;6(\allowbreak{}4)\allowbreak{}:\allowbreak{}307-\allowbreak{}12.\allowbreak{}
\bibliotitle{.\allowbreak{}Rose }G.\allowbreak{} Sick individuals and sick populations.\allowbreak{} \textit{Int J Epidemiol.\allowbreak{} }2001;30(\allowbreak{}3)\allowbreak{}:\allowbreak{} 427-\allowbreak{}32.\allowbreak{} DOI:\allowbreak{}10.\allowbreak{}1093\fshyp{}ije\fshyp{}30.\allowbreak{}3.\allowbreak{}427
\bibliotitle{.\allowbreak{}Sabaté }E.\allowbreak{} Adherencia a los tratamientos a largo plazo:\allowbreak{} pruebas para la acción.\allowbreak{} Geneva:\allowbreak{} Organización Mundial de la Salud; Organización Panamericana de la Salud; 2004.\allowbreak{}
\bibliotitle{.\allowbreak{}Sabates R,}\allowbreak{} Feinstein L.\allowbreak{} The role of education in the uptake of preventative health care:\allowbreak{} the case of cervical screening in Britain.\allowbreak{} \textit{Soc Sci Med.\allowbreak{} }2006;62(\allowbreak{}12)\allowbreak{}:\allowbreak{} 2998-\allowbreak{}3010.\allowbreak{} DOI:\allowbreak{}10.\allowbreak{}1016\fshyp{}j.\allowbreak{}socscimed.\allowbreak{}2005.\allowbreak{}11.\allowbreak{}032
\bibliotitle{.\allowbreak{}Sosa Pineda N,}\allowbreak{} Pereira S,\allowbreak{} Barreto M.\allowbreak{} Abandono del tratamiento de la tuberculosis en Nicaragua:\allowbreak{} resultados de un estudio comparativo.\allowbreak{} \textit{Rev Panam Salud Publica.\allowbreak{} }2005;17(\allowbreak{}4)\allowbreak{}:\allowbreak{}271-\allowbreak{}8.\allowbreak{} DOI:\allowbreak{}10.\allowbreak{}1590\fshyp{}S1020-\allowbreak{}49892005000400008
\bibliotitle{.\allowbreak{}Souza WV,}\allowbreak{} Albuquerque MFPM,\allowbreak{} Barcelos CC,\allowbreak{} Ximenes RA,\allowbreak{} Carvalho MS.\allowbreak{} Tuberculose no Brasil:\allowbreak{} construção de um sistema de vigilância de base territorial.\allowbreak{} \textit{Rev Saude Publica}.\allowbreak{} 2005;39(\allowbreak{}1)\allowbreak{}:\allowbreak{}82-\allowbreak{}89.\allowbreak{} DOI:\allowbreak{}10.\allowbreak{}1590\fshyp{}S0034-\allowbreak{}89102005000100011
\bibliotitle{.\allowbreak{}Ximenes RAA,}\allowbreak{} Albuquerque MFPM,\allowbreak{} Souza WV,\allowbreak{} Montarroyos UR,\allowbreak{} Diniz GT,\allowbreak{} Luna CF.\allowbreak{} Is it better to be rich in a poor area or poor in a rich area? A multilevel analysis of a case-\allowbreak{}control study of social determinants of tuberculosis.\allowbreak{} \textit{Int J Epidemiol.\allowbreak{}} 2009;38(\allowbreak{}5)\allowbreak{}:\allowbreak{}1285-\allowbreak{}96.\allowbreak{}
\bibliotitle{.\allowbreak{}World }Health Organization.\allowbreak{} Treatment of tuberculosis:\allowbreak{} guidelines for national programmes.\allowbreak{} 3.\allowbreak{} ed.\allowbreak{} Geneva; 2003.\allowbreak{}
\end{biblio}

\end{multicols}
